%TRETTONDE KAPITLET.



\chapter[Varelsers ömsesidiga slägtskap]{Organiska varelsers ömsesidiga slägtskap. Morfologi
Embryologi. Rudimentära organer.}

{\it
Klassifikation, grupper underordnade andra grupper. — Naturligt system. — Reglor och svårigheter i klassifikation förklarade enligt teorien om härstamning med modifikation. — Klassifikation af varieteter. — Härstamning alltid använd i klassifikation. — Analoga karakterer. — Affiniteter, allmänna, sammansatta, radierande. — Arters utdöende skiljer grupper och begränsar dem. — Morfologi, emellan medlemmar af samma klass, emellan delar af samma individ. — Embryologi, dess lagar förklarade genom variationernas inträffande vid en icke tidig period, och deras ärftlighet i motsvarande period. — Rudimentära organer; deras ursprung. — Sammanfattning.
}\\[0.5cm]


\section{Klassifikation.}

Sedan den mest aflägsna period i jordens historia hafva de organiska varelserna liknat hvarandra i aftagande grader, så att de kunna ordnas i grupper och undergrupper. Denna anordning är icke godtycklig som stjernornas gruppering i konstellationer. Gruppernas tillvaro skulle varit af enkel betydelse, om en grupp varit uteslutande lämpad att bebo landet och en annan vattnet, en att lefva af kött, en annan af vegetabilier och så vidare; men förhållandet är helt olika i naturen, ty det är bekant, huru allmänt medlemmar af samma grupp, till och med samma undergrupp hafva skilda lefnadsvanor. I andra och fjerde kapitlet om variation och det naturliga urvalet har jag försökt visa, att i hvarje land det är de vidt utbredda, de mycket spridda och allmänna, det är de dominerande arterna af de större slägtena i hvarje klass som variera mest. Varieteterna eller de begynnande arterna, som på detta sätt alstrats, blifva slutligen förvandlade i nya och skilda arter, och dessa skola enligt principen om ärftlighet sträfva att frambringa andra nya och dominerande former. De grupper som nu äro stora och i allmänhet innehålla många dominerande arter sträfva alltjemt att tilltaga i storlek. Jag försökte vidare att visa, att hos afkomlingarna af hvarje art finnes en benägenhet för karaktersdivergens, beroende derpå att de försöka intaga så många och skilda platser som möjligt i naturens ekonomi. Denna sista slutsats erhåller vidare stöd, om vi observera den stora olikhet i former, som på en liten yta komma i täflan, äfvensom genom vissa förhållanden vid naturalisering.

Jag försökte äfven att visa, att det finnes en konstant tendens hos de former som tilltaga i antal och divergera i karakter att uttränga och utrota de föregående mindre divergenta och mindre förädlade formerna. Jag beder läsaren återgå till det schema som illustrerar verkan af dessa olika grundsatser, såsom vi förut förklarat, och han skall se att det oundvikliga resultatet är att de modifierade ättlingarna som härstamma från en stamfar afdelas i grupper, subordinerade under andra grupper. I schemat må hvarje bokstaf på öfversta linien föreställa ett slägte innefattande flera arter; och summan af slägtena längs denna öfre linie bilda tillsammans en klass, ty alla härstamma från en uråldrig stamfar och hafva följaktligen ärft någonting gemensamt. Men de tre slägtena på venstra sidan hafva enligt samma grundsats mycket gemensamt och bilda en underfamilj, skild från den som innehåller de två slägtena närmast till höger, hvilka divergerade från en gemensam stamfar vid femte härstamningsstadiet. Dessa fem slägten hafva också mycket gemensamt, ehuru mindre; och de bilda en familj skild från den som innehåller de tio slägtena ännu längre åt höger, hvilka divergerade vid en ännu tidigare period. Och alla dessa slägten som härstamma från A bilda en ordning skild från de slägten som utgått från I. Här hafva vi således många arter härstammande från en enda stamfar grupperade i slägten och slägtena i flockar, familjer och ordningar, alla i en stor klass. Den stora lagen om alla organiska varelsers naturliga sammanförande i grupper subordinerade under andra grupper, hvilken då vi äro så förtrogna dermed icke förefaller oss egendomlig, förklaras lätt enligt mina åsigter. Otvifvelaktigt kunna organiska varelser likt alla andra föremål klassifieras på många vis, antingen artificielt efter enstaka karakterer eller naturligt enligt ett större antal karakterer. Vi veta till exempel att mineralier och de elementära ämnena kunna på detta sätt anordnas; i detta fall finnes intet sammanhang med den genealogiska successionen, och ingen orsak kan ännu uppgifvas för deras sammanfallande i grupper. Men för organiska varelser är förhållandet helt olika och de ofvan anförda åsigterna förklara deras naturliga anordnande i grupper och undergrupper och någon annan förklaring har hittills ej blifvit försökt.

Naturhistorikerna försöka såsom vi hittills hafva sett att anordna arterna, slägtena och familjerna i hvarje klass uti hvad man kallar naturligt system. Men hvad menas med detta system? Några författare betrakta det blott såsom ett schema för att ordna tillsammans de lefvande föremål som äro mest lika och skilja dem som äro mest olika, eller såsom ett konstgjordt sätt att så kort som möjligt uttala vissa satser, — det vill säga att till exempel i ett ord uppgifva alla de för däggdjur gemensamma karakterer, i ett annat de som äro gemensamma för alla rofdjur, i ett annat de som gemensamt tillhöra hundslägtet och slutligen genom att tillägga ett artnamn gifva i korthet en fullständig beskrifning på hvarje hundart. Det sinrika och nyttiga i ett sådant system är obestridligt. Men många naturforskare anse att något mera menas med ett naturligt system; de tro att det uppenbarar skaparens plan; men såvida icke det kan särskildt uppgifvas huruvida ordning i tid eller rum, eller båda, eller hvad som annars menas med skaparens plan synes mig att dermed ingenting är vunnet för vår kännedom. Sådana uttryck som Linnés ryktbara yttrande, hvilket vi ofta träffa på i mer eller mindre beslöjad form, att karaktererna icke bilda slägtet utan slägtet karaktererna, synes innebära att någonting mera än blott likhet inbegripes i våra klassifikationer. Jag tror att någonting mera deri innefattas, och att likhet i härkomst — den enda kända orsak till likhet emellan organiska varelser är det band som ehuru doldt genom olika grader af modifikation våra klassifikationer uppenbara för oss.

Låt oss nu betrakta de reglor som följas i klassifikationen och de svårigheter som påträffas enligt åsigten att klassifikationen antingen antyder någon okänd skapelseplan, eller är helt enkelt ett schema för allmänna satser och för att sammanhålla former som mest likna hvarandra. Man kunde tro (och man har i fordna tider trott) att de delar i skapnaden, som bestämde lefnadsvanorna och hvarje varelses allmänna plats i naturens ekonomi, skulle vara af stor vigt vid klassificeringen. Ingenting kan vara oriktigare. Ingen betraktar den yttre likheten emellan en råtta och en näbbmus, en dugong och en hvalfisk, en hvalfisk och en fisk såsom af någon vigt. Dessa likheter, som äro så intimt förenade med varelsens hela lif upptagas blott såsom ”adaptiva eller analoga karakterer;” men till betraktelsen af dessa likheter skola vi återkomma. Det kan äfven uppgifvas såsom en allmän regel, att ju mindre en del af organisationen står i öfverensstämmelse med särskilda vanor, ju vigtigare blir den för klasifikationen. Ett exempel: då Owen talar om dugongen (Halicore) säger han: ”Generationsorganerna, hvilkas sammanhang med ett djurs vanor och näringsämnen är så ringa, har jag alltid betraktat såsom lemnande mycket klara indikationer för dess verkliga slägtskap. I dessa organers modifikationer äro vi minst benägna att anse en blott adaptiv karakter såsom väsentlig.” Huru märkvärdigt är det icke att bland växterna vegetationsorganerna, på hvilka deras nutrition och lif beror, hafva så liten betydelse, hvaremot reproduktionsorganerna, med deras produkter, frön och embryoner äro af så stor vigt! Då vi förut afhandlade morfologiska olikheter utan fysiologisk vigt, hafva vi sett, att de ofta äro af den högsta betydelse i klassifikationen. Detta beror på deras beständighet igenom många beslägtade grupper; och beständigheten beror hufvudsakligen på att några små bildningsafvikelser i sådana delar icke blifvit skyddade och förökade genom naturligt urval, som verkar blott på nyttiga karakterer.

Att den blotta fysiologiska vigten af ett organ icke bestämmer dess klassifikatoriska värde, bevisas nästan af det faktum, att i beslägtade grupper, i hvilka samma organ, såsom vi hafva skäl att antaga, har nästan samma fysiologiska värde, dess klassifikatoriska värde är betydligt olika. Ingen naturforskare har kunnat arbeta i en grupp utan att öfverraskas af detta förhållande, och det har blifvit fullständigt erkändt i skrifter af nästan hvarje författare. Det bör vara tillräckligt att citera den största auktoriteten Robert Brown, som då han talar om vissa organer hos Proteaceæ säger att deras generiska betydelse, ”liksom alla deras organers icke blott i denna utan i hvarje naturlig familj är mycket olika och i några fall tyckes helt och hållet gå förlorad.” I ett annat arbete säger han att slägtena af Connaraceæ ”skilja sig deruti att de hafva ett eller flera fruktämnen, i närvaron eller frånvaron af fröhvita, i blombladens läge i knoppen. Några af dessa karakterer äro helt enkelt ofta af mera generisk betydelse, ehuru de äfven här om alla tagas tillsammans synas otillräckliga att skilja Cnestis från Connarus.” Jag vill gifva ett exempel bland insekterna: i en större afdelning af Hymenoptera äro antennerna såsom Westwood anmärkt mest konstanta i bildning; i en annan afdelning visa de betydliga skiljaktigheter, och olikheterna äro af underordnadt värde för klassifikationen; dock kan ingen säga att antennerna i dessa två afdelningar af samma ordning äro af olika fysiologisk vigt. En mängd exempel kunde gifvas på den varierande betydelsen för klassifikationen af samma vigtiga organ inom samma grupp af varelser.

Vidare kan ingen säga att rudimentära eller atrofierade organer äro af hög fysiologisk eller vital betydelse; dock äro otvifvelaktigt organer af denna beskaffenhet ofta af stort värde för klassifikationen. Ingen vill bestrida, att de rudimentära tänderna i unga idislares öfverkäk och vissa rudimentära ben i underbenet äro af stor vigt vid utrönandet af den nära slägtskapen emellan Ruminantia och Pachydermata. Robert Brown har strängt fasthållit vid det faktum att läget af de förkrympta blommorna hos gräsen äro af största betydelse för deras klassifikation.

Talrika exempel kunde gifvas på karakterer, som hemtas från delar af mycket obetydlig fysiologisk vigt, men hvilka i allmänhet medgifvas vara af stor betydelse för begränsningen af hela grupper; till exempel, om passagen emellan näsborrarna och munnen är öppen, den enda karakter som enligt Owen skiljer fiskar och reptilier, invikningen af underkäken hos Marsupialia, det sätt hvarpå insekternas vingar äro hoplagda, blotta färgen hos vissa alger, blotta hårigheten på blomdelar hos gräsen, beskaffenheten af hudbetäckning såsom hår och fjädrar hos ryggradsdjuren. Om Ornithorhynchus blifvit upptäckt med fjädrar i stället för hår, skulle denna yttre och obetydliga karakter af naturhistorikerna betraktas som en vigtig hjelp för bestämmandet af denna sällsamma varelses grad af slägtskap med fåglarna.

Obetydliga karakterers vigt för klassifikationen beror hufvudsakligen derpå att de genom vexelverkan stå i samband med flera andra karakterer af mer eller mindre betydelse. Värdet af en samling karakterer är mycket uppenbar i naturalhistorien. Derföre kan såsom ofta blifvit anmärkt en art afvika från sina slägtingar i flera karakterer både af hög fysiologisk vigt och nästan allmän giltighet utan att derföre lemna oss i något tvifvel, hvar den bör ställas i systemet. Derföre har man också funnit att en klassifikation som är grundad på en enstaka karakter alltid slagit fel, huru vigtig denna karakter än må vara, ty ingen del af organisationen är oföränderligen konstant. Betydelsen af en samling karakterer, äfven om ingen är i sig sjelf vigtig förklarar allena Linnés aforism, att karaktererna icke bilda slägtet, utan slägtet karaktererna; ty denna synes grundad på en uppskattning af många obetydliga likhetsförhållanden som äro för små att definieras. Vissa växter af familjen Malpighiaceæ bära både fullkomliga och förkrympta blommor; i de senare såsom A. de Jussieu har anmärkt ”försvinna större delen af de karakterer som äro egna för arten, för slägtet, för familjen, för klassen, och gäcka sålunda vår klassificering.” Men om Aspicarpa i Frankrike under flera års tid satte blott ofullständiga blommor som i en del af de vigtigaste karakterer skilde sig så underbart från ordningens egentliga typ, insåg dock Richard skarpsinnigt, såsom Jussieu anför, att slägtet ändock skulle qvarstå i familjen Malpighiaceæ. Detta exempel synes mig väl belysa andan i våra klassifikationer.

I praxis då våra naturforskare arbeta, bekymra de sig icke om de karakterers fysiologiska betydelse, hvilka de begagna vid definierandet af en grupp eller inregistrerandet af någon särskild art. Om de finna en nästan likformig karakter, som är gemensam för ett stort antal former och saknas hos andra, begagna de den som en karakter af stor vigt, om den är gemensam för ett mindre antal begagna de den såsom af underordnadt värde. Denna grundsats har blifvit i hela sin vidd erkänd af några naturhistoriker såsom sann, och af ingen tydligare än af A. S:t Hilaire. Om vissa karakterer alltid finnas förenade med andra, ehuru intet synbart föreningsband kan upptäckas emellan dem, sättes på dem ett särskildt värde. Då i de flesta djurgrupper vigtiga organer, såsom cirkulations- och respirationsorganerna eller fortplantningsorganerna befinnas vara nästan likformiga, betraktas de såsom af stor nytta vid klassifiering, men i några djurgrupper befinnas alla dessa de vigtigaste vitala organer erbjuda karakterer af underordnadt värde. Såsom Fritz Müller nyligen iakttagit är i samma grupp af karakterer Cypridina försedd med hjerta, under det i två beslägtade arter, Cypris och Cytherea intet sådant organ finnes; en art af Cypridina har väl utvecklade gälar under det andra äro i saknad af dem.

Vi kunna inse hvarföre karakterer hemtade från embryot äro äro af lika vigt som de som hemtas från den fullväxta, ty en naturlig klassificering inbegriper naturligtvis alla åldrar. Men det är ingalunda tydligt enligt den vanliga åsigten, hvarföre embryots skapnad är mera vigtigt för detta ämne än den fullväxtas, som allena spelar sin rol i naturens ekonomi. Dock hafva utmärkta naturforskare Milne Edvards och Agassiz påstått, att embryologiska karakterer äro de vigtigaste af alla, och denna lära har allmänt antagits såsom sann. Ickedestomindre har deras vigt stundom blifvit öfverskattad; för att visa detta anordnade Fritz Müller med sådana karakterers tillhjelp den stora klassen Crustacea och anordningen blef icke naturlig. Men det kan icke betviflas att karakterer hemtade från embryot i allmänhet äro af högre vigt icke blott bland djur utan äfven bland växter. De två stora afdelningarna af blommande växter äro sålunda grundade på olikheter hos embryot — på hjertbladens läge och antal samt utvecklingssättet för rotämnet och stamknoppen. Vi skola strax se, hvarföre dessa karakterer ega så stor betydelse i klassifikationen, nämligen emedan det naturliga systemet är genealogiskt i sin anordning.

Våra klassifikationer stå ofta helt och hållet under inflytande af slägtskapsserier. Ingenting kan vara lättare än att bestämma ett antal karakterer gemensamma för alla fåglar, men för krustaceerna har hittills en sådan bestämning befunnits omöjlig. Det finnes krustaceer i motsatta ändar af serien, hvilka hafva knappt en karakter gemensam, dock kunna arterna i båda ändar, från sin fullkomliga slägtskap med andra, dessa åter med andra och så vidare, igenkännas såsom obestridligen tillhörande denna klass och icke någon annan klass af Articulata.

Geografisk fördelning har ofta ehuru kanske icke fullt logiskt begagnats i klassifikation, isynnerhet i stora grupper af närbeslägtade former. Temminck framhåller nyttan eller till och med nödvändigheten häraf för vissa fågelgrupper, och den har blifvit följd af vissa ornitologer och botanister.

Slutligen hvad beträffar det komparativa värdet af de olika artgrupperna såsom ordningar, underordningar, familjer, underfamiljer och slägten, synas de vara åtminstone för det närvarande fullkomligt godtyckliga. Flera af de bästa botanister, sådana som Bentham, och andra, hafva starkt framhållit deras godtyckliga värde. Exempel kunna gifvas ibland växter och insekter på formgrupper som af erfarna naturforskare först upptagits som blott ett slägte och sedan uppstigit till rang af underfamilj eller familj; och detta har skett, icke emedan vidare forskningar hafva upptäckt vigtiga bildningsafvikelser, som först blifvit förbisedda, utan emedan talrika beslägtade arter med obetydligt olika skiljaktighetsgrad sedermera hafva blifvit upptäckta.

Alla föregående reglor och svårigheter vid klassifikationen förklaras, om jag icke misstager mig alltför mycket, enligt den åsigten att ett naturligt system är grundadt på härstamning med modifikation; att de karakterer, som naturforskare anse såsom utvisande verklig slägtskap emellan två eller flera arter, äro de som gått i arf från en gemensam stamfader, i det alla verkliga klassifikationer äro genealogiska; att gemensamheten i härkomst är det dolda band som naturforskarna omedvetet sökt och icke någon okänd skapelseplan, eller uttalandet af allmänna satser eller blotta sammanställandet eller skiljandet af mer eller mindre lika ämnen.

Men jag måste förklara hvad jag menar fullständigare. Jag tror att anordnandet af grupper inom hvarje klass i tillräcklig subordination och relation till hvarandra måste vara strängt genealogisk för att vara naturlig, men att graden af olikhet i de skilda grenarna eller grupperna, ehuru förenade i samma grad af blodsband med sin gemensamma stamfader, kan vara betydligt olika beroende på de olika grader af modifikation de hafva undergått, och detta uttryckes derigenom att formerna upptagas under skilda slägten, familjer, sektioner eller ordningar. Läsaren skall bäst förstå hvad jag menar om han gör sig mödan att återgå till schemat i fjerde kapitlet. Vi skola antaga att bokstäfverna A till L föreställa närstående slägten under den siluriska tiden, som härstamma från någon tidigare form. I tre af dessa slägten (A, F och I) hafva arterna efterlemnat modifierade ättlingar till närvarande dag, förestälda genom femton slägten (a${}^{14}$—z${}^{14}$) på den öfversta horizontala linien. Alla dessa modifierade ättlingar af en enda art äro beslägtade genom blodsband eller härstamning i samma grad; de kunna bildlikt kallas kusiner i samma millionde grad; dock skilja de sig betydligt och i olika grad från hvarandra. De från A härstammande formerna, som nu äro afdelade i två eller tre familjer, utgöra en skild ordning från dem som utgått från I, också afdelade i två familjer. Ej heller kunna de nu lefvande arterna, som härstamma från A upptagas i samma slägte som stamarten A, ej heller afkomlingarna af I tillsammans med I. Men det nu lefvande slägtet F${}^{14}$ kan antagas hafva blifvit blott obetydligt modifieradt, och det står då på samma grad som stamslägtet F, alldeles som några få nu lefvande organismer tillhöra siluriska slägten. Graden eller värdet af olikheterna emellan dessa organiska varelser, hvilka alla äro beslägtade med hvarandra i samma grad af blodsband har kommit att blifva betydligt olika. Icke destomindre förblifver deras genealogiska anordning noggrant sann icke blott i närvarande tid, utan äfven vid hvarje successiv period i härstamning. Alla de modifierade ättlingarna af A hafva ärft någonting gemensamt från deras gemensamma stamfader, likasom äfven alla ättlingarna af I, så måste äfven förhållandet vara med hvarje subordinerad gren af ättlingar på hvarje successiva stadium. Om vi likväl antaga att någon af afkomlingarna från A eller I blifvit så mycket modifierad, att den förlorat alla spår af sitt fäderne, i detta fall har äfven dess plats i systemet gått förlorad — något som tyckes hafva inträffat med några få nu lefvande organismer. Alla afkomlingarna af slägtet F utefter hela härstamningslinien antagas hafva blifvit blott litet modifierade och bilda ett enda slägte. Men detta slägte ehuru mycket isoleradt, bör ännu bibehålla sin ursprungliga intermediära belägenhet. Denna naturliga anordning visas i schemat så mycket möjligt är på papper, men på allt för enkelt sätt. Om vi icke begagnat ett förgrenadt schema utan blotta namnen på grupperna i rätliniga serier, skulle det varit ännu mindre möjligt att hafva gifvit en naturlig anordning och det är som bekant icke möjligt att i en serie på en slät yta representera de slägtskaper vi i naturen upptäcka emellan varelser af samma grupp. Enligt den åsigt jag vidhåller är det naturliga systemet i sin anordning genealogiskt liksom ett stamträd, men graden af modifikation som de olika grupperna undergått måste uttryckas genom att ordna dem i skilda så kallade slägten, underfamiljer, familjer, sektioner, ordningar och klasser.

Det kan vara mödan värdt att belysa denna åsigt om klassifieringen genom att jemföra den med språken. Om vi egde ett fullständigt stamträd för menniskoslägtet, skulle en genealogisk anordning af menniskoraserna lemna den bästa klassifikation af de olika språk som nu talas öfver jorden; och om alla utdöda tungomål och alla intermediära och långsamt vexlande dialekter deri blefvo inbegripna, så vore detta den enda möjliga anordning. Dock kan det hända, att något gammalt språk förändrats blott litet och gifvit upphof till några få nya tungomål, under det andra förändrat sig betydligt alltefter spridning, isolering och civiliseringsgraden hos de samtidigt utbildade menniskoraserna och på detta sätt gifvit upphof till många nya dialekter och tungomål. De olika skiljaktighetsgraderna emellan språken af samma stam, skulle uttryckas genom grupper och undergrupper, men den bästa eller till och med enda möjliga anordning skulle vara den genealogiska, och den skulle vara strängt naturlig, då den skulle sammanbinda alla språk, utdöda och nya, genom den närmaste slägtskap och den skulle angifva härstamning och ursprung för hvarje tungomål.

För att bekräfta dessa åsigter låt oss se på varieteternas klassifiering hvilka antagas härstamma från en enda art. Dessa äro grupperade under arter, med undervarieteter, och i några fall såsom husdufvorna behöfvas flera andra skiljaktighetsgrader. Nästan samma regler följas som vid arternas klassifiering. Författare hafva framhållit nödvändigheten af att ordna varieteter i ett naturligt i stället för artificielt system; vi varnas till exempel att icke sammanföra två varieteter af ananas blott af den orsak, att deras frukter, ehuru en vigtig del, råka vara lika; ingen ställer tillsammans den svenska rofvan och den vanliga, ehuru den ätliga, förtjockade stammen är så lika. Hvilken del som helst som befinnes vara mest konstant begagnas till att klassifiera varieteterna: den stora landbrukaren Marshall säger att hornen hos boskapen äro mycket nyttiga för detta ändamål, emedan de äro mindre föränderliga än kroppens skapnad eller färg etc. hvaremot hornen hos får äro mindre användbara emedan de äro mindre beständiga. Vid varieteters klassifikation förmodar jag att om vi hade ett verkligt stamträd, en genealogisk klassifikation i allmänhet skulle föredragas: och det har blifvit försökt i några fall. Ty vi kunna vara säkra på, att om också större eller mindre modifikation egt rum, grundsatsen om arf skulle hålla de former tillsammans som vore beslägtade i största antalet karakterer. Hos tumletter, ehuru några af undervarieteterna afvika i den vigtiga karakteren, att de hafva längre näbb, föras dock alla tillsammans derföre att de hafva den gemensamma vanan att tumla; men den korthöfdade rasen har nästan helt och hållet förlorat denna vana; utan afseende på denna olikhet föras icke destomindre dessa tumletter till samma grupp; emedan de äro beslägtade genom blodsband och lika i några andra hänseenden.

För arter i naturtillståndet har hvarje naturforskare i sjelfva verket intagit härstamning i sin klassifiering; ty han inbegriper i sin lägsta grad, arten, båda könen, och huru ofantligt dessa stundom afvika från hvarandra i karakterer, är bekant för hvarje naturhistoriker; knappt en enda karakter kan visas gemensam för de fullväxta kannarna och hermafroditerna af vissa cirripeder och dock drömmer ingen om att skilja dem åt. Så snart det blef kändt att de tre orchideerna Monachanthus, Myanthus och Catasetum, hvilka förut upptogos såsom tre skilda slägten, stundom alstrades på samma växt, betraktades de omedelbarligen såsom varieteter; och nu har jag varit i stånd att visa, att de äro maskulina, feminina och hermafroditiska former af samma art. Naturforskaren inbegriper under en art de olika larvstadierna af samma individ, huru mycket de än må skilja sig från hvarandra och den fullväxta, äfvensom Steenstrups ”vexlande generationer” som blott i en konstlad betydelse kunna betraktas såsom samma individ. Han inberäknar monstrositeter och varieteter icke på grund af deras delvisa likhet med stamformen utan emedan de härstamma derifrån.

Då härstamning i allmänhet begagnats vid sammanförandet af alla individer af samma art, ehuru hannar och honor och larver stundom är ytterligt olika, och då den begagnats att klassificera varieteterna, som undergått en viss stundom ansenlig grad af modifikation, kan då icke samma härstamningselement hafva omedvetet begagnats vid grupperingen af arter i slägten och slägten i högre grupper, allt under det så kallade naturliga systemet? Jag tror att det omedvetet begagnats; och på detta sätt allena kan jag förstå alla de reglor och hänvisningar som blifvit följda af våra bästa systematiker. Vi hafva inga skrifna stamträd; vi måste leta ut den gemensamma härstamningen ur likheter af hvarje slag. Vi välja derföre de karakterer, hvilka så vidt vi kunna döma, kunna antagas hafva minst varierat efter de lifsvilkor för hvilka hvarje art nyligen varit utsatt. Rudimentära bildningar äro från denna synpunkt lika goda, eller till och med stundom bättre än andra delar af organisationen. Vi bekymra oss icke om huru obetydlig en karakter må vara — till exempel invikningen på underkäkens vinkel, det sätt hvarpå insekters vingar hopläggas, om huden är betäckt med hår eller fjäder — om den är förherskande bland många och skilda arter, isynnerhet sådana som hafva mycket skilda lefnadsvanor, antager den stor betydelse; ty vi kunna förklara dess närvaro hos så många former med så skilda vanor blott genom arf från en gemensam stamfar. Vi kunna misstaga oss i detta hänseende i vissa moment, men om flera karakterer låt dem vara aldrig så obetydliga sammanträffa inom en stor grupp af varelser med skilda vanor, kunna vi enligt härstamningsteorien vara nästan säkra, att dessa karakterer gått i arf från en gemensam stamfader. Och vi veta att sådana af hvarandra beroende karakterer hafva särskild betydelse för klassifieringen. Vi kunna förstå hvarföre en art eller en grupp af arter kan i flera af sina vigtigaste karakterer afvika från sina slägtingar och dock med visshet sammanföras med dem. Detta kan ske utan fara, och sker ofta, så länge ett tillräckligt antal karakterer, låt dem vara aldrig så ovigtiga förråder, det dolda föreningsbandet, gemensam härkomst. Låt också två former hafva icke en enda karakter gemensam, om dessa extrema former förenas genom en kedja af intermediära grupper kunna vi med ens antaga deras gemensamma härkomst och vi föra dem alla till samma klass. Då vi finna organer af stor fysiologisk vigt — de som tjena till lifvets bevarande under de mest olika existensvilkor — i allmänhet vara de mest konstanta, fästa vi synnerlig betydelse vid dem; men om samma organer i en annan grupp eller sektion af en grupp befinnas visa stora skiljaktigheter, värdera vi dem med ens mindre för vår klassifikation. Vi skola se, hvarföre embryologiska karakterer äro af så stor vigt för systemet. Geografisk fördelning kan stundom vara nyttig för klassifikationen af stora slägten, emedan alla arter af samma slägte, som bebo någon skild och isolerad trakt efter all sannolikhet härstamma från samma föräldrar.



\section{Analoga likheter.}

Vi kunna förstå enligt ofvan utvecklade åsigter den mycket vigtiga skilnaden emellan verklig slägtskap och analoga eller adaptiva likheter. Lamarck var den första som fäste uppmärksamheten på denna skilnad och har följts af Macleay och andra. Likheten i kroppens skapnad och i de fenlika främre extremiteterna emellan dugongen som är en pachyderm och hvalen, och emellan båda dessa däggdjur och fiskarna är analogi. Bland insekter finnas härpå talrika exempel; missledd genom yttre utseende upptog Linné en homopter insekt ibland malarna. Vi se äfven någonting likartadt hos våra domesticerade varieteter, såsom i de förtjockade stammarna hos den vanliga och den svenska rofvan. Likheten emellan en vindthund och en rashäst är svårligen mera imaginär än de analogier några författare hafva uppsökt emellan vidt skilda djur. Enligt min åsigt, att karakterer äro af verklig vigt för klassifikationen blott så vida de uppenbara härstamning, kunna vi tydligen inse, hvarföre analoga eller adaptiva karakterer, ehuru af den största vigt för varelsens bestånd, äro nästan värdelösa för systematikerna. Ty djur som höra till två fullständigt skilda härstamningslinier kunna utan svårighet hafva blifvit lämpade för likartadt lefnadssätt och på detta sätt antagit en stor yttre likhet; men sådana likheter skola icke uppenbara utan snarare dölja deras blodsförvandtskap. Vi kunna på detta sätt också förstå den skenbara paradoxen, att just samma karakterer äro analoga, då en klass eller en ordning jemföres med en annan, hvilka visa verklig slägtskap, då medlemmarna af samma klass eller ordning jemföras med hvarandra: kroppens skapnad och de fenlika lemmarna äro blott analoga, om hvalar jemföras med fiskar, ty de äro i båda fallen uppkomna genom att djuren af båda klasserna gjorts lämpliga för simning i vatten; men kroppsformen och de fenlika extremiteterna tjena såsom karakterer utvisande verklig slägtskap emellan de olika medlemmarna af hvalfamiljen; ty dessa cetaceer öfverensstämma i så många karakterer, stora och små, att vi icke kunna betvifla att de hafva ärft sin allmänna kroppsform och lemmarnas bygnad från en gemensam stamfar. Så äfven bland fiskar.

Det mest anmärkningsvärda fall af analog likhet, ehuru icke beroende på lämpande efter likartade lifsvilkor, är det som Bates anfört om vissa fjärilar i trakten af Amazonfloden, hvilka så likt härma andra arter. Denna utmärkta iakttagare visar, att i ett område, der till exempel en Ithomia öfverflödar i prunkande svärmar, finnes ofta en annan fjäril, Leptalis, blandad i samma flock och liknar så nära Ithomia i hvarje skugning och färgstrimma och till och med i vingarnas form, att Bates, som under elfva år skärpt sitt öga för insektsamling, oupphörligen gäckades, ehuru han var på sin vakt. Då härmarna och de härmade fångas och jemföras, befinnas de vara helt och hållet olika i väsentliga bildningar och tillhöra icke blott skilda arter, utan till och med skilda familjer. Om denna efterapning inträffat i blott ett eller två fall, kunde den hafva förbisetts såsom ett sällsamt sammanträffande. Men om vi utgå från ett område, der en Leptalis efterapar en Ithomia, finna vi en annan härmande och härmad art tillhörande samma slägten med fullkomligt samma likhet. Tillsammans uppräknas icke mindre än tio slägten, som innehålla arter hvilka härma andra fjärilar. Härmarna och de härmade bebo alltid samma område, vi finna aldrig en härmare lefva aflägset från den form som den härmar. Härmarna äro nästan utan undantag sällsynta insekter; de härmade finnas nästan i alla fall i öfverflöd. I samma område der en art af Leptalis härmar en Ithomia finnas stundom andra fjärilar som härma samma Ithomia, så att på samma plats arter af tre fjärilslägten och till och med en mal finnas, som alla fullkomligt likna en fjäril af ett fjerde slägte. Det förtjenar särskildt anmärkas att många af de härmande Leptalisformerna äfvensom af de härmade formerna, kunna visas genom en graderad serie vara blotta varieteter af samma art; under det andra äro otvifvelaktigt skilda arter. Men hvarföre, kan man fråga, behandlas vissa former såsom de härmade och andra såsom de härmande? Bates besvarar fullt tillfredsställande denna fråga genom att visa, att den form som härmas behåller det vanliga utseendet hos den grupp till hvilken den hör, under det härmarna hafva förändrat sitt utseende och icke likna sina närmaste slägtingar.

Vi föras härnäst till att undersöka, hvilket skäl möjligen kan angifvas till att vissa fjärilar och malar så ofta antaga en annan fullständigt skild forms drägt; hvarföre har naturen till naturforskarnas förvåning nedlåtit sig till skådespelarekonster? Bates har otvifvelaktigt träffat på den rätta förklaringen. De härmade formerna, som alltid finnas i riklig mängd, måste vanligen undgå en vidsträcktare ödeläggelse; i annat fall skulle de icke kunna existera i sådana svärmar och Bates såg dem aldrig anfallas af fåglar eller vissa stora insekter som angripa andra fjärilar. Han anser sig hafva skäl att tro, att denna immunitet beror på en egendomlig och skadlig lukt, som de afgifva. De härmande formerna deremot, som bebo samma distrikt, äro jemförelsevis sällsynta, och tillhöra sällsynta grupper; de måste följaktligen vara utsatta för någon fara, ty i annat fall att sluta af antalet ägg som alla fjärilar lägga skulle de i tre eller fyra generationer svärma öfver hela landet. Om nu en medlem af dessa förföljda och sällsynta grupper kunde antaga en drägt så lik en väl skyddad arts, att den beständigt gäckade en erfaren entomologs öga, så skulle den äfven ofta bedraga rofgiriga fåglar och insekter och på detta sätt undgå mycken ödeläggelse. Bates kan nästan sägas hafva i verkligheten bevitnat den process, hvarigenom härmarna hafva kommit att så nära likna de härmade; ty han fann att några af Leptalisformerna, som härma så många andra fjärilar, varierade i ytterlig grad. I ett distrikt funnos flera varieteter och af dessa liknade en allena till en viss utsträckning den vanliga Ithomia i samma distrikt. I ett annat område funnos två eller tre varieteter, af hvilka en var mycket allmännare än de andra och denna härmade en annan form af Ithomia. Från fakta af denna beskaffenhet sluter Bates, att Leptalis först varierar; och då en varietet råkar att likna i någon mån en vanlig fjäril som bebor samma område, har denna varietet i följd af dess likhet med en frodig och föga förföljd art större utsigt att undgå förstörelse af rofgiriga fåglar och insekter och blir följaktligen oftare skyddad; — ”de mindre fullkomliga graderna af likhet blifva generation efter generation utgallrade, och blott de andra qvarlemnas för att fortplanta sitt slägte”. Här hafva vi således ett förträffligt exempel på naturligt urval.

Wallace har nyligen beskrifvit flera liknande fall af härmning hos Lepidoptera på Malayiska arkipelagen, och andra exempel kunna gifvas från andra insektordningar. Wallace har också beskrifvit ett fall af härmning bland fåglar, men vi hafva inga sådana fall bland större fyrfotadjur. Det rikligare uppträdandet af härmningar bland insekter än andra djur är sannolikt en följd af deras ringa storlek; insekter kunna icke försvara sig med undantag af sådana arter som sticka, och jag har aldrig hört något exempel på att dessa härma andra insekter, hvaremot de ofta härmas; insekter kunna icke genom flygten undkomma de större djuren; derföre äro de liksom de flesta svagare varelser inskränkta till maskering och förstälning.

Men låt oss återgå till vanligare fall af analog likhet; då medlemmar af skilda klasser ofta blifvit lämpade för att lefva under nästan lika omständigheter genom successiva små modifikationer — att till exempel bebo de tre elementen land, vatten och luft — kunna vi måhända förstå anledningen hvarföre en numerisk öfverensstämmelse stundom observerats emellan undergrupperna i skilda klasser. En naturhistoriker öfverraskad af en öfverensstämmelse af denna beskaffenhet i en klass skall genom att godtyckligt höja eller sänka värdet af grupperna i andra klasser (och vår erfarenhet visar att deras värdering hittills är godtycklig) lätt utsträcka parallelismen öfver en vid rymd och på detta sätt hafva sannolikt de ternära, qvarternära, quinära och septenära klassifikationerna uppkommit.



\section[Beskaffenheten hos slägtskapen]{Om beskaffenheten af den slägtskap som förenar
organiska varelser.}

Då de modifierade ättlingarna af dominerande arter, som tillhöra de större slägtena, sträfva att ärfva de fördelar, som gjorde den grupp till hvilken de höra stor och deras föräldrar dominerande, skola de nästan med säkerhet sprida sig vidt omkring och intaga allt flera och flera platser i naturens ekonomi. De större och mera dominerande grupperna inom hvarje klass sträfva på detta sätt fortfarande att förstoras och uttränga följaktligen många mindre och svagare grupper. Vi kunna på detta sätt förklara det förhållande att alla organismer nya och gamla innehållas i några få stora ordningar och under ännu färre klasser. Såsom bevisande huru få de högre grupperna äro till antal och huru vidsträckt de sprida sig öfver jorden är den omständigheten anmärkningsvärd, att upptäckten af Australien icke har lemnat någon insekt af någon ny klass, och att växtriket derigenom ökats blott med två eller tre familjer.

I kapitlet om geologisk succession försökte jag enligt den grundsatsen, att hvarje grupp i allmänhet divergerat mycket i karakter under den länge fortsatta modifikationsprocessen, visa orsaken hvarföre de äldre lifsformerna ofta erbjuda karakterer till en viss grad intermediära emellan nu lefvande grupper. Några få af dessa gamla och intermediära former som hafva till närvarande tid lemnat föga modifierade afkomlingar utgöra våra så kallade osculerande eller aberranta arter. Ju mera aberrant en form är, ju större måste antalet af förenande former vara som dött ut och helt och hållet gått förlorade. Och vi hafva några bevis på att aberranta grupper hafva lidit svårt af utdöende, ty de äro nästan alltid representerade af ytterst få arter, och de arter som finnas äro i allmänhet mycket olika hvarandra, hvilket förutsätter samma orsak. Slägtena Ornithorhynchus och Lepidosiren till exempel skulle icke varit mindre aberrant, om hvart och ett representerats af ett dussin arter i stället för en enda, eller en eller två. Vi kunna tror jag förklara detta förhållande blott genom att betrakta aberranta grupper såsom former, hvilka blifvit besegrade af lyckligare medkämpar, men af hvilka ett fåtal medlemmar ännu äro skyddade under ovanligt gynsamma vilkor.

Waterhouse har anmärkt att om en medlem af en djurgrupp företer slägtskap med en fullkomligt skild grupp, denna slägtskap i de flesta fall är allmän och icke speciel: enligt Waterhouse är sålunda af alla gnagare bizcacha mest närbeslägtad med marsupialia, men i de delar i hvilka den närmar sig denna ordning äro dess relationer allmänna och icke större till någon art af pungdjuren än till hvarje annan. Då slägtskapen anses vara verklig och icke blott adaptiv, måste den enligt vår åsigt bero på arf från en gemensam stamfader. Derföre måste vi antaga, antingen att alla gnagare jemte bizcacha afgrenat sig från någon uråldrig pungdjurart, hvilken af naturen varit mer eller mindre intermediär i karakter emellan gnagare och alla nu existerande pungdjur; eller också att både gnagare och pungdjur afgrenat sig från någon gemensam stam och att båda grupperna sedan undergått mycken modifikation i divergenta riktningar. Enligt båda åsigterna måste vi antaga att bizcachan genom arf bibehållit mera af sin gamla stamfars karakter än de öfriga gnagarna och derföre bör den icke vara specielt befryndad med några nu lefvande pungdjur utan indirekt med alla eller nästan alla marsupialia, då den delvis behållit karakteren af den gemensamma stamfadern eller någon tidigare medlem af gruppen. Å andra sidan liknar enligt Waterhouse af alla pungdjur Phascolomys närmast icke någon särskild gnagareart utan hela gnagareordningen. I detta fall kan likväl förmodas, att likheten är blott analog, beroende på att Phascolomys blifvit lämpad för vanor liknande gnagarnas. Den äldre de Candolle har gjort nästan liknande iakttagelser öfver den allmänna beskaffenheten af vissa växtfamiljers slägtskapsförhållanden.

Enligt grundsatsen om förökning och gradvis karaktersdivergens af de arter som härstamma från gemensamt ursprung i förening med vissa karakterers bibehållande genom arf, kunna vi förstå de ytterligt invecklade och radierande slägtskapsförhållanden, genom hvilka alla medlemmar af samma familj eller högre grupp äro förenade. Ty den gemensamma stamfadern för en hel familj, numera afdelad genom vissa medlemmars undergång i skilda grupper och undergrupper, måste hafva öfverflyttat några af sina karakterer på olika sätt och i olika grad modifierade på alla arter, och de skola följaktligen vara befryndade med hvarandra genom slägtskapslinier af olika längd (hvilket kan ses i det så ofta omnämda schemat) som uppstiga genom ett stort antal föregångare. Då det är så svårt att visa blodsförvandtskapen emellan de olika anförvandterna af någon gammal adelsfamilj äfven med tillhjelp af genealogiska stamträd och nästan omöjligt utan sådana, kunna vi lätt fatta den stora svårighet naturforskarna måst vidkännas vid att utan schema beskrifva de olika slägtskapsförhållandena som de iakttaga emellan de många lefvande och utdöda medlemmarna af samma stora naturliga klass.

Organiska varelsers undergång har såsom vi sett i fjerde kapitlet spelat en vigtig rol vid begränsandet och utvidgandet af mellanrummen emellan de olika grupperna af hvarje klass. Vi kunna således förklara hela klassers afgränsande från hvarandra — till exempel fåglar från alla vertebrerade djur — med det antagandet att många gamla lifsformer blifvit totalt förlorade, genom hvilka fåglarnas ursprungliga stamfäder fordom voro förenade med de andra och vid denna tid mindre differentierade vertebrerade klasserna. En mindre tillintetgörelse har skett bland de former, som en gång förenade fiskarna med batrachierna; den har varit ännu mindre i några andra klasser såsom Crustacea, ty här äro ännu de underbaraste former sammanlänkade genom en lång och blott delvis afbruten slägtskapskedja. Förintelsen har blott skilt grupperna åt; den har ingalunda bildat dem; ty om hvarje form som någonsin lefvat på jorden plötsligt återuppträdde, skulle ännu en naturlig klassifiering eller åtminstone ett naturligt anordnande vara möjligt, ehuru det vore fullkomligt omöjligt att gifva definitioner, genom hvilka hvarje grupp kunde skiljas. Vi skola se detta genom att återvända till schemat: bokstäfverna A till L kunna föreställa elfva siluriska slägten, af hvilka några hafva bildat stora grupper af modifierade ättlingar, med hvarje länk i hvarje gren och undergren ännu lefvande, och öfvergångarna äro icke större än emellan de finaste varieteter. I detta fall skulle det vara fullkomligt omöjligt att gifva definitioner hvarigenom alla medlemmarna i de olika grupperna kunde skiljas från sina mera omedelbara slägtingar och afkomlingar. Anordningen i schemat skulle dock ännu förblifva gällande och vara naturlig, ty enligt principen om ärftlighet skulle alla former som härstamma från A till exempel hafva någonting gemensamt. I ett träd kunna vi urskilja den eller den grenen, ehuru vid den verkliga delningen de två förena sig och sammanflyta. Vi kunna icke, såsom jag har sagt, bestämma de olika grupperna; men vi kunna utvisa typer eller former som representera de flesta af hvarje grupps karakterer, vare sig stora eller små, och på detta sätt få en allmän föreställning om värdet af olikheterna emellan dem. Härtill skulle vi komma om vi lyckades någonsin att samla alla former i någon klass, som lefvat i all tid och på hvarje punkt af jorden. Säkert är att vi aldrig skola lyckas få en så fullständig samling; icke desto mindre sträfva vi i vissa klasser till detta mål, och Milne Edvards har nyligen i en nätt afhandling framhållit den stora vigten af att fasthålla typerna, om vi också icke kunna skilja och begränsa de grupper till hvilka sådana typer höra.

Vi hafva slutligen sett att det naturliga urvalet, som härrör af kampen för tillvaron och som nästan oundvikligen leder till tillintetgörelse och karaktersdivergens hos afkomlingarna af en dominerande stamart, förklarar detta stora och allmänna drag i alla organiska varelsers slägtskapsförhållanden, nämligen deras subordination i grupp under grupp. Vi begagna härstamningselementet vid klassificering af individer af båda könen och alla åldrar under en art, ehuru de hafva blott få karakterer gemensamma; vi begagna härstamning vid klassificering af erkända varieteter, huru olika de än må vara sina stamformer; och jag tror att härstamning är det dolda föreningsbandet, som naturforskarna sökt under benämningen naturligt system. Enligt denna föreställning, att det naturliga systemet så vidt det är fullständigt är genealogiskt till anordning med graderna af olikhet uttryckta genom termerna slägte, familj, ordning m. m. kunna vi förstå de regler vi måste följa vid klassifikationen. Vi kunna inse hvarföre vi värdera vissa likheter vida mer än andra; hvarföre vi begagna rudimentära och gagnlösa organer eller andra af obetydlig fysiologisk betydelse; hvarföre vi vid sökandet efter förvandtskap emellan grupper helt sannolikt förkasta analoga eller adaptiva karakterer, och dock begagna desamma inom gränserna för hvarje grupp. Vi kunna klart inse orsaken till att alla lefvande och utdöda former kunna grupperas tillsammans inom några få stora klasser; och huru alla medlemmar af hvarje klass äro sammanbundna genom de mest invecklade och radierande slägtskapslinier. Vi skola sannolikt aldrig upplösa den outredliga väfnaden af förvandtskaper emellan medlemmarna af någon klass; men då vi hafva ett bestämdt mål i sigte, och icke söka någon okänd skapelseplan, kunna vi hoppas att göra säkra ehuru långsamma framsteg.

Professor Häckel har i sin ”Generelle Morphologie” och i flera andra arbeten nyligen användt sina vidsträckta kunskaper och sin stora förmåga på att uppställa hvad han kallar en fylogeni, eller alla organiska varelsers härstamningslinie. Vid de olika seriernas uppställning grundar han sig hufvudsakligen på embryologiska karakterer, men drager äfven nytta af homologa och rudimentära organer, äfvensom de successiva perioder i hvilka de olika formerna först uppträdde i våra geologiska formationer. Han har på detta sätt djerft tagit första steget och visat oss huru klassifikationen för framtiden bör behandlas.



\section{Morfologi.}

Vi hafva sett att medlemmar af samma klass, oberoende af deras lefnadsvanor likna hvarandra i allmän organisationsplan. Denna likhet betecknas ofta genom termen ”enhet i typ;” eller genom att säga, att delar och organer hos olika slägten af en klass äro homologa. Hela ämnet innefattas under den allmänna benämningen morfologi; det är den intressantaste afdelning af naturalhistorien och kan rent af sägas vara dess själ. Hvad kan vara besynnerligare än att en menniskohand, bildad till griporgan, en mullvads framfot, skapad för gräfning, hästens framben, delfinens fena och flädermusens vinge äro alla bildade efter samma mönster och innehålla samma ben i samma relativa läge? Geoffroy S:t Hilaire har starkt betonat den stora betydelsen af homologa delars relativa läge eller förening; de kunna vara olika till hvad grad som helst i form och storlek och förblifva dock förenade i samma oföränderliga ordning. Vi finna aldrig till exempel någon omflyttning af öfver- och underarmens ben eller af lårets och underbenets. Samma namn kunna derföre gifvas de homologa benen hos vidt skilda djur. Vi se samma stora lag i bildningen af insekternas munnar: hvad kan vara mera olika än den ofantligt långa spiralformiga snabeln hos en nattfjäril, den besynnerligt hopvikta snabeln hos ett bi eller en lus, och en skalbagges stora käkar? — dock äro alla dessa organer som tjena till helt olika ändamål bildade genom oändligt talrika modifikationer af en öfverläpp, mandibler och två par maxiller. Samma lag styr skapnaden af kräftdjurens mun och lemmar; den gäller äfven för växternas blommor.

Ingenting kan vara mera fruktlöst än att försöka förklara denna likhet i typ hos medlemmar af samma klass genom läran om ändamålsenlighet eller causæ finales. Hopplösheten af detta försök har Owen ytterligare medgifvit i sitt arbete ”Nature of Limbs.” Enligt den vanliga åsigten om hvarje varelses särskilda skapelse kunna vi blott säga att så är, — att det har behagat skaparen att bilda alla djur och växter i hvarje stor klass efter en likformig plan, men detta är icke någon vetenskaplig förklaring.

Förklaringen är uppenbar enligt teorien om urval af successiva små modifikationer, hvar och en på något sätt nyttig för den modifierade formen, men ofta genom vexelverkan inverkande på andra delar af organisationen. Förändringar af denna beskaffenhet torde hafva liten eller ingen benägenhet att förändra det ursprungliga mönstret eller att omflytta delarna. Benen i en extremitet kunna blifva korta och platta till hvad grad som helst och på samma gång invecklas i en tjock hinna, så att de kunna begagnas såsom fenor, eller en simfot kan få alla sina ben eller vissa af dem förlängda till hvad grad som helst, med den förenande hinnan förstorad, så att den kan begagnas till vinge; all denna modifikation behöfver dock icke förändra benens sammanställning eller delarnas relativa förening. Om vi antaga, att en uråldrig stamfader — architypen såsom den kan kallas — till alla däggdjur hade sina extremiteter bildade efter det allmänna mönstret, till hvad ändamål de än tjenade, kunna vi med ens fatta fulla betydelsen af lemmarnas homologa bildning genom hela klassen. Detsamma gäller insekternas munnar; vi behöfva blott antaga, att deras gemensamma stamfar hade en öfverläpp, mandibler och två par maxiller, låt vara mycket enkla till formen; och det naturliga urvalet skall förklara den oändliga vexlingen i skapnad och funktion bland insektmunnar. Icke desto mindre är det begripligt, att det allmänna mönstret för ett organ kan blifva så fördunkladt att det slutligen går förloradt genom reduktion och slutligen fullkomligt försvinnande af vissa delar, genom sammansmältning af några delar och genom fördubbling eller mångfaldigande af andra — förändringar hvilka vi veta ligga inom möjlighetens gränser. I de jättelika hafsödlornas fenor och i munnarna hos våra parasitiska krustaceer synes den allmänna bygnadsplanen delvis gått förlorad.

Det är en annan och lika egendomlig sida af vårt närvarande ämne, nämligen jemförelsen icke emellan samma delar eller organer hos olika medlemmar af samma klass, utan emellan olika delar och organer hos samma individ. De flesta fysiologer tro, att hufvudets ben äro homologa med — det är till antal och relativ förening motsvara — grunddelarna af ett visst antal kotor. De främre och bakre extremiteterna i den högre vertebratklassen äro homologa. Samma förhållande råder emellan krustaceernas underbart sammansatta käkar och ben. Det är väl bekant för nästan hvar och en, att hos en blomma det relativa läget af foderblad, kronblad, ståndare och pistill, äfvensom deras inre bygnad blifva lättast förklarade enligt åsigten att de bestå af metamorfoserade blad anordnade i spiral. Hos monströsa växter få vi ofta direkta bevis på möjligheten af ett organs öfvergång till ett annat, och vi kunna verkligen se under det tidiga eller embryonala utvecklingsstadiet hos blommor, äfvensom krustaceer och många andra djur, att organer, hvilka i fullmogen ålder äro ytterligt olika, i början äro fullkomligt lika.

Dessa fakta äro fullkomligt oförklarliga enligt den vanliga åsigten om en skapelse; hvarföre skulle hjernan inneslutas i en låda bildad af så talrika och så utomordentligt formade benstycken? Såsom Owen har anmärkt kan icke formen af skallen hos fåglar och reptilier förklaras genom den fördel som vinnes genom de olika delarnas eftergifvande under däggdjurens födelse. Hvarföre skulle likartade ben blifvit skapade för att bilda en flädermus vinge och bakben, då de begagnas för så olika ändamål? Hvarföre skulle en krustace, som har en ytterligt sammansatt mun, bildad af många delar, också hafva färre ben; eller tvärtom de med många ben hafva enklare mun? Hvarföre skulle foderblad kronblad, ståndare och pistiller i samma blomma, ehuru lämpade för så vidt skilda ändamål, alla vara bildade efter samma mönster?

Med teorien om naturligt urval kunna vi besvara alla dessa frågor. Hos ryggradsdjuren se vi en serie af inre kotor försedda med vissa utskott; hos leddjuren är kroppen delad i en serie af segment med yttre bihang, och hos blommande växter spiraler af blad. En oändlig upprepning af samma del eller organ är såsom Owen iakttagit en gemensam karakter hos alla låga eller obetydligt modifierade former; den okända stamfadern för vertebraterna egde otvifvelaktigt många ryggkotor; den okända stamfadern till leddjuren många segment, och det okända ursprunget för blommande växter många blad anordnade i en eller flera spiraler. Vi hafva förut sett att många gånger upprepade delar äro i hög grad benägna för att variera i antal och skapnad. Då sådana delar redan finnas och äro i hög grad föränderliga, skola de följaktligen lemna material för de mest olika ändamål, och de böra i allmänhet genom arf behålla tydliga spår af sin ursprungliga likhet.

I den stora klassen af blötdjur, ehuru det utan svårighet kan bevisas att delarna i skilda arter äro homologa, kunna blott få homologa serier angifvas; det vill säga, att vi äro sällan i stånd att afgöra om en del är homolog med en annan del hos samma individ. Och detta kunna vi förstå, ty hos blötdjuren, äfven de lägsta medlemmar af klassen, finna vi icke på långt när så mycken obestämd upprepning af något organ, såsom vi finna i de andra stora klasserna af djur- och växtriket.

Naturforskare tala ofta om skallen såsom bildad af förändrade kotor, kräftans käkar såsom förvandlade ben, ståndarna och pistillerna såsom metamorfoserade blad, men det vore i de flesta fall riktigare, att såsom Huxley angifvit tala om både skalle och kotor, både käkar och ben och så vidare såsom metamorfoserade icke från hvarandra såsom de nu äro, utan från något gemensamt och enklare ursprung. De flesta naturhistoriker begagna likväl ett sådant talesätt blott i figurlig betydelse; de mena dermed alldeles icke, att under en lång tid ursprungliga organer af något slag — vertebrer i ena fallet och ben i det andra — verkligen blifvit förvandlade till hufvudskålar eller käkar; naturforskarna kunna dock icke undgå att använda sådana talesätt, som hafva denna betydelse; så nära tyckes det motsvara det verkliga förhållandet. Enligt de här framstälda åsigterna kunna sådana talesätt begagnas efter orden, och det underbara faktum förklaras, att en krabbas käkar till exempel behålla många karakterer, som de sannolikt skulle bevarat genom arf, om de verkligen blifvit förvandlade från verkliga ehuru ytterst enkla ben.



\section{Utveckling och embryologi.}

Detta är ett af de vigtigaste ämnen i hela naturalhistorien. Insekternas metamorfoser, väl bekanta för hvar och en, försiggå i allmänhet plötsligt genom några få stadier, men transformationer äro i verkligheten talrika och gradvisa ehuru dolda. En viss efemer insekt (Chloeon) byter skinn under sin utveckling såsom sir J. Lubbock visat omkring tjugu gånger, och hvarje gång undergår den en viss grad af förändring; och i detta fall se vi metamorfoseringsakten försiggå på ett primärt och gradvist sätt. Många insekter och isynnerhet många krustaceer visa oss, hvilka underbara bildningsförändringar kunna försiggå under utvecklingen. Sådana förändringar nå likväl sin höjdpunkt uti några af de lägre djurens så kallade generationsvexling. Det är till exempel förvånande, att en fint förgrenad korall, beströdd med polyper och fästad vid en underhafsklippa först genom knoppning och sedan genom tvansversel delning alstrar en mängd simmande geleartade djur och att dessa alstra ägg ur hvilka kläckas simmande djur, som fästa sig vid klippor och utvecklas till förgrenade koraller, och så vidare i en oändlig cirkel. Tron på en väsentlig identitet emellan generationsvexling och vanlig metamorfos har fått betydligt stöd i Wagners upptäckt, att en larv af en fluga, Cecidomyia, på könlös väg alstrar andra liknande larver.

Det är redan visadt att olika delar och organer hos samma individ äro noggrant lika hvarandra under en tidig embryonalperiod, men i fullväxta tillståndet blifva de betydligt olika och tjena till andra ändamål. Det har vidare blifvit visat, att embryoner af de mest skilda arter inom samma klass i allmänhet äro nästan lika, men efter sin fulla utveckling blifva helt och hållet olika. Ett bättre bevis på det senare förhållandet kan icke gifvas än det af von Baer anförda, ”att embryoner af däggdjur, fåglar, ödlor och ormar och sannolikt äfven sköldpaddor äro i sina tidigaste stadier ytterligt lika hvarandra, både såsom hela och i hvarje dels utvecklingssätt, så lika, att vi i verkligheten ofta kunna skilja dem endast genom storleken. Jag har i min ego två små embryoner i sprit, hvilkas namn jag underlåtit att sätta ut, och för närvarande är det mig alldeles omöjligt att afgöra till hvilken klass de höra. De kunna vara ödlor eller små fåglar eller mycket unga däggdjur, så fullkomlig är likheten i hufvudets och bålens bildningssätt hos dessa djur. Extremiteterna saknas likväl ännu hos dessa embryoner, men äfven om de hade funnits i deras tidigaste utvecklingsstadium skulle vi icke lära någonting deraf, ty fötterna af ödlor och däggdjur, vingarna och fötterna af fåglar, så väl som händer och fötter hos menniskan utgå alla från samma grundform.” De masklika larverna af myggor, flugor, skalbaggar likna hvarandra i allmänhet mycket mera än de mogna insekterna; men i dessa fall äro embryonerna verksamma och afvika ofta betydligt från hvarandra derigenom att de blifvit lämpade efter särskilda lefnadssätt. Ett spår af lagen om embryonernas likhet qvarstår tillfälligtvis i en något senare ålder: fåglar af samma slägte och af närstående slägten likna ofta hvarandra i sin tidigaste fjäderbeklädnad såsom vi se uti de fläckiga fjädrarna hos ungar af trastgruppen. I kattfamiljen äro de flesta arter strimmiga eller fläckiga, och strimmor eller fläckar kunna tydligen varsnas hos ungarna af lejon och puma. Vi se stundom ehuru sällan någonting af detta slag hos växter; de första bladen hos Ulex och Genista och de annars fyllodiebärande nyholländska Acacierna äro fjädrade och delade liksom de vanliga bladen hos Leguminosæ.

De delar i hvilka embryoner af helt olika djur inom samma klass likna hvarandra, hafva ofta intet direkt samband med deras lifsvilkor. Vi kunna till exempel antaga, att hos embryoner af vertebrater arternas egendomliga bågiga förlopp nära branchialspringorna utbildas under liknande lifsvilkor — hos det unga däggdjuret som näres i moderns lif, i fågelägget som kläckes i nästet, och i grodrommen under vatten. Vi hafva icke mera skäl att tro på ett sådant samband än vi hafva att tro att liknande ben i menniskohanden, flädermusens vinge och delfinens fena stå i sammanhang med likartade lifsvilkor. Ingen antager att strimmorna hos en lejonunge eller fläckarna på den unga koltrasten äro af nytta för dessa djur.

Förhållandet är likväl olika om ett djur under någon del af sin embryonala bana är verksamt och måste sörja för sig sjelf. Verksamhetsperioden kan komma förr eller senare i lifvet, men närhelst den kommer är larvens lämplighet för dess lefnadsvilkor lika fullständig och vacker som det fullväxta djurets. På hvilket vigtigt sätt detta har verkat har nyligen blifvit väl visadt af sir J. Lubbock i hans anförande om den stora likheten emellan larver af några insekter, som tillhöra mycket olika ordningar, och om olikheten emellan larver af andra insekter inom samma ordning, allt efter deras lefnadsvanor. Från sådana förhållanden, isynnerhet om de innefatta en arbetsfördelning under de olika utvecklingsstadierna, såsom om samma larv under ett stadium måste söka föda, under ett annat en fästepunkt, blir likheten emellan larver af beslägtade djur stundom osynlig, och exempel kunna gifvas på larver af två artgruper, som skilja sig mera från hvarandra än de fullväxta. I de flesta fall följa likväl larverna, ehuru verksamma, mer eller mindre noga lagen för embryonernas allmänna likhet. Cirripederna lemna härpå ett godt exempel; till och med den ryktbare Cuvier insåg icke att de voro krustaceer, men en blick på larverna visar på ett omisskänneligt sätt, att detta är sant. De två hufvudafdelningarna af cirripederna, de skaftade och de sessila, så olika i yttre utseende, hafva dock larver som i alla stadier knappt kunna skiljas från hvarandra.

Under utvecklingens lopp stiger embryot i allmänhet i organisation. Jag begagnar detta uttryck, ehuru jag väl inser att det knappt är möjligt att med bestämdhet definiera hvad som menas med högre och lägre organisation. Men ingen vill väl bestrida att fjärilen är högre än kålmasken. I några fall måste likväl det fullväxta djuret anses såsom lägre i skalan än larven, såsom ibland vissa parasitiska krustaceer. Vi skola återvända till cirripederna: larverna i första stadiet hafva tre par ben, ett mycket enkelt öga och en snabelformig mun, med hvilken de upptaga ymnig föda, de tillväxa betydligt. I andra stadiet motsvarande fjärilarnas puppstadium hafva de sex par fint bygda simfötter, ett par ståtligt sammansatta ögon och ytterligt invecklade antenner; men de hafva en sluten och ofullständig mun och kunna icke upptaga någon föda: deras funktion under detta stadium är att genom sina väl utvecklade känselorganer och sin simförmåga söka en lämplig plats att fästa sig vid, der de undergå sin sista metamorfos. Då detta är fullbordadt, äro de fastväxta för hela lifvet: deras ben förvandlas nu till griporganer, de få ånyo en väl utbildad mun; men de hafva inga antenner och deras två ögon äro nu förvandlade till ett enda litet mycket enkelt öga. I detta sista fullbildade stadium kunna cirripederna betraktas såsom antingen högre eller lägre organiserade än i larvstadiet. Men i några slägten blifva larverna utvecklade antingen till hermafroditer af vanlig skapnad eller till hvad jag kallar komplementarhannar, och i senare fallet har utvecklingen helt säkert varit retrograd; ty de bestå blott af en säck som lefver en kort tid och är i saknad af mun, mage eller andra vigtiga organer med undantag af reproduktionsorganer.

Vi äro så vana vid att se bildningsolikheter emellan embryot och den fullväxta, att vi frestas att betrakta denna olikhet såsom en på sätt och vis nödvändig omständighet vid utvecklingen. Men det finnes intet skäl, hvarföre till exempel flädermusens vinge eller delfinens fena icke skulle kunnat vara med alla sina delar skizzerad i lämpliga proportioner, så snart någon bildning visade sig. I några hela djurgrupper och hos andra vissa medlemmar af andra grupper är detta förhållandet och embryot är på hvarje period helt olika den fullväxta. Owen har anmärkt angående bläckfiskarna, att ”der finnes ingen metamorfos: den cephalopoda karakteren är tydlig långt innan embryots delar äro fullbildade.” Landsnäckor och sötvattenskrustaceer äro födda med sin egendomliga form, under det de i hafvet lefvande medlemmarna af samma två stora klasser genomgå ansenliga förändringar under sin utveckling. Spindlar åter undergå knappt någon metamorfos. De flesta insektlarver genomgå ett masklikt stadium, antingen de äro aktiva eller lämpade för vissa vanor eller i overksamhet, fästade midt ibland tjenliga födoämnen, eller närda af sina föräldrar; men i några få fall såsom i slägtet Aphis se vi knappt något spår till det masklika stadiet om vi döma efter prof. Huxleys utmärkta framställning af deras utveckling.

Stundom är det blott de tidigare utvecklingsstadierna som fattas. Fritz Müller har sålunda gjort den anmärkningsvärda upptäckten, att räkliknande krustaceer (beslägtade med Penæus) först uppträda under den enkla naupliusformen och sedan de genomgått två eller flera zoeastadier och vidare genom mysisstadiet slutligen antaga sin fullmogna form. I hela den stora underklassen Malacostraca, till hvilken dessa krustaceer höra, är hittills ingen annan medlem känd för att förut utvecklas under naupliusformen, ehuru många uppträda såsom zoea; icke desto mindre angifver Müller skäl för sin tro, att alla dessa krustaceer skulle hafva uppträdt såsom nauplii, om icke något hinder i utvecklingen inträffat.

Huru kunna vi nu förklara alla dessa fakta i embryologien — den mycket allmänna ehuru icke helt och hållet ovilkorliga olikhet i skapnad emellan embryot och den fullväxta — likheten i en tidig period emellan de olika delarna hos samma individ hvilka slutligen blifva mycket olika och tjena till skilda ändamål — den allmänna men icke oföränderliga likheten emellan embryoner eller larver af de mest skilda arter i samma klass, — embryots bibehållande, så länge det finnes i ägget eller moderlifvet, af bildningar som äro af ingen nytta, vare sig i denna period eller senare i lifvet, under det embryoner i en senare period, eller larver som sjelfva måste sörja för sin utkomst, äro fullkomligt lämpade för de omgifvande vilkoren, — och slutligen det faktum att vissa larver stå högre i organisationsskalan än de fullmogna djur till hvilka de utvecklas? Jag tror att alla dessa förhållanden kunna förklaras på sätt som följer.

Det är allmänt antaget, måhända deraf att monstrositeter ofta visa sig hos embryot i en tidig period, att små variationer eller individuela afvikelser nödvändigt uppträda i en lika tidig period. Vi hafva få bevis för denna sak, men hvad vi hafva utvisar helt säkert motsatsen; ty det är bekant att uppfödare af boskap, hästar och andra djur icke bestämdt kunna säga förr än någon tid efter födseln, hvilken form och hvilka fördelar det unga djuret skall antaga. Vi se detta fullkomligt hos våra egna barn; vi kunna icke säga om ett barn skall bli af lång eller kort växt, eller hvilka dess anletsdrag skola bli. Frågan är icke på hvad period af lifvet hvarje variation orsakats, utan på hvilken period verkningarna visa sig. Orsaken kan hafva verkat, och jag tror den vanligen har verkat på den ena eller båda föräldrarna före fortplantningen. Det förtjenar anmärkas, att det är af ingen vigt för ett mycket ungt djur så länge det qvarstannar i moderlifvet eller ägget, eller så länge det växer och skyddas af sina föräldrar, om det mesta af dess karakter förvärfvas litet förr eller senare i lifvet. Det skulle för en fågel, som för att få sin föda behöfde en krökt näbb, vara af ingen betydelse, om den i ungdomen egde eller icke egde en sådan näbb, så länge den underhölls af sina föräldrar.

Jag har i första kapitlet visat, att i hvilken ålder en variation än uppträder hos fadern, i motsvarande ålder sträfvar den att uppträda hos afkomman. Vissa variationer kunna blott uppträda i motsvarande ålder, till exempel egendomligheter hos masken, kokongen eller imagon af silkesfjärilen, eller de fullväxta boskapsdjurens horn. Men variationer, hvilka efter allt hvad vi kunna se hafva uppträdt litet förr eller senare sträfva likaledes att uppträda i motsvarande ålder hos afkomlingen och fadern. Jag vill ingalunda påstå, att detta alltid är fallet, och jag kunde anföra flera undantagsfall af variationer (i ordets vidsträcktaste bemärkelse) som inträffat i en tidigare ålder hos barnet än hos fadern.

Dessa två grundsatser, att små variationer i allmänhet uppträda i en icke allt för tidig period af lifvet och ärfvas i motsvarande icke tidiga period förklara såsom jag tror alla ofvan anförda satser ur embryologien. Men låt oss först betrakta några få analoga fall bland våra domesticerade varieteter. Några författare som hafva skrifvit om hundar, påstå att vindthunden och bulldoggen, ehuru de tyckas vara så skilda, äro mycket närslägtade varieteter utgångna från samma vilda stam; derföre var jag nyfiken att se i hvad mån deras nyfödda hvalpar skilde sig från hvarandra: man sade mig att olikheterna voro lika stora som emellan de fullväxta, och att döma efter ögonmått tycktes det vara så, men efter att hafva noggrant mätt de gamla hundarna och deras sex dagar gamla hvalpar, fann jag att hvalparna icke hade antagit helt och hållet de riktiga proportionela olikheterna. Man berättade mig vidare att fålarna af draghästar och ridhästar — raser som bildats nästan helt och hållet genom urval under domesticering — skilde sig lika mycket som de fullväxta djuren; men sedan jag fått noggranna mätningar af stoen och tre dagar gamla fålar af ridhästar och tunga draghästar finner jag att detta ingalunda är förhållandet.

Då vi hafva fullgiltiga bevis, att dufraserna härstamma från en enda vild art, jemförde jag ungarna inom tolf timmar sedan de blifvit kläckta; jag mätte noggrant hos den vilda stamarten, hos kroppdufvor, påfågeldufvor, indiska, spanska dufvor, brefdufvor och tumletter proportionerna af näbbarna, munnens vidd, näsborrarnas längd och ögonlocken, fötternas storlek och benens längd. Några af dessa fåglar skilja sig i mogen ålder så utomordentligt i näbbens längd och form och i andra karakterer att de helt säkert skulle upptagas såsom särskilda arter om de funnos i vildt tillstånd. Men om man uppställer de nykläckta ungarna af alla dessa raser i en rad, så ehuru de flesta af dem genast kunde igenkännas, voro dock de proportionela olikheterna i ofvan specificerade delar ojemförligt mindre än hos de fullväxta fåglarna. Några karakteristiska skiljaktigheter såsom munnens vidd kunde svårligen upptäckas hos de unga. Men det var ett undantag från denna regel, ty ungen af den korthöfdade tumletten skilde sig i alla sina proportioner från ungen af den vilda klippdufvan och de andra raserna nästan fullt ut lika mycket som i fullväxt tillstånd.

Ofvan framstälda grundsatser förklara dessa förhållanden. Amatörer utvälja till afvel sina hundar, hästar, dufvor etc. då de äro nästan fullväxta: de bekymra sig icke om, huruvida de önskade egenskaperna förvärfvas tidigare eller senare i lifvet, blott det fullväxta djuret eger dem. Och de nu anförda exemplen isynnerhet dufvorna visa, att de karakteristiska olikheter som gifva raserna sitt värde och hvilka blifvit accumulerade genom menniskans urval i allmänhet icke hafva uppträdt i en mycket tidig lifsperiod och hafva gått i arf i en motsvarande icke tidig period.

Vi skola nu tillämpa dessa två grundsatser på arter i naturtillståndet. Låt oss taga en fågelgrupp, utgången från någon gammal form och modifierad genom naturligt urval för skilda vanor. Då de många små successiva variationerna hafva kommit öfver de olika arterna i en icke tidig period och hafva gått i arf i motsvarande ålder, skola ungarna lemnas blott litet modifierade och likna hvarandra mycket närmare än de fullväxta — just så som vi hafva sett kos dufraserna. Vi kunna utsträcka denna åsigt till vidt skilda bildningar och till hela klasser. Främre extremiteterna till exempel, som en gång tjenade såsom ben hos en uråldrig stamfar, kunna hafva blifvit genom en långvarig modifikation hos någon af de efterkommande förändrad till hand, hos en annan till fena, hos en annan till vinge, men enligt ofvan anförda grundsatser böra de främre extremiteterna icke hafva förändrats mycket hos embryot till alla dessa former; ehuru i hvar och en den embryonala extremitetformen blir betydligt olika den fullt utbildade. Hvilket inflytande dessutom länge fortsatt bruk eller overksamhet än må hafva haft vid extremiteternas eller andra delars modifiering bör detta dock hufvudsakligen eller endast hafva inverkat på dem då de blifvit fullt utbildade och måste använda all sin förmåga på att uppehålla sitt eget lif, och de häraf alstrade verkningarna skola öfvergå på afkomlingarna i motsvarande mogen ålder. Ungen bör således icke alls blifva modifierad, eller åtminstone i betydligt mindre grad.

I andra fall kunna successiva variationer hafva inträffat i en mycket tidig ålder, eller stadierna kunna gå i arf i en period tidigare än den i hvilken de förut uppträdde. I hvardera fallet skulle såsom vi hafva sett med den korthöfdade tumletten, ungen eller embryot mycket nära likna den utbildade moderformen. Och detta är utvecklingsregeln hos vissa hela grupper eller undergrupper, såsom bläckfiskar, landsnäckor, sötvattenskrustaceer, spindlar och några medlemmar af den stora insektklassen. Beträffande orsaken hvarföre ungarna i dessa grupper icke genomgå några metamorfoser, kunna vi se att detta bör härröra af följande omständigheter; nemligen att ungarna vid en mycket tidig period måste sörja för sina egna behof och att de följa samma lefnadsvanor som föräldrarna; ty i detta fall skulle det vara oundgängligt för deras existens att de modifierades på samma sätt som deras föräldrar. Hvad åter beträffar det besynnerliga förhållande att så många land- och sötvattendjur icke undergå någon metamorfos, under det hafsdjur af samma grupper genomgå olika transformationer, har Fritz Müller antagit att den långsamma modifikationsprocessen och ett djurs lämpande för att lefva på land eller i sötvatten i stället för i hafvet skulle i hög grad förenklas derigenom att det icke passerade något larvstadium, ty det är icke sannolikt, att platser väl lämpade för både larvstadiet och fullbildade stadiet under sådana nya och i hög grad förändrade lefnadsvanor kunde finnas oupptagna eller illa besatta af andra organismer. I detta fall skulle det gradvisa förvärfvandet vid en allt tidigare och tidigare period af den utbildade skapnaden gynnas af det naturliga urvalet och alla spår af fordna metamorfoser skulle slutligen gå förlorade.

Om å andra sidan det gagnade ungen af ett djur att följa lefnadsvanor obetydligt olika stamformens och att följaktligen formas på något olika sätt, eller om det gagnade larven som redan är olik sin moderform att förändras ännu mer, skulle enligt principen om arf i motsvarande ålder ungarna eller larverna kunna genom naturligt urval blifva mer och mer olika sina föräldrar till hvarje tänkbar utsträckning. Olikheter hos larverna kunna också sättas i samband med successiva utvecklingsstadier, så att larverna i första stadiet komma att afvika betydligt från larverna i andra stadiet, såsom fallet är med många djur. Den fullväxta kan också lämpas för lägen och vanor, i hvilka lokomotionsorganer eller känselorganer skulle vara gagnlösa, och i detta fall skulle metamorfosen vara retrograd.

Af nyss gjorda betraktelser kunna vi se huru genom strukturförvandlingar hos ungen i öfverensstämmelse med förändrade lefnadsvanor, tillsammans med ärftlighet i motsvarande ålder djuren i vissa fall kunna komma att genomgå utvecklingsstadier, fullkomligt olika deras ursprungliga fullväxta tillstånd. Fritz Müller, som nyligen behandlat detta ämne med mycken skicklighet, tror att stamfadern för alla insekter liknade en fullväxt insekt och att mask- eller larvstadiet, äfvensom kokong- eller puppstadiet sedan blifvit förvärfvade, men från denna åsigt afvika sannolikt många naturforskare till exempel J. Lubbock, som likaledes nyligen behandlat detta ämne. Att vissa ovanliga stadier i insekternas metamorfoser hafva blifvit förvärfvade genom inpassande uti egendomliga lefnadsvanor, kan svårligen betviflas; det första stadiet af en skalbagge, Sitaris, är sålunda såsom Fabre beskrifver en rörlig liten insekt med sex ben, två långa antenner och fyra ögon. Dessa larver kläckas i biens bon och då bihannarna gå ut från sina kulor på våren förrän honorna, hänga sig larverna fast vid dem och krypa sedan öfver på bihonorna under deras parning. Så snart honorna lägga sina ägg på ytan af den i cellerna samlade honingen, störta Sitarislarverna öfver dem och uppsluka dem. Sedermera undergå dessa larver en fullkomlig förändring; deras ögon försvinna, deras ben och antenner blifva rudimentära och de lefva på honing; så att de numera likna vanliga insektlarver; slutligen undergå de en ytterligare förvandling och framgå ändtligen såsom fullbildade skalbaggar. Om nu en insekt, som undergår förvandlingar lika Sitaris’ metamorfoser, blefve stamfader för en hel ny insektklass skulle förloppet af deras utveckling sannolikt bli helt olika hvad det nu är; och det första larvstadiet skulle helt säkert icke föreställa det fordna tillståndet af någon gammal fullvuxen insekt.

Å andra sidan är det i hög grad sannolikt, att bland många djur embryonal- eller larvstadiet visar oss mer eller mindre fullständigt stamfaderns tillstånd i sin fulla utveckling. I den stora klassen Crustacea uppträda underbart från hvarandra skilda former, nämligen sugande parasiter, cirripeder, entomostraca och till och med malacostraca först såsom larver under naupliusform, och då dessa larver nära sig och lefva i öppen sjö och icke äro lämpade för några särskilda lefnadsvanor och af andra skäl som Fritz Müller angifvit, är det sannolikt att ett oberoende fullväxt djur liknande nauplius lefde i någon mycket aflägsen tidpunkt och sedermera genom flera divergenta härstamningslinier alstrade de olika ofvannämda stora grupperna af krustaceer. Vidare är det sannolikt af hvad vi veta om däggdjurs, fåglars, fiskars och reptiliers embryoner, att dessa djur äro de modifierade ättlingarna af någon gammal stamfader, som i sitt fullväxta tillstånd var försedd med gälar, en simblåsa, fyra enkla extremiteter och en lång svans, allt lämpadt för ett aqvatiskt lif.

Då alla organiska varelser, utdöda och nya som någonsin lefvat, kunna anordnas i några få stora klasser; och då alla inom hvarje klass hafva enligt vår teori förut varit sammanbundna genom fina öfvergångar, skulle den bästa och om våra samlingar voro fullständiga den enda möjliga anordningen vara genealogisk, ty härstamning är det dolda föreningsband som naturhistorikerna sökt under benämningen naturligt system. Enligt denna åsigt kunna vi förstå hvarföre i en naturforskares ögon embryots skapnad är till och med vigtigare för klassifikationen än den fullväxtas. Om två eller flera djurgrupper, huru mycket de än skilja sig från hvarandra i skapnad och vanor, genomgå liknande embryonala stadier, kunna vi vara förvissade om, att de alla härstamma från en stamform och derföre äro nära beslägtade. Gemensamhet i embryonal bygnad uppenbarar således gemensamhet i härkomst, men olikhet i embryonal utveckling bevisar icke olikhet i härkomst, ty i en af de två grupperna kunna utvecklingsstadierna hafva blifvit undertryckta eller blifvit i så hög grad modifierade genom lämpande efter nya lefnadsvanor, att de icke längre äro igenkänneliga. Till och med i grupper i hvilka de fullväxta hafva blifvit modifierade i ytterlig grad uppenbaras ofta det gemensamma ursprunget genom larvernas skapnad; vi hafva till exempel sett att cirripederna genom sina larver med ens igenkännas tillhöra den stora klassen Crustacea. Då embryots skapnad i allmänhet visar oss mer eller mindre fullständigt bygnaden af dess mindre modifierade ursprungliga stamfader kunna vi inse, hvarföre gamla och utdöda former så ofta likna embryoner af nu lefvande arter i samma klass. Agassiz tror att detta är en allmän naturlag, och jag hoppas få se den hädanefter bevisas vara sann i de flesta fall. Den kan likväl bevisas vara sann blott i de fall i hvilka det gamla stadiet icke blifvit fullkomligt utrotadt genom att successiva variationer inträffat i en mycket tidig period af utvecklingen, eller genom att sådana variationer blifvit ärfda i en tidigare period än den i hvilken de först uppträdde. Vi måste också ihågkomma, att lagen kan vara sann och dock omöjlig att bevisa under en lång tid eller för alltid på den grund att den geologiska urkunden icke sträcker sig tillräckligt långt tillbaka i tiden. Lagen skall icke gälla i de fall i hvilka en gammal form i sitt larvstadium lämpades för något specielt lefnadssätt, och lemnade samma larvstadium i arf åt en hel grupp afkomlingar; ty dessa skulle icke i sitt larvtillstånd likna någon gammal form i dess utbildade tillstånd.

De väsentligaste fakta i embryologien, hvilka icke äro andra underlägsna i betydelse, förklaras således såsom mig synes enligt grundsatsen att modifikationer hos mängden af afkomlingar från samma stamfar uppträdt i en icke tidig lifsperiod och gått i arf i motsvarande ålder. Embryologien vinner i intresse, om vi betrakta embryot såsom en mer eller mindre dunkel teckning af stamfadern till alla medlemmar af samma stora klass, antingen i dess fullbildade eller larvstadium.



\section{Rudimentära, förkrympta och abortiva organer.}

Organer eller delar i detta sällsamma tillstånd, bärande prägel af fullkomlig obrukbarhet äro ytterligt vanliga eller till och med allmänna genom naturen. Det vore svårt att nämna ett af de högre djuren, hos hvilket icke någon del är i detta rudimentära tillstånd. Hos däggdjuren till exempel bära hannarna alltid rudimentära bröstkörtlar, hos ormar är den ena lungan förkrympt, hos fåglar kan ”bastardvingen” helt säkert betraktas såsom ett rudimentärt finger och i icke få fall kunna vingarna icke begagnas till flygt eller äro reducerade till ett rudiment. Hvad kan vara mera egendomligt än närvaron af tänder hos hvalfostren, hvilka sedan de växt upp icke hafva en tand i käken, eller de tänder som aldrig genombryta tandköttet i öfverkäken hos ofödda kalfvar?

Rudimentära organer visa sitt ursprung och sin betydelse på flera sätt. Det finnes skalbaggar af flera beslägtade arter eller till och med samma identiska art, som hafva antingen vingar af full storlek och utbildning eller blott små rudiment af en hinna, som icke sällan ligga under de fast hoplödda täckvingarna: och i detta fall är det omöjligt att betvifla att rudimenten föreställa vingar. Rudimentära organer behålla sällan sin potentialitet; detta inträffar tillfälligtvis med bröstkörtlarna hos däggdjurshannar, ty man vet att de stundom blifva väl utvecklade och afsöndra mjölk. Hos slägtet Bos finnes på jufret normalt fyra utvecklade och två rudimentära spenar, men de senare bli hos våra tama kor utbildade och lemna mjölk. Hvad växterna beträffar äro kronbladen stundom rudimentära och stundom väl utvecklade hos individer af samma art. Hos vissa dioica växter fann Kölreuter, att vid en arts kroasering, hvars hanblommor innehöllo en rudimentär pistill, med en hermafrodit växt, som naturligtvis hade väl utvecklade pistiller, var rudimentet hos den hybrida ättlingen förstoradt och detta visar klart att de rudimentära och de fullkomliga pistillerna äro väsentligt lika till natur. Ett djur kan ega flera delar i fullt utbildadt tillstånd och dock måste de i en viss mening vara rudiment, ty de äro obrukbara; ungarna af Salamandra, eller den vanliga vattenödlan hafva såsom G. H. Lewes anmärker ”gälar och tillbringa sitt lif i vatten, men Salamandraarter, som lefva högt upp på bergen framföda sina ungar fullbildade. Detta djur bor aldrig i vatten. Om vi dock öppna en drägtig hona, finna vi inuti henne ungar med fullkomligt fjäderlika gälar, och om de läggas i vatten simma de omkring likt ungar af den vanliga vattenödlan. Denna aqvatiska organisation har naturligtvis intet sammanhang med djurets framtida lif, ej heller har det något förhållande till dess embryonala vilkor; den står blott i samband med någon förfaders organisation och upprepar en phas i dess utveckling.”

Ett organ som tjenar till två ändamål kan blifva rudimentärt och abortivt för det ena, äfven det mera vigtiga och förblifva fullkomligt användbart för det andra. Hos växter är sålunda pistillens förrättning att lemna pollenrören tillfälle att nå fröämnena inom fruktämnet. Pistillen består af ett märke sittande på ett stift; men hos några Compositæ hafva hanblommorna, som naturligtvis icke kunna blifva befruktade, en rudimentär pistill utan märke; men stiftet är ännu väl utveckladt och beklädt med hår på vanligt sätt för att samla frömjölet ur de omgifvande förenade ståndarna. Ett organ kan åter blifva rudimentärt för sitt eget ändamål och begagnas till något helt annat: hos vissa fiskar synes simblåsan vara rudimentär för dess egentliga ändamål att hålla djuret flytande, men har blifvit förvandladt till början af ett andningsorgan eller lunga. Andra liknande fall kunna anföras.

Om organerna också äro obetydligt utvecklade kunna de icke kallas rudimentära, så länge de kunna användas: de kunna sägas vara under bildning och kunna af det naturliga urvalet hädanefter utvecklas till hvad grad som helst. Rudimentära organer äro å andra sidan i sjelfva verket gagnlösa, såsom tänder, hvilka aldrig genombryta tandköttet. Då de skulle vara af ännu mindre nytta, om de befunno sig i ännu mera outveckladt tillstånd, kunna de icke hafva bildats genom föränderlighet och naturligt urval, hvilket senare verkar allenast genom bevarandet af nyttiga modifikationer. De hänföras till ett tidigare tingens stadium och hafva blifvit delvis bibehållna genom ärftlighet. Det är svårt att veta, hvilka organer äro under bildning; om vi se in i framtiden kunna vi naturligtvis icke säga huru en del kommer att utvecklas och om den nu är under bildning; om vi se på det förflutna, hafva i allmänhet varelser med ett organ i detta tillstånd undanträngts af deras efterföljare med samma organ i ett mera fulländadt tillstånd och följaktligen äro de för längesedan utdöda. Pingvinens vingar äro af stor nytta, ty de tjena såsom fenor, de kunna derföre föreställa vingarnas status nascens; jag tror visserligen icke att så är förhållandet, de äro sannolikt reducerade organer modifierade för en ny funktion: vingarna hos Apteryx äro deremot fullkomligt gagnlösa och sannolikt rudimentära. De trådlika extremiteterna hos Lepidosiren äro som det tyckes i ett status nascens, ty de äro, såsom Owen uttrycker sig, ”begynnelsen till organer, som erhålla full funktionel utveckling hos de högre vertebraterna.” Mjölkkörtlarna kos Ornithorhynchus kunna i jemförelse med jufret hos kor betraktas såsom organer under bildning. Hos vissa cirripeder äro ”frena ovigera” gälar under bildning i de fall då de blott äro obetydligt utvecklade och upphört att tjena till fäste för äggen.

Rudimentära organer äro mycket benägna för att variera i utveckling och andra hänseenden hos individer af samma art. Hos beslägtade arter är dessutom den grad i hvilken samma organ blifvit reduceradt betydligt olika. På detta senare förhållande finnas många exempel i vingarnas tillstånd hos fjärilhonor af vissa grupper. Rudimentära organer kunna helt och hållet felslå, och detta innebär att hos vissa djur och växter delar helt och hållet saknas, som vi vänta oss finna och hvilka tillfälligtvis finnas hos monströsa individer. I de flesta af Scrophularieæ är den femte ståndaren helt och hållet felslagen; dock kunna vi antaga att en sådan en gång existerade, ty ett rudiment af den finnes hos många arter af familjen och detta rudiment kan händelsevis komma till full utveckling. Vid uppletandet af någon dels homologier hos olika medlemmar af samma klass är ingenting vanligare eller till och med nödvändigare för att fullständigt inse delarnas relationer, än upptäckandet af rudiment. Detta visas förträffligt i de af Owen lemnade teckningarna öfver benen i hästens och noshörningens underben.

Det är ett vigtigt faktum att rudimentära organer, såsom tänderna i öfverkäken hos hvalar och idislare, ofta kunna upptäckas hos embryot men sedermera helt och hållet försvinna. Det är också såsom jag tror en allmän regel, att en rudimentär del är af större omfång i förhållande till andra delar hos embryot än hos den fullväxta, så att organet i denna tidiga ålder är mindre rudimentärt eller till och med alldeles icke kan sägas vara rudimentärt. Rudimentära organer hos den fullväxta sägas derföre ofta hafva bibehållit sin embryonala karakter.

Jag har nu anfört det väsentligaste rörande de rudimentära organerna. Då man reflekterar öfver dem, måste hvar och en känna sig förvånad, ty samma slutledningsförmåga, som säger oss att de flesta delar och organer äro fullständigt utbildade för vissa ändamål, säger oss med samma säkerhet att dessa rudimentära och förkrympta organer äro ofullständiga och gagnlösa. I naturhistoriska arbeten sägas i allmänhet rudimentära organer hafva blifvit skapade ”för symmetrien” eller för att ”fullända naturens schema.” Men detta är icke någon förklaring, blott en upprepning af sjelfva saken. Ej heller är den öfverensstämmande med sig sjelf; Boa constrictor har till exempel rudiment af bakre extremiteter och ett bäcken, och om man kan säga att dessa bibehållits för att ”fullständiga naturens schema”, hvarföre, frågar professor Weismann, hafva de icke bibehållits hos andra ormar, hvilka ega icke ens ett spår af dessa ben? Hvad skulle man tänka om en astronom, som påstod att månarna gå i elliptiska banor omkring sina planeter ”för symmetri”, emedan planeterna röra sig så omkring solen? En utmärkt fysiolog förklarar närvaron af rudimentära organer genom att antaga att de tjena till att afsöndra öfverflödiga eller för organisationen skadliga ämnen; men kunna vi antaga, att den lilla papillen, som ofta representerar pistillen hos hanblommor och blott består af cellväfnad kan verka på sådant sätt? Kunna vi antaga att rudimentära tänder, som sedermera absorberas, äro nyttiga för det tillväxande kalffostret genom att aflägsna så dyrbara ämnen som fosforsyrad kalk? Då ett menniskofinger amputerats har man sett ofullständiga naglar växa upp på stumpen, och jag kunde lika väl tro att dessa spår af naglar utvecklats för att afsöndra hornämne, som att de rudimentära naglarna på fenorna hos sjökon (Manatus) hafva bildats för detta ändamål.

Enligt åsigten om härstamning med modifikation är ursprunget till de rudimentära organerna mycket enkelt. Vi hafva en mängd fall af rudimentära organer hos våra domesticerade alster — såsom svansstumpen hos svanslösa raser — spåret af ett öra hos öronlösa fårraser — uppträdandet af små i huden hängande horn hos våra hornlösa boskapsraser, isynnerhet enligt Youatt hos unga djur — och hela blommans tillstånd hos blomkålen. Vi se ofta rudiment af olika delar hos missfoster. Men jag tviflar att något af dessa fall sprider något ljus öfver ursprunget till de rudimentära organerna annat än genom att visa att sådana kunna åstadkommas, ty jag tror icke att arter i naturtillståndet undergå hastiga förändringar. Jag tror att bristande användning varit det hufvudsakligen verksamma, att deraf har under successiva generationer kommit den gradvisa reduktionen af olika organer till dess de blifvit rudimentära — såsom förhållandet är med ögonen hos djur som bebo mörka grottor och med vingarna hos fåglar som bo på oceanöar, hvilka sällan tvingats af rofdjur att taga till flykten och slutligen förlorat förmågan att flyga. Ett under vissa vilkor nyttigt organ kan åter blifva skadligt under andra vilkor, såsom vingarna hos skalbaggar som lefva på små fristående öar, och i detta fall kan det naturliga urvalet fortfarande långsamt reducera organet till dess det blifvit oskadligt och rudimentärt.

Hvarje förändring i skapnad och funktion som kan åstadkommas genom omärkligt små steg ligger inom området för det naturliga urvalet, så att ett organ som genom förändring i lefnadsvanor blifvit obrukbart eller skadligt för ett ändamål kan modifieras och användas för ett annat ändamål. Ett organ kan behållas för en enda af dess fordna förrättningar. Ett organ som ursprungligen bildats genom det naturliga urvalet kan, om det blir obrukbart, lätt blifva föränderligt, ty dess variationer kunna icke längre förhindras af det naturliga urvalet. I hvilken period af lifvet antingen bristande användning eller urval reducerar ett organ, och detta inträffar i allmänhet då varelsen uppnått mogen ålder och måste utöfva all sin verksamhetskraft, skall grundsatsen om ärftlighet i motsvarande period reproducera organet i dess förkrympta skick vid samma mogna ålder, men sällan beröra det hos embryot. Vi kunna således begripa den betydliga storleken af rudimentära organer hos fostret i förhållande till andra delar och dess mindre storlek hos den fullväxta. Men om hvarje steg af reduktionsprocessen skulle gå i arf, icke i motsvarande ålder utan i en mycket tidig lifsperiod skulle den rudimentära delen småningom gå förlorad och vi skulle hafva ett fall af fullständigt försvinnande. Grundsatsen om organisationens ekonomi som i ett föregående kapitel utvecklats, enligt hvilken materialet för ett organs utveckling så mycket som möjligt inbespares om organet icke längre är af nytta för egaren, kan ofta härvid vara verksam och behjelplig till den totala förlusten af ett rudimentärt organ.

Då närvaron af rudimentära organer således beror på benägenheten hos hvarje del af organisationen som länge egt bestånd att gå i arf, kunna vi inse i öfverensstämmelse med den genealogiska åsigten om klassificering orsaken hvarföre systematikerna hafva ansett rudimentära organer lika nyttiga eller stundom till och med nyttigare än delar af hög fysiologisk vigt. Rudimentära organer kunna jemföras med bokstäfver i ett ord, som ännu bibehållas vid stafning; ehuru obrukbara vid ordets uttalande, äro de till nytta för ordets härledning. Enligt åsigten om härstamning med modifikation kunna vi antaga, att tillvaron af organer i ett rudimentärt, ofullständigt och obrukbart skick eller helt och hållet felslagna, långt ifrån att erbjuda någon större svårighet, som de helt visst göra för den vanliga skapelseläran, kunde till och med blifvit härledd a priori i öfverensstämmelse med här utvecklade åsigter.



\section{Sammanfattning.}

I detta kapitel har jag försökt att visa, att alla organiska varelsers anordning från alla tider i grupp under grupp — att beskaffenheten af den slägtskap som förenar alla nu lefvande och utdöda organismer genom invecklade, radierande slägtskapslinier till ett fåtal stora klasser — de reglor naturhistorikerna följa och de svårigheter de påträffa vid sina klassifikationer — det värde som gifvits karakterer, om de äro beständiga och förherskande, vare sig de äro af stor eller ringa betydelse, eller såsom rudimentära organer icke ega någon betydelse alls — den stora motsatsen emellan analoga eller adaptiva karakterer och verkliga slägtskapskarakterer och andra sådana reglor — att allt detta är en naturlig följd af antagandet om beslägtade formers gemensamma härstamning jemte deras modifikation genom naturligt urval med åtföljande förintelse och karaktersdivergens. Då vi öfvertänka denna åsigt om klassificeringen, få vi komma ihåg, att härstamningselementet allmänt begagnats vid könens, åldrarnas, dimorfa formers och erkända varieteters ordnande under samma art, huru olika de än må vara i kroppsbygnad. Om vi vidare utsträcka bruket af härstamning — den enda med säkerhet kända orsaken till likhet emellan de organiska varelserna — skola vi inse hvad som menas med ett naturligt system; det är genealogiskt i sin plan med graderna af förvärfvad olikhet betecknade genom uttrycken varietet, art, slägte, familj, ordning och klass.

Enligt samma åsigt om härstamning med modifikation blifva alla fakta ur morfologien begripliga — vare sig vi fästa oss vid samma mönster, utveckladt hos olika arter af samma klass i homologa organer, till hvad ändamål de än användas, eller vid de homologa delarna hos hvarje djur- eller växtindivid.

Enligt den grundsatsen att successiva små variationer icke nödvändigt eller i allmänhet inträffa i en mycket tidig period af lifvet och gå i arf i motsvarande period kunna vi förstå hufvudsatserna i embryologien, nämligen den nära likheten hos fostret emellan delar som äro homologa, och hvilka i mogen ålder blifva betydligt skilda till bygnad och funktion; och likheten hos beslägtade ehuru skilda arter emellan deras homologa delar eller organer, ehuru de hos den fullväxta tjena till ändamål så skilda som möjligt. Larver äro foster i verksamhet, hvilka blifvit specielt modifierade i större eller mindre grad efter sina lefnadsvanor och hvilkas modifikationer gå i arf i motsvarande ålder. Enligt samma grundsatser — och om vi ihågkomma att då organer reduceras i storlek antingen af bristande användning eller genom naturligt urval, detta i allmänhet sker i den period af lifvet, då varelsen måste sörja för sina egna behof, samt ihågkommande, huru kraftig ärftlighetslagen är — kunde tillvaron af rudimentära organer till och med tänkas på förhand. Vigten af de embryologiska karaktererna och af rudimentära organer för klassifikationen inses enligt den föreställningen att en naturlig anordning måste vara genealogisk.

Alla de förhållanden som varit föremål för våra betraktelser i detta kapitel synas mig fullständigt förklara, att de oräkneliga arter, slägten och familjer, af hvilka denna jord är bebodd, alla inom hvarje klass eller grupp härstamma från gemensamma föräldrar och hafva alla blifvit modifierade under utvecklingens gång, så att jag utan tvekan skulle antaga denna åsigt äfven om den icke stöddes af några andra fakta och bevis.


