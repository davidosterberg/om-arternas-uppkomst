%FEMTE KAPITLET.





\chapter{Lagarna för variationen.}

{\it
Verkan af de yttre förhållandena. — Organernas användning och hvila i förening med naturligt urval; flyg- och synorganerna. — Acklimatisering. — Utvecklingens vexelverkan. — Utvecklingens kompensering och ekonomi. — Falsk vexelverkan. — Multipla, rudimentära och lågt organiserade bildningar äro föränderliga. — Delar, som utvecklats på ovanligt sätt, äro i hög grad föränderliga; artkarakterer mer föränderliga än slägtkarakterer; sekundära könskarakterer föränderliga. — Arter af samma slägte variera på analogt sätt. — Återgång till längesedan förlorade karakterer. Sammanfattning.
}\\[0.5cm]

Jag har hittills stundom talat om föränderlighet — så vanlig och mångfaldig hos organiska varelser under domesticering, i mindre grad i naturtillståndet — såsom om den berott på en slump. Detta är naturligtvis ett fullkomligt oriktigt uttryckssätt, som blott tjenar till att visa vår fullkomliga obekantskap med orsaken till hvarje speciel afvikelse. Några författare tro, att det tillhör de reproduktiva organernas funktioner lika mycket att frambringa individuela olikheter eller helt små bildningsafvikelser, som att göra barnet likt föräldrarna. Men den vida större föränderligheten, likasom det talrikare förekommandet af monstrositeter under domesticering än under naturtillståndet föranleder mig att tro, att sådana formafvikelser direkt bero på beskaffenheten af de lifsvilkor, under hvilka föräldrarna och deras mer aflägsna förfäder hafva lefvat under flera generationer. I första kapitlet har jag försökt att visa, att förändrade förhållanden verka på två sätt, direkt på hela organisationen eller på vissa delar allena, och indirekt genom reproduktionssystemet. I alla fall äro två faktorer verksamma, organismens beskaffenhet, som är vigtigast, och förhållandenas beskaffenhet. De förändrade förhållandenas direkta verkan leder till bestämd och obestämd modifikation. I senare fallet synes organisationen vara plastisk och vi hafva en mycket vexlande föränderlighet. I förra fallet är organismens beskaffenhet sådan, att den lätt gifver vika, så snart den utsättes för vissa omständigheter, och alla eller nästan alla individer förändras på samma sätt.

Huru stor den direkta verkan af vexlingar i klimat, föda etc., är på en organisk varelse, är ytterligt svårt att afgöra. Min tro är, att verkan är ytterligt liten på djur, men större på växter. Vi kunna åtminstone med säkerhet säga, att sådana inflytelser icke kunna hafva alstrat de många i ögonen fallande, invecklade förhållanden, som vi se öfverallt i naturen, emellan en organisk varelses kroppsbygnad och en annans. Någon ringa inflytelse kunna vi tillskrifva klimat, föda med mera: så yttrar Edward Forbes med bestämdhet att snäckor vid södra gränsen för deras utbredningsområde och i grundt vatten variera och antaga mera glänsande färger än längre norrut och på större djup, men dessa påståenden hafva sedermera betviflats. Gould tror, att fåglar af samma art äro prydda med mera lysande färger under en ständigt klar himmel, än om de lefva på öar eller vid kusterna. Wollaston är äfven öfvertygad om, att vistandet nära sjöar inverkar på insekternas färger och Moquin-Tandon meddelar en förteckning på växter, som vid sjöstränder få mer eller mindre köttiga blad, ehuru de på land ingalunda äro köttiga. Många andra sådana exempel kunde gifvas.

Det faktum, att varieteter af en art, om de komma in på området för en annan art, ofta i ringa grad antaga dennas karakterer, öfverensstämmer med vår åsigt, att arterna af alla slag blott äro väl utpräglade och permanenta varieteter. De snäckarter som äro inskränkta till de tropiska och grundare sjöarna äro i allmänhet prydda med mera lysande färger än de som blott bebo kalla och djupare sjöar. Fåglar som blott hålla sig på fastland äro enligt Gould mera bjert färgade än fåglarna på öar. Insektarter, som endast bebo sjöstränder äro, såsom hvarje samlare vet, ofta bronsfärgade och mörka; växter som uteslutande lefva vid vatten äro mycket benägna att antaga köttiga blad. Den som tror på hvarje arts särskilda skapelse måste säga till exempel, att den ena snäckan blef skapad med lysande färger för en varm sjö, men att den andra fick sin lifliga färg genom förändring då den flyttades till varmare och grundare vatten.

Om en förändring är af den ringaste fördel för en varelse, så kunna vi icke säga, huru mycket deraf bör tillskrifvas det naturliga urvalets accumulativa verkan eller de yttre lefnadsförhållandena. Så veta pelshandlare mycket väl, att djuren hafva tätare och vackrare pels, ju strängare klimatet varit under hvilket de lefvat; men hvem kan säga, i hvad mån detta beror på att de varmast klädda individerna hafva under många generationer blifvit gynnade och skyddade, eller på det stränga klimatets direkta verkan? Ty det synes väl, att klimatet har en viss direkt verkan på våra tämda fyrfotadjurs hårbeklädnad.

Exempel kunna gifvas på samma varietets uppkomst under de mest olika yttre förhållanden och å andra sidan på olika varieteters uppkomst af samma art under fullkomligt samma omständigheter. Sådana fakta visa, huru indirekt lifsvilkoren verka. Hvarje naturhistoriker känner äfven oräkneliga exempel på arter, som bibehålla sig äkta utan alla variationer, ehuru de lefva under de mest olika klimat. Sådana betraktelser föranleda mig att lägga blott ringa vigt på lifsvilkorens direkta verkan. Indirekt synas de, såsom redan är anmärkt, spela en vigtig rol genom att orsaka rubbningar i reproduktionssystemet och sålunda inleda föränderlighet, och det naturliga urvalet samlar sedan alla fördelaktiga, ehuru små variationer, till dess de blifva fullt utvecklade och för oss förnimbara.

I vidsträcktare mening kan man säga, att lefnadsförhållandena icke blott orsaka föränderlighet, utan äfven innefatta det naturliga urvalet; ty på lefnadsomständigheterna beror, om den ena eller den andra varieteten bibehålles. Men menniskans metodiska urval visar oss, att dessa två variationselementer äro väsentligt skilda; lefnadsförhållandena förorsaka föränderlighet och menniskans vilja samlar medvetet eller omedvetet förändringarna i en eller annan bestämd riktning.



\section[Verkan av ett organs bruk]{Verkan af ett organs bruk och bristande användning.}

De förhållanden vi omnämt i första kapitlet lemna föga tvifvel om att flitig användning stärker och utvidgar vissa delar hos våra husdjur, under det att samma delar försvagas, om de icke användas, och att sådana modifikationer äro ärftliga. Under fritt naturtillstånd hafva vi ingen måttstock för jemförelse att bedöma verkningarna af ett länge fortsatt bruk af vissa organer eller deras hvila, ty vi känna icke stamformerna, men många djur visa bildningar, hvilka kunna förklaras såsom följder af bristande användning. Såsom professor Owen anmärker, finnes ingen större anomali i naturen än en fågel som ej kan flyga; dock finnas flera i denna belägenhet. En sydamerikansk andart (Anas brachyptera) kan blott fladdra öfver vattenytan och har sina vingar nästan i samma tillstånd som den tama Aylesbury andrasen. Då de stora på marken betande fåglarna sällan flyga annat än för att undvika någon fara, tror jag, att de nästan vinglösa fåglar, som nu bebo eller fordom bebott öarna i oceanen, der de icke oroas af några rofdjur, hafva för sin nuvarande skapnad att tacka den omständigheten, att de icke användt sina vingar. Strutsen bebor visserligen fast land och är utsatt för faror som den ej kan undvika genom flygt, men med sina starka fötter kan den försvara sig för fienden lika väl som något af de mindre fyrfotadjuren. Vi kunna föreställa oss att strutsens stamfader hade ungefär samma vanor som trapparna och att i samma mån som det naturliga urvalet under successiva generationer förökade storleken och vigten af dess kropp, i samma mån begagnades benen mer och vingarna mindre, till dess de blefvo fullkomligt odugliga till flygt.

Kirby har anmärkt (och jag har äfven sjelf observerat samma faktum) att de främre tarserna eller fötterna på många skalbaggar som lefva i gödselhögar ofta äro afbrutna; han undersökte sjutton exemplar i sin egen samling och icke en enda hade ett spår qvar deraf. Onites apelles är så ofta i saknad af framtarserna att den beskrifves såsom saknande dem. I några andra slägten finnas de, men i mycket rudimentärt tillstånd. Hos Ateuchus, Egypternas heliga skalbagge, saknas de helt och hållet. Det är icke tillräckligt bevisadt, att tillfälliga stympningar gå i arf. Jag vill heldre förklara den totala frånvaron af främre tarser hos Ateuchus och deras rudimentära tillstånd hos några andra slägten såsom följd af bristande användning hos deras förfäder, ty då tarserna nästan alltid saknas hos många skalbaggar, som lefva i gödselhögar, måste de förloras under en tidig lefnadsperiod och kunna derföre icke vara af mycken nytta eller mycket begagnas af dessa insekter.

I några fall kunna vi lätt tillskrifva bristande användning vissa bildningsmodifikationer, som helt och hållet eller till en stor del bero på det naturliga urvalet. Wollaston har upptäckt det märkvärdiga förhållandet, att af 550 på Madeira boende skalbaggsarter 200 hafva så ofullkomliga vingar, att de ej kunna flyga, och att af tjugunio inhemska slägten icke mindre än tjugutre hafva alla sina arter i denna belägenhet! Flera sådana förhållanden, såsom att skalbaggarna i många delar af verlden af vinden drifvas ut i sjön och omkomma, att skalbaggarna på Madeira efter Wollastons iakttagelser ligga mycket dolda tills vinden är stilla och solen skiner, att antalet vinglösa skalbaggar är större på de stora öde öarna än till och med på Madeira, och särskildt det märkvärdiga faktum som Wollaston så starkt framhåller, att vissa stora grupper af skalbaggar der fullkomligt fattas, hvilka annars äro utomordentligt talrika, och hvilkas vanor nästan tvinga dem att flyga; — alla dessa betraktelser hafva kommit mig att tro, att vingarnas rudimentära tillstånd hos så många af Madeiras skalbaggar till en stor del beror på det naturliga urvalets verkningar, sannolikt i förening med bristande användning. Ty under tusentals generationer har hvarje individ, som flugit minst antingen derföre att dess vingar varit minst utvecklade eller på grund af sina lefnadsvanor, haft största utsigten att öfverlefva alla andra, emedan den icke af vinden kunnat drifvas ut i sjön, och å andra sidan hafva de skalbaggar, som gerna flögo, oftast drifvit ut i sjön och omkommit.

De insekter på Madeira, som ej lefva på marken och hvilka vanligen måste använda sina vingar för att finna sitt lifsuppehälle, såsom de af blommor lefvande skalbaggarna och fjärilarna, hafva enligt Wollastons förmenande alls icke förkrympta, utan snarare starkt utvecklade vingar. Detta står i full öfverensstämmelse med det naturliga urvalet. Ty när en ny insekt först kommer på en ö, beror det naturliga urvalets bemödande att förstora eller reducera vingarna på huruvida ett större antal individer skulle bevaras genom att lyckligt strida med vindarna eller genom att uppgifva försöket och sällan eller aldrig flyga. Förhållandet är alldeles detsamma som med matroser från ett skeppsbrutet fartyg: för de goda simmarna skulle det varit fördelaktigare, om de kunnat simma ännu längre, för de dåliga simmarna, om de alls icke kunnat simma och hållit sig vid vraket.

Mullvaden och några gräfvande gnagare hafva ögon, som äro rudimentära i storlek och i några fall fullkomligt täckta af hud och hår. Detta ögonens tillstånd är sannolikt resultatet af en småningom skeende reduktion på grund af bristande användning, men härvid spelar måhända äfven det naturliga urvalet någon rol. En Sydamerikansk gnagare, Tuco-tuco, eller Ctenomys har ännu mera underjordiska vanor än mullvaden, och en spanior, som ofta fångat sådana, försäkrade mig, att de ej sällan voro fullkomligt blinda; en som jag fick lefvande var helt säkert, såsom undersökningen tillkännagaf, blind i följd af inflammation i blinkhinnan. Då inflammationer i ögonen måste vara skadliga för djuret, och ögonen icke äro oumbärliga för ett djur med underjordiska vanor, så kan i sådant fall en förminskning af deras storlek, sammanväxning af ögonlocken och beklädnad med hår vara en ren fördel, och om detta är förhållandet, så understödes det naturliga urvalets verkan af organets bristande användning.
Det är väl bekant, att flera djur af de mest olika klasser, som bebo hålorna i Kärnthen och Kentucky, äro blinda. Hos några krabbor finnes ögonstjelken qvar, ehuru ögonen äro borta; teleskopställningen finnes qvar, ehuru teleskopet med sina glas fattas. Det är svårt att föreställa sig, att ögon ehuru onyttiga kunde vara till någon skada för djur som lefva i mörkret, och derföre tillskrifver jag deras förlust helt och hållet bristen på användning. Professor Silliman hade fångat två hålråttor (Neotoma), en blind djurart, en half engelsk mil innanför ingången till den underjordiska hålan, och således ej i dess innersta del; deras ögon voro stora och glänsande och Silliman meddelar, att sedan de under en månads tid småningom vänts vid starkt ljus, fingo de en svag synförmåga och begynte blinka.

Det är svårt att föreställa sig mera identiska lefnadsvilkor än djupa kalkstensgrottor i nästan samma klimat, så att, om man utgår från den vanliga åsigten, att blinda djur hafva blifvit särskildt skapade för Amerikas och Europas grottor, man äfven kunde vänta en stor öfverensstämmelse i organisation och slägtskapsförhållanden. Någon sådan likhet emellan de båda djurformerna förefinnes i det hela alldeles icke, och Schiödte anmärker med afseende på insekterna allena, att hela fenomenet bör betraktas såsom rent lokalt, och den likhet, som finnes emellan några invånare i Mammuthålan i Kentucky och i Kärnthenhålorna, är blott helt enkelt följd af den analogi, som öfverhufvud finnes emellan Europas och Nordamerikas fauna. Enligt min tanke måste man antaga, att amerikanska djur, som hade vanlig synförmåga, i en följd af generationer småningom från den yttre verlden trängt allt djupare och djupare in i de aflägsnaste vinklar i Kentuckyhålorna på samma sätt som europeiska djur i Kärnthenhålorna. Vi hafva några bevis för en sådan gradvis förändring af lefnadsvanorna; Schiödte anmärker: ”Vi betrakta derföre dessa underjordiska faunor såsom små ned i jorden trängande förgreningar af den närmaste omgifningens geografiskt begränsade faunor, hvilka i samma mån de tränga djupare ned i mörkret lämpa sig efter de omgifvande förhållandena; djur som föga afvika från den vanliga formen bilda öfvergången från ljus till mörker; derpå följa de djur som äro bildade för skymning och slutligen de som äro bestämda för totalt mörker, hvilkas skapnad är helt egendomlig.” Dessa Schiödtes anmärkningar hänföras icke blott till en enda, utan på helt skilda arter. Då ett djur efter otaliga generationer nått de djupaste vinklarna, har bristande användning mer eller mindre fullständigt tillintetgjort synorganet, och det naturliga urvalet har ofta åstadkommit andra förändringar, såsom en tillväxt i antennerna eller känseltrådarna till ersättning för synen. Oaktadt dessa modifikationer kunna vi ännu vänta att hos Amerikas grottdjur finna slägtskap med de andra invånarna i samma kontinent, och hos Europas grottdjur med de öfriga europeiska djuren. Detta är verkligen fallet med några af Amerikas grottdjur, såsom jag hör af Professor Dana, och några europeiska grottinsekter äro mycket nära beslägtade med den omgifvande traktens insekter. Det skulle vara mycket svårt att gifva en rationel förklaring öfver de blinda grottdjurens slägtskap med de andra invånarna i de två verldsdelarna, om man utgår från den vanliga åsigten om deras särskilda skapelse. Att några af den gamla och nya verldens grottinvånare äro nära beslägtade kunna vi vänta från den välkända slägtskapen emellan de flesta andra af deras alster. Då en blind Bathysciaart funnits i stort antal på skuggiga klippor utanför hålorna, så står sannolikt icke synens förlust hos de arter af samma slägte som bebo grottorna i något samband med boningsplatsens mörker, och det är lätt begripligt att en redan förut blind insekt lätt skall finna sig i att bebo en mörk håla. Ett annat blindt slägte, Anophthalmus, erbjuder den märkvärdiga egendomligheten, att såsom Murray anmärkte dess olika arter bo i åtskilliga hålor i Europa likaväl som i Kentuckys hålor, och att slägtet öfverhufvud blott lefver i grottor. Det är dock möjligt att stamfadern eller stamfäderna till dessa skilda slägten förut varit försedda med ögon, varit vidt utbredda i båda verldsdelarna och (liksom båda verldsdelarnas elefanter) dött ut med undantag af de i sina trånga fängelser nu lefvande arterna. Långt ifrån att öfverraskas deraf, att några af grottdjuren äro mycket anomalt bildade, såsom Agassiz anmärker om den blinda fisken Amblyopsis, och såsom förhållandet är med den blinda reptilen Proteus, jemförd med Europas öfriga reptilier, förvånas jag snarare deraf, att icke flera spillror af det gamla lifvet bevarats, då invånarna i dessa mörka boningar icke kunna hafva varit utsatta för en så särdeles sträng täflan.



\section{Acklimatisering.}

Vana är ärftlig hos växter, såsom blomningstid, nödig regnmängd för groningsprocessen, softid och så vidare och detta föranleder mig att här säga något om acklimatisering. Då det är mycket vanligt, att arter af samma slägte bebo mycket heta så väl som mycket kalla trakter, och då enligt min åsigt alla arter af samma slägte härstamma från en enda stamart, så måste, om detta är riktigt, acklimatisering utan svårighet under en lång följd af generationer kunna åstadkommas. Det är väl bekant, att hvarje art är lämpad efter klimatet i dess eget hemland: arter från en arktisk eller tempererad trakt kunna icke tåla ett tropiskt klimat och tvärtom. Många saftiga växter kunna icke uthärda ett fuktigt klimat. Men graden af arters lämplighet för det klimat under hvilket de lefva har ofta blifvit för högt uppskattad. Vi kunna sluta oss dertill af vår oförmåga att förutsäga, huruvida en importerad växt skall uthärda vårt klimat eller ej, och från det stora antalet växter och djur, som blifvit förflyttade till oss från varmare klimat och trifvas väl. Vi hafva skäl för det antagandet, att arternas utbredning i naturtillståndet begränsas lika mycket, om icke mer, af täflan med andra organiska varelser, än genom deras lämplighet för ett visst klimat. Må nu denna lämplighet i allmänhet vara noga afpassad eller icke, så hafva vi dock för några få växtarter bevis på, att de redan af naturen äro till en viss grad vända vid olika temperaturer eller acklimatiserade; arter af Pinus och Rhododendron, uppdragna af frön som Hooker samlat af träd, växande vid olika höjder på Himalaya, visa i England olika förmåga att uthärda köld. Herr Thwaites har upplyst mig om, att han observerat likartade fakta på Ceylon och liknande observationer har H. C. Watson gjort på Europeiska arter af växter, som införts från Azoriska öarna till England. Hvad djuren beträffar kunde trovärdiga fall uppgifvas af arter, som under den historiska tiden spridt sig långt omkring från varmare trakter till kallare och tvärtom; men vi veta icke bestämdt, om dessa djur voro väl afpassade efter klimatet i deras hemort, ehuru i alla vanliga fall vi antaga detta; ej heller veta vi om de sedermera blifvit acklimatiserade i sina nya hemland.

Då vi kunna antaga, att våra husdjur ursprungligen utvalts af ociviliserade menniskor derföre att de voro nyttiga och utan svårighet fortplantade sig i fångenskapen och ej derföre att de sedermera befunnos i stånd att uthärda långa förflyttningar, så tror jag att våra husdjurs allmänna och utomordentliga förmåga, att icke blott tåla vid de mest olika klimat, utan att äfven förblifva fullkomligt fruktsamma (ett mycket starkare skäl) kan tjena som argument för, att en stor del andra djur som nu äro i vildt tillstånd kunde utan svårighet bringas att uthärda vidt skilda klimat. Vi få dock icke drifva detta argument för långt, emedan några af våra husdjur sannolikt härstamma från flera vilda stammar, så att i våra tama hundraser till exempel blodet af en arktisk och en tropisk varg eller vild hund kan vara blandadt. Råttor och möss kunna icke betraktas som husdjur, men de hafva af menniskan blifvit förflyttade till många delar af verlden och hafva nu vida större utbredning än någon annan gnagare, de lefva fritt under Färöarnas kalla himmel i norden; på Falklandsöarna i söder och många öar i de heta zonerna. Derföre är jag böjd att anse lämpligheten för ett visst klimat för en egenskap, som utan svårighet kan inympas på en medfödd böjlighet i konstitution, som är gemensam för de flesta djur. Menniskans och hennes husdjurs förmåga att uthärda de mest olika klimat och sådana fakta, som att de fordna arterna af elefanten och noshörningen voro i stånd att uthärda nordens klimat, under det att de nu lefvande arterna äro tropiska eller subtropiska i sitt lefnadssätt, böra enligt denna åsigt icke betraktas såsom oregelbundenheter, utan blott såsom exempel på en mycket vanlig böjlighet i konstitution, som under vissa omständigheter gör sig gällande.

En mycket svår fråga är att afgöra huru mycket af arternas acklimatisering för ett visst klimat beror blott på vana, och huru mycket på naturens urval af varieteter med medfödd olika kroppsbeskaffenhet. Att vana och öfning derpå har ett visst inflytande, måste jag tro både från analogi och från oupphörliga varningar i arbeten öfver landthushållning, till och med i de gamla kinesiska encyklopedierna, att vara mycket försigtig vid ett djurs förflyttning från ett område till ett annat; ty det är icke sannolikt, att menniskan med så mycken framgång skulle hafva utvalt så många raser och underraser med kroppskonstitution särskildt lämpad för det af henne sjelf bebodda området: resultatet måste i min tanke tillskrifvas vanan. Å andra sidan kan jag ej se någon anledning till tvifvel, att det naturliga urvalet fortfarande sträfvar att bevara de individer som äro födda med en kroppskonstitution särdeles passande för deras hemland. I skrifter öfver åtskilliga slag af odlade växter sägas vissa varieteter bättre än andra kunna tåla vid vissa klimat: detta är på ett mycket slående sätt visadt i arbeten öfver fruktträd i Förenta Staterna, i hvilka vissa varieteter vanligen äro rekommenderade för de nordliga, andra för de sydliga staterna; och då de flesta af dessa varieteter äro nybildade, så kan man icke tillskrifva vanan deras konstitutionella olikheter. Jerusalemärtskockan, som i England aldrig fortplantar sig med frön och följaktligen aldrig lemnat nya varieteter, anföres såsom bevis att acklimatisering ej kan åstadkommas, ty den är ännu lika känslig som någonsin: den turkiska bönan har likaledes ofta blifvit framhållen i samma afsigt och med ännu mera eftertryck; men förr än någon under ett tjog generationer har sått sina bönor så tidigt, att en mycket stor del blifvit förstörd af frost, och sedan samlat frön af de få öfverlefvande, undvikande omsorgsfullt tillfälliga kroaseringar, och sedan med samma försigtighetsmått samlat frön från dessa plantor, förr kan man icke ens säga, att experimentet blifvit försökt. Och vi kunna icke heller antaga, att olikheter i konstitution icke förekomma emellan dessa plantor, ty uppgifter hafva blifvit offentliggjorda, som visa att vissa plantor synas vara vida mera resistenta än andra och jag har sjelf derpå observerat mycket slående exempel.

Öfverhufvudtaget tror jag vi kunna sluta oss till, att vana, användning och bristande öfning i vissa fall hafva spelat en ansenlig rol i modifikationen af konstitutionen och vissa organers struktur, men att verkningarna af bruk och bristande användning ofta i vidsträckt grad blifvit förenade med och stundom öfverväldigade af det naturliga urvalet.



\section{Utvecklingens vexelverkan.}

Med detta uttryck menar jag, att hela organisationens alla delar dess tillväxt och utveckling äro så nära förenade, att om små variationer uppkomma i en del och af det naturliga urvalet förökas, så förändras äfven andra delar. Detta är en mycket vigtig punkt, men ännu föga känd, och helt och hållet skilda klasser af fakta kunna här lätt förvexlas med hvarandra. Vi skola strax se att enkelt arf ofta falskeligen erbjuder utseendet af en vexelverkan. Det tydligaste exempel är, att modifikationer som samlas blott för ungens eller larvens bästa inverkar på de fullväxta individernas skapnad, på samma sätt som någon missbildning, som träffar det tidiga embryot, allvarsamt inverkar på den fullt utbildade individens organisation. De mångfaldiga delar af kroppen, som äro homologa och hvilka vid en tidig embryonal period äro identiska, synas vara benägna att variera på likartadt sätt: vi se detta deruti, att den högra och venstra sidan af kroppen förändras på samma vis, likaså framben och bakben, och vi se till och med käkarna variera på samma sätt som extremiteterna, ty några anse ju underkäken vara homolog med extremiteterna. Jag tviflar dock icke, att dessa tendenser mer eller mindre fullständigt beherskas af det naturliga urvalet; så har det en gång gifvits en hjortfamilj med blott ett horn, och om denna afvikelse hade varit af någon större nytta för afkomlingarna, är det sannolikt, att det naturliga urvalet gjort den bestående.

Såsom några författare hafva anmärkt, visa homologa delar en benägenhet att växa ihop; detta observeras ofta hos monströsa växter och ingenting är vanligare än en förening af homologa delar till en normal bildning, såsom föreningen af kronbladen i blomkronan till ett rör. Hårda delar synas inverka på formen af närliggande mjuka delar; några författare tro, att olikheten i bäckenets skapnad hos fåglar förorsakar den anmärkningsvärda skilnaden i njurarnas skapnad. Andra tro att bäckenets form hos qvinnan genom tryck inverkar på formen af barnets hufvud. Enligt Schlegel bestämmes hos ormarna läget af vissa vigtiga inelfvor af kroppens skapnad och sättet att svälja.

Beskaffenheten af detta sammanhang är ofta mycket dunkel. Isidore Geoffroy S:t Hilaire har med eftertryck framhållit, att vissa missbildningar mycket ofta förekomma tillsammans, andra sällan, och detta utan att vi äro i stånd att angifva något skäl dertill. Hvad kan vara mer egendomligt än sambandet emellan blå ögon och fullkomlig döfhet hos katten, färgen på sköldpaddhonans skal och hennes kön, fjäderbeklädnaden på fötterna och huden emellan de yttre tårna hos dufvorna eller emellan närvaron af mer eller mindre dun hos de unga nykläckta fåglarna och deras framtida fjäderbeklädnads färg, eller åter förhållandet emellan håret och tänderna hos den nakna turkiska hunden, ehuru här otvifvelaktigt homologi kommer med i spelet. Med afseende på detta slag af vexelverkan synes det mig knappt såsom en tillfällighet, att de två däggdjursordningar hvilka mest afvika i sin hudbeklädnad, nämligen Cetacea (hvalar) och Edentata (tandlöse, bältor, myrkottar), också äro de mest afvikande i sin tandbygnad.

Jag känner intet fall, som vore lämpligare att visa vigten af lagarna för vexelverkan, oberoende af fördel och således äfven af det naturliga urvalet, än skilnaden emellan de yttre och inre blommorna hos vissa växter af familjerna Compositæ och Umbelliferæ. Hvar och en känner skilnaden emellan kantblommorna och de centrala blommorna hos till exempel tusenskön (Bellis) och denna skilnad är ofta åtföljd af förkrympning af vissa blomdelar. Men hos vissa Compositæ skilja sig äfven frukterna i storlek och skapnad och till och med fruktämnet med dess bidelar visar olikheter såsom Cassini har beskrifvit. Dessa skilnader hafva af några författare blifvit tillskrifna tryck och skapnaden af strålblommornas frukter hos några Compositæ understödja denna åsigt. Men bland umbellaterna är det, såsom Hooker upplyser mig, ingalunda hos de tätaste blomhufvuden, som de inre och yttre blommorna oftast visa skiljaktigheter. Man kunde tro, att utvecklingen af de i blomflockens rand befintliga kronbladen skulle beröfva vissa andra blomdelar deras näring och sålunda förorsaka deras förkrympning, men i vissa Compositæ finnas olikheter i frukterna af de yttre blommorna, utan någon skilnad i blomkronan. Möjligt är, att dessa olikheter kunna stå i sammanhang med ett olika tillflöde af näringsvätska till de yttre blommorna: vi veta åtminstone, att af oregelbundna blommor de som stå närmast axeln oftast visa benägenhet att blifva regelbundna. Såsom ett exempel härpå och tillika såsom ett slående fall af vexelverkan, vill jag tillägga att jag nyligen i några trädgårdspelargonier observerat, att centralblomman i knippet ofta förlorar de mörka fläckarna på de öfre kronbladen, och att om detta händer det vidhängande honingsgömmet försvinner; om färgen är borta blott från det ena af de två öfre kronbladen, är honingsgömmet blott mycket förkortadt.

Med afseende på olikheterna i blomkronan af de yttre och centrala blommorna af ett blomhufvud eller en blomflock, vill jag nämna, att jag alls icke anser Sprengels åsigt vara så sökt, som den först synes, att kantblommorna tjena till att ditlocka insekter, hvilkas verksamhet är särdeles vigtig för befruktningen af dessa två familjers växter: och om det är af nytta, kommer väl äfven det naturliga urvalet med i spelet. Deremot synes det knappt möjligt att skilnaden emellan både den inre och yttre skapnaden af fröen som icke alltid är förenad med olikheter i blommorna, kan vara af någon fördel för växten; dock äro dessa olikheter hos Umbelliferæ af så synbar vigt, att den äldre de Candolle grundade sina stora indelningar af denna familj på sådana olikheter. Vi se sålunda att modifikationer, som af systematikern betraktas såsom af högt värde, kunna helt och hållet bero på okända lagar för utvecklingens vexelverkan och ej erbjuda arterna, så vidt vi kunna se, den ringaste fördel.

Vi kunna ofta oriktigt tillskrifva utvecklingens vexelverkan bildningar, som äro gemensamma för hela artgrupper och hvilka i sjelfva verket blott bero på ärftlighet; ty en gammal stamfader kan genom naturligt urval hafva förvärfvat någon egendomlighet i kroppsbildning och efter tusen generationer någon annan, helt och hållet oberoende förändring och dessa två förändringar, som sedan gått i arf till en hel grupp af ättlingar med skilda lefnadsvanor, kunna helt naturligt anses stå i något nödvändigt samband med hvarandra. Så tviflar jag icke heller derpå, att vissa skenbara vexelverkningar, som möta genom hela ordningar, blott och bart bero på det sätt, hvarpå det naturliga urvalet kan verka. Alphonse de Candolle har till exempel anmärkt, att vingade frön aldrig förekomma i frukter som icke öppnas; jag skulle förklara denna regel så, att fröen icke kunna småningom antaga vingar genom det naturliga urvalet annat än i öppnade frukter, så att växtindivider med frön, som voro blott litet bättre egnade att föras vida omkring, kunde få någon fördel öfver andra, som lemna frön mindre tjenliga att spridas, och denna process kan omöjligen försiggå i en frukt som icke öppnar sig.



\section{Utvecklingens kompensering och ekonomi.}

Den äldre Geoffroy och Goethe uppstälde ungefär samtidigt sin lag om utvecklingens kompensering eller jemvigt, eller såsom Goethe uttryckte den, ”för att slösa på ena sidan måste naturen hushålla på den andra”. Till en viss grad passar detta i min tanke in på våra kulturalster: om näring tillflödar en del eller ett organ i öfvermått, så strömmar den sparsamt eller åtminstone icke i öfverflöd till en annan del; det är derföre svårt, att få en ko att gifva mycket mjölk och på samma gång blifva fet. Samma kålvarietet kan icke lemna en riklig mängd födande blad och en stor mängd oljrika frön. Då fröen i våra frukter förkrympa, vinner frukten sjelf i storlek och beskaffenhet. Hos våra höns åtföljes vanligen en stor fjädertofs på hufvudet af en förminskad kam och ett stort skägg af förminskade köttflikar. Dock kan man näppeligen antaga, att denna lag eger i full utsträckning sin tillämpning på arter i naturtillståndet, ehuru många goda iakttagare, isynnerhet botanister, tro på dess sanning. Här vill jag dock ej anföra några exempel, ty jag kan svårligen finna ett medel att skilja emellan följderna af å ena sidan en dels stora utveckling genom det naturliga urvalet och de närliggande delarnas förkrympning af samma orsak eller af bristande användning, och å andra sidan ett organs förkrympning genom brist på näringsämnen i följd af andra närliggande delars utomordentliga utvidgning.

Jag förmodar också, att några anförda fall af kompensering kunna jemte några andra fakta bringas under en mera allmän grundsats, att nämligen det naturliga urvalet alltjemt försöker att vara sparsamt i hvarje del af organisationen. Om under förändrade lefnadsvilkor ett förut fördelaktigt organ blifver mindre nyttigt, så företages väl en förminskning, ehuru ringa, i dess utveckling af det naturliga urvalet, ty det skulle icke vara gynsamt för individen att förslösa sin näring på att uppbygga ett onyttigt organ. Blott på detta sätt kan jag förstå ett faktum, som mycket förvånade mig, då jag undersökte cirripederna, och många exempel kunde gifvas derpå, nämligen att en cirriped, som lefver såsom parasit inuti ett annat djur och på detta sätt är skyddad, förlorar mer eller mindre fullständigt sitt eget skal. Detta är fallet med hannen af Ibla och på ett i sanning utomordentligt sätt med Proteolepas; alla andra cirripeders pansar består af de tre ofantligt utvecklade främre segmenterna af hufvudet och är försedt med starka nerver och muskler, men hos den parasitiska och skyddade Proteolepas är hela främre delen af hufvudet reducerad till ett blott rudiment fästadt vid basen af gripantennerna. Inbesparandet af ett stort och inveckladt organ, som blifvit öfverflödigt genom Proteolepas’ parasitiska lefnadssätt, om det också skedde blott småningom, skulle vara en afgjord fördel för hvarje efterkommande individ af arten; ty under den kamp för sin tillvaro, som hvarje djur måste utkämpa, skulle hvarje proteolepasindivid hafva bättre utsigt att försvara sig, om en mindre mängd näring bortslösades till utbildande af ett numera fullkomligt onyttigt organ.

På detta sätt tror jag det alltid lyckas för det naturliga urvalet i längden att reducera och inbespara hvarje del af organisationen, så snart den blifvit öfverflödig, utan att derföre på något sätt gifva anledning till en annan dels större utveckling i motsvarande grad. Och tvärtom lyckas det lika väl för det naturliga urvalet att bringa ett visst organ till stor utveckling utan att dertill såsom en nödvändig kompensering behöfva reduktionen af några närliggande delar.



\section[Rudimentära bildningar]{Multipa, rudimentära och lågt organiserade bildningar
äro föränderliga.}

Såsom Isidore Geoffroy S:t Hilaire anmärker, synes det vara en regel både ibland varieteter och arter, att om en del eller ett organ förekommer i mångfaldigt antal hos samma individ (såsom kotorna hos ormarna och ståndarna hos polyandriska blommor) antalet är föränderligt, hvaremot, om antalet är mindre, det är mera konstant. Samma författare och några botanister hafva äfven anmärkt, att mångtaliga organer äro mycket benägna för föränderlighet i sin inre bygnad. Så vida denna ”vegetativa upprepning”, för att begagna prof. Owens uttryck, är ett tecken till låg organisation, synes ofvanstående iakttagelse stå i sammanhang med den bland naturforskarna mycket allmänna åsigten, att varelser som stå lågt på naturens skala äro mera föränderliga än de högre. Med låg organisation menar jag i detta fall, att de olika delarna af organisationen äro blott i ringa mån specielt utbildade för vissa förrättningar; så länge samma del har att fullborda olikartade värf, kunna vi måhända inse, hvarföre den skulle förblifva föränderlig, det vill säga, hvarföre det naturliga urvalet icke med samma omsorg skulle hafva bevarat eller förkastat hvarje liten formafvikelse, som om organet vore bildadt blott för ett särskildt ändamål. En knif, som är ämnad att skära hvad som helst, kan hafva nästan hvarje godtycklig form, då deremot ett verktyg för ett visst ändamål lämpligare har en viss bestämd form. Vi få aldrig glömma, att det naturliga urvalet verkar på hvarje varelses särskilda organer blott i och för individens nytta.

Ofullkomligt utvecklade, rudimentära organer äro också i hög grad benägna för föränderlighet, såsom vissa författare påstå, i min tanke med fullt skäl. Vi skola återkomma till de rudimentära och abortiva organerna i allmänhet och här vill jag blott tillägga, att deras föränderlighet synes bero på deras gagnlöshet, och på den grund har det naturliga urvalet ingen förmåga att förhindra afvikelser i deras bygnad. Rudimentära organer äro således prisgifna åt det fria inflytandet af utvecklingens olika lagar, följderna af en länge fortsatt overksamhet och benägenheten till återgång.



\section[Variation i ovanliga grader]{En hos en art i ovanlig grad eller på ovanligt sätt
utvecklad del har i jemförelse med samma del
hos en beslägtad art en stor benägenhet att
variera.}

För många år sedan öfverraskades jag högeligen af ett likartadt yttrande af Waterhouse; en observation af prof. Owen öfver armarnas längd hos orang-outang gör det sannolikt, att äfven han har kommit till nästan samma åsigt. Jag kan icke hoppas öfvertyga någon om riktigheten af detta påstående utan att framlägga den långa lista af fakta som jag samlat, men som ej här kan införas. Jag kan blott uttala min öfvertygelse, att denna regel är mycket allmän. Flera källor till fel har jag visserligen blifvit varse, men jag tror mig hafva fästat tillbörligt afseende vid dem. Framför allt är att observera, att denna regel icke har sin tillämpning på någon del, om den ock är ovanligt utvecklad, med mindre dess utveckling är ovanlig i jemförelse med samma del hos närbeslägtade arter. Flädermössens vingar äro visserligen en ytterst abnorm bildning hos däggdjursklassen, men här har regeln ingen tillämpning, då vingbildningen tillhör en hel större grupp; den kunde blott tillämpas, om någon art af flädermöss hade sina vingar utbildade på ett ovanligt sätt i jemförelse med de andra arterna af samma slägte. Regeln har en mycket sträng tillämpning på ”sekundära sexualkarakterer” om de äro ovanligt utbildade. Med termen ”sekundära sexualkarakter” menar Hunter sådana karakterer som tillhöra ett enda kön, hanne eller hona, utan att stå i omedelbart sammanhang med fortplantningen. Regeln gäller både hannar och honor, men då honorna mera sällan erbjuda anmärkningsvärda sekundära sexualkarakterer, så kan regeln mera sällan tillämpas på dem. Att regeln är så fullständigt användbar på dessa karakterer, beror väl på deras stora föränderlighet, vare sig de äro ovanligt utbildade eller icke, ett förhållande hvarom föga tvifvel kan hysas. Men att vår regel icke är inskränkt härtill visa de hermafroditiska cirripederna, och jag vill tillägga, att jag vid undersökningen af denna ordning lade särskildt märke till Waterhouses yttrande, och jag är fullkomligt öfvertygad att regeln nästan oföränderligen gäller hos cirripederna. I ett kommande arbete skall jag gifva en lista på mera anmärkningsvärda fall, här vill jag blott i korthet gifva ett, som belyser regeln i dess vidsträcktaste användning. Täckvalvlerna hos de sessila cirripederna (Balanider) äro i hvarje mening mycket vigtiga organer och skilja sig ytterst litet hos skilda slägten; men hos flera arter af ett slägte, Pyrgoma, visa dessa valvler en märkvärdig grad af olikhet: de homologa valvlerna äro hos skilda arter stundom fullkomligt olika formade; och graden af afvikelse hos individerna af flera arter är så stor, att det icke är någon öfverdrift att påstå, att varieteterna skilja sig mera från hvarandra i karakteren af dessa vigtiga valvler, än andra arter af skilda slägten.

Då fåglar inom samma trakt variera utomordentligt litet, har jag fästat särskild uppmärksamhet vid dem och regeln synes mig helt visst hafva stor giltighet hos denna klass. Jag kan icke visa, att den har tillämpning på växterna och detta skulle allvarsamt hafva skakat min tro på dess sanning, om icke växternas stora föränderlighet öfverhufvud hade gjort det särdeles svårt att jemföra den relativa föränderlighetsgraden.
Om vi hos en art se en del eller ett organ utveckladt på anmärkningsvärdt sätt eller i ovanlig grad, så ligger det nära till hands att antaga, att organet är af stor vigt för denna art, och dock är delen i detta fall utomordentligt benägen för föränderlighet. Hvarpå kan detta bero? Från den åsigten, att hvarje art blifvit särskildt skapad med alla sina delar sådana dessa äro, kan jag icke se någon förklaring. Men vår åsigt, att grupper af arter hafva härstammat från andra arter och hafva blifvit modifierade af det naturliga urvalet, tror jag kan sprida något ljus häröfver. Om hos våra husdjur någon del eller hela djuret försummas och något urval ej användes, så skall en sådan del (såsom kammen hos Dorkinghönsen) eller hela rasen upphöra att bevara sin nästan likformiga karakter. Rasen säges då hafva urartat. I rudimentära organer och sådana, som blott i ringa grad blifvit specielt bildade för ett visst ändamål och måhända äfven i polymorfa grupper se vi ett nästan liknande fall i naturen, ty i sådana fall har det naturliga urvalet icke kunnat komma i full verksamhet och således lemnas organisationen i en vacklande ställning. Men hvad som häraf mera särskildt angår oss är, att hos våra husdjur de delar, som för närvarande undergå hastiga förändringar genom ett fortsatt urval, också äro i hög grad benägna att variera. Man må betrakta de olika dufraserna, se denna märkvärdiga olikhet i näbben hos de olika tumletterna, i näbben och hudvårtorna hos brefdufvorna, i påfågeldufvornas hållning och stjert, etc., och dessa äro de karakterer vid hvilka de engelska amatörerna hufvudsakligen fästa afseende. Äfven underraserna, såsom den korthöfdade tumletten, äro svåra att få fullkomliga och ofta födas individer som vidt skilja sig från typen. Man kan derföre i sanning säga att en beständig strid fortgår emellan å ena sidan så väl benägenheten för återgång till ett mindre modifieradt stadium som en medfödd sträfvan till vidare föränderlighet af alla slag, och å andra sidan inflytandet af fortfarande urval till att bibehålla rasen ren. I längden vinner urvalet seger och vi frukta icke att få en vanlig tumlett af en ädel korthöfdad ras. Men så länge urvalet fortgår hastigt, kunna vi vänta att finna mycken obeständighet hos de organer som undergå modifikationer. Det förtjenar vidare att anmärkas, att dessa föränderliga karakterer, som menniskan med sitt urval framkallar, stundom af fullkomligt okända orsaker mera hålla sig till ena könet än det andra, i allmänhet till hannarna, såsom brefdufvans hudvårtor och kroppdufvans utvidgade kräfva.
Låt oss nu återvända till naturen. Om ett organ har blifvit utveckladt på något utomordentligt sätt hos någon art, jemförd med andra arter af samma slägte, kunna vi antaga, att detta organ har undergått en ansenlig förändring sedan den tid, då arten afskilde sig från slägtets gemensamma stamfader. Denna period ligger sällan ytterligt långt tillbaka, då arter i allmänhet ej bibehålla sig längre än under en geologisk period. En utomordentlig grad af modifikation förutsätter en ovanligt stor och länge fortsatt föränderlighet, som det naturliga urvalet länge användt till artens bästa. Men då det ovanligt utbildade organets föränderlighet har varit så stor och länge fortfarit under en ej så ytterligt aflägsen period, kunna vi såsom en allmän regel vänta att ännu finna en större föränderlighet i sådana delar än i andra delar af organisationen, hvilka under en mycket längre period hållit sig beständiga. Och detta är enligt min öfvertygelse verkliga förhållandet. Att striden emellan det naturliga urvalet å ena sidan och benägenheten till återgång och föränderligheten å den andra under tidens lopp skall upphöra och att äfven de mest abnormt bildade organer kunna blifva beständiga, derpå ser jag intet skäl att tvifla. Om derföre ett organ, huru abnormt det än är, har i ungefär samma tillstånd öfvergått på många modifierade ättlingar, såsom flädermössens vingar, måste det enligt min teori hafva funnits redan under en ofantligt lång period i nästan samma tillstånd, och derföre är det nu icke mera föränderligt än något annat organ. Det är blott i de fall, i hvilka modifikationen är jemförelsevis ny och utomordentligt stor, som vi skola i hög grad ännu finna denna generativa föränderlighet, som vi kunna kalla den. Ty i dessa fall skall föränderligheten blott sällan ännu hafva blifvit tillintetgjord genom ett fortsatt urval af de individer som varierade på lämpligt sätt och i lämplig grad, och genom ett fortsatt afskiljande af dem som sträfvade att återgå till en tidigare och mindre modifierad form.



\section[Artkaraktarer äro föränderligare]{Artkarakterer äro föränderligare än slägtkarakterer.}

Den grundsats, som innefattas i denna tankegång, kan vidare utsträckas, ty det är bekant att artkarakterer äro föränderligare än slägtkarakterer; ett enkelt exempel förklarar hvad jag härmed menar. Om några arter i ett stort växtslägte hafva blå blommor och andra röda, så är blommornas färg blott en artkarakter och ingen skulle öfverraskas deraf att någon af de blå arterna varierade med röda blommor eller tvärtom, men om alla arterna hade blå blommor, då vore färgen en slägtkarakter och dess föränderlighet vore en mera ovanlig omständighet. Jag har valt detta exempel, emedan derpå icke kan tillämpas den förklaring som de flesta vetenskapsmän eljest pläga framhålla, att nämligen specifika karakterer äro mera föränderliga än slägtkarakterer, emedan de hemtas från delar af mindre fysiologisk vigt än de delar, som vanligen begagnas att klassificera slägtena. Jag tror visserligen att denna förklaring delvis, ehuru blott indirekt, är riktig, jag skall likväl återkomma härtill i vårt kapitel om klassifikationen. Det torde vara nästan öfverflödigt att anföra exempel till bevis för ofvanstående sats, att artkarakterer äro mera föränderliga än slägtkarakterer; men jag har upprepade gånger i naturhistoriska verk funnit, att om någon författare med öfverraskning iakttagit att ett vigtigt organ, som i allmänhet i stora artgrupper är mycket konstant, hos närbeslägtade arter visade betydliga skiljaktigheter så var det också föränderligt hos individerna bland några af dessa arter. Och detta förhållande visar, att en karakter, som i allmänhet har värdet af slägtkarakter, om den faller i värde och blir blott af en artkarakters värde, ofta blifver föränderlig, ehuru dess fysiologiska vigt förblir densamma. Något likartadt äger rum vid monstrositeter: åtminstone tyckes Isidore Geoffroy S:t Hilaire icke tvifla på, att ju större olikheter ett organ visar i skilda arter af samma grupp, ju mera är det underkastadt individuela anomalier.

Enligt den vanliga åsigten, att hvarje art är en oberoende skapelse, huru skulle den del af organisationen, som skiljer sig från samma del hos andra särskildt skapade arter af samma slägte, kunna vara mera föränderlig än de delar som äro nästan öfverensstämmande hos skilda arter? Jag kan icke deraf finna någon förklaring häröfver. Men om vi utgå från den åsigten, att arter äro blott väl utpräglade och permanenta varieteter, kunna vi med all säkerhet vänta att finna dem fortfarande variera i de delar af organisationen, hvilka inom en relativt ny period hafva varierat och på detta sätt kommit att förete afvikelser. Eller för att framställa förhållandet i andra ord: — de kännetecken, hvaruti alla arter af ett slägte likna hvarandra, och hvaruti de skilja sig från arterna af något annat slägte, kallas slägtkarakterer, och dessa karakterer tillskrifver jag i allmänhet arf från någon gemensam stamfader; ty det bör väl sällan hafva händt, att det naturliga urvalet på fullkomligt samma vis modifierat flera arter, lämpade för mer eller mindre olika lefnadsvanor: och då dessa så kallade slägtkarakterer hafva gått i arf från en aflägsen tid, då arterna först afskilde sig från deras gemensamma ursprung och följaktligen icke hafva varierat något eller blott obetydligt, är det icke sannolikt att de skulle variera för det närvarande. Å andra sidan, de kännetecken som skilja arterna af samma slägte från hvarandra kallas artkarakterer, och då dessa sedan den tid då de afskilde sig från sin gemensamma stamfar hafva varierat, så är det sannolikt, att de ofta ännu skola till en viss grad vara föränderliga, åtminstone mera föränderliga än de delar af organisationen, som en längre tid förblifvit konstanta.



\section[Sekundära sexualkarakterier]{Sekundära sexualkarakterer äro mer föränderliga
än artkarakterer.}

I sammanhang med detta ämne vill jag göra blott två andra anmärkningar. Jag tror man skall medgifva, utan att jag ingår i några detaljer, att sekundära sexualkarakterer äro mycket föränderliga; jag tror också man skall medgifva, att arter af samma grupp skilja sig från hvarandra vida mer i sina sekundära sexualkarakterer än i andra delar af sin organisation; jemför man till exempel graden af olikhet emellan hannarna af hönsfåglarna, hos hvilka de sekundära könskaraktererna äro starkt utvecklade, med graden af olikhet emellan deras honor, så inses lätt sanningen af detta påstående. Orsaken till den ursprungliga föränderligheten hos de sekundära könskaraktererna är icke tydlig, men vi kunna inse, hvarföre dessa karakterer icke hafva blifvit så beständiga och likformiga som andra delar af organisationen, ty sekundära sexuela karakterer hafva blifvit samlade af det sexuela urvalet, som icke är så strängt i sin verkan som det naturliga urvalet, då det icke förstör de mindre gynnade hannarna, utan blott gifver dem en fåtaligare afföda. Ehvad orsaken än må vara till de sekundära sexualkarakterernas föränderlighet, då de nu engång äro föränderliga, har det sexuela urvalet ett stort spelrum för sin verksamhet och har således utan svårighet gifvit arterna af samma grupp en större grad af olikhet i sina sexuela karakterer än i andra delar af deras organisation.

Det är ett anmärkningsvärdt faktum, att de sekundära sexuela skilnaderna emellan de två könen af samma art i allmänhet tillkomma samma delar af organisationen, i hvilka de olika arterna af samma slägte skilja sig från hvarandra. På detta förhållande vill jag gifva två upplysande exempel, som tillfälligtvis stå först på min lista, och då olikheterna i detta fall äro af mycket ovanlig beskaffenhet, kan förhållandet knappt vara tillfälligt. Flera mycket stora grupper af skalbaggar hafva såsom gemensam karakter ett lika antal leder i tarserna, men hos familjen Engidæ varierar antalet betydligt efter Westwoods iakttagelser; och antalet varierar likaledes hos de två könen af samma art; hos de gräfvande Hymenoptera är kärlförgreningen i vingarna en karakter af största vigt, gemensam för stora grupper, men hos vissa slägten finnas olikheter i nervförgreningen hos olika arter och på samma sätt hos de båda könen af samma art. Lubbock har för kort tid sedan iakttagit, att några små krustaceer lemna förträffliga bevis för denna lag. Hos Pontanella till exempel är det egentligen de främre känseltrådarna och det femte benparet som lemna sexualkarakterer och samma organer erbjuda äfven de vigtigaste artskilnader. Detta förhållande har en klar betydelse enligt mitt åskådningssätt: jag betraktar nämligen alla arter af samma slägte såsom afkomlingar från samma stamfader, likasom de båda könen af hvarje art. Följaktligen, om en del af den gemensamma stamfaderns eller hans tidigare ättlingars organisation började variera, skulle sannolikt dels det naturliga urvalet, dels det sexuela urvalet gynnat denna dels förändringar för att göra de olika arterna passande hvar och en för sin plats i naturens hushållning, och likaledes för att göra de båda könen lämpliga för hvarandra eller för att egna hannar och honor till olika lefnadssätt, eller slutligen för att sätta hannarna i stånd att kämpa med andra hannar om sina honor.

Ändtligen drager jag nu den slutsatsen, att den större föränderligheten hos de specifika karaktererna, som skilja art från art, än hos de generiska, som skilja slägte från slägte, och äro gemensamma för flera arter; — att den ofta ytterliga föränderligheten af någon del, som hos en art är utvecklad på ovanligt sätt i jemförelse med samma del hos dess samslägtingar, och den ringa föränderligheten hos en del, huru ovanligt utvecklad den än må vara, blott den är gemensam för en hel grupp af arter; — att den stora föränderligheten af sekundära könskarakterer, och den stora olikheten uti samma karakterer hos närbeslägtade arter; — att de sekundära sexualkarakterernas utveckling i samma delar af organisationen, som utgöra vanliga artkarakterer, — att alla dessa förhållanden äro grundsatser, som stå i nära sammanhang med hvarandra. Alla bero de hufvudsakligen derpå, att arterna af samma grupp hafva utgått från en gemensam stamfader, af hvilken de fått mycket gemensamt i arf, — vidare derpå, att delar som under sednare tid i hög grad hafva varierat visa större benägenhet att fortfarande undergå förändringar, än sådana som under en lång tid blifvit ärfda oförändrade, — derpå, att det naturliga urvalet har mer eller mindre fullständigt, alltefter tidens längd, öfverväldigat benägenheten till återgång och till vidare förändring, — derpå, att det sexuela urvalet är mindre strängt än det vanliga, — och derpå, att variationer i samma delar hafva blifvit samlade af det naturliga och sexuela urvalet, och således blifvit lämpade för sekundära sexuela och för vanliga specifika ändamål.



\section[Variation hos skilda arter]{Skilda arter förete analoga variationer, och en varietet
af en art antager ofta karaktererna af en närbeslägtad
art eller återgår till en tidig stamfaders karakterer.}

Dessa satser inser man utan svårighet vid betraktande af våra domesticerade raser. De mest skilda dufraser erbjuda i de mest skilda trakter undervarieteter med tillbakavända fjädrar på hufvudet och fjädrade fötter, karakterer, som icke finnas hos den ursprungliga klippdufvan; detta är alltså analoga variationer hos två eller flera skilda raser. Det vanliga förekommandet af fjorton eller till och med sexton stjertfjädrar hos kroppdufvan kan betraktas såsom en variation representerande den normala skapnaden af en annan ras, påfågeldufvan. Jag förmodar ingen vill betvifla, att alla sådana analoga variationer bero derpå, att alla dufraserna hafva ifrån en gemensam stamfader ärft samma konstitution och benägenhet för variation, då de utsättas för likartade inflytelser. I växtriket hafva vi ett fall af analog variation uti de utvidgade stammarna, eller rötterna såsom de vanligen kallas, af den svenska rofvan och Ruta baga, växter hvilka flera botanister betrakta såsom varieteter, frambragta genom odling af en gemensam stamfader: om detta icke är förhållandet, så hafva vi ett fall af analog variation i två så kallade bestämda arter, och till dessa kan läggas en tredje, den vanliga rofvan. Enligt den vanliga åsigten om hvarje arts oberoende skapelse skulle vi tillskrifva denna likhet i stammarnas utvidgning hos dessa tre växter icke vera causa, gemensamt ursprung och en deraf följande benägenhet att variera på samma sätt, utan tre särskilda ehuru nära beslägtade skapelseakter. Många liknande fall af analog variation hafva observerats af Naudin i den stora gurkfamiljen och af andra författare hos våra sädesarter. Likartade fall ibland insekter i deras naturtillstånd hafva nyligen med mycken skicklighet behandlats af mr Walsh, som har hänfört dem under sin lag om likartad föränderlighet.

Hos dufvor påträffa vi ännu ett annat fall, nämligen ett tillfälligt uppträdande hos alla raser af skifferblå fåglar med två svarta band å vingarna, hvit gump, ett band i kanten af stjerten och de yttersta stjertfjädrarna vid basen försedda med hvit kant. Då alla dessa kännetecken äro karakteristiska för stamfadern, klippdufvan, tror jag ingen vill betvifla, att detta är ett fall af återgång till stamfaderns karakter och icke af en ny analog variation hos alla raserna. I min tanke kunna vi sätta full lit till denna slutsats, emedan såsom vi hafva sett dessa färgkarakterer gerna uppträda hos bastarderna af två skilda och olika färgade raser; och i detta fall finnes ingenting i de yttre lifsvilkoren, som kan förorsaka den skifferblå färgens återuppträdande med de öfriga färgteckningarna, annat än kroaseringsaktens inflytande på lagarna för ärftligheten.

Otvifvelaktigt är det mycket öfverraskande, att finna karakterer uppträda, som varit förlorade under många, sannolikt hundratals generationer. Men då en ras blifvit kroaserad blott en gång med en annan ras, visar afkomman tillfälligtvis en benägenhet att återgå till den främmande rasens karakter under många generationer — enligt några tolf eller till och med tjugu generationer. Efter tolf generationer är, enligt det vanliga uttryckssättet, proportionen af den främmande stamfaderns blod 1 till 2048 och dock antages, såsom vi se, denna lilla qvantitet främmande blod qvarhålla en benägenhet till återgång. Hos en ras, som icke blifvit kroaserad, men hos hvilken båda föräldrarna hafva förlorat några karakterer som stamfadern egde, kan benägenheten att antaga den förlorade karakteren, vare sig den är stark eller svag, enligt hvad vi förut anfört bibehålla sig under nästan huru stort antal generationer som helst. Om en karakter, som hos en ras gått förlorad, efter ett stort antal generationer åter framträder, är det sannolikaste antagandet, icke att afkomman plötsligt sträfvar att likna en flera hundra generationer aflägsen stamfader, utan att hos hvarje generation den ifrågavarande karakteren legat latent och slutligen under okända gynsamma förhållanden blifvit utvecklad. Hos den indiska dufvan till exempel, som mycket sällan lemnar ett blått exemplar, finnes sannolikt en latent benägenhet hos hvarje generation att antaga denna färg. Möjligheten af karakterers bevarande i latent tillstånd under en lång tid kan förklaras enligt hypotesen om pangenesis, som jag framstält i ett annat verk. Den abstrakta osannolikheten af en sådan benägenhets latenta bibehållande under ett stort antal generationer är icke större än osannolikheten af fullkomligt onyttiga och rudimentära organers bibehållande på samma sätt. En benägenhet att frambringa rudimentära organer är verkligen ofta ärftlig.

Då alla arter enligt vår teori antagas härstamma från en gemensam stamfader, kunde vi vänta att de tillfälligtvis skulle variera på samma vis, så att varieteterna af två eller flera arter skulle likna hvarandra eller att en varietet af en art i vissa karakterer skulle likna en annan, bestämd art, då denna art enligt vårt åskådningssätt icke är annat än en väl markerad och permanent varietet. Men karakterer, som på detta sätt vunnits, skola sannolikt vara af ovigtig beskaffenhet, ty närvaron af alla mindre vigtiga karakterer regleras af det naturliga urvalet i öfverensstämmelse med arternas skilda vanor, och lemnas icke åt den ömsesidiga verkan af organismens beskaffenhet och lifsvilkoren. Man kan vidare vänta, att arterna af samma slägte skulle tillfälligtvis visa benägenhet att antaga stamfaderns längesedan förlorade karakterer. Då vi likväl aldrig fullständigt känna karaktererna hos den gemensamma stamfadern för en naturlig grupp, kunna vi icke skilja dessa två fall: om vi till exempel icke kände, att klippdufvan har hvarken fjäderbeklädda fötter eller omvända fjädrar, så kunde vi icke säga, om dessa karakterer hos våra tama raser vore återgång eller analoga variationer; men vi kunde hafva antagit, att den blå färgen vore ett fall af återgång på grund af antalet kännetecken, som äro förenade med denna färg, hvilka sannolikt icke skulle uppträda alla tillsammans såsom blott variation. Med mera säkerhet kunde vi hafva antagit detta på grund deraf, att den blå färgen och de öfriga kännetecknen så ofta uppträda, då skilda raser af olika färg kroaseras. Ehuru det i naturtillståndet i allmänhet måste lemnas oafgjordt, hvilka fall äro återgång till fordom existerande karakterer och hvilka fall äro blott analoga variationer, så böra vi dock enligt vår teori stundom finna en arts varierande afkomma antaga karakterer (antingen genom regress eller genom analog variation), som redan finnas hos andra medlemmar af samma grupp. Och detta är otvifvelaktigt förhållandet.

Svårigheten att i våra systematiska verk igenkänna en föränderlig art beror till en stor del derpå att dess varieteter liksom härma andra arter af samma slägte. Man kunde också uppställa en ansenlig lista på former, som stå midt emellan tvänne andra former, hvilka sjelfva hafva rang af blott tvifvelaktiga arter, och detta visar, att den ena under sin variation har antagit några af den andras karakterer och sålunda bildat mellanformerna, så vida icke man vill anse alla dessa former såsom särskildt skapade arter. Men det bästa beviset lemna delar eller organer af vigtig och i allmänhet likformig beskaffenhet, hvilka variera så att de till en viss grad antaga samma dels eller organs karakter hos en närbeslägtad art. Jag har samlat en lång lista af sådana fakta, men här, såsom förut, är det mig tyvärr omöjligt att anföra dem; jag kan blott upprepa, att sådana fall med säkerhet inträffa, och de synas mig vara mycket anmärkningsvärda.

Jag vill dock anföra ett egendomligt och inveckladt fall, som visserligen icke har afseende på någon vigtig karakter, men som förekommer hos flera arter af samma slägte, dels under domesticering, dels i naturtillståndet. Det är nästan säkert ett fall af återgång. Åsnan har stundom mycket tydliga tvärband å benen liksom zebran; man har försäkrat, att de framträda starkast hos fölet, och på grund af egna undersökningar tror jag detta vara sant. Linien på skuldran är stundom dubbel och mycket föränderlig i längd och kontur. En hvit åsna, icke en albino, har blifvit beskrifven utan både rygglinie och skulderlinie, och dessa linier äro stundom mycket mörka eller saknas alldeles hos mörkfärgade åsnor. Man påstår sig äfven hafva sett Pallas’ Kulan med en dubbel linie å skuldrorna; mr Blyth har sett en Hemionus med en tydlig skulderlinie, ehuru den egentligen icke har någon, och Poole har meddelat mig att denna arts fålar i allmänhet äro strimmiga på benen och svagt strimmiga på skuldrorna. Quaggan, som på kroppen är lika strimmig som zebran, saknar tvärband på benen, men Gray har aftecknat ett specimen med mycket tydliga zebraliknande band på hasorna.

Hvad hästen beträffar har jag i England samlat fall af ryggliniens förekomst hos hästar af de mest olika raser och alla färger: tvärband å benen äro icke sällsynta hos bruna, gråbruna och en gång har jag påträffat dem hos kastanjebruna; en svag linie å skuldran synes stundom hos bruna hästar och ett spår deraf har jag funnit hos en rödbrun. Min son har meddelat mig en noggrann undersökning och afteckning af en brun belgisk draghäst med en dubbel linie å skuldran och med tvärband å benen; jag har sjelf sett en brun devonshirepony, och en liten brun walliserpony har blifvit noga beskrifven, båda med tre parallela linier å skuldran.

Hästrasen Kattywar i nordvestra delen af Ostindien är i allmänhet strimmig, så att en häst utan strimmor icke anses vara af ren ras, såsom Poole har meddelat mig, hvilken på regeringens uppdrag undersökte rasen. Ryggen är alltid försedd med en linie, benen i allmänhet med tvärband och linien å skuldran, som stundom är dubbel eller till och med tredubbel, är mycket vanlig; dessutom äro ansigtets sidor stundom strimmiga. Linierna äro oftast starkast hos fålen, och försvinna stundom helt och hållet hos gamla hästar. Poole har sett både gråa och rödbruna kattywarhästar försedda med strimmor, då de voro nyss födda. Enligt ett meddelande af W. W. Edwards har jag skäl att tro, att hos de engelska kapplöpningshästarna linien å ryggen är mycket vanligare hos fålen än hos det fullväxta djuret. Jag har sjelf nyligen förskaffat mig en fåle genom parning af ett mörkbrunt sto (dotter af en turkomanisk hingst och ett flamlänskt sto) och en rödbrun engelsk rashäst; då fålen var en vecka gammal, hade den på bakre delen af ryggen och i pannan flera mycket smala, mörka zebraliknande band och benen voro svagt strimmiga; alla linierna försvunno snart helt och hållet. Utan att ingå i vidare detaljer vill jag tillägga, att jag har samlat fall af ben- och skulderlinier hos hästar af mycket olika raser i skilda trakter från England till östra China och från Norge i norr till malayiska arkipelagen i söder. I alla delar af verlden påträffas dessa linier oftast hos bruna och gråbruna; under benämningen bruna (”duns”) innefattas en hel rad färgnyanser från svartbrun ända till nära gräddfärgad.

Jag vet att Hamilton Smith, som skrifvit öfver detta ämne, antager, att de olika hästraserna härstamma från flera stamarter, af hvilka en, den bruna, var strimmig; och att de ofvan beskrifna företeelserna bero på tidigare kroasering med den bruna stammen. Men denna åsigt kan utan fara förkastas, ty det är högst osannolikt att den tunga belgiska draghästen, walliserponyn, den spensliga kattywarrasen m. fl. hvilka alla bebo de mest skilda delar af verlden, skulle hafva blifvit kroaserade med en antagen stamart.

Låt oss taga i betraktande verkningarna af kroasering emellan de olika arterna af hästslägtet. Rollin försäkrar, att den vanliga mulåsnan af häst och åsna har en särdeles stor benägenhet att få linier på sina ben; enligt Gosse hafva i Förenta Staterna nio mulåsnor af tio strimmiga ben. En gång såg jag en mulåsna, hvars ben voro till den grad strimmiga, att den kunde tagas för en hybrid zebra, och W. C. Martin har i sitt utmärkta arbete om hästen lemnat en figur öfver en liknande mulåsna. I fyra färglagda teckningar af bastarder af åsna och zebra, som jag sett, voro tvärbanden å benen starkare utvecklade än å den öfriga delen af kroppen, och hos en af dem fans en dubbel skulderlinie. Hos lord Mortons beryktade bastard af ett kastanjebrunt sto och en quaggahanne, äfvensom hos de rena ättlingar, som stoet sedan gaf med en svart arab, voro benen mycket starkare tvärbandade än hos sjelfva quaggan. Ett annat anmärkningsvärdt fall är en bastard af åsna och hemionus, som Gray afbildat, och han har uppgifvit att han känner äfven ett annat fall deraf; ehuru åsnan blott tillfälligtvis och hemionus aldrig har strimmor å benen och den senare icke en gång någon linie å skuldran, hade denna bastard alla fyra benen försedda med tvärband och tre korta linier å skuldran, liksom den bruna devonshire- och walliserponyn, samt några zebraliknande strimmor på ansigtets sidor. I detta sista fall var jag så öfvertygad om, att icke ens en färgstrimma beror på hvad man kallar slump, att jag blott af tillvaron af ansigtslinierna hos denna hybrid föranleddes att fråga Colonel Poole, om sådana strimmor någonsin förekomma hos den starkt strimmiga kattywarrasen, och svaret blef såsom vi hafva sett bekräftande.

Hvad hafva vi nu att säga om alla dessa fakta? Vi se flera väl skilda arter af hästslägtet blifva genom enkel variation strimmiga på benen liksom en zebra eller på skuldran liksom en åsna. Hos hästen se vi denna benägenhet stark, så ofta en brun färgnyans framträder, — en nyans, som närmar sig den vanliga färgen hos de andra arterna af slägtet. Strimmornas uppträdande är icke åtföljdt af någon formförändring eller någon annan ny karakter. Vi se denna benägenhet för strimmighet starkast utvecklad hos bastarder af de mest skilda arter. Låt oss jemföra härmed hvad vi hafva sett hos dufraserna: de härstamma från en stamart (jemte två eller tre underarter eller geografiska raser) af blå färg med vissa band och andra tecken, och om en ras i följd af enkel variation antager en blå färg, uppträda oföränderligen dessa linier och andra kännetecken utan någon annan förändring i form eller karakter. Om de äldsta och renaste raserna af olika färg kroaseras, se vi en stark benägenhet hos bastarderna att antaga den blå färgen med de svarta linierna och de öfriga kännetecknen. Jag har påstått, att det sannolikaste antagandet att förklara sådana gamla karakterers återuppträdande är det, att hos ungarna af hvarje generation finnes en tendens att antaga de längesedan förlorade karaktererna och att denna tendens af okända orsaker stundom bryter fram. Och vi hafva just sett, att hos flera arter af hästslägtet strimmorna äro antingen starkare eller uppträda oftare hos ungarna än hos de gamla. Om vi kalla för arter alla dessa dufraser, af hvilka några hafva bibehållit sig rena under århundraden, så är förhållandet fullkomligt lika med hvad vi hafva sett hos arterna af hästslägtet. Jag för min del ser med full tillförsigt tusen sinom tusen generationer tillbaka, och jag ser i ett djur, strimmigt liksom zebran, ehuru måhända af mycket olika skapnad för öfrigt, den gemensamma stamfadern för våra tama hästar (vare sig dessa härstamma från en eller flera vilda arter), för åsnan, hemionus, quaggan och zebran.

Den som tror, att hvarje art af hästslägtet blifvit skapad särskildt, skall förmodligen påstå att hvarje art har blifvit skapad med benägenhet att variera både i naturtillståndet och i kulturtillståndet på detta sätt, så att den ofta blir strimmig liksom andra arter af slägtet; och att hvar och en äfven blifvit skapad med en stark benägenhet att efter kroasering med arter som bebo de mest skilda delar af verlden frambringa bastarder som i sina strimmor likna icke sina föräldrar utan andra arter af samma slägte. Att antaga en sådan åsigt vore, tyckes mig, att förkasta den verkliga för en icke verklig eller åtminstone okänd orsak. Den gör Guds verk till blott gäck och bedrägeri; lika väl skulle jag kunna tro med de gamla, okunniga kosmogonisterna, att fossila snäckor aldrig lefvat utan blifvit skapade i sten för att härma de på hafsstranden lefvande snäckdjuren.



\section{Sammanfattning.}

Vi sväfva i djup okunnighet om lagarna för föränderligheten. Icke i ett fall af hundra kunna vi påstå oss känna skälet, hvarföre den eller den delen har undergått förändring. Men hvarhelst vi hafva medel att anställa jemförelse, synas samma lagar hafva medverkat till frambringande af de ringare olikheterna emellan varieteterna af samma art och de större olikheterna emellan arter af samma slägte. Förändrade lefnadsförhållanden inleda i allmänhet blott en ostadig föränderlighet, men stundom hafva de mera direkta och bestämda verkningar; och dessa kunna under tidernas lopp blifva starkare utpräglade, ehuru vi icke hafva tydliga bevis i detta hänseende. Vanan synes i många fall varit mäktig i sina verkningar genom att frambringa egendomligheter i konstitution, och likaså organernas användning och overksamhet, den förra genom att stärka dem, den senare genom att försvaga och förminska dem. Homologa delar visa benägenhet att variera på samma sätt. Modifikationer i hårda och yttre delar inverka stundom på mjukare och inre delar. Om en del är starkt utvecklad, har den möjligen benägenhet att taga näring från de närliggande delarna och hvarje del skyddas, om den kan behållas utan skada för individen. Förändringar i skapnad i en tidig period kunna inverka på delar som senare utveckla sig, och många fall af vexelverkan i variation äro otvifvelaktiga, ehuru vi icke kunna begripa beskaffenheten deraf. Multipla delar variera i antal och skapnad, hvilket måhända beror derpå, att sådana delar icke hafva blifvit specialiserade för någon viss förrättning, så att deras modifikationer icke blifvit använda vid det naturliga urvalet. Af samma orsak följer, att de lägre organismerna äro mera föränderliga än de som stå högre i skalan och som hafva hela sin organisation mera specialiserad. Rudimentära organer, som äro obrukbara, stå icke under det naturliga urvalets inflytande och äro derföre variabla. Artkarakterer — det är sådana karakterer, som hafva börjat visa afvikelser då de olika arterna af samma slägte afskilde sig från den gemensamma stamfadern — äro mera föränderliga än slägtkarakterer, eller de som under en lång tid gått i arf och icke under samma period hafva varierat. I dessa anmärkningar hafva vi talat om särskilda organer eller delar som ännu äro föränderliga, emedan de nyligen hafva varierat och sålunda hafva blifvit olika, men vi hafva sett i andra kapitlet, att samma princip kan tillämpas på hela individen, ty på ett område, der många arter af ett slägte förekomma — det är, der fordom mycken variation och differentiering försiggått, eller der fabrikationen af nya arter har varit liflig — på ett sådant område och ibland dessa arter finna vi nu öfverhufvudtaget de flesta varieteter. Sekundära sexualkarakterer äro i hög grad föränderliga och sådana karakterer visa stora olikheter hos arterna af samma grupp. Föränderlighet i samma delar af organisationen har i allmänhet användts till att gifva könen af samma art sexuela karakterer och artskilnader åt arterna af samma slägte. En del eller ett organ, som är utveckladt till en utomordentlig storlek eller på ett ovanligt sätt i jemförelse med samma del eller organ hos närslägtade arter måste hafva undergått en utomordentlig grad af modifikation sedan slägtet bildades; och på detta sätt kunna vi förstå hvarföre ett sådant organ fortfarande varierar i högre grad än andra delar; ty variation är en länge forsatt och långsam process, och det naturliga urvalet har i sådana fall ännu ej haft tillräcklig tid att öfvervinna benägenheten för variation och återgång till ett mindre modifieradt stadium. Men då en art med något utomordentligt utveckladt organ har blifvit stamfader till många modifierade afkomlingar — hvilket enligt vår åsigt måste vara en mycket långsam process, som fordrar en lång tidrymd, — i sådant fall har det naturliga urvalet lyckats att gifva organet en bestämd karakter, huru ovanligt det än må hafva blifvit utveckladt. Arter, som ärfva nästan samma konstitution från en gemensam stamfader och äro underkastade samma inflytelser, sträfva naturligen att lemna analoga variationer, eller också kunna samma arter tillfälligtvis antaga någon af deras gamla stamfäders karakterer. Ehuru nya och vigtiga modifikationer icke kunna uppkomma från återgång och analog variation, så bidraga dock sådana modifikationer till naturens sköna och harmoniska mångfald.

Hvad orsaken än må vara till hvarje ringa olikhet emellan föräldrar och afkomma — och en orsak måste i hvarje fall finnas — så har dock det naturliga urvalets ständiga samlande af fördelaktiga afvikelser gifvit upphof till alla dessa modifikationer i skapnad, hvilka äro af största vigt för hvarje arts tillvaro.


