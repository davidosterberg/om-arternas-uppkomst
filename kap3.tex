%TREDJE KAPITLET.


\chapter{Kampen för tillvaron.}

Dess förhållande till det naturliga urvalet. — Uttryckets användning i vidsträckt mening. — Förökningens geometriska progression. — Naturaliserade växters och djurs hastiga förökning. — Hinder för förökningen. — Allmän konkurrens. — Klimatets verkningar. — Skydd genom individernas antal. — Inveckladt förhållande emellan alla djur och växter i naturen. — Kampen för tillvaron häftigast emellan individer och varieteter af samma art, ofta äfven emellan arter af samma slägte. — En organisms förhållande till en annan organism är det vigtigaste af alla förhållanden.


Innan vi företaga oss behandlingen af detta ämne, måste jag förutskicka några anmärkningar för att visa i hvad förhållande kampen för tillvaron står till det naturliga urvalet. Vi hafva funnit i föregående kapitel, att organismerna i naturtillståndet ega en viss individuel föränderlighet, och jag tror ej att detta någonsin blifvit förnekadt. För oss är det af ingen vigt, om en mängd tvifvelaktiga former kallas för arter, underarter eller varieteter, eller hvad rang de två- eller trehundra tvifvelaktiga formerna i Storbritanien äro berättigade att antaga, blott existensen af utpräglade varieteter erkännes. Men blotta tillvaron af en individuel föränderlighet och några välutpräglade varieteter, om de också äro en nödvändig grundval för våra betraktelser, hjelpa oss föga att inse, huru arterna i naturen uppstått. Huru har denna märkvärdiga öfverensstämmelse uppstått emellan olika delar af organisationen, mellan dem och de yttre lefnadsförhållandena och emellan olika organiska varelser? Denna öfverensstämmelse se vi utomordentligt tydligt hos hackspetten och misteln och föga mindre tydligt hos de lägsta parasiterna, som hänga sig fast vid ett däggdjurs hår eller en fågels fjädrar, vi se den i bygnaden af skalbaggen, som dyker i vattnet, på de hårbeklädda frön, som af den minsta vindfläkt föras bort, med ett ord vi se denna öfverensstämmelse öfverallt och i hvarje del af den organiska verlden.

Vidare kan man fråga, kvaraf kommer det, att varieteterna, som jag kallat begynnande arter, slutligen öfvergå i väl begränsade arter, hvilka i de flesta fall uppenbarligen äro mera skilda från hvarandra än varieteterna af samma art? Huru uppstå dessa grupper af arter, hvilka betecknas med benämningen slägten och sinsemellan visa större olikheter än arterna af dessa slägten? Alla dessa verkningar bero, såsom vi i nästa kapitel fullständigare skola se, på kampen för tillvaron. Hvarje afvikelse, huru liten den än är och på hvad sätt den än har uppkommit, om den blott i någon mån är af nytta för individen af en art, måste under individens oändligt invecklade relationer till andra varelser och till den yttre naturen sträfva att skydda denna individ och att öfvergå på afkomlingarna. Afkomlingarna skola äfven hafva mera utsigt att öfverlefva alla andra individer af samma art, hvilka oupphörligen födas men af hvilka blott ett ringa antal kan blifva vid lif. Denna princip, hvarigenom hvarje liten men fördelaktig afvikelse bibehålles, har jag kallat det naturliga urvalet för att beteckna dess förhållande till menniskans urval. Vi hafva sett, att menniskan genom urval, genom hopande af små men nyttiga afvikelser, som naturen erbjuder henne, kan vinna stora resultat och är i stånd att göra de organiska varelserna passande för sina behof. Men såsom vi framdeles få se, är det naturliga urvalet oupphörligt verksamt och så ofantligt öfverlägset menniskans svaga bemödanden, som naturens verk i allmänhet stå öfver konstens.

Vi skola nu litet mera i detalj studera kampen för tillvaron; i mitt senare arbete skall den såsom den förtjenar i större omfång behandlas. Den äldre De Candolle och Lyell hafva filosofiskt och fullständigt bevisat, att alla organiska varelser äro underkastade en svår konkurrens. Hvad växterna beträffar har ingen med mera snille och skicklighet behandlat detta ämne än W. Herbert, dekanus i Manchester, tydligen på grund af sin vidsträckta erfarenhet i hortikultur. Ingenting är lättare än att i ord bekänna sanningen af den allmänna kampen för tillvaron, men ingenting är svårare än att ständigt hafva den i tanken. Med mindre vi fast inpräglat den i vårt minne, skola vi blott oklart fatta eller alldeles missförstå naturens hushållning, med dess fördelningssätt, sällsynthet eller öfverflöd, utslocknande och förändring. Vi se naturen stråla i sin fulla glans, vi se ofta öfverflöd på näringsmedel, men vi se icke eller förgäta, att fågeln, som sorglöst låter sin sång ljuda omkring oss, lefver af insekter eller frön och således ständigt mördar ett lif, eller vi glömma, huru många af dessa sångare eller deras ägg eller deras bon oupphörligt förstöras af roffåglar och andra fiender; vi tänka ej på, att om näringsämnen nu finnas i öfverflöd, detta ej på hvarje tid och i alla år är förhållandet.



\section[Uttrycket för kampen]{Uttrycket kampen för tillvaron i vidsträcktare betydelse.}

Detta uttryck begagnar jag i en vidsträckt och metaforisk betydelse, innefattande de lefvande varelsernas beroende af hvarandra och (hvad som är vigtigare) skyddandet icke blott af individens lif utan äfven individernas fortplantning. Med rätta kan man säga, att två rofdjur af hundfamiljen kämpa med hvarandra för sin näring och sitt lif under tider af brist, men man kan äfven säga att växten vid öknens gräns kämpar med torkan för sin tillvaro, ehuru det vore riktigare att säga, att dess tillvaro är beroende af fuktigheten. En växt, som årligen frambringar tusende frön, af hvilka i medeltal blott ett kommer till utveckling, kan med skäl sägas kämpa för sin tillvaro med andra växter af samma eller andra arter, som redan bekläda jorden. Mistelns tillvaro är beroende af äppleträdet och några andra trädarter, dock kan man blott i mycket öfverförd bemärkelse säga, att hon kämpar med dessa träd, ty om för många parasiter växa på samma träd, skall det förtorka och dö; men växa de på samma gren, kan man med mera sanning säga, att de kämpa med hvarandra. Då mistelns frön utspridas af vissa fåglar, så är dess tillvaro beroende af dessa fåglars existens och man kan säga, att hon kämpar med andra fruktbärande växter för att locka fåglarna att äta af hennes frukter och sålunda hellre utströ hennes frön än andra växters. I dessa olika betydelser, som öfvergå i hvarandra, använder jag för beqvämlighets skull uttrycket kampen för tillvaron.



\section{Förökningens geometriska progression.}

En kamp för tillvaron följer omedelbarligen af organismernas sträfvan att föröka sig i hög grad. Hvarje varelse, som under sin naturliga lifstid framalstrar flera ägg eller frön, måste under någon period af sitt lif eller på en viss årstid eller något år vara utsatt för förstörande orsaker, annars skulle antalet individer raskt tillväxa i geometrisk progression i så hög grad, att intet område skulle vara i stånd att underhålla dem alla. Om derföre flera individer framfödas än som kunna blifva vid lif, så måste en kamp för tillvaron uppstå antingen emellan individerna af samma art eller emellan flera arter eller emellan dem och de yttre lefnadsvilkoren. Det är Malthus’ lära, med mångdubblad kraft öfverflyttad på hela växt- och djurriket, ty i detta fall kan ej blifva fråga om någon artificiel förökning af lifsmedel eller någon försigtig afhållsamhet från äktenskap. Om derföre några arter nu tilltaga mer eller mindre raskt i antal, så kunna dock icke alla samtidigt föröka sig efter samma måttstock, ty verlden skulle ej kunna rymma dem.

Det är en regel utan undantag, att hvarje organisk varelse till den grad förökar sig, att jorden snart skulle vara befolkad af afkomlingarna af ett enda par, om icke en förstöring inträdde. Till och med menniskan, som blott långsamt förökar sig, skulle på tjugufem år fördubbla sitt antal och om förökningen fortgick i samma proportion, skulle jorden bokstafligen inom några tusen år ej hafva plats för hennes afkomlingar. Linné har redan beräknat, att om en ettårig växt sätter blott två frön (och det finnes knappt någon växt, som är så litet produktiv), och om telningarna af dessa frön under nästa år åter satte två frön och så vidare, skulle på tjugu år antalet af växtens efterkommande hafva stigit till en million. Man betraktar elefanten såsom den bland alla kända djur som långsammast förökar sig, och jag har sökt beräkna det sannolika minimum af dess förökning; dervid har jag antagit, att den börjar fortplanta sig först vid sitt trettionde år och fortsätter dermed till sitt nittionde och att den under denna tid bringar till verlden tre par ungar. Vid sådant förhållande skulle redan efter femhundra år finnas femton millioner elefanter, alla härstammande från ett och samma par.

Men vi hafva säkrare bevis för denna sak än blott teoretiska betraktelser, nämligen de talrikt anförda fallen af vissa djurarters förvånande hastiga förökning i naturtillståndet, då omständigheterna hafva varit gynnsamma under två eller tre års tid. Ännu mera slående äro de bevis, som lemnas oss af våra i vissa trakter förvildade husdjur; om uppgifterna om nötkreaturens och hästarnas förökning i Sydamerika och Australien ej voro till visshet bekräftade, skulle de hafva varit alldeles otroliga, och dock föröka sig dessa djurarter långsamt. Växterna visa samma förhållande; man kan uppräkna fall af införda växter, som på hela öar på mindre än tio år blifvit allmänna. Några växter som nu äro utbredda i sådant antal öfver de visträckta slätterna i Laplataområdet, att de der utesluta nästan alla andra växter, äro införda från Europa och likaså gifves det växter i Orienten, såsom jag hört af dr Falconer, hvilka nu äro utbredda från Cap Comorin till Himalaya och dock blifvit införda dit efter Amerikas upptäckande. I dylika fall, hvarpå otaliga exempel skulle kunna anföras, skall ingen antaga att sådana växters och djurs fruktsamhet plötsligen och tidtals har tilltagit i så hög grad. Den naturligaste förklaringen är, att de yttre förhållandena varit mycket gynsamma, att förstöringen af både unga och gamla varit ringa och att nästan alla afkomlingar varit i stånd att fortplanta sig. I sådant fall är redan den geometriska progressionen med dess förvånande resultat tillräcklig att förklara på ett enkelt sätt den utomordentligt hastiga förökningen och införda naturalsters vidsträckta utbredning i deras nya hemland.

I naturtillståndet frambringa nästan alla växter årligen frön och bland djuren finnas blott få, som icke årligen para sig. Vi kunna derföre med tillförsigt påstå, att alla växter och djur föröka sig i geometrisk progression, att de äro i stånd att med hast befolka hvarje trakt, i hvilken de kunna existera, och att denna sträfvan att föröka sig i geometrisk progression måste på någon tid af deras lif lida afbräck genom förstörande inflytelser. Vår noggranna kännedom om de större husdjuren skulle visserligen kunna missleda vår åsigt i detta hänseende, då vi icke se dem träffas af någon större förstörelse; men vi glömma, att tusende årligen slagtas till vår näring och att i naturtillståndet ett lika stort antal väl kan antagas duka under.

Den enda skilnaden emellan de organiska varelser, som årligen frambringa tusende ägg eller frön, och dem, som blott afgifva några få, är den, att de senare behöfva några år mer för att under gynsamma förhållanden befolka ett område, vore detta aldrig så stort. Condoren lägger två ägg och strutsen tjugu och dock torde i samma trakt condoren lätt bli den allmännare af de två. Stormfågeln (Procellaria glacialis) lägger blott ett ägg och dock tror man, att han är den talrikaste fågeln i verlden. Den ena flugan lägger hundra ägg, den andra, till exempel Hippobosca, blott ett, men på detta allena beror icke antalet individer, som på samma område kunna uppehålla sig. Ett stort antal ägg är af vigt för en art, för hvilken tillgången på lifsmedel ofta vexlar, ty det sätter den i stånd att på kort tid rikligt föröka sig. Men den väsentligaste fördelen af ett stort antal ägg eller frön ligger deri, att derigenom utjemnas följderna af en större ödeläggelse, som inträffar under någon lifsperiod, och denna period är i allmänhet tidig. Kan ett djur på något sätt skydda sina ägg eller ungar, så framfödes ett mindre antal och medeltalet af individer blir detsamma — men om många ägg och ungar duka under, så måste ett större antal framfödas, om icke arten skall slockna ut. Ett träd som lefver i tusen år behöfver för att bibehålla antalet individer lika på tusen år sätta blott ett frö, förutsatt att detta enda ej förstöres och kommer på ett ställe som är lämpligt för dess utveckling. Medeltalet individer af en växt- eller djurart beror således blott indirekt på antalet af deras frön eller ägg.

Vid betraktandet af naturen är det nödvändigt att alltid behålla dessa omständigheter för ögonen och aldrig glömma, att vi om hvarje organisk varelse omkring oss kunna säga, att han sträfvar att föröka sitt antal till ytterlighet, men att hvar och en under någon period af sitt lif är invecklad i en strid med fiendtliga förhållanden och att en ödeläggelse oundvikligen träffar de unga eller gamla individerna i hvarje generation eller i återkommande perioder. Så snart ett hinder öfvervinnes eller ödeläggelsen minskas, så ökas nästan ögonblickligen individernas antal.



\section{Hinder för förökningen.}

De hinder som motverka hvarje arts naturliga sträfvan att föröka sitt individantal äro föga kända. Betraktar man de frodigaste arterna, så skall man finna, att ju större deras individantal är, desto mer tilltaga deras bemödanden att vidare föröka sig. Vi känna icke ens i ett enstaka fall dessa hinder. Detta skall dock icke öfverraska någon som betänker huru litet vi i detta hänseende veta om menniskoslägtet, som dock är vida mera kändt än någon annan djurart. Detta föremål har redan af flera skriftställare blifvit ganska noggrant behandladt och i mitt senare arbete skall jag med större utförlighet omnämna flera af dessa hinder och särskildt närmare belysa de vilda djuren i Sydamerika. Här vill jag blott anföra några få anmärkningar för att återkalla i läsarens minne några hufvudpunkter. Ägg och unga djur synas lida mest, dock är denna regel icke utan undantag. En ofantlig mängd af växternas frön gå visserligen förlorade, men efter flera af mig anstälda iakttagelser tror jag, att sådden lider mest af att gro i en mark, som redan är tätt beväxt med andra växter; de förstöras äfven i stor mängd af hvarjehanda fiender. Så observerade jag på ett litet jordstycke af tre fots längd och två fots bredd, huru alla frön af våra inhemska örter sköto upp och af 357 blefvo icke mindre än 295 förstörda hufvudsakligen af snäckor och insekter. Om en gräsvall, som ofta blir slagen (och förhållandet blir detsamma om den afbetas af kreatur), får obehindradt växa, så skola de kraftigare växterna småningom döda de mindre kraftiga äfven om de äro fullt utbildade, och af tjugu arter, som växte tillsammans på ett blott tre eller fyra fots område, tillintetgjordes i ett sådant fall nio under det de öfriga växte så mycket yppigare.

Tillgången på näringsmedel för en art bestämmer naturligtvis den yttersta gränsen för artens tillväxt, men i många fall är det icke vinnandet af tillräcklig föda som bestämmer en djurarts medelantal, utan detta beror ofta på, att individerna af en art sjelfva tjena till föda åt en annan. Det synes derföre icke vara tvifvel underkastadt, att mängden af rapphöns och hjerpar, harar och dylikt på stora gods hufvudsakligen beror på utrotandet af de små rofdjuren. Om i England under de följande tjugu åren intet vildbråd sköts, men rofdjuren ej heller utrotades, så skulle efter all sannolikhet tillgången på vildt minskas, ehuru vildbråden nu fällas i hundratusental. Å andra sidan gifves det äfven många fall, såsom till exempel elefanten och noshörningen, der någon sådan förstöring från rofdjurens sida ej kommer i fråga, ty till och med den indiska tigern vågar sällan anfalla en elefantunge, som skyddas af sin moder.

Klimatet har vidare en väsentlig andel i bestämmandet af medelantalet individer af en art, och jag tror, att en periodiskt inträffande ytterlig köld eller torka är ett af de mest verksamma hinder för förökningen. Jag beräknade, att vintern 1854—55 på mitt eget gods ödelade fyra femtedelar af alla fåglar (beräknadt hufvudsakligen efter det ringa antalet bon under påföljande vår), och detta är en fruktansvärd ödeläggelse, om vi betänka att för menniskorna är redan en dödlighet af 10 procent vid epidemier mycket stor. Klimatets verkan synes vid första påseende vara alldeles oberoende af kampen för tillvaron, men då klimatet hufvudsakligen förminskar tillgången på näring, föranleder det den häftigaste kamp emellan de individer, som söka samma föda vare sig de tillhöra samma art eller icke. Äfven om klimatet verkar omedelbart, såsom till exempel en ytterlig köld, så skola de svagaste lida mest eller de som vid vinterns annalkande hafva förskaffat sig minst föda. Om vi vandra från söder till norr eller från en fuktig trakt till en torr, skola vi städse se några arter blifva allt mer och mer sällsynta och slutligen alldeles försvinna, och då klimatombytet i detta fall är uppenbart, så skola vi frestas att tillskrifva hela resultatet dess omedelbara inverkan. Och dock är detta en falsk åsigt; vi glömma att hvarje art, till och med der den är allmännast, under någon tid af sitt lif är utsatt för fiender eller konkurrenter om födoämnen eller för brist, och om dessa fiender eller konkurrenter blott det minsta gynnas genom någon vexling i klimatet, så tilltaga de i mängd och då hvarje trakt redan är fullt bebodd, så måste den andra arten vika. Om vi under vår marsch mot söder se en art i aftagande, så kunna vi vara säkra, att orsaken dertill ligger lika mycket deruti, att andra arter gynnas, som deruti att denna skadas; förhållandet är detsamma, om vi gå norrut, ehuru i ringare grad, emedan antalet arter och följaktligen antalet medtäflare aftager mot norr. Deraf kommer det, att, om vi gå mot norr eller bestiga ett berg, vi oftare påträffa förkrympta former, som härröra från klimatets omedelbart skadliga inflytande, än om vi gå mot söder eller stiga utför ett berg. Då vi slutligen nå de arktiska regionerna eller de snöbetäckta bergstopparna eller fullkomliga ödemarker, så utkämpas striden för tillvaron hufvudsakligen med elementerna.

Att klimatets verkningar företrädesvis äro indirekta genom andra arters gynnande, inses tydligt af den fabelaktiga mängd växter i våra trädgårdar, hvilka visserligen äro fullkomligt i stånd att uthärda vårt klimat, men aldrig kunnat naturaliseras, af det skäl att de hvarken kunna täfla med våra inhemska växter eller motstå ödeläggelsen genom våra inhemska djur.

Om en art genom mycket gynsamma omständigheter på ett litet område har förökat sig till ett öfvermåttan stort antal, uppträda ofta epidemier — så är åtminstone vanligen fallet med våra husdjur — och här hafva vi ett hinder, som är oberoende af kampen för tillvaron. Dock synes åtminstone en del af dessa så kallade epidemier häröra af parasitiska maskar, hvilka af någon orsak i hög grad gynnas, möjligen genom lättheten att utbreda sig på de tätt samlade djuren, och så få vi äfven här i viss mån en kamp emellan parasitdjuren och de djur på hvilka de lefva.

Å andra sidan är i många fall ett stort antal individer af en art i förhållande till fiendens antal nödvändigt för artens bestånd. Man kan derföre utan fara så säd, roffrön m. m. i mängd på våra fält, emedan deras frön då finnas i stort öfverskott i förhållande till de fåglar, som lefva deraf; dock kunna dessa fåglar, om de hafva öfverflöd på näringsmedel under en årstid, ej föröka sig i proportion till tillgången, ty på vintern kan ej hela antalet finna sin utkomst. Hvar och en som försökt det vet deremot, huru mödosamt det är att i trädgård uppdraga frön af hvete eller andra sådana växter, och i sådana fall har jag förlorat hvarje sädeskorn. Denna åsigt om nödvändigheten af ett stort antal individer af en art förklarar i min tanke några egendomliga fall i naturen, såsom att mycket sällsynta växter stundom uppträda i utomordentligt stor mängd på de fläckar der de förekomma och att många växter, som lefva liksom i samhällen, äfven just på yttersta gränsen af sitt område bilda sådana samhällen, det vill säga äro rika på individer. I sådana fall kan man tro, att en växtart blott kan ega bestånd der lefnadsvilkoren äro så gynsamma, att många individer kunna lefva tillsammans och på detta sätt skydda arten för ytterlig ödeläggelse. Härvid vill jag tillägga, att i några af dessa fall äfven den goda verkan af kroasering och de dåliga följderna af befruktning emellan beslägtade individer böra tagas i betraktande, men jag vill icke här fördjupa mig i detta invecklade ämne.



\section[Växter och djurs ömsesidiga förhållande]{Alla djurs och växters ömsesidiga förhållande i kampen för tillvaron.}

Man anför många exempel, hvilka visa, huru oförmodade de ömsesidiga förhållandena äro emellan de organiska varelserna, som på samma trakt hafva att kämpa med hvarandra. Jag vill anföra blott ett sådant exempel, som ehuru enkelt har interesserat mig. I Staffordshire på en slägtings egendom, der jag hade godt tillfälle till undersökning, fans en stor, ytterst ofruktbar hed, som aldrig blifvit rörd af menniskohand; men några hundra acres af densamma af fullkomligt samma beskaffenhet hade tjugufem år förut blifvit inhägnade och planterade med skotska furor. Förändringen i den ursprungliga vegetationen på det inhägnade stället var ytterst märkvärdig, större än man vanligen iakttager, om man öfvergår från en mark till en annan af helt och hållet olika beskaffenhet. Icke blott talförhållandena emellan hedväxterna voro alldeles förändrade, utan på planteringen växte äfven tolf sådana arter (starr och gräs oberäknade), som icke voro att finna på heden. Verkan på insekterna måste hafva varit ännu större, ty på planteringen voro sex arter insektätande fåglar allmänna, hvilka icke syntes på heden, som deremot besöktes af två eller tre andra sådana arter. Här se vi huru mäktigt införandet af blott en enda trädart varit, då ingenting annat blifvit gjordt, med undantag af inhägnandet, så att boskapen var utestängd derifrån. Men att äfven inhägnandet är ett vigtigt moment, har jag tydligt sett i närheten af Farnham i Surrey. Der finnas vidsträckta hedar med några grupper gamla tallar på de aflägsna kullarna; på de sista tio åren hade ansenliga sträckor blifvit inhägnade och inom inhägnaderna sköto upp en mängd sjelfsådda unga tallar, så tätt hoppackade, att alla ej kunde blifva vid lif. Då jag fått veta, att dessa unga tallar ej blifvit sådda eller planterade, blef jag så förvånad öfver deras antal, att jag strax vände mig åt flera håll för att undersöka hundra acres af den ej inhägnade heden och jag kunde finna bokstafligen icke en enda tall med undantag af de gamla planterade grupperna. Men då jag såg mig närmare omkring bland växterna på den fria heden, fann jag en mängd telningar och små träd, hvilka hade oupphörligt blifvit afbetade af boskapen. På en fläck af en qvadratalns storlek på flera hundra alnars afstånd från de gamla trädgrupperna räknade jag trettiotvå afbetade små träd, af hvilka ett visade tjugusex årsringar och således under åratal hade försökt att höja sig öfver hedens örter, men förgäfves. Det var då icke underligt, att landet så snart det inhägnades, blef så tätt beväxt med kraftiga unga tallar. Och heden var dock så ytterligt ofruktbar och vidsträckt, att ingen skulle hafva trott, att boskapen med denna påföljd skulle här hafva sökt bete.

Här se vi tallens förekomst absolut beroende af boskapen; i andra verldstrakter är boskapens tillvaro åter beroende af vissa insekter. Paraguay bildar måhända det märkvärdigaste exempel härpå, ty här äro aldrig nötkreatur, hästar eller hundar förvildade, ehuru de söder och norr derom svärma omkring i vildt tillstånd. Azara och Rengger hafva visat att orsaken dertill ligger i tillvaron af en liten i Paraguay allmän fluga, som lägger sina ägg i nafveln på de nyfödda ungarna af dessa djurarter. Förökningen af dessa så talrikt förekommande flugor måste i allmänhet motarbetas af någon orsak, sannolikt af andra parasitiska insekter. Om derföre vissa insektätande fåglar i Paraguay aftogo i antal, så skulle de parasitiska insekterna sannolikt förökas, och häraf skulle antalet af dessa flugor förminskas, nötkreaturen och hästarna skulle förvildas och detta skulle åter föranleda en betydlig förändring i vegetationen, något som jag verkligen på några ställen i Sydamerika har iakttagit. Detta måste nu i hög grad inverka på insekterna och härigenom såsom vi sett i Staffordshire på de insektätande fåglarna och så vidare i allt vidsträcktare kretsar. Vi hafva börjat dessa serier med insektätande fåglar och vi sluta med dem. I naturen äro förhållandena dock ej alltid så enkla som här. Strid på strid med vexlande utgång måste oafbrutet återkomma; men under tidernas lopp hålla de olika krafterna mot hvarandra jemnvigt så noga, att naturen under långa perioder behåller ett oförändradt utseende, ehuru ofta den obetydligaste småsak är tillräcklig att gifva en organisk varelse segern öfver en annan. Det oaktadt är vår okunnighet så stor, att vi förvånas, då vi höra talas om en organisk varelses undergång, och då vi icke se orsakerna, så åberopa vi syndafloder till verldens förstöring eller uppfinna lagar för de lefvande formernas varaktighet.

Jag är särdeles benägen att genom ytterligare exempel visa huru sådana växter och djur, som på naturens skala stå vidt skilda från hvarandra, äro förenade genom en väfnad af invecklade relationer. Jag skall framdeles få tillfälle att visa, att den utländska Lobelia fulgens i denna del af England aldrig besökes af insekter och följaktligen på grund af sin egendomliga blombygnad aldrig kan sätta frukt. Nästan alla våra orchidéer måste ovilkorligen besökas af insekter för att med deras tillhjelp befruktas. Genom försök har jag öfvertygat mig, att humlor äro oumbärliga till befruktningen af styfmorsblomman (Viola tricolor, Pensé), ty andra bin slå sig aldrig ned på dessa blommor. Likaså har jag funnit, att biens besök äro nödvändiga för befruktningen af flera af våra klöfverarter. Tjugu blomhufvud af hvitklöfver (Trifolium repens) lemnade mig till exempel 2,290 frön, under det tjugo andra växter af samma art, som voro oåtkomliga för bien, icke lemnade ett enda. Hundra hufvud af rödklöfver (Trifolium pratense) lemnade likaledes 2,700 frön och samma antal skyddade för humlor lemnade intet! Humlor allena besöka denna sista klöfverart, ty andra bin kunna ej komma åt dess honingssaft. Man har äfven förmodat, att motten bidrager till klöfverns befruktning, men för min del tviflar jag åtminstone på, att de kunna verka något på den röda klöfvern, emedan de icke äro tunga nog för att trycka ned blomkronans sidoblad. Man har derföre skäl att antaga, att om hela humleslägtet blef sällsynt eller utrotadt i England, måste äfven styfmorsblomman och rödklöfvern blifva mycket sällsynta eller helt och hållet försvinna. Antalet af humlor i ett distrikt är åter i hög grad beroende af antalet åkerråttor, som förstöra deras bon och honingskakor, och öfverste Newman, som länge har observerat humlans lefnadssätt, tror att mer än två tredjedelar af dem förstöras i hela England. Men såsom hvar och en vet beror råttornas antal på antalet kattor och Newman säger, att han i närheten af byar och små städer har funnit största antalet humlor, hvilket han tillskrifver mängden af kattor, som utrota råttorna. Det är derföre väl troligt, att närvaron i stor mängd af ett kattartadt djur på ett område indirekt genom råttor och bin kan hafva inflytande på mängden af vissa växter derstädes.

Vid hvarje art äro sannolikt olika momenter verksamma såsom hinder dels på olika perioder af lifstiden, dels på olika årstider; några kunna verka kraftigare än andra, men alla tillsammans betinga medeltalet individer eller artens existens. I många fall låter det visa sig, att helt olika orsaker på olika trakter hafva inflytande på samma arts förekomst. Om vi betrakta buskar och örter, som bekläda en strand, så äro vi benägna att på tillfällighetens räkning skrifva så väl deras proportionela antal som deras arter. Men huru falsk är icke denna åsigt. Hvar och en har hört, att om en skog nedhugges i Amerika, uppträder en fullkomligt olika vegetation, och man har dock iakttagit, att träden, som växa på de gamla indianska ruinerna i de södra staterna af Nordamerika, hvilka en gång måste hafva blifvit beröfvade sin vegetation, nu visa samma brokiga mångfald och samma artförhållanden som de omgifvande orörda skogarna. Hvilken kamp måste icke här hafva egt rum emellan de olika trädarterna, som alla afkasta årligen sina frön i tusental, hvilket krig emellan insekter, snokar och andra djur med fåglar och rofdjur, hvilka alla sträfva att föröka sig, hvilka alla lefva af hvarandra eller af träden och deras frön och telningar eller af andra växter, som i början beklädde marken och härigenom hindrade trädens utveckling. Kastar man en hand full fjädrar upp i luften, så måste alla falla ned efter bestämda lagar, men huru enkelt är icke detta fallproblem att lösa i jemförelse med verkan och återverkan af de tallösa växter och djur, som under loppet af århundraden bestämt art- och talförhållandena af de träd som nu växa på de gamla indianska ruinerna.

Ett förhållande emellan tvänne organiska varelser, sådant som parasiters beroende af den art på hvilken de lefva, eger i allmänhet rum emellan sådana arter, som stå långt ifrån hvarandra på naturens skala; detta är ofta fallet med sådana, om hvilka man riktigt kan säga, att de kämpa med hvarandra för sin tillvaro, såsom de växtätande däggdjuren och gräshopporna. Men den striden blir nästan utan undantag den häftigaste, som föres emellan individer af samma art som bebo samma område, söka samma födoämnen och äro utsatta för samma faror. Emellan varieteter af samma art blir kampen i allmänhet lika så häftig och stundom se vi striden afgjord på kort tid. Om vi till exempel så olika hvetevarieteter ibland hvarandra och åter utså de blandade fröna, så skola några varieteter, som lämpa sig efter klimatet och jordmånen eller som af naturen äro de fruktsammaste, besegra de öfriga, lemna mera frön och således redan efter få år helt och hållet undantränga de öfriga. För att uppdraga ett blandadt förråd af så närbeslägtade varieteter, som de olika färgade Lathyrus odoratus måste man hvarje år uppsamla dem särskildt och derefter blanda dem på nytt i tillbörligt förhållande, om icke de svagare år från år skola aftaga och slutligen dö ut. Så är förhållandet äfven med fårraserna; man har påstått, att vissa bergvarieteter af får bringa andra att dö ut, så att de icke kunna hållas tillsammans. Samma resultat har man äfven fått, då man hållit flera varieteter blodiglar tillsammans. Man kan till och med betvifla, att varieteterna af någon kulturväxt eller något husdjur ega så noga samma styrka, vana och konstitution, att de ursprungliga talförhållandena i en blandning af dem skulle kunna bibehålla sig ens under ett halft dussin generationer, om man lät dem kämpa med hvarandra såsom de organiska varelserna i naturtillståndet och icke årligen sorterade fröna eller ungarna.



\section[Individer av samma art]{Kampen för tillvaron häftigast emellan individer och varieteter af samma art.}

Då arterna af ett slägte vanligen ehuru ej alltid hafva mycken likhet med hvarandra i vana och konstitution och alltid i skapnad, så blir kampen emellan arter af ett slägte, om de komma i täflan med hvarandra, vanligen hårdare än emellan arter af olika genera. Vi se detta i en svalarts utbredning öfver Förenta Staterna, hvarest den utträngde en annan art. Utbredningen af misteltrasten (Turdus viscivorus) i några delar af Skottland har föranledt talltrastens (Turdus musicus) minskning. Huru ofta höra vi icke, att en råttart i de mest olika klimat har intagit en annans plats. I Ryssland har den lilla kakerlackan (Blatta orientalis) öfverallt drifvit framför sig sina större anförvandter. Våra bin, som nu blifvit införda i Australien, äro på väg att utrota de små inhemska bien, som ej hafva gadd. En art åkersenap har utträngt en annan, och så vidare i andra fall. Vi kunna oklart inse, hvarföre striden är häftigast emellan de mest beslägtade formerna, som nära på intaga samma plats i naturens hushållning, men sannolikt skola vi i intet enda fall vara i stånd att angifva huru det tillgått, att i den stora kampen för tillvaron den ena har vunnit segern öfver den andra.
Af föregående framställning kan man draga en slutsats af största vigt, att hvarje organisk varelses bildning står i det innerligaste, ehuru ofta fördolda sammanhang med skapnaden af alla andra organiska varelser, med hvilka den kommer i strid om näring eller bostad, för hvilka den måste fly och af hvilka den lefver. Detta är klart af bygnaden af tigerns tänder och klor, af formen på benen och fötterna hos de parasiter, som hänga fast på tigerns hårbeklädda kropp. På lejontandens vackert befjädrade frön och på vattenbaggens afplattade, hårbeklädda ben synes genast deras relation inskränkt till luften och vattnet. Men fördelen af fjäderförsedda frön står utan tvifvel i nära förhållande till landets beskaffenhet, som är så tätt beväxt, att fröna först måste drifvas vida omkring i luften för att kunna falla ned på en ännu fri mark. Vattenbaggen är genom skapnaden af sina ben lämpad till dykning, hvarigenom han sättes i stånd att täfla med andra vatteninsekter, att jaga sitt eget byte och att fly undan för andra djur, som anfalla honom.

Det förråd af näringsämnen, som är nedlagdt i många växters frön synes vid första påseende ej stå i något förhållande till andra växter. Men de unga telningarnas lifliga tillväxt, som uppspira ur sådana frön (såsom ärter och bönor) om de sås ut midt ibland högt gräs, gifver mig anledning att tro, att detta förråd hufvudsakligen är bestämdt att påskynda den unga telningens tillväxt under det den har att kämpa med andra, kraftiga, omgifvande växter.

Hvarföre fördubblar och fyrdubblar icke en växt sitt antal i sitt utbredningsområde? Vi veta, att den rätt väl kan uthärda litet mer eller mindre hetta och köld, torka och fuktighet, ty i alla händelser utbreder den sig i något varmare eller kallare, torrare eller fuktigare områden. I detta fall inse vi väl, att om vi ville i tanken gifva växten förmåga att vidare utbreda sig, så måste vi gifva den någon fördel öfver de med henne täflande växterna, eller de djur, som lefva af henne. Vid gränsen af hennes geografiska utbredning skulle en efter klimatet lämpad förändring i konstitution uppenbarligen vara af väsentlig fördel för våra växter, men vi hafva skäl att tro, att få växter och djur utbreda sig så långt, att de förstöras genom klimatets stränghet allena. Blott då vi komma till lifvets yttersta gränser i allmänhet, i de arktiska regionerna eller vid den torra öknens gräns, blott der upphör denna täflan. Landet må vara hur kallt och torrt som helst, alltid skola några arter eller några individer af samma art täfla med hvarandra om de varmaste eller fuktigaste fläckarna.
Om en växt- eller djurart i någon ny trakt försättes ibland nya medtäflare, så måste äfven de yttre lefnadsförhållandena väsentligen förändras, om också klimatet förblifver detsamma som i den gamla hemorten. Ville vi höja denna arts medeltal i den nya hemorten, så måste vi låta den undergå förändringar olika med dem, som i dess egentliga hemland hafva samma verkan; ty vi måste gifva den fördelar öfver medtäflare eller fiender af ett helt annat slag.

Det är lätt nog att i tanken gifva en viss växt- eller djurform någon fördel öfver en annan, men sannolikt skulle vi i praxis icke i något enda fall veta hvad som borde göras för att vinna detta mål. Det skulle öfvertyga oss om vår obekantskap med alla organiska varelsers ömsesidiga förhållande, en öfvertygelse, som är likaså nödvändig, som den tyckes svår att vinna. Allt hvad vi kunna göra, är att städse hålla i minnet, att hvarje organisk varelse sträfvar att föröka sitt antal i en geometrisk progression, att hvar och en under någon period af sitt lif, på någon viss årstid, i hvarje generation eller med oregelbundna mellantider måste kämpa för sin tillvaro och är utsatt för tillintetgörelse. Om vi reflektera öfver denna kamp för tillvaron, så kunna vi trösta oss med den tanken, att naturens krig icke är oafbrutet, att all fruktan är främmande, att döden i allmänhet är snabb, och att det är det kraftigare, det friskare, det lyckligare slägtet, som lefver qvar och förökar sig.


