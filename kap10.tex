%TIONDE KAPITLET.


\chapter[Succesivt uppträdande]{De organiska varelsernas uppträdande i successiva
geologiska perioder.}

{\it
Nya arters långsamma och successiva uppträdande. — Olika förändringsgrad. — En gång försvunna arter uppträda icke mer. — Artgrupper följa samma allmänna regel i sitt uppträdande och försvinnande som enstaka arter. — Förintelse. — Samtidiga förändringar i lifsformer öfver hela jorden. — Utdöda arters slägtskap med hvarandra och med nu lefvande arter. — De gamla formernas utvecklingsgrad. — Samma typers successiva uppträdande inom samma område. — Sammanfattning af detta och föregående kapitel.
}\\[0.5cm]

Vi skola nu se till, huruvida de förhållanden och lagar som gälla för de organiska varelsernas geologiska succession bättre öfverensstämma med den vanliga åsigten om arternas oföränderlighet eller med åsigten om deras långsamma och gradvisa modifikation genom härstamning och naturligt urval.

Nya arter hafva uppträdt mycket långsamt både till lands och vatten. Lyell har visat att det näppeligen är möjligt att bestrida klarheten i de bevis härför som de tertiära lagren innehålla, och hvarje år sträfvar att fylla luckorna emellan lagren och göra procentförhållandet emellan förgångna och nya former mera graderadt. I några af de nyaste bäddarna, ehuru äfven de otvifvelaktigt af hög ålder, beräknade efter år, äro blott en eller två arter utdöda och blott en eller två nya, hvilka hafva uppträdt för första gången antingen der på platsen eller så vidt vi känna på jordytan. De sekundära formationerna äro mera brutna; men såsom Bronn har anmärkt, har hvarken uppträdandet eller försvinnandet af dess många arter inbäddade i samma formation inträffat samtidigt.

Arter af olika slägten och klasser hafva icke förändrats i samma förhållande eller i samma grad. I de äldre tertiära bäddarna kunna ännu några få nu lefvande snäckor finnas midt ibland en mängd utdöda former. Falconer har gifvit ett slående exempel på ett likartadt faktum, nämligen en ännu lefvande krokodilart, som finnes blandad med många utdöda däggdjur och reptilier i de subhimalayiska lagren. Den siluriska Lingula skiljer sig obetydligt från de nu lefvande arterna af detta slägte, hvaremot de flesta af de andra siluriska molluskerna och alla krustaceer hafva i hög grad förändrats. Landets alster synas undergå förändringar i en hastigare progression än hafvets, hvarpå ett slående exempel nyligen iakttagits i Schweitz. Det finnes skäl för det antagandet, att organismer som stå högt i skalan förändras hastigare än de lägre stående, ehuru undantag gifvas från denna regel. Graden af förändring af organismer är enligt Pictet icke densamma i hvarje successiv så kallad formation. Om vi likväl jemföra några af de mest sammanhängande formationer, skola vi finna att alla arterna hafva undergått någon förändring. Om en art en gång har försvunnit från jordytan, hafva vi intet skäl att tro att samma form åter kommer att uppträda. Det kraftigaste skenbara undantag från denna regel är Barrande’s så kallade ”kolonier”, som under en period intränga midt i en äldre formation, och sålunda tillåta den förr existerande faunan att återuppträda; men Lyells förklaring, att de äro fall af tillfällig flyttning från en skild geografisk provins, synes mig fullt tillfredsställande.

Alla dessa förhållanden öfverensstämma väl med vår teori, som icke innefattar någon oföränderlig lag om utveckling, som har till följd att alla invånarne på en yta förändras hastigt eller samtidigt eller i lika grad. Modifikationsprocessen måste vara långsam och bör i allmänhet angripa blott ett fåtal arter på samma gång, ty hvarje arts föränderlighet är fullkomligt oberoende af alla andra arters. Om dessa variationer eller individuela afvikelser samlas genom naturligt urval i större eller mindre grad, på detta sätt förorsakande en större eller mindre grad af permanent modifikation, det beror på många invecklade förhållanden, — på variationernas nytta, på den fria kroaseringen, på traktens långsamt föränderliga fysiska förhållanden, på nya kolonisters inflyttningar och på beskaffenheten af de andra inbyggarna, med hvilka de varierande arterna komma i täflan. Det är derföre ingalunda öfverraskande, att en art bibehåller samma identiska form mycket längre än en annan, eller att den, om den förändras, modifieras i mindre grad. Vi finna likartade förhållanden emellan invånarna i skilda distrikt; landsnäckorna och skalbaggarna på Madeira skilja sig till exempel betydligt från deras närmaste slägtingar på Europas fastland, under det hafssnäckorna och fåglarna hafva blifvit oförändrade. Vi kunna måhända förstå den skenbart hastigare proportionen af förändring hos landdjur och högre organiserade alster i jemförelse med hafsdjur och lägre organiserade varelser, om vi fästa afseende vid de mera invecklade förhållandena emellan de högre organiserade varelserna och deras organiska och oorganiska lefnadsvilkor, som vi utvecklat i ett föregående kapitel. Då många af invånarna på en yta hafva blifvit modifierade och förädlade, kunna vi enligt grundsatsen om kampen för tillvaron och de vigtiga förhållandena emellan organismerna i denna kamp inse, att en form som icke blifver i någon mån modifierad och förädlad går sin undergång till mötes. Vi inse deraf hvarföre alla arter i en trakt till slut blifva modifierade (om vi taga tillräckligt lång tidrymd), ty eljest skulle de snart dö ut.

Hos medlemmar af samma klass bör graden af förändring under långa och lika tidrymder öfverhufvudtaget vara nästan densamma, men då samlandet af formationer, rika på fossilier och af lång varaktighet, beror derpå att stora massor af sediment blifvit aflagrade på sjunkande ytor, hafva våra formationer nästan nödvändigt bildats på vidsträckta tidrymder med oregelbundna afbrott, och följaktligen är graden af förändring som de i på hvarandra följande lager inbäddade fossilierna visar icke lika. Hvarje formation betecknar enligt denna åsigt icke en ny och fullständig skapelseakt, utan blott en tillfällig scen tagen nästan på slump i ett långsamt försiggående drama.

Vi kunna tydligen inse, hvarföre en art som en gång gått förlorad icke vidare bör uppträda, äfven om samma lifsvilkor, organiska eller oorganiska skulle återkomma. Ty ehuru ättlingarna af en art kunna vara lämpade (och otvifvelaktigt har detta händt i otaliga fall) att intaga en annan arts plats i naturens hushållning, och på detta sätt ersätta den, så böra dock icke de två formerna, den gamla och den nya, vara identiskt lika, ty båda skola helt säkert ärfva afvikande karakterer från sina skilda stamfäder, och organismer som redan äro skilda variera på olika vis. Det vore till exempel möjligt, om alla våra påfågeldufvor dogo ut, att amatören kunde bilda en ny ras som svårligen kunde skiljas från den nuvarande stammen; men om stamfadern klippdufvan likaledes vore utdöd, och i naturtillståndet hafva vi all anledning att tro att stamformer i allmänhet utträngas och ersättas af sina modifierade ättlingar, är det otroligt att en påfågeldufva, identisk med den nu existerande rasen, skulle bildas från en annan dufart, eller ens från en annan väl utbildad ras af tama dufvan, ty de successiva variationerna skulle helt säkert vara till en viss grad skiljaktiga och den nybildade varieteten skulle sannolikt ärfva från sin stamfader några karakteristiska afvikelser.

Artgrupper, det är slägten och familjer, följa samma allmänna reglor i sitt uppträdande och försvinnande som enstaka arter, förändra sig mer eller mindre hastigt och i högre eller mindre grad. Om en grupp en gång dött ut, uppträder den aldrig mer, det är, dess tillvaro är kontinuerlig, så länge den räcker. Jag vet väl, att det finnes några skenbara undantag från denna regel, men undantagen äro märkvärdigt få, så få att E. Forbes, Pictet och Woodward (ehuru alla motståndare till mina åsigter) medgifva dess sanning, och regeln är i full öfverensstämmelse med min teori. Ty alla arter af samma grupp, huru länge den än har egt bestånd, äro modifierade ättlingar af hvarandra och af någon gemensam stamfader. Hos slägtet Lingula måste till exempel de arter som successivt uppträdt i alla perioder hafva varit förenade genom en oafbruten serie af generationer från det understa siluriska lagret till närvarande tid.

Vi hafva sett i föregående kapitlet att många arter af en grupp stundom falskeligen synas hafva uppträdt plötsligt i massa, och jag har försökt att gifva en förklaring öfver detta förhållande, som om det vore sant skulle vara särdeles olycksbringande för min teori. Men sådana fall äro helt säkert undantagsfall, och den allmänna regeln är ett gradvist tilltagande i antal, till dess gruppen nått sitt maximum, och derefter förr eller senare ett gradvist aftagande. Om antalet arter som innefattas i ett slägte eller antalet slägten i en familj föreställes genom en vertikal linie af vexlande tjocklek, uppstigande genom de successiva geologiska formationerna i hvilka arterna äro funna, synes linien begynna i dess nedre ända icke i en skarp spets utan afbruten; den tilltager sedan småningom i tjocklek uppåt, ofta under en viss utsträckning oförändrad, och afsmalnar slutligen i de öfre lagren, utmärkande arternas aftagande och slutliga försvinnande. Denna gradvisa tillökning i en grupps artantal är i full öfverensstämmelse med teorien, ty arterna af samma slägte och slägtena af samma familj kunna förökas blott långsamt och gradvis; modifikationsprocessen och bildandet af ett antal beslägtade former är nödvändigt en långsam och småningom skeende process, i det en art först gifver upphof till två eller tre varieteter, dessa blifva långsamt förvandlade till arter, som i sin ordning genom lika långsamma steg lemna andra varieteter och arter och så vidare likasom grenar från en stor stam, till dess gruppen hunnit sin storlek.

\section{Arters undergång.}

Vi hafva hittills blott tillfälligtvis talat om arters och artgruppers försvinnande. Enligt teorien om naturligt urval äro gamla formers utslocknande och produktionen af nya och förädlade former nära förbundna med hvarandra. De gamla åsigterna att alla jordens invånare blifvit bortsopade genom revolutioner under successiva perioder hafva blifvit uppgifna äfven af sådana geologer, såsom Elie de Beaumont, Murchison, Barrande, m. fl. hvilkas allmänna åsigter naturligen borde föra dem till sådana slutsatser. Tvärtom hafva vi från studiet af de tertiära formationerna allt skäl att tro, att arter och grupper af arter gradvis försvinna, den ena efter den andra, först från en fläck, sedan från en annan och slutligen från jorden. I några få fall kan likväl denna utrotningsprocess hafva gått mycket hastigt, såsom vid genombrytandet af ett näs och den deraf följande invasionen af en mängd nya invånare i en närgränsande sjö, eller genom sänkningen af en ö. Både enstaka arter och hela artgrupper lefva under mycket olika långa perioder; några grupper hafva såsom vi hafva sett räckt ifrån lifvets första kända gryning ända till nuvarande tid, några hafva försvunnit redan före slutet af den palæozoiska perioden. Inga oföränderliga lagar synas reglera den tidslängd under hvilken en enstaka art eller en artgrupp kan ega bestånd. Vi hafva skäl att tro, att en artgrupps undergång i allmänhet är en långsammare process än dess bildning; om dess uppträdande och försvinnande såsom förut betecknas med en vertikal linie af olika tjocklek, befinnes linien afsmalna mera långsamt emot sin öfre ända, som betecknar utrotningsprocessen, än i sin nedre del, som utmärker gruppens uppkomst och den tidiga förökningen i arternas antal. I några fall har likväl hela gruppers utslocknande varit märkvärdigt hastigt, såsom ammoniternas utslocknande emot slutet af den sekundära perioden.

Utslocknandet af arterna har varit insvept i den mest oförtjenta hemlighet. Några författare hafva till och med antagit, att likasom individen har en bestämd lifslängd, så hafva äfven arterna en bestämd varaktighet. Ingen kan mer än jag hafva förundrat sig öfver arternas undergång. Då jag i La Plata fann en hästtand inbäddad ibland qvarlefvor af Mastodon, Megatherium, Toxodon och andra utdöda vidunder, hvilka alla existerade samtidigt med ännu lefvande snäckor i en mycket aflägsen geologisk period, var jag full af förvåning, ty då jag såg att hästen efter dess införande af spaniorerna i Sydamerika har spridt sig i vildt tillstånd öfver hela trakten och har förökat sitt antal i ojemförlig proportion, frågade jag mig, hvad som kunde hafva så nyligen tillintetgjort den fordna hästen under lifsvilkor som syntes så gynsamma. Men min förvåning var grundlös. Professor Owen har visat, att tanden tillhörde en utdöd art, ehuru den var så lik den nu lefvande hästens. Om denna häst ännu lefvat men i någon mån sällsynt, skulle ingen naturhistoriker känt sig det minsta öfverraskad af dess sällsynthet, ty sällsynthet är en egenskap, som tillhör ett stort antal arter af alla klasser i alla länder. Om vi fråga oss hvarföre den ena eller den andra arten är sällsynt, svara vi att det är något ofördelaktigt i dess lifsvilkor, men hvad detta något är, kunna vi knappt någonsin uppgifva. Med det antagandet att den fossila hästen ännu existerade såsom en sällsynt art, kunna vi hafva antagit för säkert af analogier från alla andra däggdjur, äfven elefanten, som så långsamt förökar sig, och från den tama hästens naturalisation i Sydamerika, att den inom få år skulle hafva befolkat hela kontinenten. Men vi skulle icke kunna säga, hvilka ogynsamma förhållanden förhindrade dess förökning, antingen ett eller flera, eller på hvilken period af hästens lif eller i hvad grad de verkade. Om förhållandena fortfarande men långsamt blifvit allt mindre och mindre gynsamma, skulle vi helt säkert icke hafva märkt detta förhållande, och dock skulle den fossila hästen blifvit allt mer och mer sällsynt och slutligen dött ut, under det dess plats intagits af någon lyckligare medtäflare.

Det är alltid särdeles svårt att ihågkomma, att hvarje varelses förökning beständigt motverkas af omärkliga fiendtliga inflytelser, och att dessa samma omärkliga inflytelser äro fullt tillräckliga att göra en art sällsynt och slutligen utrota den. Detta ämne är så litet kändt, att jag upprepade gånger hört personer uttala sin förvåning öfver att sådana vidunder som Mastodon och den äldre Dinosaurus hafva dött ut, liksom om blott en kroppslig styrka gåfve segern i kampen för tillvaron. Blotta storleken deremot kan i många fall, såsom Owen anmärkt, förorsaka en hastigare utrotning, då den kräfver en större tillgång på föda. Förrän menniskan bebodde Indien och Afrika, måste någon orsak hafva motverkat den lefvande elefantens förökning. En särdeles tillförlitlig auktoritet, dr Falconer, tror, att insekterna, genom att oupphörligt oroa och försvaga elefanten i Indien, förhindra dess förökning, och till samma åsigt kom Bruce angående den afrikanska elefanten i Abyssinien. Säkert är, att insekter och blodsugande flädermöss inverka på de större naturaliserade däggdjurens tillvaro i flera delar af Sydamerika.

Vi se i många fall i de nyare tertiära formationerna, att sällsynthet föregår utslocknande; och så har äfven förloppet varit för de djur som antingen lokalt eller helt och hållet blifvit utrotade genom menniskans åtgörande. Jag upprepar hvad jag sade år 1845: att medgifva, att arter i allmänhet blifva sällsyntare innan de utrotas, att icke förvånas öfver en arts sällsynthet och dock finna det besynnerligt om arten upphör att existera, det är fullkomligt detsamma som att medgifva att sjukdom hos individen är en förelöpare till döden, att icke öfverraskas af en sjukdom och dock, om den sjuka dör, förvånas och misstänka att han dog någon vådadöd.

Teorien om det naturliga urvalet är grundad på den åsigten, att hvarje ny varietet, och slutligen hvarje ny art bildas och bibehålles derigenom att den har någon fördel öfver dem med hvilka den kommer i täflan, och den följande utrotningen af de mindre väl utrustade formerna följer deraf nästan oundvikligt. Det är samma förhållande med våra kulturalster; då en ny och något förädlad varietet uppkommit, uttränger den först de mindre förädlade varieteterna i det närmaste grannskapet; då den nått en högre grad af förädling flyttas den vida omkring likt vår korthornsboskap och intager andra rasers plats i andra trakter. Uppträdandet af nya former och försvinnandet af de gamla, vare sig naturliga eller med konst bildade, äro oskiljaktigt förenade med hvarandra. I frodiga grupper har antalet af nya specifika former som uppkommit på en gifven tid sannolikt under några perioder varit större än antalet gamla specifika former som dött ut; men vi veta, att arterna icke hafva förökats oändligt åtminstone under de senare geologiska epokerna, så att vi kunna tro, om vi betrakta de senare tiderna, att bildandet af nya former har haft till följd försvinnandet af ungefär lika många gamla former.

Täflan bör i allmänhet såsom vi förut hafva utvecklat och belyst med exempel vara strängast emellan de former som mest likna hvarandra i alla hänseenden. De förädlade och modifierade ättlingarna af en art böra derföre i allmänhet utrota stamarten, och om många nya former hafva utvecklats från samma art, äro de närmaste slägtingarna till denna art, det vill säga andra arter af samma slägte, de närmaste offren. Följaktligen bör såsom jag tror ett antal nya arter som härstamma från en art, det är ett nytt slägte, uttränga ett gammalt slägte af samma familj. Men det måste ofta hafva händt, att en ny art tillhörande en viss grupp har gripit in på det område som tillhörde en art af en annan grupp och på detta sätt orsakat dess undergång. Om många beslägtade former utvecklas från den lyckliga inkräktaren, måste många andra vika tillbaka från sina platser, och i allmänhet de beslägtade formerna, som lida af någon gemensamt ärfd underlägsenhet. Men vare sig det är arter af samma eller af skilda klasser som hafva vikit tillbaka för andra modifierade och förädlade arter, några få af de svagare kunna ofta skyddas under en lång tid deraf att de äro lämpade för något särskildt lefnadssätt eller deraf att de bebo något aflägset eller isoleradt område, der de undgått en svårare täflan. I de australiska hafven lefva till exempel ännu några arter af Trigonia, ett stort snäckslägte i de sekundära formationerna, och några få medlemmar af den stora och nästan utdöda fiskgruppen ganoiderna bebo ännu våra insjöar. Den totala utrotningen af en grupp är derföre i allmänhet såsom vi hafva sett en långsammare process än dess bildning.

Med afseende på det plötsliga försvinnandet af hela familjer eller ordningar såsom trilobiterna vid slutet af den palæozoiska perioden och ammoniterna vid slutet af den sekundära perioden måste vi komma ihåg hvad som redan blifvit sagdt om de sannolikt ofantliga tidrymder som förflutit emellan våra formationer; och på dessa mellantider kan ett långsamt försvinnande försiggått. Dessutom, om vid en hastig inflyttning eller ovanligt rask utveckling många arter af en ny grupp hafva tagit i besittning ett område, böra många af de äldre arterna dö ut i motsvarande grad, och de sålunda utgående formerna äro vanligen beslägtade, ty de ega gemensamt samma underlägsenhet.

Det sätt hvarpå enstaka arter och hela artgrupper dö ut öfverensstämmer väl med teorien om det naturliga urvalet. Vi böra icke förundra oss öfver denna utrotning; om vi måste undra, så må det vara öfver vår egen förmätenhet att för ett ögonblick föreställa oss, att vi begripa de många invecklade förhållanden hvarpå hvarje arts existens beror. Om vi ett ögonblick glömma, att hvarje art sträfvar att föröka sig oändligt, och att vissa hinder alltid äro verksamma, ehuru sällan för oss märkbara, blir hela naturens ekonomi ytterligt dunkel. När vi någonsin kunna säga med bestämdhet, hvarföre den ena arten är rikare på individer än den andra, hvarföre den ena kan naturaliseras i ett gifvet land och icke en annan, då, och icke förr än då kunna vi med rätta förvåna oss öfver att vi icke kunna förklara en särskild art eller artgrupps undergång.

\section[Om lifsformernas samtidiga vexling]{Om lifsformernas samtidiga vexling öfver hela jorden.}

Svårligen är någon palæontologisk upptäckt mera öfverraskande än det förhållande att lifsformerna vexla nästan samtidigt öfver hela jorden. Sålunda kan vår europeiska kritaformation igenkännas i många skilda delar af jorden under de mest olika klimat, der ej ett spår af kritan sjelf kan finnas, nämligen i Nordamerika, i det tropiska Sydamerika, på Eldslandet, vid Goda Hoppsudden och på indiska halfön. På dessa skilda punkter förete nämligen de organiska qvarlefvorna i vissa lager en omisskännelig likhet med dem som finnas i kritan. Visserligen påträffas icke samma arter, ty i några fall är icke en enda art identiskt densamma, men de höra till samma familjer och slägten och slägtafdelningar och visa stundom likartade karakterer i de mest obetydliga punkter, såsom blott vid den ytliga skulpturen. Vidare påträffas andra former, som icke finnas i den europeiska kritan, men i formationerna deröfver och derunder, i samma ordning i dessa skilda verldsdelar. I flera successiva palæozoiska formationer i Ryssland, Vestra Europa och Nordamerika hafva flera iakttagare anmärkt samma parallelism emellan lifsformerna och enligt Lyell är detta förhållandet med flera europeiska och nordamerikanska aflagringar. Äfven om vi afse från de få fossila arter som äro gemensamma för gamla och nya verlden, är den allmänna parallelismen ännu tydlig emellan de successiva lefvande formerna i de palæozoiska och tertiära lagren och de olika formationerna kunna utan svårighet jemföras med hvarandra.

Dessa iakttagelser röra likväl jordens hafsinvånare; vi hafva icke tillräckliga data att bedöma om landets och insjöarnas produkter på skilda trakter förändras på samma parallela vis. Vi kunna betvifla det; om Megatherium, Mylodon, Macrauchenia och Toxodon hade förts från La Plata till Europa utan någon upplysning om deras geologiska läge, skulle ingen hafva anat att de lefvat samtidigt med hafssnäckor som ännu lefva; men då dessa besynnerliga vidunder lefde tillsammans med Mastodon och häst, kunna vi åtminstone antaga att de lefvat under en af de senare tertiära perioderna.

Då vi säga, att de marina formerna hafva förändrats samtidigt öfver hela jorden, får man icke antaga, att detta uttryck gäller samma tusental eller tiotusental år eller ens att det har en mycket sträng geologisk betydelse; ty om alla de nu i Europa lefvande hafsformer och alla som lefde i Europa under pleistocenperioden (en mycket aflägsen period, beräknad efter år, innefattande hela istiden) jemfördes med de nu i Sydamerika och Australien lefvande, skulle näppeligen den skickligaste naturhistoriker vara i stånd att säga, antingen de närvarande eller de pleistocena invånarna i Europa mest liknade den södra hemisferens bebyggare. Flera kompetenta iakttagare påstå till exempel, att de nu existerande produktionerna i Förenta Staterna äro mera beslägtade med dem som lefde i Europa under vissa sena tertiära perioder än med de nuvarande invånarna i Europa; och om så är förhållandet är det tydligt att fossilförande lager som nu afsättas på Nordamerikas kuster hädanefter komma att ställas i jembredd med något äldre europeiska lager. Om vi se framåt till en aflägsen framtida epok, kunna vi icke destomindre hysa föga tvifvel om att alla de mera moderna marina formationerna, de öfre pliocena, pleistocena och de nyaste lagren i Europa, Nord- och Sydamerika och Australien skola betraktas såsom samtidiga i geologisk mening, derföre att de innehålla till en viss grad beslägtade organiska qvarlefvor och icke de former som finnas blott i de äldre underliggande aflagringarna.

Det förhållandet att de organiska formerna vexla samtidigt i ofvan anförda vidsträckta mening i olika delar af jorden har i hög grad öfverraskat några utmärkta iakttagare, de Verneuil och d’Archiac. Sedan de anfört parallelismen emellan de palæozoiska organiska formerna i olika delar af Europa, tillägga de: ”Om vi rikta vår uppmärksamhet på Nordamerika och der upptäcka en serie af analoga företeelser, synes det säkert, att alla dessa arternas modifikationer, deras utslocknande, och införandet af nya icke kan bero på blotta förändringar i hafsströmmar eller andra mer eller mindre lokala och temporära orsaker, utan på allmänna lagar som regera hela djurriket”. Äfven Barrande har gjort liknande iakttagelser med samma påföljd. Det är i sjelfva verket alldeles onyttigt att betrakta hafsströmmar, klimat eller andra förhållanden såsom orsak till dessa stora förändringar i de organiska formerna öfver jordytan under de mest olika klimat. Vi måste såsom Barrande anmärkt, söka någon särskild lag. Vi skola inse detta tydligare då vi behandla de organiska varelsernas nuvarande utbredning, och finna huru ringa förhållandet är emellan olika trakters fysiska förhållanden och beskaffenheten af dess invånare.

Detta stora faktum, de organiska formernas parallela succession öfver hela jorden, är förklarligt med teorien om det naturliga urvalet. Nya arter bildas derigenom att de hafva något företräde framför andra äldre former, och de former som redan äro dominerande eller hafva något företräde öfver andra former i sitt eget land skulle vara mest i stånd att alstra det största antalet nya varieteter eller begynnande arter. Vi hafva tydliga bevis härpå deruti att de växter som äro dominerande, det är de som äro allmännast och hafva största spridningen, alstra största antalet nya varieteter. Det är äfven naturligt att de dominerande, varierande arterna med stor spridning, hvilka redan hafva till en viss utsträckning inkräktat på andra arters område, skola vara de, som hafva största utsigten att sprida sig ännu mera och i nya områden gifva upphof till andra nya varieteter och arter. Spridningsprocessen är ofta mycket långsam beroende på klimatiska och geografiska förändringar, sällsamma tillfälligheter och nya arters gradvisa acklimatisering i de olika klimat de måste passera, men under tidernas lopp lyckas i allmänhet de dominerande formerna att sprida sig och vinna slutligen öfverhand. Spridningen är sannolikt långsammare för landinvånare i skilda kontinenter än för hafsinvånarna i ett sammanhängande haf. Vi kunna derföre vänta att finna, såsom vi också göra, en mindre grad af parallelism i successionen af landets produktioner än i hafvets.

Den parallela och i en vidsträckt mening samtidiga successionen af samma organiska former öfver hela jorden synes mig således väl öfverensstämma med den grundsatsen, att nya arter bildats af dominerande arter som sprida sig vidsträckt och variera; de på detta sätt bildade nya arterna blifva sjelfva dominerande på den grund att de hafva haft någon fördel öfver deras redan förut dominerande stamfäder likasom öfver andra arter, och sprida sig i sin ordning, variera och bilda nya former. De gamla formerna som besegras och vika undan för de nya segrande formerna äro i allmänhet förenade i grupper, emedan de gemensamt ärft någon underlägsenhet; och derföre då nya och förädlade grupper sprida sig öfver jorden, försvinna gamla grupper och formernas succession sträfvar öfverallt efter motsvarighet både i deras första uppträdande och slutliga försvinnande.

Det är ett annat anmärkningsvärdt förhållande, som står i sammanhang med detta ämne. Jag har angifvit skälen för min åsigt, att de flesta af våra på fossilier rika stora formationer aflagrats under en sänkningsperiod; och att mellantider af lång varaktighet, saknande fossilier egde rum under de perioder då hafsbotten var antingen stillastående eller stadd i höjning, och likaledes om sediment icke samlades tillräckligt hastigt för att begrafva och skydda de organiska qvarlefvorna. Under dessa långa och fattiga mellantider antager jag att invånarna i hvarje område undergingo en ansenlig grad af förändring och utrotning och att flyttningen var stor från andra delar af jorden. Då vi hafva skäl att tro, att stora ytor äro underkastade samma rörelse, är det sannolikt att fullkomligt samtidiga formationer ofta blifvit samlade öfver mycket vidsträckta ytor i samma delar af verlden; men vi hafva långt ifrån rätt till den slutsatsen, att detta oföränderligen varit förhållandet, och att stora ytor oföränderligen hafva undergått samma rörelse. Om två formationer hafva afsatt sig i två områden nästan, men icke fullkomligt under samma period, skulle vi i båda finna samma allmänna succession i de organiska formerna af skäl som utvecklats i föregående paragrafer, men arterna skulle icke vara fullt motsvariga; ty i det ena området skulle tiden varit litet längre än i det andra för modifikation, utslocknande och inflyttning.

Jag förmodar, att fall af denna beskaffenhet förekomma i Europa. Prestwich har i sitt utmärkta arbete öfver de eocena aflagringarna i England och Frankrike dragit upp en allmän parallel emellan de successiva lagren i dessa två länder; men då han jemför vissa lager i England och Frankrike, så ehuru han i båda finner en märkvärdig öfverensstämmelse i antalet af de arter som höra till samma slägte, skilja sig dock dessa arter på ett mycket svårförklarligt sätt i betraktande af de båda områdenas ringa afstånd, såvida man icke får antaga, att ett näs skilde två haf som voro samtidigt bebodda af skilda faunor. Lyell har gjort liknande iakttagelser på några af de senare tertiära formationerna. Barrande visar också, att det finnes en öfverraskande allmän parallelism i de successiva siluriska aflagringarna i Böhmen och Skandinavien; icke destomindre finner han en sällsam grad af olikhet i arter. Om de olika formationerna i dessa trakter icke blifvit afsatta under fullkomligt samma perioder, — om en formation i en region motsvarar ett tomrum i en annan — och om i båda områdena arterna hafva fortfarande långsamt förändrats under de olika formationernas bildning och under den långa tiden dem emellan; i detta fall kunde de olika formationerna i de två regionerna ordnas på samma sätt i öfverensstämmelse med den allmänna successionen af organiska former, och ordningen skulle ehuru orätt synas vara noggrant parallel; icke desto mindre skulle arterna icke vara fullkomligt desamma i de skenbart motsvariga lagren i de två områdena.

\section[Slägtskap emellan utöda och lefvande]{Om slägtskapen emellan utdöda arter och emellan
dem och de nu lefvande.}

Vi skola nu taga i betraktande den ömsesidiga slägtskapen emellan utdöda former och nu lefvande arter. De tillhöra alla några få stora klasser och detta förhållande förklaras med ens enligt grundsatsen om härstamning. Ju äldre en form är, ju mera skiljer den sig i regel från lefvande former. Men alla utdöda arter kunna, såsom Buckland för längesedan yttrade, antingen insättas i ännu existerande grupper eller emellan dem. Att de utdöda organiska formerna hjelpa till att fylla upp luckorna emellan nu lefvande slägten, familjer och ordningar kan icke bestridas. Ty om vi fästa vår uppmärksamhet antingen vid de lefvande allena eller blott vid utdöda, är serien långt mindre fullkomlig än om vi förena båda till ett allmänt system. Hela sidor skulle kunna fyllas med exempel från Owen, som visa huru utdöda ryggradsdjur passa in emellan lefvande grupper. Cuvier uppstälde Ruminantia och Pachydermata såsom två de mest skilda däggdjursordningar, men Owen har upptäckt så många fossila föreningslänkar, att han måst förändra klassifikationen och ställa vissa pachydermer i samma underordning som ruminantia: han har till exempel genom fina gradationer upplöst det skenbart stora mellanrummet emellan svin och kamel. En annan utmärkt palæontolog Gaudry visar att ganska många af de af honom i Attika upptäckta fossila däggdjuren på det fullkomligaste vis förena nu lefvande slägten. Äfven det stora mellanrummet emellan fåglar och reptilier har af prof. Huxley visats delvis uppfyllas på det mest oväntade sätt genom å ena sidan strutsen och den utdöda Archæopteryx å andra sidan Compsognathus, en af Dinosaurii, hvilken grupp innefattar de mest gigantiska af alla landreptilier. Om vi nu taga i betraktande de ryggradslösa djuren, försäkrar oss Barrande, och en större auktoritet kan icke nämnas, att han med hvarje dag lärt att ehuru de palæozoiska djuren helt säkert kunna upptagas i nu lefvande grupper, dessa grupper dock icke voro vid denna period så skilda som de nu äro.

Några skriftställare hafva uttalat sig emot den åsigten, att en utdöd art eller artgrupp betraktas såsom mellanform emellan lefvande arter eller grupper. Om med dessa uttryck menas att en utdöd form är direkt i alla sina karakterer ett mellanstadium emellan två lefvande former, har protesten fullt värde. Men i ett naturligt system stå helt säkert många utdöda former emellan nu lefvande arter och några utdöda slägten emellan lefvande slägten, äfven emellan slägten af skilda familjer. Det naturligaste fallet isynnerhet för mycket skilda grupper såsom fiskar och reptilier synes vara att om vi till exempel antaga dem numera vara skilda genom ett dussintal karakterer, de gamla grupperna äro skilda genom ett mindre antal karakterer, så att två grupper ehuru äfven förr fullkomligt skilda på denna period stodo hvarandra närmare än numera.

Det är en allmän tro, att ju äldre en form är, desto mera sträfvar den att genom sina karakterer förena grupper som nu äro vidt skilda från hvarandra. Denna åsigt måste otvifvelaktigt inskränkas till de grupper som under de geologiska periodernas förlopp hafva undergått stora förändringar, och det skulle vara svårt att bevisa sanningen af detta påstående, ty hvarje nu och då lefvande djur såsom Lepidosiren befinnes hafva slägtskap med mycket skilda grupper. Om vi dock jemföra de äldre reptilierna och batrachierna, de äldre fiskarna, de äldre cephalopoderna och de eocena däggdjuren med de nyare formerna af samma klasser, måste vi medgifva att det är sanning i den satsen.

Låt oss då se till huru dessa fakta och meningar öfverensstämma med teorien om härstamning med modifikation. Då ämnet är något inveckladt måste jag bedja läsaren återgå till schemat i fjerde kapitlet. Vi kunna antaga att de numererade bokstäfverna föreställa slägten och de punkterade linierna som divergera från dem, arterna i hvarje slägte. Schemat är allt för enkelt, allt för få slägten och arter äro gifna, men det är ovigtigt för oss. De horizontala linierna kunna föreställa de geologiska formationerna och alla former under den öfversta linien betraktas såsom utdöda. De tre nu lefvande slägtena a${}^{14}$, p${}^{14}$, q${}^{14}$ bilda en liten familj; b${}^{14}$ och f${}^{14}$ en beslägtad familj eller underfamilj, och o${}^{14}$, e${}^{14}$ och m${}^{14}$ en tredje familj. Dessa tre familjer tillsammans med de många utdöda slägtena på de olika härstamningslinierna, som utgå från stamformen (A) bilda en ordning; ty alla skola hafva ärft någonting gemensamt från sin fordna gemensamma stamfader. Enligt grundsatsen om en fortsatt sträfvan till karaktersdivergens, som förut illustrerades med detta schema, måste en form i allmänhet så mycket mer skilja sig från sin gamla stamfar, ju nyare den är. Derföre kunna vi förstå den regeln, att de äldsta fossilierna skilja sig mest från de nu lefvande formerna. Vi måste likväl icke antaga att karaktersdivergensen är en nödvändighet; den beror helt enkelt derpå, att en art derigenom blir i stånd att intaga många och skilda platser i naturens ekonomi. Derföre är det möjligt såsom vi hafva sett i några siluriska former, att en art kan blifva obetydligt modifierad efter dess obetydligt ändrade lifsvilkor, och dock under en lång period bibehålla samma allmänna karakterer. Detta betecknas i figuren genom bokstafven F${}^{14}$.

Alla de många former, utdöda och nya, som härstamma från A bilda såsom förut nämdt en ordning; och denna ordning har genom de fortsatta verkningarna af tillintetgörelse och karaktersdivergens blifvit delad i flera underfamiljer, af hvilka några antagas hafva dukat under på olika tider och några lefva qvar intill denna dag.

Om vi betrakta schemat kunna vi se, att om många af de utdöda formerna, som antagas vara inbäddade i de successiva formationerna, upptäcktes på olika punkter långt ned i serien, de tre lefvande familjerna på den öfversta linien skulle blifva mindre skilda från hvarandra. Om till exempel slägtena a${}^{1}$, a${}^{5}$, a${}^{10}$, f${}^{8}$, m${}^{3}$, m${}^{6}$, m${}^{9}$ blefvo uppgräfda, skulle dessa tre familjer så nära förenas med hvarandra, att de sannolikt skulle hafva sammanförts i en stor familj, såsom det har gått med idislarna och vissa pachydermer. Den som icke ville kalla de utdöda slägtena, som på detta sätt förenade de lefvande slägtena af tre familjer, för mellanformer i karakter, han skulle hafva rätt, ty de äro icke direkt mellanformer utan blott genom en lång omväg genom många vidt skilda former. Om många utdöda former skulle upptäckas öfver en af de mellersta horizontala linierna eller geologiska formationerna — till exempel öfver N:o VI — men ingen under denna linie, skulle blott två af familjerna (de på venstra sidan, a${}^{14}$ etc. och b${}^{14}$ etc.) kunna förenas i en, och två familjer skulle stå qvar, hvilka voro icke mindre skilda från hvarandra än de voro före fossiliernas upptäckande. Vidare, om de tre familjerna som bildas af åtta slägten (a${}^{14}$—m${}^{14}$) på öfversta linien antagas skilja sig från hvarandra i ett halft dussin vigtiga karakterer, skulle de familjer som vid den med VI betecknade perioden helt säkert hafva skilt sig från hvarandra genom ett ringare antal karakterer; ty de skulle på detta tidiga stadium hafva divergerat i mindre grad från sitt gemensamma ursprung. Deraf kommer det att gamla och utdöda former äro ofta i en viss ringa grad mellanformer i karakterer emellan deras modifierade afkomlingar eller emellan sina sidoslägtingar.

I naturen blir saken mera invecklad än schemat föreställer, ty grupperna hafva blifvit talrikare, deras varaktighet har varit ytterligt olika och de hafva modifierats i olika grad. Då vi ega blott sista volymen af det geologiska arkivet, och det i mycket dåligt skick, hafva vi icke rätt att vänta oss kunna utom i några få fall fylla upp de vidsträckta luckor i det naturliga systemet och på detta sätt förena skilda familjer eller ordningar. Allt hvad vi hafva rätt att vänta är att de grupper, som hafva inom kända geologiska perioder undergått mycken modifikation, skulle i de äldre formationerna något närma sig hvarandra, så att de äldre medlemmarna skilja sig mindre från hvarandra i några af sina karakterer än de nu lefvande medlemmarna af samma grupper; och enligt våra bästa palæontologers samstämmiga bevis är detta ofta förhållandet.

Enligt teorien om härstamning med modifikation förklaras således på tillfredsställande sätt de stora fakta som röra de utdöda lifsformernas slägtskapsförhållanden till hvarandra och till lefvande former. Och de äro fullkomligt oförklarliga enligt hvarje annan åsigt.

Enligt samma teori är det tydligt att faunan i en stor period af jordens historia bör vara i sina allmänna karakterer ett mellanstadium emellan den som föregick och den som följde derefter. De arter som lefde på det sjette stora stadiet i schemat äro de modifierade ättlingarna af dem som lefde på femte stadiet och stamfäder till dem som lefde i det sjunde ännu mera modifierade; derföre måste de i karakterer stå emellan de lefvande formerna ofvan och under. Vi måste likväl medgifva några föregående formers totala undergång och i vissa trakter inflyttning af nya former från andra områden och en hög grad af modifikation under det långa tomrummet emellan de successiva formationerna. Med dessa medgifvanden är hvarje geologisk periods fauna otvifvelaktigt i karakterer ett mellanstadium emellan den föregående och den efterföljande faunan. Jag behöfver blott gifva ett exempel, nämligen det sätt hvarpå fossilierna i devoniska systemet, då detta först upptäcktes, med ens af palæontologer betraktades såsom intermediära i karakterer emellan fossilierna i de ofvanliggande stenkolsformationerna och det underliggande siluriska systemet. Men hvarje fauna är icke nödvändigt exakt ett mellanstadium, då olika långa tider hafva förflutit emellan de följande formationerna.

Att vissa slägten göra undantag från denna regel, det är icke någon verklig invändning emot sanningen af det påstående, att hvarje periods fauna såsom ett helt är nästan intermediär i karakter emellan den föregående och den efterföljande faunan. Mastadonter och elefanter anordnade af Falconer i två serier först efter deras slägtskapsförhållanden och sedan efter deras period öfverensstämma icke i dessa serier. Den i karakter yttersta arten är icke den äldsta eller yngsta; ej heller äro de som äro mellanformer i karakter intermediära i tid. Men om vi för ett ögonblick antaga i detta och andra fall, att vår kännedom om artens första uppträdande och försvinnande vore fullkomlig, hafva vi intet skäl att antaga, att former som sedermera alstrades nödvändigt räckte motsvarande lång tid: en mycket gammal form kunde tillfälligtvis räcka mycket längre än en form som sedermera uppkom, särskildt af landets alster som bebo skilda distrikt. Må vi jemföra små ting med stora: om de förnämsta lefvande och utdöda raser af tamdufvan anordnades i en serie så väl ske kan efter slägtskap skulle denna anordning icke öfverensstämma med en serie uppstäld efter tiden för deras bildning och ännu mindre med en serie efter tiden för deras undergång; ty stamfadern, klippdufvan, lefver ännu, och många varieteter emellan klippdufvan och brefdufvan hafva dött ut; och brefdufvor som visa en ytterlighet i den vigtiga karakteren, en lång näbb, hafva uppkommit förr än de kortnäbbade tumletterna, hvilka i detta hänseende stå vid motsatta ändan af serien.

I nära sammanhang med denna sats, att de organiska qvarlefvorna från en intermediär formation äro till en viss grad mellanformer i karakter, står det faktum som alla palæontologer framhålla, att fossilier från två efter hvarandra följande formationer äro vida närmare beslägtade än fossilier från två aflägsna perioder. Pictet anför ett väl kändt exempel, den allmänna likheten af de organiska qvarlefvorna från flera stadier af kritaformationen, ehuru arterna äro skilda i hvarje lager. Detta faktum allena tyckes genom sin allmännelighet hafva rubbat professor Pictet i sin tro på arternas oföränderlighet. Han som är så förtrogen med de lefvande arternas fördelning öfver jordytan vill icke försöka förklara den stora likheten emellan skilda arter i nära på hvarandra följande formationer dermed att de fysiska förhållandena på de gamla områdena förblifvit oförändrade. Vi få komma ihåg, att de lefvande formerna, åtminstone de som bebo hafvet, hafva förändrats nästan samtidigt öfver hela jordytan och följaktligen under de mest skilda klimat och förhållanden. Vi böra observera de talrika vexlingarna i klimat under den pleistocena perioden, som innefattar hela istiden, och anmärka huru litet artformerna som bebo hafvet hafva deraf rönt något inflytande.

Enligt teorien om härstamning visas lätt fulla betydelsen deraf att de fossila qvarlefvorna från tätt efter hvarandra följande formationer äro nära beslägtade ehuru ansedda såsom skilda arter. Då bildandet af hvarje formation ofta har blifvit afbruten och långa tomrum finnas emellan de successiva formationerna, böra vi icke vänta oss att finna, såsom jag försökt visa i sista kapitlet, i en eller två formationer alla de intermediära varieteterna emellan arterna som uppträdde vid början och slutet af dessa perioder, men vi böra finna efter mellantider, mycket långa om de beräknas efter år, men blott måttligt långa efter geologisk beräkning, beslägtade former eller såsom de af några hafva kallats representerande arter, och dessa finna vi helt säkert. Vi finna i korthet sagdt så tydliga bevis på långsam och knappt märkbar förändring af artformer, som vi hafva rätt att vänta.



\section[Utveckling hos gamla och nya former]{Om graden af utveckling hos de gamla formerna
jemförda med de lefvande.}

Vi hafva sett i fjerde kapitlet att graden af organernas differentiering och specialisering hos alla organiska varelser i mogen ålder är den bästa hittills funna måttstocken på graden af fulländning. Vi hafva också sett, att då delarnas och organernas specialisering är en fördel för hvarje varelse, så sträfvar det naturliga urvalet att göra hvarje varelses organisation mera specialiserad och fullkomlig och i denna mening högre, ehuru det dock kan lemna många varelser med enkla och oförädlade bildningar lämpliga för enklare lifsvilkor, och i några fall torde det naturliga urvalet äfven degradera eller förenkla organisationen och dock lemna sådana degraderade varelser i ett tillstånd som gör dem lämpliga för sitt nya lefnadssätt. På ett annat och mera allmänt sätt blifva nya arter öfverlägsna sina föregångare, ty de måste i kampen för tillvaron besegra alla de äldre formerna med hvilka de komma i täflan. Vi kunna derföre antaga, att om under ett nästan lika klimat de eocena invånarna på jorden kunde bringas i täflan med de nu lefvande, skulle de förra besegras och utrotas af de senare, liksom de sekundära af de eocena och de palæozoiska af de sekundära formerna. Genom detta bevis på seger i kampen för tillvaron, äfvensom enligt måttstocken, organernas specialisering, borde nya former enligt teorien för naturligt urval stå högre än de gamla formerna. Är detta fallet? Ett stort flertal palæontologer skulle besvara frågan med ja och jag antager att svaret måste medgifvas vara riktigt ehuru svårt att fullkomligt bevisa.

Det är icke någon kraftig invändning mot denna slutsats, att vissa brachiopoder hafva blifvit blott obetydligt modifierade från en ytterligt aflägsen geologisk period. Ej heller är det något oöfvervinneligt inkast att foraminifererna icke hafva, såsom Carpenter framhåller, fortskridit i organisation sedan den laurentiska perioden; ty några organismer måste hafva qvarstått lämpade för enkla lefnadsvilkor, och hvilka voro bättre lämpade derför än dessa lågt organiserade protozoer? Det är icke heller någon svår invändning som prof. Phillips framhåller, att insjösnäckorna hafva förblifvit nästan oförändrade från tiden för deras första uppträdande intill denna dag; ty dessa snäckor hafva haft att utstå en vida mindre sträng kamp än de mollusker som bebo öppna hafvet med dess otaliga inbyggare. Sådana invändningar som de ofvanstående skulle vara olycksbringande för hvarje åsigt som innnefattar fortskridande i organisation såsom en nödvändighet. De skulle äfven vara farliga för min teori om foraminifererna kunde bevisas hafva uppträdt först under den laurentiska perioden eller brachiopoderna först under den cambriska formationen, ty i detta fall skulle de icke hafva haft tid att uppnå den ståndpunkt de då innehade. Då formerna en gång hafva kommit till en viss grad af utveckling, finnes ingen nödvändighet i det naturliga urvalet för deras fortgående utveckling, ehuru de under hvarje successiv period måste undergå små förändringar för att kunna bibehålla sina platser under förändrade lefnadsförhållanden. Alla sådana invändningar stranda på den frågan, om vi verkligen känna huru gammal verlden är och på hvad period de olika lifsformerna uppträdde, och derom kunna meningarna vara delade.

Frågan huruvida organisationen på det hela taget har gått framåt är i många hänseenden svår att lösa. Den geologiska urkunden, på alla tider ofullständig, sträcker sig icke långt nog tillbaka som jag tror för att med obestridlig klarhet bevisa, att inom jordens kända historia organisationen gått synnerligen framåt. Äfven i närvarande tid, då vi betrakta medlemmar af samma klass, stämma naturforskarna icke öfverens hvilka former böra anses för de högsta; några anse till exempel Selachii eller hajarna för de högsta fiskarna på grund af deras slägtskap med reptilierna i några vigtiga delar, andra anse Teleostei, benfiskarna, för de högsta. Ganoiderna stå midt emellan selachier och benfiskar, de senare äro för det närvarande vida öfverlägsna i antal, men fordom existerade endast ganoider och selachier och i detta fall kan man säga att fiskarna hafva gått framåt eller tillbaka i organisation allt efter den måttstock man väljer. Att försöka jemföra i detta hänseende former af olika typer synes hopplöst, hvem kan afgöra om en bläckfisk är högre än ett bi — denna insekt som von Baer ansåg vara ”i sjelfva verket högre organiserad än en fisk, ehuru efter en annan typ?” I den invecklade kampen för tillvaron är det troligt att krustaceerna, som icke stå mycket högt i sin egen klass, kunna besegra cephalopoder, de högsta bland blötdjuren, och sådana krustaceer skulle stå mycket högt ibland de ryggradslösa djuren, om de bedömdes efter det mest afgörande af alla kriterier, kampen för tillvaron. Afsedt från dessa svårigheter att afgöra hvilka former äro de mest framskridna i organisation böra vi icke blott jemföra de högsta lemmarna af en klass på två perioder — ehuru detta otvifvelaktigt är ett och kanhända det vigtigaste element vid afvägningen — utan vi böra jemföra alla medlemmarna, höga och låga vid de två perioderna. Vid en uråldrig tidpunkt svärmade de högsta mollusker, cephalopoder och brachiopoder omkring i riklig mängd, i närvarande tid äro båda ordningarna i hög grad reducerade, under det andra ordningar, intermediära i organisation hafva rikligen tilltagit; följaktligen antaga några vetenskapsmän att blötdjuren voro högre utvecklade förr än nu; men ett starkare skäl kan framhållas på motsatta sidan, nämligen den ofantliga inskränkningen i antalet af de lägsta blötdjur, och det faktum att de nu lefvande cephalopoderna ehuru få i antal äro högre organiserade än deras fordna representanter. Vi böra också jemföra det relativa antalet af högre och lägre klasser på jorden under två perioder: om till exempel för det närvarande femtio tusen slag af vertebrerade djur finnas, och om vi kände, att i någon förfluten period blott tiotusen slag lefvat, böra vi anse denna förökning i antal inom den högsta klassen, hvilket innefattar ett undanträngande af lägre former såsom ett afgjordt framåtskridande i organisationen. Vi se således huru hopplöst svårt det är att jemföra med fullkomlig ärlighet under så ytterligt invecklade förhållanden graden af organisation hos de ofullständigt kända faunor i de successiva perioderna.

Vi skola ännu klarare inse denna svårighet om vi betrakta vissa nu lefvande faunor och floror. Med stöd af den utomordentliga hastighet, med hvilken europeiska produkter nyligen spridt sig öfver Nya Zeeland och inkräktat på områden som förut innehades af andra alster, måste vi tro, att, om alla djur och växter i Storbritannien lemnades fria på Nya Zeeland, under tidernas lopp en mängd britiska former skulle blifva fullkomligt naturaliserade der och uttränga många af infödingarna. Å andra sidan, då knappt en enda inbyggare från södra hemisferen blifvit vild i någon del af Europa, kunna vi betvifla, att om alla Nya Zeelands alster försattes i fritt tillstånd i England något ansenligt antal skulle vara i stånd att inkräkta på de områden som nu innehafvas af våra inhemska växter och djur. Från denna synpunkt stå Storbritanniens alster mycket högre i skalan än Nya Zeelands. Dock skulle icke den skickligaste naturhistoriker efter en undersökning af arterna i de två länderna hafva förutsett detta resultat.

Agassiz och några andra auktoriteter framhålla, att fordna djur till en viss grad likna embryonerna af de nyare djuren af samma klasser, och att den geologiska successionen af utdöda former är nästan parallel med de nu lefvande formernas embryologiska utveckling. Denna åsigt öfverensstämmer märkvärdigt väl med min teori. I ett följande kapitel skall jag försöka visa att den fullväxte skiljer sig från sitt embryo på grund af variationer som inträffa på en icke tidig period och gå i arf i motsvarande ålder. Denna process, som lemnar embryot nästan oförändradt, gifver den fullväxta under loppet af successiva generationer allt större och större afvikelser. Embryot kommer således att lemnas qvar såsom ett slags ritning, skyddad af naturen, af djurets fordna och mindre modifierade tillstånd. Denna åsigt kan vara sann och kan dock aldrig fullt bevisas. Om vi till exempel se att de äldsta kända däggdjur, reptilier och fiskar strängt tillhöra sina särskilda klasser, ehuru några af dessa gamla former äro i ringa grad mindre skilda från hvarandra än de typiska medlemmarna af samma grupper i närvarande tid, skulle det vara fåfängt att söka efter djur som hafva vertebraternas gemensamma embryologiska karakter, förr än fossilrika lager upptäckas långt under det lägsta siluriska lagret — en upptäckt som har föga sannolikhet för sig.



\section[Senare tertiärperioden]{Om successionen af samma typer inom samma område
under den senare tertiärperioden.}

För många år sedan visade Clift, att de fossila däggdjuren från australiska grottorna voro nära beslägtade med denna kontinents lefvande pungdjur. I Sydamerika är ett likartadt förhållande tydligt äfven för ett ovant öga i de gigantiska sköldar som funnits i flera delar af La Plata; och prof. Owen har på det mest slående sätt visat att de flesta af de fossila däggdjuren som der äro begrafda äro beslägtade med de sydamerikanska typerna. Detta slägtskapsförhållande blir ännu mera klart af den underbara samling af ben som Lund och Clausen funno i Brasiliens grottor. Dessa fakta inverkade på mig till den grad att jag år 1839 och 1845 framhöll med eftertryck denna ”lag om typernas succession”, — ”detta underbara slägtskapsförhållande i samma kontinent emellan de döda och de lefvande”. Prof. Owen har sedermera utsträckt samma lag till den gamla verldens däggdjur. Vi se samma lag i denna författares restaurerade utdöda gigantiska fåglar på Nya Zeeland. Vi se den också hos fåglarna i Brasiliens grottor. Woodward har visat att samma lag gäller för hafssnäckorna, men i följd af de flesta molluskslägtens vidsträckta fördelning är den icke väl utpräglad hos dem. Andra fall kunde tilläggas såsom förhållandet emellan de utdöda och lefvande saltvattensnäckorna i Aral och Kaspiska hafvet.

Hvad betyder väl denna märkvärdiga lag om succesionen af samma typ inom samma område? Den skulle vara väl djerf, som efter att hafva jemfört klimatet i Australien och några delar af Sydamerika under samma latitud ville försöka att förklara å ena sidan genom olikheten i fysiska förhållanden olikheten emellan invånarna på dessa båda kontinenter, och å andra sidan genom likhet i yttre förhållanden öfverensstämmelsen i typer under den senare tertiära perioden. Ej heller kan man påstå att det är en oföränderlig lag, att pungdjur skulle bildats hufvudsakligen eller endast i Australien och tandlöse (Edentata) och andra amerikanska typer blott i Sydamerika. Ty vi veta att Europa i fordna tider var befolkadt af talrika pungdjur och i ofvan antydda afhandlingar har jag visat, att i Amerika lagen för landdäggdjurens utbredning förr varit helt olika mot nu. Nordamerika var fordom i hög grad delaktig af den nuvarande karakteren hos kontinentens södra hälft och den södra hälften var förr mera än nu beslägtad med den norra halfvan. Af Falconers och Cautleys upptäckter veta vi likaledes, att norra Indien fordom företedde större likhet i däggdjur med Afrika än i närvarande tid. Analoga fakta kunde gifvas angående hafsdjurens fördelning.

Enligt teorien om härstamning med modifikation förklaras med ens den stora lagen om typernas varaktiga men icke oföränderliga succession inom samma yta, ty invånarna i hvarje fjerdedel af jorden sträfva naturligtvis att på denna fjerdedel lemna under den näst följande perioden beslägtade ehuru i viss mån modifierade afkomlingar. Om invånarna på en kontinent förr voro betydligt olika inbyggarna på en annan kontinent, så måste deras modifierade ättlingar ännu visa olikheter på samma vis och i nästan samma grad. Men efter mycket långa mellantider och efter stora geografiska förändringar som tillåta mycken inflyttning måste de svagare vika för de mera dominerande formerna och lagarna för den fordna och nuvarande fördelningen blifva icke oföränderliga.

Man kan på skämt fråga, om jag tror att Megatherium och andra beslägtade ofantliga vidunder som fordom lefde i Sydamerika hafva lemnat efter sig sengångare, bältor och myrslokar såsom sina vanslägtade efterkommande. Vi kunna icke för ett ögonblick antaga det. Dessa ofantliga djur hafva fullständigt dött ut utan att lemna några efterkommande. Men i Brasiliens grottor finnas många utdöda arter som i storlek och i alla andra karakterer äro beslägtade med de i Sydamerika ännu lefvande arterna, och några af dessa fossilier kunna vara de verkliga stamfäderna till lefvande arter. Vi få icke förglömma att enligt vår teori alla arter af samma slägte äro afkomlingar af en art, så att om sex slägten, hvart och ett med åtta arter finnas i en geologisk formation och i en följande formation finnas sex andra beslägtade eller representerande genera, hvart och ett med samma antal arter, då kunna vi antaga att i allmänhet blott en art af hvart och ett äldre slägte har lemnat efter sig modifierade afkomlingar, hvilka utgöra arterna af de nya slägtena; de andra sju arterna af hvarje slägte hafva dött ut utan att lemna efter sig några afkomlingar. Eller också, och detta är sannolikt det vanligaste förhållandet, äro två eller tre arter af blott två eller tre af de sex äldre slägtena stamfäder till de nya slägtena, i det de andra arterna och hela de andra slägtena dött ut helt och hållet. I undergångna ordningar med slägten och arter som aftaga i antal såsom de sydamerikanska Edentata lemna ännu färre slägten och arter modifierade afkomlingar efter sig.



\section[Sammanfattning]{Sammanfattning af detta och föregående kapitel.}

Jag har försökt att visa att de geologiska urkunderna äro ytterligt ofullständiga; att blott en liten del af jorden har blifvit geologiskt undersökt med noggranhet, att blott vissa klasser af organiska varelser hafva blifvit bibehållna i fossilt tillstånd, att antalet af specimen och arter som finnas i våra muséer är rakt ingenting i jemförelse med antalet generationer som måste hafva dukat under till och med under en enda formation, att då sänkning varit ett nästan nödvändigt vilkor för samlandet af aflagringar rika på fossila arter af många slag och tillräckligt tjocka att motstå en följande afnötning, stora mellantider måste hafva förflutit emellan de flesta af våra successiva formationer, att sannolikt en större utrotning försiggått under sänkningsperioderna och mera variation under höjningsperioderna och att från de senare urkunderna äro minst fullständiga, att hvarje enstaka formation icke blifvit afsatt utan afbrott, att durationen af hvarje formation sannolikt är kort i jemförelse med durationen af de specifika formerna, att inflyttning har spelat en vigtig rol i de nya formernas första uppträdande på en yta eller i en formation; att arter med stor spridning äro de som variera mest och oftast varit källa till nya arter; att varieteter först hafva varit lokala och slutligen att ehuru hvarje art måste hafva passerat talrika öfvergångsstadier, de perioder under hvilka modifikationsprocessen försiggått, ehuru många och långa beräknade efter år, hafva varit korta i jemförelse med de perioder under hvilka hvarje art förblifvit i oförändradt skick. Alla dessa omständigheter tillsammanstagna böra i hög grad förklara hvarföre vi — ehuru vi finna många föreningslänkar — icke finna oändliga varieteter som förena alla utdöda och lefvande former genom de finaste grader. Vi måste också beständigt ihågkomma, att några förenande varieteter emellan två eller flera former, som kunna finnas, upptagas såsom lika många nya och skilda arter, så vida icke hela kedjan kunde fullbordas; ty ingen kan påstå att vi hafva ett säkert kriterium enligt hvilket arter och varieteter med visshet kunna skiljas.

Den som förkastar denna åsigt om de geologiska urkundernas ofullständighet, bör med rätta förkasta hela teorien. Ty han kan förgäfves fråga, hvar alla dessa talrika öfvergångsformer finnas som fordom måste hafva förenat de beslägtade eller representerande arterna, som finnas i de successiva stadierna i samma stora formation. Han kan betvifla de långa tider som förflutit emellan de successiva formationerna; han kan förbise den vigtiga rol inflyttning har spelat, om han betraktar formationerna i en trakt allena såsom Europa; han kan framhålla det tydligt men ofta skenbart plötsliga uppträdandet af hela artgrupper. Han kan fråga: hvar finnas qvarlefvorna af alla dessa otaliga organismer som måste hafva funnits långt innan det cambriska systemet afsattes? Vi veta att åtminstone ett djur då lefde; men jag kan besvara denna fråga blott genom att antaga att der våra oceaner nu sträcka sig, der hafva de under ofantliga perioder funnits och der våra oscillerande kontinenter nu stå, der hafva de stått sedan början af det cambriska systemet; men att långt före denna period jorden företedde ett helt annat utseende, och att de äldre kontinenterna bildade af formationer äldre än de af oss kända nu existera blott såsom qvarlefvor i ett metamorfiskt tillstånd eller ligga begrafda under oceanen.

Om vi lemna dessa svårigheter synas mig de andra stora fakta ur palæontologien enkelt följa af teorien om härstamning med modifikation genom naturligt urval. Vi kunna deraf förstå orsaken hvarföre nya arter uppträda långsamt och efter hvarandra; hvarföre arter af skilda klasser icke nödvändigt förändras samtidigt eller i samma proportion eller i samma grad; ehuru de under tidernas lopp alla undergå modifikationer i någon mån. Försvinnandet af gamla former är en nästan oundviklig följd af nya formers bildande. Vi kunna fatta hvarföre en art aldrig återkommer sedan den en gång försvunnit. Grupper af arter tilltaga långsamt i antal och räcka olika lång tid, ty modifikationsprocessen är nödvändigt långsam och beror på många invecklade förhållanden. De dominerande arterna af stora dominerande grupper sträfva att lemna många modifierade ättlingar som bilda nya grupper och undergrupper. Då dessa äro bildade försvinna tillsammans arterna af de mindre kraftiga grupperna, ty de hafva ärft sin underlägsenhet från en gemensam stamfader, och lemna inga afkomlingar efter sig. Men en hel artgrupps ytterliga utrotande har stundom varit en långsam process, i det några få ättlingar lefva qvar i skyddade och isolerade områden. Då en grupp en gång helt och hållet dött ut, återkommer den ej, ty generationskedjan är bruten.

Vi kunna förstå, hvarföre de dominerande formerna som sprida sig vida och lemna största antalet varieteter sträfva att befolka verlden med beslägtade och modifierade ättlingar och dessa skola i allmänhet lyckas att undantränga de grupper som äro dem underlägsna i kampen för tillvaron. Efter långa mellantider tyckas derföre jordens alster hafva förändrats samtidigt.

Vi kunna förstå hvarföre alla lifsformer, gamla och nya, bilda tillsammans ett fåtal klasser, ty alla äro åtminstone genom generation förenade med hvarandra. Vi kunna förstå af den fortsatta sträfvan till karaktersdivergens, hvarföre ju äldre en form är, ju mera skiljer den sig i allmänhet från de nu lefvande, hvarföre gamla och utdöda former ofta sträfva att fylla tomrummet emellan nu existerande former, stundom förenande till en två grupper som fordom upptogos såsom skilda; men oftare förande dem blott litet närmare tillsammans. Ju äldre en form är, ju oftare står den till en viss grad emellan nu skilda grupper, ty ju äldre en form är ju mera beslägtad är den med, och ju mera liknar den följaktligen den gemensamma stamfadern till grupper som sedan blifvit vidt skilda. Utdöda former äro sällan direkta mellanformer emellan nu lefvande former utan först genom en lång omväg genom andra utdöda och afvikande former. Vi kunna klart se, hvarföre de organiska qvarlefvorna af tätt på hvarandra följande former äro nära beslägtade, ty de äro nära sammanbundna genom generation. Vi kunna tydligt inse hvarföre qvarlefvorna i en intermediär formation äro intermediära i karakter.

Invånarna af hvarje successiv period i jordens historia måste hafva besegrat sina föregångare i kampen för tillvaron och stå så till vida högre i naturens skala, och deras bygnad har i allmänhet blifvit mera specialiserad; och detta kan förklara den hos så många palæontologer allmänna åsigten att organisationen öfverhufvudtaget gått framåt. Utdöda och gamla former likna till en viss grad embryonerna af nyare djur som tillhöra samma klasser och detta underbara faktum får en enkel förklaring i våra åsigter. Successionen af samma typer inom samma områden under de senare geologiska perioderna upphör att vara hemlighetsfull och blir begriplig enligt grundsatsen om ärftlighet.

Om nu den geologiska urkunden är så ofullständig som många tro, och man kan åtminstone försäkra att den icke kan bevisas vara mera fullständig, förminskas eller försvinna de stora svårigheterna för teorien om naturligt urval. Å andra sidan tyckas mig alla palæontologiens hufvudsatser proklamera, att arterna hafva uppkommit genom vanlig generation, i det de gamla formerna hafva ersatts af nya och förädlade lifsformer, resultaten af variation och de mest gynnade varieteternas bestånd.


