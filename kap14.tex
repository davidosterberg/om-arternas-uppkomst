%FJORTONDE KAPITLET.


\chapter{Rekapitulation och slut.}

{\it
Rekapitulation af invändningarna emot teorien om naturligt urval. — Rekapitulation af de allmänna och enskilda omständigheter, som tala till dess fördel. — Orsaker till den allmänna tron på arternas oföränderlighet. — Huru långt teorien om naturligt urval kan utsträckas. — Verkningarna på naturhistoriens studium af dess antagande. — Slutanmärkningar.
}\\[0.5cm]

Då detta arbete är ett enda långt argument, kan det vara fördelaktigt för läsaren att hafva de ledande fakta och slutledningarna i korthet rekapitulerade.

Att många allvarsamma invändningar kunna göras emot teorien om härstamning med modifikation genom naturligt urval, vill jag icke förneka. Jag har bemödat mig att gifva dem deras fulla kraft. Ingenting kan i första ögonblicket synas svårare att tro, än att de mest invecklade organer och instinkter fått sin fullkomlighet icke genom krafter öfverlägsna ehuru analoga med menniskoförnuftet, utan blott genom accumulation af oräkneliga små variationer, hvar och en nyttig för sin egare. Icke destomindre kan icke denna svårighet betraktas såsom verklig, huru stor vår inbilning än må göra den, om vi medgifva följande satser, att alla delar af organismen och alla instinkter åtminstone förete individuela olikheter — att en kamp för tillvaron existerar, som leder till bevarande af gynsamma afvikelser i kroppsbygnad eller instinkt och slutligen att gradationer i fullkomligheten af hvarje organ kunna hafva existerat, hvar och en god för sitt slag. Sanningen af dessa satser kan som jag tror icke bestridas.

Det är otvifvelaktigt mycket svårt till och med att gissa genom hvilka gradationer många bildningar nått sin fulländning, isynnerhet ibland genombrutna och bristfälliga grupper af organiska varelser, som mycket lidit af förintelsen; men vi se i naturen så många sällsamma grader, att vi böra vara ytterst försigtiga i att säga, att ett organ eller en instinkt eller hela kroppsbygnaden icke kunde hafva kommit i dess nuvarande skick genom många gradvisa steg. Det finnes, det måste medgifvas, fall af synnerlig svårighet för teorien om det naturliga urvalet och ett af de mest egendomliga af dessa är närvaron af två eller tre kaster af arbetare eller sterila honor i samma myrsamhälle, men jag har försökt att visa, huru dessa svårigheter besegras.

Hvad beträffar den nästan allmänna ofruktsamheten emellan arter vid kroasering, som bildar en så anmärkningsvärd motsats till den nästan allmänna fruktsamheten hos varieteter vid kroasering, måste jag hänvisa läsaren till rekapitulationen af de fakta vid slutet af åttonde kapitlet, som synas mig visa, att denna ofruktsamhet icke mera är någon speciel begåfning, än omöjligheten för två skilda träd att ympas tillsammans, utan sammanfaller med olikheter inskränkta till reproduktionsorganerna hos de kroaserade slägtena. Vi se sanningen af denna slutsats i den stora olikheten i resultaten af två arters ömsesidiga kroasering, om en art först begagnats till fader och sedan till moder: analogien från betraktelserna om de dimorfa växterna leder tydligen till samma resultat, ty om formerna befruktas illegitimt, gifva de få eller inga frön och deras afkomlingar blifva mer eller mindre ofruktsamma; och dessa former som otvifvelaktigt tillhöra samma art skilja sig i intet hänseende från hvarandra, annat än i sina reproduktionsorganer och deras funktioner.

Ehuru fruktsamheten hos varieteter vid kroasering och hos deras mestiser af så många författare försäkrats vara allmän, kan detta icke anses vara riktigt efter fakta gifna af sådana auktoriteter som Gärtner och Kölreuter. Ej heller är den så allmänna fruktsamheten vid varieteters kroasering öfverraskande, om vi ihågkomma, att deras reproduktionssystem antagligen icke blifvit i så hög grad modifierade. Vidare hafva de flesta varieteter som underkastats experiment varit alster af domesticering; och då domesticering (jag menar icke blott instängning) nästan säkert sträfvar att aflägsna ofruktsamhet, böra vi icke vänta att den också skall åstadkomma ofruktsamhet.

Bastarders ofruktsamhet är ett fall helt och hållet skildt från ofruktsamhet vid första kroaseringen, ty bastardernas reproduktionsorganer äro mer eller mindre funktionelt impotenta, hvaremot vid första kroaseringen båda arternas organer naturligtvis äro fullkomliga. Då vi fortfarande se att organismer af alla slag blifva i någon mån sterila deraf att de utsättas för förändrade vilkor, behöfva vi icke förvånas öfver att bastarder äro till en viss grad ofruktsamma, ty deras konstitution bör nästan ovilkorligen vara i oordning i följd af två skilda organisationers sammanblandning, men om detta är den verkliga orsaken till deras ofruktsamhet vill jag icke afgöra. Ofvanstående jemförelse har ett stöd i en annan parallel, men rakt motsatt klass af fakta, nämligen att kraften och fruktsamheten hos alla organiska varelser förökas genom små förändringar i deras lifsvilkor och att afkomman af obetydligt modifierade former eller varieteter vid kroasering förvärfva ökad styrka och fruktsamhet. En ansenlig förändring i lefnadsförhållanden och kroasering emellan betydligt modifierade former minska fruktsamheten, under det å andra sidan mindre förändringar i lifsvilkoren och kroasering emellan mindre modifierade former öka fruktsamheten.

Om vi vända oss till den geografiska fördelningen äro svårigheterna rätt allvarsamma som möta teorien om härstamning med modifikation. Alla individer af samma art och alla arter af samma slägte eller till och med högre grupp måste härstamma från gemensamma föräldrar, och derföre i huru aflägsna och isolerade delar af jorden de nu förekomma, måste de under successiva generationer hafva flyttat från någon punkt till alla andra. Vi äro ofta ur stånd att ens gissa huru detta skulle tillgått. Då vi likväl hafva skäl att tro, att några arter hafva behållit samma specifika form för mycket långa tidsperioder omöjliga att beräkna efter år, bör icke för mycken vigt läggas på samma arts vidsträckta spridning; ty under långa perioder finnas alltid många utsigter för flyttning på något sätt. En afbruten utbredning kan ofta förklaras genom arternas undergång i de mellanliggande områdena. Det kan icke förnekas att vi äro hittills mycket okunniga om fulla vidden af de olika klimatiska och geografiska förändringar, som jorden undergått under nyare tider och sådana förändringar kunna naturligtvis hafva underlättat flyttning. Såsom exempel har jag försökt att visa, huru mäktigt istidens inflytande varit på fördelningen af beslägtade arter öfver jorden. Vi hafva ännu ingen kännedom om de många tillfälliga transportmedlen. Beträffande den omständigheten att skilda arter af samma slägte bebo aflägsna och isolerade områden, så måste alla flyttningssätt varit möjliga under en lång period, då modifikationsprocessen nödvändigt varit långsam; och följaktligen är svårigheten uti arternas vidsträckta utbredning som tillhöra samma slägte i någon mån minskad.

Då enligt teorien om det naturliga urvalet ett oändligt antal mellanformer måste hafva existerat förenande alla arter i hvarje grupp genom öfvergångar så fina som våra nuvarande varieteter, kan man fråga: hvarföre se vi icke dessa mellanformer öfverallt omkring oss? Hvarföre äro icke alla de organiska varelserna omkring oss sammanblandade i ett outredligt kaos? Hvad de nu lefvande formerna angår, få vi ihågkomma, att vi icke hafva rätt att vänta (utom i sällsynta fall) att upptäcka direkt förenande länkar emellan dem utan blott emellan hvar och en och någon utdöd, undanträngd form. Äfven på en vidsträckt yta, som under en lång period förblifvit sammanhängande och hvars klimat och andra förhållanden ändras omärkligt vid förflyttandet från ett område som bebos af en art till ett annat som innehafves af en beslägtad art, hafva vi icke rätt att vänta att ofta finna mellanvarieteter i de mellanliggande zonerna. Ty vi hafva skäl att tro att blott ett fåtal arter af ett slägte undergå förändringar; de andra blifva totalt utrotade och lemna inga modifierade ättlingar. Af de arter som förändras, variera blott få i samma trakt liktidigt och alla andra modifikationer försiggå långsamt. Jag har också visat, att de intermediära varieteterna som sannolikt först uppträdde i de mellanliggande zonerna, skulle med lätthet utträngas af de beslägtade formerna å endera sidan; ty då de senare som existera i större mängd i allmänhet modifieras och förädlas i hastigare proportion än mellanvarieteterna, hvilka existerade i mindre antal, så skulle intermediära formerna under tidernas lopp utträngas och utrotas.

Enligt denna lära om utrotandet af en oändlig mängd föreningslänkar emellan nu lefvande och utdöda invånare på jorden och i hvarje successiv period emellan de utdöda och ännu äldre arterna, hvarföre är icke hvarje geologisk formation fyld af sådana länkar? Hvarföre lemnar icke hvarje samling af fossila qvarlefvor fulla bevis på lifsformernas öfvergångar och förvandlingar? Ehuru geologiska undersökningar otvifvelaktigt uppenbarat den fordna tillvaron af många föreningslänkar och på detta sätt närmare förenat talrika organiska former, hafva de dock icke framlagt de oändligt många öfvergångarna emellan fordna och närvarande arter som teorien fordrar; och detta är den svåraste af de många invändningar som kunna uppställas emot henne. Hvarföre synas åter hela grupper af beslägtade arter, ehuru ofta blott skenbart, hafva oförmodadt uppträdt i de på hvarandra följande geologiska stadierna? Ehuru vi nu veta att organiska varelser uppträdde på jorden i en oberäkneligt aflägsen period, långt innan de lägsta bäddarna af de cambriska lagren afsattes, hvarföre finna vi icke under detta system stora hopar af lager belastade med qvarlefvor af stamfäderna till de cambriska fossilierna? Ty enligt teorien måste sådana lager någonstädes afsatts i dessa gamla ytterligt okända perioder i jordens historia.

Jag kan besvara dessa frågor och invändningar blott med antagandet att den geologiska urkunden är långt mer ofullständig än de flesta geologer tro. Antalet af exemplar i alla våra museer är så godt som ingenting i jemförelse med de tallösa arter hvilka helt säkert existerat. Stamformen till två eller flera arter skulle icke vara i alla sina karakterer en direkt mellanform emellan dess modifierade afkomlingar, icke mer än klippdufvan är direkt intermediär i kräfva och stjert emellan dess afkomlingar påfågeldufvan och kroppdufvan. Vi äro ur stånd att igenkänna en art såsom stamfader till en annan modifierad art, om vi skulle undersöka båda aldrig så noga, så länge vi icke ega alla mellanformer, och på grund af ofullständigheten i den geologiska urkunden hafva vi icke rätt att vänta oss finna så många föreningslänkar. Om två eller tre eller till och med flera föreningslänkar upptäcktes, skulle de helt enkelt upptagas såsom lika många nya arter, isynnerhet om de funnos i olika geologiska formationer, deras olikheter må vara aldrig så små. Talrika nu existerande tvifvelaktiga former kunde nämnas, som sannolikt äro varieteter, men hvem kan påstå att i framtiden så många fossila länkar kunna upptäckas, att naturforskarna skola vara i stånd att afgöra huruvida dessa tvifvelaktiga former böra eller icke böra kallas varieteter. Blott en liten del af jorden har blifvit geologiskt undersökt; blott af vissa klasser kunna organiska varelser bevaras i fossilt tillstånd åtminstone i något större antal. Många arter då de en gång äro bildade undergå icke vidare några förändringar och dö ut utan att lemna några modifierade afkomlingar; och de perioder, under hvilka arterna hafva undergått modifikation, ehuru de äro långa beräknade efter år, hafva sannolikt varit korta i jemförelse med de perioder under hvilka de hafva behållit samma form. Det är de dominerande och vidt utbredda formerna som variera mest och oftast och varieteter äro ofta först lokala — två orsaker som göra upptäckten af mellanformer i någon formation mindre sannolik. Lokala varieteter sprida sig icke i andra och aflägsna områden förr än de blifvit ansenligt modifierade och förädlade; och då de hafva spridt sig och upptäckas i en geologisk formation, skola de synas som oförmodadt skapade der, och de skola helt enkelt upptagas som nya arter. De flesta formationer hafva blifvit samlade med afbrott; och deras varaktighet har sannolikt varit kortare än artformernas varaktighet öfverhufvud. Successiva formationer äro i de flesta fall skilda från hvarandra genom tomma långvariga mellantider; ty fossilförande formationer tjocka nog att motstå afnötning kunna i regel samlas blott der mycket sediment afsättes på hafvets botten. Under de vexlande perioderna af ytans höjning och stillastående böra urkunderna i allmänhet vara tomma. Under de senare perioderna bör föränderligheten sannolikt vara stor bland de organiska formerna; under sänkningsperioden deremot träder förintelsen i verksamhet.

Med afseende på frånvaron af fossilrika lager under den cambriska formationen kan jag blott återgå till den hypotes jag framstält i nionde kapitlet. Att den geologiska urkunden är ofullständig till den grad vår teori fordrar, det torde få vara benägna att antaga. Om vi betrakta tillräckligt långa mellantider, visar geologien att alla arter hafva förändrats och de hafva förändrats på erforderligt sätt, ty det har skett långsamt och med grader. Vi se detta tydligt deruti, att de fossila qvarlefvorna i tätt följande formationer alltid äro mycket närmare beslägtade med hvarandra än fossilier från vidt skilda formationer.

Detta är summan af de hufvudsakliga invändningarna och svårigheterna som rättmätigt kunna uppställas emot teorien: och jag har nu i korthet redogjort för de svar och förklaringar som kunna gifvas. Jag har sjelf allt för väl i många år känt tyngden af dessa svårigheter för att betvifla deras vigt, men det förtjenar särskildt anmärkas att de vigtigare invändningarna röra frågor om hvilka vi äro fullkomligt okunniga; ej heller veta vi huru okunniga vi äro. Vi känna icke alla möjliga öfvergångsformer emellan de enklaste och mest fulländade organer; man kan icke påstå, att vi känna alla de olika fördelningsmedlen under årens lopp, eller att vi veta huru ofullständig den geologiska urkunden är. Allvarsamma äro visserligen alla dessa invändningar, men de äro dock icke så vidt jag kan döma tillräckliga att kullkasta teorien om härstamning med modifikation.

Låt oss nu vända oss till den andra sidan af argumentet. Under domesticering se vi mycken föränderlighet, orsakad eller åtminstone väckt af förändring i lefnadsvilkor. Denna föränderlighet styres af många invecklade lagar — vexelverkan, användning och overksamhet och den bestämda verkan af omgifvande förhållanden. Det är mycket svårt att förvissa sig om i hvad mån våra domesticerade produkter hafva blifvit modifierade, men vi kunna tryggt antaga, att det har skett i hög grad och att modifikationerna kunna ärfvas under långa perioder. Så länge lifsvilkoren blifva desamma, hafva vi skäl att tro, att en modifikation, som redan gått i arf i många generationer, kan fortfarande gå i arf under ett nästan oändligt antal generationer. Å andra sidan hafva vi bevis på, att föränderlighet, då den en gång kommit i verksamhet icke upphör någon längre tid i kulturtillståndet; ty nya varieteter bildas ännu tillfälligtvis af våra äldsta kulturalster.

I sjelfva verket åstadkommer menniskan aldrig föränderlighet, hon blott försätter utan afsigt de organiska varelserna i nya lefnadsförhållanden, och naturen verkar då på organisationen och orsakar föränderlighet. Men menniskan kan utvälja de af naturen henne gifna variationerna och på detta sätt samla dem i en önskad riktning. Hon lämpar således djur och växter för sin egen nytta eller nöje. Hon kan göra det metodiskt eller omedvetet genom att skydda de för henne nyttigaste eller angenämaste individer utan någon afsigt att förändra rasen. Hon kan förvisso betydligt inverka på karakteren af en ras genom att i successiva generationer utvälja individuela olikheter så små att de ej äro märkbara för ett ovandt öga. Detta urval har varit den mest verksamma orsak vid bildandet af de mest utmärkta och nyttiga raser af kulturalster. Att många raser som menniskan frambragt hafva i hög grad karakteren af naturliga arter, visas genom det outplånliga tviflet, om många af dem äro varieteter eller ursprungligen skilda arter.

Det finnes intet tydligt skäl, hvarföre dessa grundsatser, så verksamma vid domesticering, icke skulle kunna verka under naturtillståndet. I de gynnade individernas och rasernas bestånd uti den ständigt återkommande kampen för tillvaron se vi en kraftig och alltid verkande form af urvalet. Kampen för tillvaron följer omedelbarligen från alla organiska varelsers förökning i geometrisk progression. Denna förökningens geometriska progression bevisas genom beräkning, genom den hastiga tillväxten bland många växter och djur under en följd af egendomliga årstider eller vid naturalisering i nya länder. Flera individer födas än som möjligen kunna fortlefva. Ett gran i vågskålen kan afgöra, hvilka individer skola lefva och hvilka skola dö, hvilka varieteter eller arter skola förökas i antal och hvilka skola aftaga och slutligen dö ut. Då individerna af samma art i alla hänseenden komma i närmaste täflan med hvarandra, måste kampen vara häftigast emellan dem; den bör vara nästan lika svår emellan varieteter af samma art, och närmast i häftighet emellan arter af samma slägte. Å andra sidan bör kampen ofta vara häftig emellan varelser som stå långt ifrån hvarandra i naturens skola. Den minsta fördel hos vissa individer, i någon ålder eller årstid, öfver dem med hvilka de komma att täfla eller bättre lämplighet för de omgifvande fysiska förhållanden i huru ringa grad som helst skall sänka vågskålen.

Bland djur med skilda kön måste i de flesta fall en kamp uppstå emellan hannarna om besittningen af honorna. De kraftigaste hannarna eller de som varit lyckligast i striden med lifsvilkoren skola lemna största mängd afföda. Men framgången skall ofta bero på att hannarna hafva särskilda vapen eller försvarsmedel eller behag och en ringa förmån skall lända till seger.

Då geologien tydligt förklarar, att hvarje land har undergått stora fysiska förändringar, hafva vi kunnat vänta att organiska varelser hafva varierat i naturtillståndet på samma sätt som under domesticering. Och om någon föränderlighet egde rum i naturtillståndet, så vore det oförklarligt om icke ett naturligt urval kom i verksamhet. Man har ofta försäkrat, men försäkringen kan icke bevisas, att graden af variation under naturtillståndet är strängt begränsad. Menniskan kan ehuru verkande på yttre karakterer allena och ofta nyckfullt inom en kort period vinna stora resultat genom att addera små individuela afvikelser hos sina domesticerade alster och hvar och en medgifver att arter förete individuela olikheter. Men utom sådana olikheter medgifva alla naturforskare att varieteter existera som anses tillräckligt skilda för att upptagas i ett system. Ingen har uppdragit en tydlig skilnad emellan individuela olikheter och små varieteter eller emellan tydligare utbildade varieteter och underarter och arter. På skilda kontinenter och på skilda delar af samma kontinent, om den är afdelad genom stängsel af något slag, och på aflägsna öar, hvilken mängd af former existera ej, som mången erfaren naturforskare skulle upptaga såsom varieteter, andra såsom geografiska raser eller underarter, och andra såsom skilda ehuru närslägtade arter.

Om nu växter och djur variera, låt vara aldrig så litet eller långsamt, hvarföre skulle vi betvifla att variationerna eller individuela olikheter, som äro på något sätt fördelaktiga, kunna skyddas och bevaras genom naturligt urval? Om menniskan kan med tålamod utvälja för henne nyttiga varieteter, hvarföre skulle icke under förändrade och invecklade lifsvilkor variationer ofta uppkomma nyttiga för naturens lefvande alster och skulle icke dessa kunna skyddas eller utväljas? Hvilken gräns kan man väl sätta för denna förmåga, verksam under hela tidsåldrar och strängt granskande hela konstitutionen, skapnaden och vanorna hos hvarje varelse, — gynnande det som är godt och förkastande det som är dåligt? Jag kan icke se någon gräns för denna kraft, långsamt och skönt bearbetande hvarje form, så att den blir lämplig för de mest invecklade lefnadsförhållanden. Teorien om naturligt urval, äfven om vi icke fästa oss vid något annat, synes i och för sig sjelf antaglig. Jag har redan rekapitulerat så väl jag har kunnat de deremot uppstälda svårigheter och invändningar. Vi skola nu vända oss till de särskilda fakta och argument som tala till förmån för teorien.

Enligt den åsigten att arter äro blott väl utpräglade och permanenta varieteter och att hvarje art först existerade såsom varietet, kunna vi se hvarföre icke någon demarkationslinie kan uppdragas emellan arter, som allmänt antagas uppkomma genom särskilda skapelseakter, och varieteter, som erkännas bildade genom sekundära lagar. Enligt samma åsigt kunna vi förstå orsaken hvarföre i hvarje region, der många arter af ett slägte bildats och der de nu frodas, samma arter förete många varieteter; ty der artfabrikationen varit verksam kunna vi såsom allmän regel vänta att ännu finna den i full gång, och detta är verkliga förhållandet om varieteter äro begynnande arter. Arterna af de större slägtena, hvilka lemna största antalet varieteter eller begynnande arter behålla dessutom till en viss grad karakteren af varieteter; ty de afvika från hvarandra genom en mindre grad af skilnad än arterna af mindre slägten. De närslägtade arterna af de större slägtena hafva också tydligen inskränkt utbredning och äro i sina slägtskapsförhållanden samlade i små grupper omkring andra arter, — två hänseenden hvaruti de äfven likna varieteter. Dessa förhållanden äro sällsamma enligt åsigten om hvarje arts oberoende skapelse, men förklaras lätt om hvarje art först existerade såsom varietet.

Då hvarje art genom sin förökning i geometrisk progression sträfvar att tilltaga i oändligt antal, och då de modifierade ättlingarna af hvarje art skola vara i stånd att föröka sig i samma mån mera, som de äro olika i vanor och skapnad, så att det är dem möjligt att intaga många och helt olika platser i naturens hushållning, måste hos det naturliga urvalet finnas en beständig sträfvan att skydda de mest divergerande ättlingarna af någon art. Under en länge fortsatt modifikationsprocess sträfva derföre de små skiljaktigheterna, karakteristiska för varieteter af samma art, att förökas till större afvikelser, karakteristiska för arterna af hvarje slägte. Nya och förädlade varieteter skola ovilkorligen undantränga och utrota de äldre, mindre förädlade och intermediära varieteterna; och på detta sätt blifva arterna i stor utsträckning bestämda och skilda föremål. Dominerande arter som tillhöra de större grupperna inom hvarje klass, sträfva att bilda nya dominerande former, så att hvarje stor grupp sträfvar att blifva allt större och större och på samma gång mera divergent i karakter. Men då alla grupper icke kunna på detta sätt med framgång tilltaga i storlek, ty jorden skulle icke rymma dem, besegra de mera dominerande grupperna de mindre. Detta sträfvande hos de större grupperna att alltjemt tilltaga i storlek och divergera i karakter jemte den nästan oundvikliga förintelsen förklarar anordnandet af alla lifsformer i grupper underordnade andra grupper, alla inom ett fåtal stora klasser, som varit förherskande i alla tider. Denna stora sats, alla organiska varelsers grupperande under ett så kalladt naturligt system, är omöjlig att förklara enligt teorien om oberoende skapelseakter.

Då det naturliga urvalet verkar blott genom att samla små, successiva gynsamma variationer, kan det icke åstadkomma någon stor eller hastig modifikation; det kan verka blott genom korta och långsamma steg. Regeln ”natura non facit saltum”, som hvarje nytt bidrag till våra kunskaper gör allt mera sann, är lätt begriplig enligt denna teori. Vi kunna se hvarföre igenom hela naturen samma allmänna mål vinnes genom en nästan oändlig mängd vexlande medel; ty hvarje egendomlighet, då den en gång är förvärfvad, går länge i arf och bildningar som redan blifvit olikartade på många vägar kunna lämpas för samma ändamål. Vi kunna, i korthet sagdt, inse hvarföre naturen är slösande på förändringar men njugg på nyheter. Men hvarföre detta skulle vara en naturlag om hvarje art blifvit särskildt skapad, kan ingen menniska förklara.

Många andra förhållanden kunna, såsom mig synes, förklaras enligt denna teori. Nog är det besynnerligt, att en fågel under formen af en hackspett skulle blifvit skapad för att jaga insekter på marken; att landgåsen, som aldrig eller sällan simmar, blifvit skapad med simfötter; att en trastartad fågel skapats för att dyka och lefva af insekter som finnas under vatten; att en stormfågel blifvit bildad med vanor och skapnad, lämpliga för en tordmules lefnadssätt! Och så vidare i otaliga andra fall. Men enligt vår åsigt, att hvarje art beständigt förökas i antal, under det det naturliga urvalet alltid är redo att göra de långsamt varierande ättlingarna af hvar och en lämpliga för någon ej besatt eller illa besatt plats i naturens hushållning, upphöra alla dessa fakta att vara besynnerliga och kunde till och med insetts på förhand.

Vi kunna förstå orsaken hvarföre en sådan harmonisk skönhet råder i allmänhet uti naturen. Att undantag finnas enligt vårt begrepp om skönhet, lär väl ingen vilja betvifla, om han tänker på de giftiga ormarna, några fiskar och några förskräckliga flädermöss med förvridna menniskoansigten. Sexuelt urval har gifvit de mest lysande färger och andra prydnader åt hannarna, men stundom åt båda könen af många fåglar, fjärilar och några få andra djur. Fåglarnas röst har urvalet ofta gjort musikalisk såväl för honornas som för våra öron. Blommor och frukter hafva gjorts märkbara genom prunkande färger i motsats till det gröna löfverket, så att blommorna med lätthet kunna ses, besökas och befruktas af insekter och fröen spridas af fåglar. Slutligen hafva några lefvande föremål blifvit vackra blott genom symmetrisk utveckling.

Då det naturliga urvalet verkar genom täflan, gör det invånarna i hvarje trakt fullkomliga blott i förhållande till de andra invånarna; så att vi icke böra öfverraskas af att arterna i något land, ehuru de enligt den allmänna åsigten antagas hafva blifvit skapade och särskildt lämpade för denna trakt, blifva besegrade och utträngda af naturaliserade alster från andra länder. Ej heller få vi förundra oss öfver, att alla inrättningar i naturen så vidt vi kunna döma icke äro absolut fullkomliga, eller att några af dem helt och hållet strida emot vårt begrepp om ändamålsenlighet. Vi få icke förvånas öfver att biets gadd en gång begagnad emot en fiende ofta förorsakar biets död; att drönare framfödas i så stor mängd för en enda förrättning och sedan dödas af sina sterila systrar; öfver den oändliga mängd frömjöl som våra furor utsprida, öfver bivisarnas instinktmässiga hat mot sina egna fruktsamma döttrar, öfver ichneumonidernas utveckling inuti lefvande löfmaskar och andra sådana fall. Det mest märkvärdiga enligt teorien om naturligt urval är att flera fall af bristande fullkomlighet icke blifvit iakttagna.

De invecklade och föga kända lagar som beherska erkända varieteter äro så vidt vi kunna se desamma som de lagar, hvilka hafva reglerat bildandet af så kallade specifika differenser. I båda fallen tyckas fysiska förhållanden hafva åstadkommit någon direkt och bestämd verkan, men huru mycken kunna vi icke säga. Således om varieteter intränga i något nytt område, antaga de gerna någon af de karakterer, som äro egendomliga för arterna i detta område. Hos både varieteter och arter tyckes användning och overksamhet hafva åstadkommit ansenliga verkningar; ty det är omöjligt att bestrida sanningen häraf, om vi se till exempel på den ofvannämda andarten, hvars vingar äro odugliga till flygt och i samma tillstånd som den tama andens; eller den gräfvande tucutucu, som ofta är blind, och vissa mullvadar som vanligen äro blinda och hafva sina ögon betäckta med hud; eller om vi betrakta de blinda djur som bo i mörka hålor i Amerika och Europa. Hos varieteter och arter synas vexelverkande variationer hafva spelat en vigtig rol, så att om en del blifvit modifierad andra delar nödvändigt också blifvit modifierade. Hos både varieteter och arter förekommer återgång till längesedan förlorade karakterer. Enligt teorien om särskilda skapelser är det fullkomligt oförklarligt detta tillfälliga uppträdande af strimmor på skuldran och benen hos flera arter af hästslägtet och deras bastarder. Huru enkelt förklaras icke detta förhållande, om vi antaga att dessa arter alla härstamma från en strimmig stamfader på samma sätt som de olika dufraserna härstamma från den blåa streckade klippdufvan?

Enligt den vanliga åsigten om hvarje arts oberoende skapelse, hvarföre skulle de specifika karaktererna, de som skilja arterna af samma slägte från hvarandra, vara mera föränderliga än slägtkaraktererna i hvilka alla öfverensstämma? Hvarföre skulle till exempel en blommas färg vara mera benägen att variera i en art af ett slägte, om de andra arterna, alster af oberoende skapelser, hafva olika färgade blommor, än om alla arterna af slägtet hafva lika färgade blommor? Om arter blott äro väl utpräglade varieteter, hvilkas karakterer blifvit i hög grad beständiga, kunna vi förstå detta förhållande, ty de hafva redan varierat sedan de afskilde sig från en gemensam stamfar i sina karakterer, genom hvilka de blifvit specifikt skilda från hvarandra; dessa samma karakterer skulle vara mera benägna att fortfarande variera än de generiska karaktererna som utan förändring gått i arf under enorma tidsperioder. Enligt skapelseteorien är det oförklarligt, hvarföre en del, som hos någon art af ett slägte är på ett mycket ovanligt sätt utvecklad och derföre såsom vi naturligen kunna antaga af stor vigt för arten, skulle vara i hög grad benägen för variationer; men enligt vår åsigt har denna del sedan de olika arterna afgrenade sig från en gemensam stamfar undergått en ovanlig grad af förändring och modifikation och derföre kunna vi vänta, att denna del i allmänhet bör ännu vara föränderlig. Men en del kan utvecklas på det mest ovanliga sätt, liksom flädermusens vinge, utan att vara mera föränderlig än andra bildningar, om organet är gemensamt för många underordnade former, det är om det blifvit ärfdt under mycket lång tid; ty i detta fall har det blifvit beständigt genom ett länge fortsatt naturligt urval.

Om vi se på instinkterna, så underbara många äro, erbjuda de icke någon större svårighet än kroppsliga bildningar för teorien om naturligt urval af små, gradvisa men gynsamma modifikationer. Vi kunna på detta sätt förstå, hvarföre naturen rör sig genom gradvisa steg vid förlänandet af hvarje djurs olika instinkter. Jag har försökt att visa, huru mycket ljus grundsatsen om gradation sprider öfver kupbiets beundransvärda arkitektoniska förmåga. Vanan kommer otvifvelaktigt att verka vid instinkternas modifikation; men den är säkert icke oundgänglig, såsom vi hafva sett hos de könlösa insekterna, hvilka icke lemna några arfvingar till verkningarna af länge fortsatta vanor. Enligt åsigten att alla arter härstamma från en gemensam stam och hafva ärft mycket gemensamt, kunna vi förstå orsaken, hvarföre beslägtade arter, då de försättas under helt olika lifsvilkor, dock följa nästan samma instinkter; hvarföre trastarna i det tropiska och tempererade Sydamerika till exempel betäcka sina bon med dy likt våra britiska arter. Enligt åsigten att instinkter långsamt förvärfvats genom naturligt urval, behöfva vi icke förundra oss att några instinkter tydligen icke äro fullkomliga och utsatta för fel, och att många instinkter förorsaka andra djur lidanden.

Om arter äro blott väl markerade och permanenta varieteter, kunna vi med ens se, hvarföre deras kroaserade ättlingar följa samma invecklade lagar i graden och arten af likhet med deras föräldrar, — att sammansmälta i hvarandra genom successiv kroasering och i andra sådana punkter — som de kroaserade ättlingar af erkända varieteter. Denna likhet skulle vara oförklarlig, om arterna äro alster af oberoende skapelser och om varieteter hafva uppkommit enligt sekundära lagar.

Om vi medgifva att den geologiska urkunden är ofullständig till ytterlig grad, så skola de fakta som urkunden gifver i hög grad stödja teorien om härstamning med modifikation. Nya arter hafva långsamt och tid efter annan trädt fram på skådebanan; och graden af förändring efter olika tids förlopp är helt olika i skilda grupper. Arters och hela artgruppers undergång, som har spelat en så vigtig rol i den organiska verldens historia, följer nästan oundvikligt från grundsatsen om naturligt urval; ty gamla former ersättas af nya och förädlade former. Hvarken enstaka arter eller artgrupper uppträda ånyo, om den vanliga generationskedjan en gång är bruten. Det gradvisa försvinnandet af dominerande former med den långsamma modifikationen af deras ättlingar är anledningen till, att lifsformerna efter långa mellantider synas oss samtidigt förändrade öfver hela jorden. Den omständigheten att de fossila qvarlefvorna i hvarje formation äro i någon mån intermediära i karakterer emellan fossilierna i de ofvan och under liggande formationerna förklaras helt enkelt genom deras intermediära läge i härstamningskedjan. Det kända faktum, att alla utdöda arter kunna klassificeras tillsammans med alla de nyare varelserna, följer naturligt deraf, att de lefvande och utdöda varelserna äro ättlingar af gemensamma föräldrar. Då arter i allmänhet hafva divergerat i karakter under sin långa härstamnings- och modifikationskurs, kunna vi inse, hvarföre de äldre formerna, eller de tidiga stamfäderna till hvarje grupp, så ofta intaga ett läge som till en viss grad är intermediärt emellan lefvande grupper. Nya former betraktas i allmänhet såsom varelser öfverhufvudtaget högre i organisationsskalan än äldre former, och de måste vara högre så till vida, att de senare och mera förädlade formerna hafva i kampen för tillvaron besegrat de äldre och mindre förädlade; de hafva också i allmänhet fått organer mera specialiserade för skilda förrättningar. Detta förhållande låter väl förlika sig dermed att talrika arter ännu behålla enkla och föga förädlade bildningar, lämpliga för enkla lifsvilkor; det är likaledes förenligt dermed att några former hafva gått tillbaka i organisation, då de vid hvarje härstamningsstadium hafva blifvit bättre lämpade för förändrade och lägre lefnadsvanor. Slutligen förklaras äfven den underbara lagen om beslägtade formers långa varaktighet på samma kontinent — pungdjur i Australien, tandlösa i Sydamerika, och andra dylika fall — ty i allmänhet äro inom samma trakt de nu lefvande och de utdöda formerna beslägtade genom härstamning.

Om vid betraktande af den geografiska fördelningen vi medgifva att under tidernas lopp mycken flyttning egt rum från en del af jorden till en annan, beroende på fordna klimatiska och geografiska förändringar och på de många tillfälliga och okända spridningsmedlen, då kunna vi förstå enligt teorien om härstamning med modifikation de flesta vigtigaste förhållanden vid fördelningen. Vi kunna inse, hvarföre en så öfverraskande parallelism skulle ega rum i fördelningen af organiska varelser öfver ytan och deras geologiska succession i tiden; ty i båda fallen hafva varelserna varit förenade genom vanlig generation och modifikationssätten hafva varit de samma. Vi se fulla betydelsen af det märkvärdiga faktum, som öfverraskat hvarje resande, nämligen att på samma kontinent under de mest skiljaktiga förhållanden, under hetta och köld, på berg- och lågland, i öknar och träsk, de flesta invånare inom hvarje större klass äro fullkomligt beslägtade, ty de äro afkomlingar af samma förfäder och tidigare nybyggare. Enligt samma grundsats om forntida flyttningar i de flesta fall förenad med modifikation kunna vi förklara med biträde af isperioden identiteten af några få växter och den nära slägtskapen af så många andra på de mest aflägsna berg och i de norra och södra tempererade zonerna; och likaledes den nära slägtskapen af några af invånarna i hafvet vid nordliga och sydliga tempererade latituder, ehuru skilda af hela oceanen emellan vändkretsarna. Ehuru två länder kunna förete fysiska förhållanden så nära lika som samma art möjligen kan erfordra, böra vi icke förvånas öfver att deras invånare äro olika, om de under en lång tid varit söndrade från hvarandra; ty då relationen organismer emellan är den vigtigaste af alla relationer och då två länder fått sina bebyggare på olika tider och i olika proportioner från något annat land eller från hvarandra, bör modifikationens lopp i de båda ytorna ovilkorligen varit olika.

Enligt denna åsigt om flyttning med följande modifikation inse vi, hvarföre oceanöar bebos blott af ett fåtal arter, men af dessa många äro egendomliga eller endemiska former. Vi kunna tydligen inse, hvarföre arter som tillhöra de djurgrupper, hvilka icke kunna öfverskrida stora oceansträckor, såsom grodor och landdäggdjur, icke bebo oceanöar; och hvarföre å andra sidan nya och egendomliga arter af flädermöss, djur som kunna passera oceanen, så ofta finnas på öar på långt afstånd från någon kontinent. Sådana fall som närvaron af egendomliga arter af flädermöss på oceanöar och frånvaron af alla andra landdjur äro ytterst oförklarliga fakta enligt teorien om oberoende skapelseakter.

Tillvaron af nära beslägtade eller representerande arter på två områden innebär enligt teorien om härstamning med modifikation, att samma stamformer förr bebodde båda områdena, och vi finna nästan oföränderligen, att hvarhelst många beslägtade arter bebo två områden, några andra arter äro gemensamma för båda. Hvarhelst många närslägtade ehuru skilda arter påträffas, finnas likaledes tvifvelaktiga former och varieteter tillhörande samma grupper. Det är en regel af stor allmänhet, att invånarna på hvarje yta äro beslägtade med invånarna i närmaste trakt, som kunnat tjena till källa för flyttningar. Vi se detta i den slående slägtskapen emellan nästan alla växter och djur på Galapagosöarna, Juan Fernandez och andra amerikanska öar och växterna och djuren på Amerikas angränsande fastland, och emellan inbyggarna på Cap Verdöarna samt andra Afrikanska öar och arterna på Afrikas fastland. Man måste medgifva, att dessa förhållanden icke få någon förklaring enligt skapelseteorien.

Det kända förhållandet, att alla förgångna och lefvande organiska varelser kunna anordnas i ett fåtal klasser, i grupp under grupp, och att de utdöda formerna ofta passa in emellan de nyare grupperna, kan förklaras enligt teorien om naturligt urval med dess biträden förintelse och karaktersdivergens. Enligt samma grundsatser se vi hvarföre de ömsesidiga slägtskapsförhållandena emellan formerna i hvarje klass äro så invecklade och måste sökas på omvägar. Vi inse hvarföre vissa karakterer äro mera än andra brukbara för klassifikationen — hvarföre adaptiva karakterer, ehuru af ofantlig vigt för varelserna, äro af knappt någon vigt för klassificeringen; hvarföre karakterer som hemtas från rudimentära organer ehuru af ingen nytta för varelserna ofta äro af högt klassifikatoriskt värde; och hvarföre embryologiska karakterer ofta äro de värdefullaste af alla. Den verkliga slägtskapen emellan alla organiska varelser i motsats till deras adaptiva likheter beror på arf eller gemensam härstamning. Det naturliga systemet är en allmän genealogisk anordning med de förvärfvade graderna af olikhet betecknade med termerna varietet, art, slägte, familj etc.; och vi måste söka härstamningslinien uti de mest permanenta karakterer, hvilka de än må vara och af huru liten vital betydelse.

Likheten i benstomme hos menniskohanden, flädermusvingen, delfinens fena och hästens ben, det lika antalet kotor i girafens och elefantens hals, och otaliga andra dylika fall förklara sig sjelfva enligt teorien om härstamning med långsam och successiv modifikation. Likheten i modell emellan flädermusens vinge och bakben, ehuru begagnade till så olika ändamål, — emellan benen och käkarna hos kräftan — emellan kronbladen, ståndarna och pistillerna i en blomma är likaledes förklarlig enligt åsigten om gradvis modifikation af delar eller organer, hvilka ursprungligen voro lika hos en ursprunglig stamfader till alla dessa klasser. Enligt grundsatsen att successiva variationer icke alltid inträffa i en tidig period och gå i arf i en motsvarande icke tidig period, inse vi tydligt, hvarföre foster af ett däggdjur eller fågel med luftrespiration hafva gälspringor och bågformiga arterer liksom fiskarna, hvilka måste andas den i vatten upplösta luften med tillhjelp af väl utvecklade gälar.

Organers overksamhet, stundom med biträde af naturligt urval har ofta reducerat organer, om de blifvit obrukbara under förändrade lifsvilkor, och vi kunna klarligen förstå enligt denna åsigt betydelsen af rudimentära organer. Men bristande användning och urval verkar i allmänhet på hvarje varelse, då den kommer till mogen ålder och spelar sin fulla rol i kampen för tillvaron, och har derföre ringa magt öfver ett organ under en tidigare lifsperiod; derföre blir intet organ reduceradt eller rudimentärt vid denna period. Kalfven har till exempel tänder, som aldrig genomskära tandköttet i öfverkäken, ärfda från en tidig stamfar som hade väl utvecklade tänder; och vi kunna tro, att tänderna hos det mogna djuret blifvit reducerade under successiva generationer genom bristande användning, eller derigenom att tungan och gommen eller läpparna blifvit lämpade genom naturligt urval till tuggning utan tändernas biträde; hvaremot hos kalfven tänderna blifvit oberörda af urvalets eller overksamhetens följder och enligt grundsatsen om ärftlighet i motsvarande ålder, hafva gått i arf från en aflägsen tid till den närvarande. Enligt åsigten att hvarje organisk varelse med alla dess särskilda delar blifvit särskildt skapad, huru ytterst oförklarligt är det icke, att organer så ofta förekomma med prägel af fullkomlig gagnlöshet, såsom tänderna hos kalffostret eller de skrynklade vingarna under de hoplödda täckvingarna hos många skalbaggar. Naturen kan sägas hafva bemödat sig att uppenbara sitt modifikationsschema genom rudimentära organer, embryologiska och homologa bildningar, men vi skola dock icke fatta detta schema.

Jag har nu rekapitulerat de fakta och betraktelser som fullkomligt öfvertygat mig att arterna blifvit modifierade under en lång serie af generationer hufvudsakligen genom det naturliga urvalet af talrika små, gynsamma variationer. Jag kan icke tro, att en falsk teori skulle så fullständigt som teorien om naturligt urval förklara de stora klasser af ofvan specificerade fakta. Det är icke någon kraftig invändning, att vetenskapen hittills icke sprider något ljus öfver det vida högre problemet, lifvets väsende och ursprung. Hvem kan förklara väsendet af attraktionen eller gravitationen? Ingen vägrar att antaga resultaten af detta okända element, attraktionen; oaktadt Leibnitz fordom anklagade Newton för att hafva infört ”dolda egenskaper och under i filosofien”.

Jag ser intet giltigt skäl hvarföre de i detta arbete framlagda åsigterna skulle stöta någons religiösa känslor. För att visa huru öfvergående sådana intryck äro är det tillräckligt att komma ihåg, att den största upptäckt af en menniska, attraktionslagen, också angreps af Leibnitz såsom omstörtande den naturliga och följaktligen äfven den uppenbarade religionen. En ryktbar andlig författare har skrifvit till mig, att ”han småningom lärt sig inse, att det är en lika ädel uppfattning af gudomligheten att tro, att han skapade några få originalformer mäktiga af utveckling till andra nödvändiga former, än att tro, att han behöfde en ny skapelseakt för att fylla de tomrum som uppkommit genom verkningarna af hans egna lagar.”

Man kan fråga, hvarföre hafva nästan alla de mest utmärkta lefvande naturhistoriker och geologer förkastat denna åsigt om arternas föränderlighet intill sista tider. Man kan icke påstå, att organiska varelser i naturtillståndet icke äro underkastade någon variation; det kan icke bevisas, att graden af variation under loppet af långa perioder är begränsad; ingen tydlig gräns har hittills blifvit dragen eller kan någonsin uppdragas emellan arter och väl markerade varieteter. Det kan icke försvaras, att arter vid kroasering oföränderligen äro sterila och varieteter oföränderligen fruktsamma, eller att ofruktsamheten är en särskild begåfning och tecken på skapelsen. Tron att arter äro oföränderliga produkter var nästan oundviklig så länge jordens historia ansågs vara af kort varaktighet; och nu då vi hafva förvärfvat oss någon föreställning om tidens längd, äro vi allt för förståndiga, att utan bevis antaga, att den geologiska urkunden är så fullständig, att den skulle lemnat oss fulla bevis på arternas förvandling, om de hafva undergått någon sådan.

Men den väsentliga orsaken till vår naturliga obenägenhet att medgifva, att en art gifvit upphof till andra och skilda arter, är den, att vi alltid äro långsamma att medgifva någon stor förändring, hvars grader vi ej se. Svårigheten är densamma som så många geologer kände, då Lyell först påstod att långa rader af fastlandsklippor blifvit bildade och stora dalar urhålkade genom omständigheter som vi ännu se i verksamhet. Själen kan omöjligen fatta betydelsen af tio millioner år, den kan icke lägga tillsamman och begripa verkningarna af många små variationer samlade under en nästan oändlig mängd generationer.

Ehuru jag är fullkomligt öfvertygad om sanningen af de i detta arbete under formen af ett utkast gifna åsigter, väntar jag mig ingalunda att kunna öfvertyga erfarna naturforskare, hvilkas kunskaper utgöras af en mängd fakta, alla under många år sedda från en synpunkt rakt motsatt min. Det är så lätt att dölja vår okunnighet bakom uttrycken ”skapelseplan”, ”ändamålets enhet” och dylika, och att anse oss gifva en förklaring, då vi blott relatera ett faktum. Den som af naturen är böjd att fästa mera vigt vid oförklarade svårigheter än vid förklaringen af ett större antal förhållanden skall helt säkert förkasta min teori. Några få mera mottagliga naturforskare, som redan begynt betvifla arternas oföränderlighet kunna röna något intryck af detta arbete, men jag ser med förtröstan in i framtiden, på unga nybildade naturforskare som kunna med opartiskhet se båda sidorna af frågan. Den som har anledning att tro på arternas föränderlighet skall göra en god tjenst med att samvetsgrant uttala sin öfvertygelse; ty blott på detta sätt kunna vi bortskaffa mängden af fördomar, som öfverväldiga detta ämne.

Några utmärkta naturhistoriker hafva nyligen uttalat sin tro, att en mängd förklarade arter i hvarje slägte icke äro verkliga arter, men att andra deremot äro verkliga arter, det vill säga hafva blifvit oberoende skapade. Denna slutsats synes mig mycket besynnerlig. De medgifva att en mängd former, som de sjelfva ända till nu ansett såsom skilda skapelser och som af flertalet naturhistoriker ännu betraktas såsom sådana och hvilka följaktligen hafva alla yttre karakteristiska drag af verkliga arter, — de medgifva att dessa uppkommit genom variationer, men de vägra att utsträcka samma åsigt till andra obetydligt olika former. Likväl påstå de icke att de kunna bestämma eller ens gissa, hvilka äro de skapade lifsformerna och hvilka uppkommit enligt sekundära lagar. De medgifva variation såsom en vera causa i ett fall, förkasta den godtyckligt i ett annat fall utan att angifva någon skilnad på dessa båda fall. Den dag skall komma, då detta skall anföras såsom en egendomlig belysning af blindheten i förutfattade meningar. Dessa författare synas icke mera förskräckta för en underbar skapelseakt än för vanlig födelse. Men tro de verkligen att vid otaliga perioder i jordens historia vissa elementära atomer hafva blifvit kommenderade att flyga tillsammans till lefvande väfnader? Tro de att vid hvarje skapelseakt en eller flera individer bildades? Voro alla oändligt talrika slag af djur och växter skapade såsom ägg eller frön, eller såsom fullväxta? och hvad däggdjuren beträffar, skapades de med de falska märkena af födelse från moderlifvet? Otvifvelaktigt kunna icke dessa samma frågor besvaras af dem som tro på uppträdandet eller skapandet af blott några få lifsformer eller af samma form allena. Flera författare hafva påstått, att det är lika lätt att tro på etthundra millioner varelsers skapelse som på en; men Maupertuis filosofiska axiom om ”den minsta verksamhet” leder menniskan mera villigt att antaga det mindre antalet; och säkert behöfva vi icke tro att otaliga varelser inom hvarje stor klass hafva blifvit skapade med fullkomliga, ehuru bedrägliga märken af härstamning från en enda stamfar.

Man kan fråga huru långt jag utsträcker läran om arternas modifikation. Frågan är svår att besvara, ty ju mera skilda formerna äro som vi betrakta, så mycket mera förlora argumenten i styrka. Men några argument af största vigt gå ännu längre. Alla medlemmarna af hela klasser äro förenade genom en slägtskapskedja och alla kunna klassificeras enligt samma grundsats i grupp under grupp. Fossila qvarlefvor sträfva stundom att fylla upp mycket stora mellanrum emellan existerande ordningar. Organer i rudimentärt tillstånd visa klart, att en tidig stamfar hade organet i fullt utbildadt tillstånd, och detta innebär i några fall en enorm grad af modifikation hos afkomlingarna. Igenom hela klasser formas särskilda bildningar efter samma mönster och i en mycket tidig period likna embryonerna hvarandra. Derföre kan jag icke betvifla att teorien om härstamning med modifikation omfattar alla medlemmar af samma klass. Jag tror att alla djur härstamma från på sin höjd fyra eller fem urformer, och växterna från ett lika eller mindre antal.

Analogien skulle föra mig ett steg längre, nämligen till det antagandet, att alla växter och djur härstamma från en prototyp. Men analogien kan vara en bedräglig ledsagare. Icke desto mindre hafva alla lefvande varelser någonting gemensamt i sin kemiska sammansättning, sin cellulära bygnad, sina utvecklingslagar och sin mottaglighet för skadliga inflytelser. Vi se detta till och med i ett så obetydligt faktum, som att samma gift ofta lika inverkar på växter och djur eller att giftet som afsöndras af galläplebildande insekter bildar vidunderliga utväxter på den vilda rosen och eken. Bland alla organiska varelser synes den sexuela fortplantningen vara väsentligt lika. Hos alla är så vidt vi hittills känna fröblåsan densamma; så att alla organismer utgå från gemensam början. Om vi nu äfven betrakta de två stora hufvudafdelningarna, växt- och djurriket äro vissa låga former så till vida intermediära i karakter, att naturhistorikerna hafva stridt om till hvilket rike de skulle höra och såsom prof. Asa Gray uttalat, ”hafva sporerna af många bland de lägre algerna i början en karakteristiskt djurisk och sedermera en otvetydigt vegetabil tillvaro”. Derföre synes det icke otroligt enligt grundsatsen om naturligt urval med karaktersdivergens, att från någon så låg och intermediär form både växter och djur hafva kunnat utvecklas, och om vi medgifva detta, måste vi likaledes medgifva, att alla organiska varelser som någonsin lefvat på jorden kunna hafva härstammat från någon ursprunglig form. Men detta antagande är hufvudsakligen grundadt på analogi och det är oväsentligt om det antages eller icke. Otvifvelaktigt är det möjligt såsom G. H. Lewes har påstått, att vid lifvets första början många olika former utbildades, men om så är förhållandet, kunna vi antaga att blott några få hafva lemnat modifierade ättlingar. Ty såsom jag nyligen anmärkt angående de andra medlemmarna af hvarje stor klass, såsom Vertebrata, Articulata etc., hafva vi tydliga bevis i deras embryologa, homologa och rudimentära bildningar, att inom hvarje klass alla härstamma från en enda stamfar.

Om de af mig i detta arbete samt af Wallace i Linnean Journal framstälda eller med dem analoga åsigter om arternas uppkomst allmänt antagas, kunna vi dunkelt förutse, att en ansenlig omhvälfning i naturalhistorien måste inträffa. Systematikerna skola blifva i stånd att fortsätta sina arbeten såsom hittills, men de skola icke oupphörligt jägtas af det dunkla tviflet om den eller den formen är en äkta art. Att detta skall vara en stor lindring tager jag för gifvet, det bjuder mig min egen erfarenhet. De ändlösa tvisterna, om de femtio britiska Rubusformerna äro eller icke äro goda arter, skola upphöra. Systematikerna skola blott hafva att afgöra (och detta är dock icke lätt) om en form är tillräckligt bestämd och konstant för att tillåta en definition, och om den är definierbar, om skiljaktigheterna äro tillräckligt vigtiga för att göra den förtjent af artnamnet. Detta sista skall vara vida väsentligare än det nu är; ty skiljaktigheter emellan två former, ehuru små, om de icke förenas genom intermediära gradationer, betraktas af de flesta naturhistoriker såsom tillräckliga för att höja båda formerna till rang af art. Hädanefter skola vi tvingas att erkänna att den enda skilnaden emellan arter och väl utpräglade varieteter är att de senare säkert eller antagligen äro förenade för det närvarande genom gradationer, hvaremot arterna fordom voro det. Derföre skola vi utan att förkasta afseendet på tillvaron i närvarande tid af öfvergångar emellan två former ledas till ett noggrannare öfvervägande och högre värderande af den verkliga graden af skilnad emellan dem. Det är väl möjligt att former, som nu allmänt erkännas vara varieteter, hädanefter kunna anses förtjena namnet art; och i detta fall skola det vetenskapliga språket och det vanliga språket komma öfverens. Med ett ord, vi skola behandla arterna på samma sätt som de naturhistoriker behandla slägtena, hvilka erkänna att slägtena äro blott artificiela kombinationer för beqvämlighets skull. Detta må nu icke vara någon behaglig utsigt, men vi skola åtminstone vara befriade från det fåfänga sökandet efter det oupptäckta och oupptäckbara väsendet af begreppet art.

De öfriga och mera allmänna afdelningarna af naturalhistorien skola vinna betydligt i intresse. De af naturhistorikerna använda termerna slägtskap, gemensam typ, morfologi, adaptiva karakterer, rudimentära och felslagna organer m. fl. skola upphöra att vara metaforiska och få sin fullständiga betydelse. Då vi icke mera betrakta en organisk varelse såsom en vilde betraktar ett skepp, såsom någonting utöfver vår fattningsförmåga, då vi vunnit den öfvertygelsen att hvarje naturprodukt har en lång historia; då vi betrakta hvarje invecklad bygnad och instinkt såsom summan af många uppfinningar, hvar och en af nytta för dess egare, på samma sätt som en stor mekanisk uppfinning är summan af många arbetares möda, erfarenhet, öfverlägning och till och med blunder, om vi från denna synpunkt betrakta hvarje organisk varelse, huru mycket mera intresse — så är åtminstone min erfarenhet — bör icke studiet af naturalhistorien erbjuda!

Ett stort och nästan obeträdt fält för forskningen är härmed öppnadt, orsakerna och lagarna för variationen, vexelverkan, verkningarna af organers bruk och bristande användning, den direkta verkan af yttre förhållanden och så vidare. Studiet af våra domesticerade alster skall vinna i värde. En ny af menniskor skapad varietet skall vara ett vigtigare och intressantare ämne för studier än en art mera till den oändliga mängden redan beskrifna arter. Våra klassificeringar skola blifva så vidt ske kan genealogier; och de skola då säkert visa, hvad man kan kalla skapelseplan. Reglorna för klassificering skola otvifvelaktigt blifva enklare om vi hafva ett bestämdt mål i sigte. Vi ega inga stamträd eller vapenböcker, och vi måste upptäcka och uppspåra de många divergerande härstamningslinierna i våra naturliga genealogier genom karakterer af hvarje slag som länge gått i arf. Rudimentära organer skola tala sanningens språk om beskaffenheten af för längesedan förlorade bildningar. Arter och artgrupper som äro kallade aberranta och hvilka med någon inbillningskraft kunna kallas lefvande fossilier skola vara oss behjelpliga vid uppgörandet af en teckning af de gamla lifsformerna. Embryologien skall ofta uppenbara för oss den mer eller mindre fördunklade bilden af prototyperna till hvarje stor klass.

Då vi blifvit öfvertygade att alla individer af samma art och alla nära beslägtade arter af de flesta slägten hafva inom en icke mycket aflägsen period utgått från en gemensam stam och flyttat ut från en födelseort, och då vi bättre känna de olika flyttningssätten, skola vi med det ljus geologien nu sprider och fortfarande skall sprida öfver fordna klimatförändringar och vexlingar i landets höjd säkerligen blifva iståndsatta att på ett beundransvärdt sätt spåra fordna flyttningar af invånarna på hela jorden. Äfven för närvarande genom jemförelse emellan hafsinvånarna på motsatta sidor af en kontinent och beskaffenheten af de olika inbyggarna på denna kontinent med afseende på deras synbara flyttningsmedel kan något ljus spridas öfver den gamla geografien.

Den ädla vetenskapen geologi förlorar något af sin glans genom ofullständigheten i dess urkund. Jordskorpan med dess inbäddade qvarlefvor kan icke betraktas såsom ett väl fyldt museum utan såsom en fattig samling gjord på slump och med långa mellantider. Hvarje stor fossilförande formations bildning visar sig hafva berott på ett ovanligt sammanträffande af gynsamma omständigheter och de blanka tomrummen emellan de successiva lagren hafva varit af lång varaktighet. Men vi skola med någon säkerhet kunna mäta längden af dessa mellantider genom att jemföra de föregående och efterföljande organiska formerna. Vi måste vara försigtiga att försöka upptaga såsom strängt samtidiga två formationer, hvilka icke innesluta många identiska arter, enligt den allmänna successionen af lifsformerna. Då arter bildas och förstöras genom långsamt verkande och ännu existerande orsaker och icke genom underbara skapelseakter och katastrofer, och då den vigtigaste af alla orsaker till organisk förändring är nästan oberoende af förändrade och möjligen hastigt förändrade fysiska vilkor, nämligen den ömsesidiga relationen emellan organismer — i det förädling af en organism bestämmer andras förädling eller utrotning — så följer, att graden af organisk förändring af fossilierna från en formation till en annan sannolikt tjenar såsom en god måttstock på tidens verkliga lopp. Ett antal arter som hålla sig tillsammans i massa kunna för en lång tid blifva oförändrade, under det inom samma period flera af dessa arter genom att flytta i nya länder och komma i täflan med främmande medtäflare kunna blifva förändrade, så att vi icke få öfverskatta noggranheten af de organiska förändringarna såsom tidmätare. Under tidiga perioder af jordens historia, då lifsformerna sannolikt voro färre och enklare var förändringens fortgång sannolikt långsammare, och vid lifvets första gryning, då mycket få former af den enklaste bildning existerade kan förändringens fortgång hafva varit långsam till en ytterlig grad. Jordens historia, ehuru så vidt den nu är känd af omätlig längd, skall hädanefter anses som kort i jemförelse med de tider som måste hafva förflutit sedan de första organiska varelserna, stamfäder till otaliga utdöda och lefvande ättlingar, framträdde på skådebanan.

I den aflägsna framtiden ser jag fält öppna för vigtigare forskningar. Psykologien skall hvila på en ny grund, hvarje själsförmögenhets nödvändiga förvärfvande genom grader. Ljus skall äfven spridas öfver menniskans ursprung och hennes historia.

Skrifställare af första rangen synas vara fullt belåtna med den åsigten att hvarje art blifvit oberoende skapad. För mig öfverensstämmer det bättre med hvad vi känna om de af Skaparen åt materien gifna lagarna, att bildandet och förstörandet af de förgångna och närvarande inbyggarna på jorden bero på sekundära orsaker likartade med dem som bestämma individers födelse och död. Då jag icke betraktar alla varelser såsom skilda skapelser, utan såsom ättlingar i rätt nedstigande linie af några få varelser som lefde långt innan de första siluriska lagren afsattes, synas de mig blifvit adlade. Att döma efter det förflutna kunna vi antaga, att icke en af de nu lefvande arterna skall öfverlemna sin oförändrade afbild till en aflägsen framtid. Och af de nu lefvande arterna skola blott få lemna ättlingar af något slag in i en långt aflägsen framtid, ty det sätt hvarpå alla organiska varelser äro grupperade visar att det större antalet arter i hvarje slägte och alla arterna i många slägten hafva icke lemnat några efterkommande utan blifvit helt och hållet utrotade. Vi kunna kasta en profetisk blick in i framtiden, så mycket att vi kunna förutsäga, att de allmänna och vidt spridda arterna af de större och dominerande grupperna i hvarje klass skola slutligen blifva de herskande och skapa nya dominerande arter. Då alla lefvande organiska former äro ättlingar i rätt nedstigande linie från dem som lefde långt före den siluriska tiden, kunna vi vara säkra, att den vanliga generationsföljden aldrig blifvit bruten och att ingen syndaflod har ödelagt hela jorden. Derföre kunna vi med ett visst förtroende se på en säker framtid af lika oberäknelig längd. Och då det naturliga urvalet arbetar blott i och för hvarje varelses bästa, skola alla kroppsliga och intellektuela egenskaper fortgå emot fulländning.

Det är angenämt att betrakta ett stycke land, beklädt med många växter af många slag, med fåglar sjungande i buskarna, med kringfladdrande insekter, med maskar krälande i den fuktiga jorden, och att tänka sig att dessa så omsorgsfullt utarbetade former, så olika hvarandra och beroende af hvarandra på ett så inveckladt sätt, att de alla äro bildade af lagar som ännu äro verksamma omkring oss. Dessa lagar i vidsträcktaste bemärkelse heta: utveckling och fortplantning; ärftlighet som nästan innefattas i fortplantningen; föränderlighet i följd af lifsvilkorens indirekta och direkta verkan och af hvila och verksamhet, en förökning i sådan proportion att den leder till en kamp för tillvaron och såsom en följd deraf till naturligt urval i förening med karaktersdivergens och förstörande af mindre förädlade former. Sålunda följer af naturens krig, af hunger och död det mest upphöjda ämne vi äro i stånd att fatta, bildandet af de högre djuren. Det ligger storhet i denna åsigt, att lifvet med dess olika förmögenheter af Skaparen ursprungligen blifvit inblåst i några få former eller blott en enda; och att, under det denna planet har fortsatt sitt kretslopp enligt gravitationens oföränderliga lag, från en så enkel början utvecklats och alltjemt bildas otaliga de skönaste och mest underbara former.
