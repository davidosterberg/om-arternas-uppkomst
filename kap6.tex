%SJETTE KAPITLET.





\chapter{Svårigheter för teorien.}

{\it
Svårigheter för teorien om härstamning med modifikation. — Öfvergångar. — Brist på öfvergångsvarieteter eller deras sällsynthet. — Öfvergångar i lefnadsvanor. — Olika vanor hos samma art. — Arter med vanor, som äro vidt skilda från deras slägtingars. — Organer af den yttersta fullkomlighet. — Medel till öfvergångar. — Svåra fall. — Natura non facit saltum. — Organer af ringa vigt. — Organer, som icke i alla fall äro absolut fullkomliga. — Lagen om typens enhet och existensvilkoren innehållen i teorien om det naturliga urvalet.
}\\[0.5cm]

Långt innan läsaren hunnit till denna del af mitt arbete, torde en mängd svårigheter hafva mött honom. Några af dem äro af så allvarsam beskaffenhet, att jag ännu i dag knappt kan tänka på dem utan att blifva tvehågsen, men så vidt jag kan döma äro de flesta af dem blott skenbara, och de som äro verkliga kullkasta som jag tror icke min teori.

Dessa svårigheter och inkast kunna ordnas under följande rubriker: för det första, om arter hafva uppstått ur andra arter genom omärkligt fina modifikationer, hvarföre se vi icke öfverallt otaliga öfvergångsformer? Hvarföre är icke hela naturen ett virrvarr af former i stället för en samling väl skilda arter såsom vi i verkligheten se?

För det andra: är det möjligt, att ett djur med till exempel flädermusens skapnad och lefnadssätt skulle kunnat bildas genom modifikation af ett annat djur med helt olika lefnadssätt och kroppsbildning? Kunna vi tro, att det naturliga urvalet skulle å ena sidan skapa organer af så ringa vigt, som till exempel girafens svans, som blott tjenar till flugsmälla, och å andra sidan organer af så underbar bygnad som ögat, hvars oförlikneliga fulländning vi ännu i dag blott ofullständigt begripa?
För det tredje: kan instinkten förvärfvas och modifieras genom det naturliga urvalet? Hvad hafva vi att säga om en sådan instinkt, som drifver bien att bygga celler, hvarigenom de hafva gått djupsinniga matematikers upptäckter i förväg?

För det fjerde: huru skola vi förklara, att arter, då de kroaseras, äro sterila eller alstra en ofruktsam afkomma, under det varieteter vid kroasering visa en oförminskad fruktsamhet?

De båda första afdelningarna skola behandlas i detta kapitel, — åt instinkten och bastardbildningen egna vi särskilda kapitel.



\section[Bristen på öfvergångsvarieteter]{Bristen på öfvergångsvarieteter eller deras
sällsynthet.}

Då det naturliga urvalet verkar blott genom bibehållandet af gynsamma modifikationer, så sträfvar hvarje ny form i ett fullt befolkadt område att intaga den plats, som förut innehades af dess egen mindre förädlade stamform och andra mindre väl utrustade former med hvilka den kommer i beröring, och till slut undanträngas dessa helt och hållet. Det naturliga urvalet går således hand i hand med denna utrotning. Om vi derföre betrakta hvarje art såsom härstammande från någon okänd stamform, böra i allmänhet både stamfadern och alla öfvergångsvarieteter hafva blifvit utrotade redan genom den nya formens bildnings- och fulländningsprocess.

Men då enligt denna teori oräkneliga bildningsformer måste hafva existerat, hvarföre finna vi dem icke inbäddade i oändlig mängd i jordskorpan? Det torde vara mera lämpligt att behandla denna fråga i kapitlet om våra geologiska urkunders ofullständighet och här vill jag blott anföra, att svaret ligger, såsom jag tror, deruti, att geologiens urkunder äro vida mindre fullständiga än vanligen antages. Jordskorpan är ett vidsträckt museum, men naturens samlingar äro ofullständiga och gjorda blott med långa mellantider.

Man kan invända, att om flera närslägtade arter bebo samma område, vi säkerligen skulle i närvarande tid finna många öfvergångsformer. Låt oss taga ett enkelt fall; om vi resa öfver en kontinent från norr till söder, finna vi i allmänhet tid efter annan närbeslägtade eller hvarandra ersättande arter, hvilka uppenbarligen intaga samma plats i landets naturhushållning. Dessa representativa arter gränsa ofta intill hvarandra och gripa in på hvarandras område, och i samma mån som den ena blir sällsyntare och sällsyntare, i samma mån blir den andra allt mera allmän, till dess den senare helt och hållet ersätter den andra. Om vi jemföra dessa arter med hvarandra å de ställen, der de äro blandade, äro de i allmänhet lika väl skilda från hvarandra i hvarje del af sin organisation, som exemplar tagna ifrån hvarderas egentliga område. Enligt min teori härstamma dessa beslägtade arter ifrån en gemensam stamfader och under modifikationsprocessen har hvar och en blifvit lämpad efter lefnadsförhållandena i dess eget område och har ersatt och utrotat sin ursprungliga stamform och alla öfvergångsvarieteter emellan dess fordna och närvarande tillstånd. Derföre kunna vi icke vänta att i närvarande tid påträffa talrika öfvergångsformer i hvarje trakt, ehuru de måste hafva funnits der och kanhända äro inbäddade i fossilt tillstånd. Men hvarföre finna vi icke närstående mellanformer i det närliggande området, der lefnadsförhållandena bilda en öfvergång från det ena områdets till det andras? Detta inkast förbryllade mig för en lång tid, men jag tror, att det till en stor del kan förklaras.

Framförallt skola vi vara ytterst försigtiga i det antagandet, att en trakt har varit sammanhängande under en lång period, derföre att den nu är det. Geologien gifver oss anledning att tro, att de flesta fastland hafva varit afdelade i öar äfven under den senare tertiärperioden; och på sådana öar kunna skilda arter hafva bildats utan möjligheten att lemna mellanformer i de mellanliggande områdena. Genom förändringar i landets form och klimat måste också de sammanhängande hafven ofta i nyare tider hafva varit mindre sammanhängande och likformiga än nu. Dock vill jag icke på detta sätt undvika svårigheten, ty jag tror, att många fullkomligt skilda arter hafva bildats på helt och hållet sammanhängande områden, ehuru jag icke tviflar att de nu sammanhängande områdenas fordna delning har spelat en vigtig rol vid bildandet af nya arter, särskildt med afseende på vandrande djur, som fritt kroasera sig.

Om vi betrakta arterna såsom de nu äro fördelade öfver en vidsträckt yta, finna vi dem i allmänhet temligen talrika öfver ett stort område, derefter blifva de hastigt allt mer och mer sällsynta mot gränserna och försvinna slutligen helt och hållet. Det neutrala området emellan två representerande arter är derföre vanligen litet i jemförelse med de af hvardera bebodda distrikterna. Samma förhållande finna vi, då vi bestiga ett berg och ofta är det särdeles anmärkningsvärdt, huru hastigt en allmän fjellväxt försvinner, såsom Alphonse de Candolle iakttagit. E. Forbes har anmärkt samma förhållande vid draggningar på hafsbotten. För dem som betrakta klimatet och de fysiska lifsvilkoren såsom de enda bestämmande för fördelningen torde dessa förhållanden vara särdeles öfverraskande, då klimat och höjd eller djup blott omärkligt och småningom förändras. Men om vi komma ihåg, att nästan hvarje art äfven i sitt egentliga hemland skulle förökas till oändligt antal, om den icke vore i oupphörlig strid med andra arter, att nästan alla arter antingen lefva af andra eller tjena andra till föda, med ett ord, att hvarje organisk varelse antingen direkt eller indirekt är beroende af andra organiska varelser, så måste vi inse, att invånarnas utbredning på ett område ingalunda uteslutande beror på omärkliga förändringar i de fysiska förhållandena, utan till en stor del på närvaron af andra arter, på hvilka de lefva eller af hvilka de förtäras eller med hvilka de komma i täflan; och då dessa arter redan hafva en bestämd begränsning och icke mera omärkligt öfvergå i hvarandra, så måste utbredningen af en art blifva skarpt begränsad. Dessutom måste hvarje art på gränserna af dess utbredningsområde, der den finnes i minskadt antal, under förändringar i mängden af fiender eller byte eller på vissa årstider vara utsatt för en hög grad af förstöring, och detta bör ännu mera skarpt begränsa dess geografiska utbredning.

Om min åsigt är riktig, att beslägtade eller representerande arter, då de bebo en sammanhängande yta, äro så fördelade, att hvardera har ett vidsträckt utbredningsområde med ett jemförelsevis smalt neutralt område emellan dem, i hvilket de hastigt blifva allt mer och mer sällsynta, så bör samma regel kunna tillämpas på varieteter, då dessa icke väsentligt skilja sig från arter; och om vi tänka oss en föränderlig art beboende en mycket vidsträckt yta, så böra vi äfven antaga tvänne varieteter, passande för två stora områden, och en tredje för en liten mellanliggande zon. Denna mellanvarietet bör följaktligen finnas i mindre antal, då den bebor ett trängre och mindre område; och så vidt jag har kunnat finna, gäller denna regel för varieteter i naturtillståndet. Jag har påträffat slående exempel på denna regel hos varieteter, hvilka utgjort mellanformer emellan andra utpräglade varieteter af Balanus-arter. Och enligt meddelanden af mr Watson, doktor Asa Gray och mr Wollaston tyckas varieteter som bilda mellanformer emellan två andra former vara numeriskt sällsyntare än de former, hvilka de förena. Om vi nu kunna för riktiga antaga dessa fakta och bevis och deraf draga den slutsatsen, att varieteter som sammanbinda tvänne andra i allmänhet hafva funnits i ringare antal än de former, som de förena, så kunna vi i min tanke förklara, hvarföre mellanformer icke kunna ega bestånd under längre perioder, hvarföre såsom en allmän regel de skulle utrotas och försvinna förr än de former, hvilka de ursprungligen förenade.

Ty en form som finnes i ringa antal löper, såsom vi hafva sett, större fara att utrotas än en form som finnes i större mängd, och i detta särskilda fall torde mellanformen vara särdeles utsatt för fiendtliga infall af de närbeslägtade formerna å båda sidor. Men en mycket vigtigare betraktelse är såsom jag tror, att under den föregående modifikationsprocessen, genom hvilken tvänne varieteter antagas förvandlas till två skilda arter, de två som finnas i större antal och bebo ett större område hafva en stor fördel öfver mellanformen, som i ringa antal bebor ett trångt mellanliggande område. Ty former som finnas i mängd hafva alltid en större sannolikhet att under en gifven period lemna fördelaktiga variationer för det naturliga urvalet, än de sällsyntare former som finnas i mindre antal. I kampen för tillvaron skola derföre de allmänna formerna sträfva att undantränga och ersätta de mindre allmänna, ty dessa modifieras och förädlas blott långsamt. Det är samma grundsats, enligt hvilken, såsom vi hafva sett i andra kapitlet, de allmänna arterna öfverhufvudtaget förete ett större antal väl utpräglade varieteter än de sällsyntare. För att förklara min mening vill jag antaga tre fårvarieteter, en passande för vidsträckta bergstrakter, en annan passande för en jemförelsevis liten backig trakt och en tredje lämplig för vidsträckta områden vid kullarnas fot, och jag vill vidare antaga, att invånarna försöka att genom urval förädla sina stammar med lika mycken ihärdighet och skicklighet. Sannolikheten för ett godt resultat är vida större för de större hjordegarna på bergen eller på slätterna, emedan de kunna förädla sina raser vida hastigare än de mindre hjordegarna i den mellanliggande backiga trakten, och följaktligen bör bergrasen eller slättrasen snart undantränga den mindre förädlade rasen i det mellanliggande området, och de två raser, som ursprungligen funnos i större antal, skola komma i nära beröring med hvarandra, utan att vidare vara skilda genom den undanträngda varieteten.

Med ett ord, jag tror, att arterna kunna vara väl begränsade utan att på någon tid förete ett oupplösligt kaos af varierande mellanformer: 1) emedan nya varieteter bildas mycket långsamt, ty variation är en mycket långsam process och det naturliga urvalet kan icke uträtta någonting så vida icke fördelaktiga individuela olikheter eller variationer uppträda och så vida icke någon plats i traktens naturhushållning kan bättre fyllas genom modifikation af någon eller flera af dess invånare. Och sådana nya platser bero på långsamma förändringar i klimat, eller på tillfälliga inflyttningar af nya invånare och sannolikt i ännu högre grad derpå, att några af de gamla invånarna småningom blifva modifierade, hvarvid de sålunda bildade nya formerna och de gamla ömsesidigt inverka på hvarandra. Derföre böra vi i hvarje trakt och på hvarje tid se blott några få arter, som förete ringa till en viss grad permanenta modifikationer i kroppsbildning; och detta är helt visst förhållandet.

För det andra måste många nu sammanhängande ytor inom en ny period hafva varit afdelade i skilda portioner, i hvilka många former, särskildt bland de arter som para sig för hvarje befruktning och vandra mycket, kunna hafva blifvit tillräckligt skilda för att anses såsom representerande arter. I detta fall måste mellanformer emellan de olika representerande arterna och deras gemensamma stamfader en gång hafva existerat inom hvarje afskild portion af landet, men dessa länkar hafva under det naturliga urvalets fortgång blifvit ersatta och utrotade, så att de icke längre finnas i lifvet.

För det tredje, om två eller flera varieteter hafva bildats i skilda delar af en fullkomligt sammanhängande yta, så böra mellanformer sannolikt först hafva uppkommit i de mellanliggande områdena, men de böra i allmänhet hafva haft en kort varaktighet. Ty af redan angifna skäl (nämligen af hvad vi veta om beslägtade eller representerande arters äfvensom erkända varieteters verkliga utbredning) måste dessa mellanformer förekomma i ringare antal än de varieteter som de förena. Redan af denna orsak böra mellanformerna vara utsatta för tillfälliga förödelser, och under det naturliga urvalets fortgående modifikationsprocess böra de nästan säkert undanträngas och ersättas af de former de sammanbinda; ty då dessa förekomma i större antal skola de lemna flera varieteter och sålunda blifva vidare förädlade af det naturliga urvalet och vinna större fördelar.

Slutligen måste, icke på en viss tid utan alltid, om min teori är sann, helt säkert tallösa varieteter hafva funnits sammanbindande alla arter af samma grupp, men det naturliga urvalet sträfvar oupphörligen, såsom ofta nämts, att utplåna stamformerna och mellanformerna. Bevis på deras fordna tillvaro kan således finnas blott ibland fossila qvarlefvor, hvilka, såsom vi skola försöka visa i ett följande kapitel, äro bevarade i en ytterligt ofullständig och osammanhängande urkund.

\section[Om uppkomsten av särskilda vanor]{Om uppkomsten af organiska varelser med särskilda
vanor och kroppsbildning och öfvergångar
dem emellan.}

Motståndare till mina åsigter hafva frågat, huru till exempel ett landrofdjur har kunnat förvandlas till ett hafsrofdjur, ty huru skulle djuret kunna ega bestånd i något öfvergångsstadium? Det vore lätt att visa, att inom samma grupp rofdjur förekomma, som intaga hvarje stadium emellan äkta landdjur och vattendjur, och då hvar och en består genom kampen för tillvaron, är det klart, att hvar och en är i sina vanor väl passande för sin plats i naturen. Så har till exempel Mustela vison i Nordamerika simhud på fötterna och liknar en utter i pelsen, de korta benen och svansens form; hela sommaren dyker djuret i vatten och lefver af fisk; men under den långa vintern lemnar det det frusna vattnet och lefver af råttor och landdjur liksom andra vesslor. Om man hade valt ett annat fall och frågat huru en insektätare blifvit förvandlad till en flygande flädermus, då hade frågan varit vida svårare att besvara. Dock tror jag, att sådana invändningar hafva föga betydelse.

Här liksom vid andra tillfällen lider jag af en stor olägenhet, nämligen att jag af de många öfvertygande fall jag har samlat blott kan gifva ett eller två exempel på öfvergångar i lefnadsvanor och kroppsbildning hos närslägtade arter af samma slägte och af öfvergående eller beständiga olikheter i vanor hos samma art. Och det synes mig, att ingenting annat än en lång lista af sådana fall är tillräckligt att minska svårigheten i vissa enstaka fall, såsom det ofvan anförda angående flädermusen.

Om vi betrakta ekorrarnas familj, så finna vi der de finaste öfvergångar från djur med svagt utbredda svansar och från andra, som enligt J. Richardsons iakttagelse hafva bakre delen af kroppen något bredare och huden starkare utvecklad, ända till de så kallade flygande ekorrarna. Dessa hafva extremiteterna och början af svansen förenade genom en utbredning af huden, som tjenar som en fallskärm och sätter dem i stånd att sväfva i luften från träd till träd på förvånande afstånd. Vi kunna icke betvifla, att hvarje bildning är af nytta för hvarje slags ekorre i dess eget område genom att sätta honom i stånd att undfly fåglar eller rofdjur, eller att lättare samla föda, eller såsom vi hafva skäl att tro genom att minska faran för tillfälliga fall. Men deraf följer icke att hvarje ekorres bildning är den bästa han kan antaga under alla naturliga förhållanden. Låt klimat och växtlighet förändras, låt andra täflande gnagare eller nya rofdjur inflytta eller gamla modifieras, så skall all analogi föra oss på den tanken, att åtminstone några af ekorrarna skola aftaga i antal eller blifva utrotade, så vida de icke också blifva modifierade och förädlade på motsvarande sätt. Derföre kan jag icke se någon svårighet för det antagandet, att i synnerhet under förändrade lefnadsförhållanden individer med allt starkare och starkare sidohud bibehållits, då denna karakter varit nyttig och gått i arf, till dess genom det naturliga urvalets accumulativa verkningar en fullständig så kallad flygande ekorre bildats.

Vi skola nu taga i betraktande Galeopithecus eller den flygande Lemur, som fordom räknats till flädermössen. Den har en utomordentligt vidsträckt sidohud, som sträcker sig från underkäkens vinkel ända till svansen och omfattar extremiteterna och de förlängda fingrarna; sidohuden är äfven försedd med en sträckmuskel. Ehuru numera inga former finnas, som sammanbinda den flygande Galeopithecus med de öfriga medlemmarna af Lemuridernas familj, så är det icke någon svårighet att antaga, att sådana länkar fordom funnits, och att hvar och en blifvit bildad på samma sätt genom gradvisa öfvergångar, som de mindre fulländade flygande ekorrarna, och att hvarje bildning varit af någon nytta för sin egare. Ej heller kan jag se någon oöfvervinnelig svårighet att tro, att den hinna, som sammanbinder fingrarna och underarmen hos Galeopithecus, småningom kan blifva förstorad genom det naturliga urvalet och på detta sätt hvad flygorganerna beträffar förvandla honom till en flädermus. Hos vissa flädermöss, hos hvilka vinghinnan är utsträckt från skuldran ända till svansen och omfattar bakbenen, se vi måhända ännu verkliga spår af en apparat, som ursprungligen var mera egnad för glidning genom luften än för verklig flygt.

Om ungefär ett dussin fågelslägten hade dött ut eller voro okända, hvem skulle då våga misstänka, att fåglar finnas till, hvilka begagna sina vingar blott såsom åror liksom Micropterus (Eyton), eller såsom fenor i vatten och såsom framben i land, liksom pingvinen, eller såsom segel liksom strutsen, eller slutligen hvilkas vingar alls icke göra tjenst såsom Apteryx? Dock är hvar och en af dessa fåglars skapnad fördelaktig för honom i de lefnadsförhållanden under hvilka den lefver, ty hvar och en lefver i en oupphörlig kamp, men den är derföre icke den bästa möjliga under alla möjliga lefnadsförhållanden. Från dessa anmärkningar få vi icke sluta till, att några af de ofvan anförda vingformerna verkligen angifva de grader, genom hvilka fåglarna hafva vunnit sin fullkomliga förmåga att flyga, ty de kunna bero på bristande användning af vingarna, men de kunna åtminstone visa, hvilka olika öfvergångssätt äro möjliga.

Då vi se, att ett ringa antal af de djur, som lefva och andas i vatten, såsom molluscer och crustaceer, är i stånd att lefva på land, och då vi se, att vi hafva flygande fåglar, flygande däggdjur, flygande insekter af de mest olikartade typer, att vi fordom haft flygande reptilier, så kunna vi äfven tänka oss, att de flygfiskar, som nu sväfva genom luften, obetydligt höjande sig med tillhjelp af fladdrande fenor, hafva blifvit modifierade till fullständigt bevingade djur. Om detta hade inträffat, hvem skulle då inbillat sig, att de i ett tidigt öfvergångsstadium hade bebott den öppna oceanen och hade begagnat sina outvecklade flygorganer, såsom nu är fallet, blott till att undgå andra fiskars käftar?

Om vi se något organ högeligen utbildadt till ett visst ändamål såsom fågelns vingar, få vi komma ihåg, att djur i de tidigare öfvergångsstadierna till en sådan bildning sällan finnas till i närvarande tid, ty de hafva blifvit undanträngda af sina efterträdare, som småningom blifvit mera fulländade genom det naturliga urvalet. Vidare kunna vi antaga, att öfvergångsstadier emellan organer lämpade för mycket olika lefnadsvanor sällan hafva utvecklats i en tidig period i stort antal och under många underordnade former. För att återgå till vårt antagna exempel flygfiskarna, synes det sannolikt, att verkligt flygande fiskar i många former utvecklats för att förfölja många slags byte på många vägar, både till lands och sjös, till dess deras flygorganer hunnit en så hög grad af fulländning, att de fått en afgjord öfvervigt öfver andra djur i kampen för tillvaron. Möjligheten att i fossilt tillstånd upptäcka arter med öfvergångsformer är derföre alltid liten, då de hafva funnits i ringare antal än arterna med fullt utbildade organer.

Jag vill nu gifva två eller tre exempel på förändrade vanor hos individerna af samma art. I hvarje fall är det lätt för det naturliga urvalet att göra ett djurs skapnad lämplig för dess förändrade lefnadssätt, eller särskildt för en af dess vanor. Det är likväl svårt och för oss af ringa vigt att afgöra, antingen lefnadssättet förändras först och organisationen sedan, eller om små modifikationer i organisationen leda till förändring i lefnadssättet; sannolikt förändras båda samtidigt. Bland fall af förändradt lefnadssätt vill jag blott anföra, att många af de britiska insekterna nu lefva på utländska växter eller på konstprodukter. Många exempel kunde anföras på föränderliga lefnadsvanor: jag har ofta sett en flugsnappare i Sydamerika (Saurophagus sulphuratus) sväfvande lik en tornfalk öfver en fläck och sedan öfver en annan, och vid andra tillfällen har jag sett honom stå stilla vid vattenranden och plötsligt dyka ned i vattnet lik en kungsfiskare (Alcedo). I våra egna trakter se vi ofta talgoxen (Parus major) klättra i träd lik en trädkrypare (Certhia); stundom dödar den små fåglar genom slag i hufvudet liksom törnskatan (Lanius) och jag har ofta hört och sett den bulta fröen af idegran emot en gren och på detta sätt bräcka dem likasom nötkrakan (Nucifraga caryocatactes). I Nordamerika såg Hearne ofta den svarta björnen i flera timmars tid simma omkring i vattnet med vidöppen mun för att fånga vatteninsekter likasom en hval.

Då vi ofta se individer af en art föra ett lefnadssätt helt olika andra individers af samma art och andra arters af samma slägte, kunna vi vänta, att sådana individer skola gifva upphof till nya arter med andra lefnadsvanor och med en mer eller mindre ansenligt modifierad kroppsbildning, olik deras ursprungliga typ. Och sådana fall påträffa vi i naturen. Kan ett mera slående exempel anföras på ändamålsenligheten i ett djurs organisation än hackspettens skapnad, som sätter honom så väl i stånd att klättra i träd och gripa insekter i barkens remnor? Dock finnas i Nordamerika hackspettar som lefva hufvudsakligen af frukter och andra med långa vingar, som jaga insekter i flygten. På La Platas slätter, der intet träd växer, finnes en hackspett (Colaptes campestris), som har två tår riktade framåt och två bakåt, en lång spetsig tunga, spetsiga stjertfjädrar, tillräckligt styfva att hålla fågeln i vertikal riktning, ehuru ej så styfva som hos de typiska hackspettarna, och en rak, stark näbb. Näbben är likväl icke så rak eller stark som hos de typiska hackspettarna, men den är stark nog att borra i träd. Denna Colaptes är således i alla väsentliga delar af sin organisation en hackspett. Äfven sådana obetydliga karakterer som färgen, den sträfva tonen och den vågformiga flygten, allt tillkännagifver dess nära slägtskap med vår vanliga hackspett; dock kan jag försäkra, icke blott efter mina egna, utan äfven efter den samvetsgranna Azaras iakttagelser, att den aldrig klättrar i träd. Såsom ett annat exempel på olikartade vanor inom samma familj kan anföras, att de Saussure har beskrifvit en mexikansk Colaptes, som borrar hål i hårda träd för att lägga upp ett förråd af ekollon, till hvad ändamål är okändt.
Stormfåglarna äro bland alla fåglar de som mest flyga och äro mest bundna vid hafvet, men i de lugna sunden vid Eldslandet skulle en art, Puffinaria Berardi, i sina vanor, i sin märkvärdiga förmåga att dyka, i sitt sätt att simma och flyga, då den nödgas dertill, kunna tagas för en tordmule (Alca) eller en lom (Colymbus); icke desto mindre är den i sina väsentliga delar en stormfågel, men med många delar af sin organisation i hög grad modifierade efter dess nya lefnadssätt, hvaremot hackspetten i La Plata hade undergått blott obetydliga modifikationer. Den noggrannaste undersökning af strömstarens (Cinclus) döda kropp skulle aldrig hafva kommit någon att ana dess vid vattnet bundna lefnadssätt, och dock förskaffar sig denna till trastfamiljen hörande fågel hela sitt underhåll genom dykning, genom att begagna sina vingar under vatten och plocka undan stenar med sina fötter. Alla lemmarna i den stora ordningen Hymenoptera bland insekterna äro landdjur med undantag af slägtet Proctotrupes, hvilket enligt Sir John Lubbocks nya iakttagelser är vattendjur i sitt lefnadssätt; den går ofta ned i vatten och dyker icke med benens utan vingarnas tillhjelp och stannar ända till fyra timmar under ytan; dock kan icke den minsta modifikation i skapnad upptäckas i öfverensstämmelse med sådana abnorma vanor.

Den som tror, att hvarje varelse har blifvit skapad sådan som vi nu se den, måste någon gång känt sig öfverraskad, då han påträffat ett djur, hvars lefnadssätt och kroppsbildning icke stått i öfverensstämmelse med hvarandra. Hvad kan vara tydligare, än att anden och gåsen blifvit skapade med simhud emellan tårna för att simma? Dock finnas gäss med simhud på fötterna, hvilka sällan eller aldrig gå i vattnet, och utom Audubon har ingen sett fregattfågeln (Tachypetes) sänka sig ned på hafsytan, ehuru den har alla sina fyra tår förenade med simhud. Å andra sidan äro doppingarna (Podiceps) och sothönan (Fulica) afgjordt vattendjur, ehuru deras tår blott äro försedda med hudflikar. Hvad synes vara antagligare än att vadarnas långa tår utan simhud äro dem gifna för att gå öfver sumpig mark och flytande vattenväxter, och dock är sumphönan (Gallinula chloropus) nästan lika mycket vattendjur som sothönan (Fulica), och ängsknarren (Crex pratensis) deremot lika mycket landdjur som vakteln eller rapphönan. I sådana fall har lefnadssättet förändrats utan motsvarande förändring i skapnad; gåsens simhud kan sägas vara rudimentär till funktion men icke till form. Hos fregattfågeln deremot har den djupt klufna hinnan mellan tårna redan börjat att förändras.
Den som tror på särskilda och oräkneliga skapelseakter kan säga, att i dessa fall det har behagat skaparen att bestämma en varelse af en typ för den plats, som en varelse af annan typ borde intaga, men detta synes mig blott vara ett upprepande af sjelfva saken i andra ordalag. Den som tror på kampen för tillvaron och på det naturliga urvalet skall erkänna, att hvarje organisk varelse oupphörligen sträfvar att tilltaga i antal, och att, om någon varierar aldrig så litet vare sig i kroppsbildning eller lefnadsvanor och på detta sätt vinner någon fördel öfver någon annan invånare i samma trakt, att denna varelse skall inkräkta på den andras plats, huru olika den än må vara dess egen. Det skall derföre icke öfverraska honom att finna gäss och fregattfåglar med simhud på fötterna, hvilka lefva på torra landet och blott sällan stiga i vattnet, eller att finna en med långa tår försedd ängsknarr, som lefver på ängar i stället för i träsk, eller att finna en hackspett der intet träd växer, eller att finna dykande trastar och dykande Hymenoptera och stormfåglar med doppingars lefnadsvanor.



\section[Väl utvecklade organer]{Väl utvecklade och sammansatta organer.}

Jag erkänner öppet, att det synes i högsta grad orimligt att antaga, att ögat med alla dess oefterhärmliga inrättningar för accomodation, för moderering af ljusstyrkan och för korrektion af den sferiska och kromatiska aberrationen skulle hafva bildats genom naturligt urval. Då man först uttalade den satsen, att solen står stilla och att jorden vänder sig omkring sin axel, förklarade hela menniskoslägtets sunda förstånd, att denna sats var falsk, men det gamla ordspråket vox populi, vox dei är i vetenskapen intet gällande. Mitt förnuft säger mig likväl, att om talrika öfvergångsformer kunna visas ifrån ett ofullkomligt och enkelt öga till ett fullkomligt och sammansatt och att hvarje form är nyttig för sin egare, hvilket helt säkert låter sig göra; om vidare ögat obetydligt varierar och variationerna gå i arf, hvilket äfven är förhållandet, och om sådana variationer kunde vara nyttiga för något djur under föränderliga lefnadsvilkor, att i sådant fall någon verklig svårighet icke finnes att tro, att ett fullkomligt och sammansatt öga kan bildas genom det naturliga urvalet. Huru en nerv kan vara känslig för ljus, det bekymrar oss icke mer än lifvets första ursprung, men då några af de lägsta organismerna, hos hvilka inga nerver kunna upptäckas, likväl äro känsliga för ljus, synes det icke omöjligt, att vissa element i sarkoden, hvaraf de hufvudsakligen bestå, kunna utbildas till nerver begåfvade med denna särskilda känslighet.

Då vi söka de grader, hvilka ett organ under sin fulländning genomgått, böra vi taga i betraktande uteslutande dess stamfäder i rät linie; men detta är knappt någonsin möjligt och vi tvingas att fästa oss vid andra arter och slägten af samma grupp, det är vid sidoslägtingarna, som härstamma från samma stamform, för att se, hvilka grader äro möjliga och om några lägre former hafva gått i arf i oförändradt eller föga förändradt tillstånd. Men organets utbildning i skilda klasser kan äfven sprida något ljus öfver de stadier, genom hvilka det har blifvit fulländadt hos en viss art.

Det enklaste organ som kan kallas ett öga består af en synnerv omgifven af pigmentceller och betäckt med genomskinlig hud, men utan lins eller någon ljusbrytande kropp. Enligt M. Jourdain kunna vi likväl gå ett steg djupare och finna samlingar af pigmentceller uppenbarligen tjenande såsom synorgan, men utan någon nerv och hvilande helt enkelt i sarkodmassan. Ögon af sådan enkel bygnad äro icke mäktiga af bestämdt seende och tjena blott att skilja ljus från mörker. Hos vissa sjöstjernor finnas små intryckningar i pigmentlagret omkring synnerven, hvilka äro fylda med en genomskinlig gelatinös massa med en framskjutande konvex yta likasom hornhinnan hos de högre djuren. Den ofvan anförde författaren, som har lemnat dessa uppgifter, antager att denna anordning icke tjenar till att åstadkomma en bild utan blott att koncentrera ljusstrålarna och göra ljusförnimmelsen så mycket lättare. I denna ljusstrålarnas koncentrering finna vi det första och aldra vigtigaste steget framåt till ett verkligt öga; ty vi behöfva blott sätta synnervens nakna ända, som hos några de lägre djuren ligger djupt dold inuti kroppen och hos andra nära dess yta, på behörigt afstånd från den koncentrerande apparaten, och en verklig bild aftecknas på synnerven.

I leddjurens stora klass kunna vi utgå från en synnerv helt enkelt beklädd med pigment, som stundom bildar en pupill, men utan lins eller annan optisk inrättning. Angående insekterna känner man numera, att de talrika facetterna på hornhinnan hos det stora sammansatta ögat bilda verkliga linser och att konerna innesluta egendomligt modifierade nervtrådar. Men dessa organer äro så olikartade hos klassen Articulata, att Müller fordom uppstälde tre stora klasser af sammansatta ögon med sju underafdelningar, jemte en fjerde stor klass af aggregerade enkla ögon.

Om vi tänka öfver dessa fakta, som här äro anförda i korthet, med afseende på denna stora, olikartade och graderade serie af ögon hos de lägre djuren, och om vi komma ihåg, huru ringa antalet af de nu lefvande formerna måste vara i jemförelse med de utdöda, blir svårigheten icke så stor att tro, att det naturliga urvalet har förvandlat en enkel synnerv omgifven med pigment och beklädd med en genomskinlig hinna till ett optiskt instrument, så fulländadt som hos några af leddjurens stora klass.

Den som vill gå ännu längre behöfver icke tveka att taga steget ut, om han finner efter genomläsandet af detta arbete, att massor af fakta, som på annat sätt äro oförklarliga, lätt låta förklara sig genom descendensteorien; han kan medgifva, att ett organ, så fulländadt som örnens öga, kan bildas genom det naturliga urvalet, ehuru han i detta fall icke känner öfvergångsformerna. Man har invändt, att för att modifiera ögat och ändock bibehålla det såsom ett fullkomligt instrument skulle många förändringar hafva försiggått samtidigt, hvilket antages icke kunna ske genom det naturliga urvalet, men såsom jag har försökt visa i mitt arbete om husdjurens förändringar är det icke nödvändigt att antaga, att alla modifikationer försiggå samtidigt, om de äro små och gradvisa. Äfven i den högst organiserade afdelningen af djurriket, Vertebrata, kunna vi utgå från ett öga så enkelt som hos lancettfisken, hvilket består af en liten säck af genomskinlig hud, försedd med en nerv och beklädd med pigment men utan hvarje annan optisk inrättning. Hos både fiskar och reptilier är enligt Owen ”serien af olika utvecklade ljusbrytande organer mycket stor.” Det är ett betydelsefullt faktum, att äfven hos menniskan enligt den stora auktoriteten Virchow den sköna kristallinsen hos embryot bildas genom en samling af epidermisceller, som ligga i ett säckformigt veck af huden, och att glaskroppen bildas af embryonal underhudscellväf. Om vi vilja komma till en riktig slutsats angående ögats bildning med alla dess beundransvärda, fullkomliga karakterer, är det oundgängligen nödvändigt att låta förnuftsskäl öfvervinna våra föreställningar, men jag har sjelf känt denna svårighet allt för väl att blifva öfverraskad, om någon tvekar att gifva grundsatsen om det naturliga urvalet en så förvånande utsträckning.

Man kan knappt undvika att jemföra ögat med ett teleskop. Vi veta, att detta instrument har nått sin fulländning genom de högsta menskliga snillekrafters långvariga ansträngningar och vi kunna deraf tänka oss, att ögat har blifvit bildadt genom en i viss mån analog process. Månne icke detta antagande är väl förmätet? Hafva vi rätt att antaga, att skaparen verkar medelst intellektuela krafter liksom menniskan? Om vi måste jemföra ögat med ett optiskt instrument, måste vi i inbillningen antaga ett tjockt lager af genomskinlig väfnad genomdränkt af vätska en för ljus känslig nerv derunder, och vi måste vidare antaga, att hvarje del af detta lager småningom och oupphörligt undergått förändringar i täthet, så att det afdelat sig i flera lager af olika tjocklek och täthet på olika afstånd från hvarandra, och att hvarje lagers yta småningom förändrat sin form. Vi måste vidare antaga, att det gifves en kraft, det naturliga urvalet, som uppmärksamt iakttager hvarje obetydlig förändring i de genomskinliga lagren, och omsorgsfullt bevarar hvar och en, som under förändrade omständigheter på något sätt eller i någon mån är i stånd att frambringa en tydligare bild. Vi måste antaga, att hvarje ny form af instrumentet mångfaldigats i milliontal, och att hvar och en bevarats till dess en bättre uppkommit och att alla de gamla då blifvit förstörda. Hos de lefvande organismerna är det variationen som åstadkommer de små förändringarna, generationen reproducerar dem nästan i oändlighet och det naturliga urvalet utsöker med aldrig felande skicklighet hvarje förbättring. Låt denna process fortgå i millioner år och bland millioner individer af många slag; kunna vi då icke tro, att ett lefvande optiskt instrument på detta sätt kan bildas, lika öfverlägset ett konstgjordt af glas, som skaparens verk i allmänhet äro öfverlägsna menniskans?



\section{Öfvergångsätt.}

Om man kunde bevisa, att det finnes ett inveckladt organ, som omöjligen har kunnat bildas genom otaliga successiva små modifikationer, vore min teori fullkomligt kullslagen. Men jag kan icke finna något sådant fall. Otvifvelaktigt finnas många organer, hvilkas öfvergångsstadier vi icke känna, särskildt om vi fästa afseende vid de mera isolerade arterna, omkring hvilka enligt min teori en stor ödeläggelse har egt rum. Samma är förhållandet, om vi taga ett organ gemensamt för alla medlemmarna af en stor klass, ty i detta sednare fall måste organet ursprungligen hafva bildats i en mycket aflägsen period efter hvilken alla medlemmarna hafva blifvit utvecklade; och för att upptäcka de tidiga öfvergångsstadier genom hvilka organet har passerat, borde vi taga i betraktande mycket gamla stamformer, som för längesedan dött ut.

Vi böra vara ytterst försigtiga i att antaga, att ett organ icke kan hafva bildats genom öfvergångsstadier af något slag. Talrika fall kunde gifvas bland de lägre djuren af ett organ, som på samma tid uträttar fullkomligt skilda funktioner; så till exempel fullgör näringskanalen tre förrättningar hos larver till trollsländan och hos fisken Cobitis, den respirerar, digererar och bortför excrementer. Om en hydra vändes ut och in, så fullgör den yttre ytan matsmältningen och den inre andas. I sådana fall torde det naturliga urvalet, om någon fördel dermed vunnes, för en enda funktion lämpa hela det organ eller en del deraf, som förut fullgjorde två, och på detta sätt genom omärkliga öfvergångar i hög grad förändra dess natur. Man känner många växter, hvilka reguliert på samma gång frambringa olika bildade blommor, och om sådana växter skulle alstra blott ett slag blommor, skulle i några fall en stor förändring inträffa i artens karakterer. Det kan också visas, att bildandet af två slags blommor på samma växt har åstadkommits genom vissa grader. Å andra sidan kunna två skilda organer hos samma individ samtidigt utföra samma förrättning och detta är ett i hög grad vigtigt medel till öfvergångar; det finnes till exempel fiskar med gälar, som andas den i vattnet upplösta luften på samma gång de andas fri luft i simblåsan, som genom kärlrika skiljeväggar är afdelad i små delar och är försedd med en luftgång för att fylla den med luft. Vi hafva ett annat exempel från växtriket: växter klättra på tre skilda sätt, genom spiralvridning, genom att omfatta ett stöd med sina klängen och genom att utsända luftrötter; dessa tre metoder finnas vanligen i skilda grupper, men några få växter använda två af dessa medel eller till och med alla tre förenade hos samma individ. I alla sådana fall kan ett af de två organerna, som utföra samma förrättning, modifieras och fulländas, så att det kan utföra hela arbetet, blott under modifikationsprocessen biträdt af det andra organet och detta andra organ kan modifieras till något annat, fullkomligt olika ändamål eller bli helt och hållet utplånadt.

Simblåsan hos fiskarna är ett godt exempel, emedan det tydligen visar oss det högst vigtiga faktum, att ett organ, som ursprungligen är bildadt till ett ändamål nämligen för simningen, kan förvandlas till ett organ för en helt annan förrättning, respirationen. Simblåsan har också blifvit använd såsom ett biorgan till fiskarnas hörselapparat. Alla fysiologer medgifva att simblåsan är homolog med eller ”ideelt lik” de högre vertebrerade djurens lungor i läge och struktur, och derföre finnes intet skäl till tvifvel, att simblåsan verkligen har blifvit ombildad till lungor, eller ett organ som uteslutande användes för respiration.

Enligt denna åsigt kunna vi påstå, att alla vertebrerade djur med verkliga lungor härstamma på fortplantningens vanliga väg ifrån en gammal och okänd prototyp, försedd med en simblåsa. Vi kunna på detta sätt, såsom jag sluter af Owens interessanta beskrifning på dessa delar, förstå det märkvärdiga förhållande, att hvarje partikel af födan eller drycken som vi svälja måste passera öfver luftstrupens öppning med fara att komma ned i lungorna, så vida icke den vackra inrättning funnes, som tillsluter ljudspringan. Hos de högre ryggradsdjuren hafva gälarna helt och hållet försvunnit — men hos embryot utmärkes deras förra läge af de så kallade gälspringorna på sidorna af halsen och blodkärlens bågiga förlopp. Men det är klart, att de nu helt och hållet försvunna gälarna hafva småningom blifvit bearbetade af det naturliga urvalet till något särskildt ändamål; gälarna och ryggpansaret på ringmaskarna antagas till exempel vara homologa med insekternas flygvingar och täckvingar, och det är icke osannolikt, att hos våra nu lefvande insekter organer, som i en förfluten period tjenade till respiration, verkligen hafva förvandlats till flygorganer.

Vid betraktandet af organers förvandling är det af så stor vigt att hålla i minnet möjligheten af förvandling från en funktion till en annan, att jag vill gifva ett annat exempel. De skaftade cirripederna hafva två små hudveck, som jag kallat frena ovigera, hvilka afsöndra en klibbig vätska, med hvars tillhjelp de qvarhålla äggen till dess de äro kläckta. Dessa cirripeder hafva inga gälar, utan hela kroppsytan jemte de små vecken tjenar till respiration. Balaniderna eller de sessila cirripederna å andra sidan hafva inga frena ovigera, utan äggen ligga lösa i botten af äggsäcken inom det väl slutna skalet; men å de ställen som motsvara frena ovigera hafva de stora veckiga hinnor, hvilka fritt kommunicera med äggsäckens och den öfriga kroppens blodförande lacuner, hvilka hinnor af prof. Owen och andra naturforskare som behandlat ämnet blifvit ansedda för gälar. Nu tror jag icke någon vill bestrida, att frena ovigera hos den ena familjen äro homologa med gälarna hos den andra, och de öfvergå i sjelfva verket i hvarandra. Vi kunna derföre icke betvifla att de små hudvecken, som ursprungligen tjenade såsom frena ovigera men som äfven i ringa grad deltogo i respirationsakten, hafva småningom ombildats till gälar genom det naturliga urvalet, helt enkelt genom tillväxt i storlek och körtlarnas försvinnande. Om alla skaftade cirripeder hade dött ut, och de hafva redan dött ut i större mängd än de sessila, hvem skulle då hafva kunnat tänka sig, att gälarna i den sednare familjen ursprungligen hade existerat såsom organer, hvilkas förrättning varit att skydda äggen från att spolas ut ur äggsäcken.



\section[Speciella svårigheter]{Speciela svårigheter för teorien om det naturliga
urvalet.}

Ehuru vi måste vara ytterst försigtiga i det antagandet, att något organ icke har kunnat bildas genom successiva gradvisa förändringar, gifvas otvifvelaktigt fall, som äro särdeles svåra att förklara och några af dem vill jag behandla i ett kommande arbete.

Ett af de svåraste äro de könlösa insekterna, hvilka ofta visa en skapnad helt olika både hannarnas och de fruktsamma honornas; detta fall skall behandlas i nästa kapitel. Fiskarnas elektriska organer erbjuda ett annat fall af synnerlig svårighet, ty det är omöjligt att begripa, genom hvad slags förändringar dessa underbara organer hafva bildats. Såsom Owen har anmärkt finnes mycken analogi emellan dem och vanliga muskler, i deras verkningssätt, i nervkraftens inflytande på dem äfvensom inverkan af vissa retmedel såsom stryknin och enligt några äfven i den finare bygnaden. Vi känna icke ens, till hvad nytta dessa organer äro, ehuru de otvifvelaktigt hos Gymnotus och Torpedo tjena såsom starka försvarsmedel och möjligen för att gripa rof. Men hos rockorna finnes såsom Matteucci nyligen iakttagit ett analogt organ i stjerten, hvilket om det retas starkt utvecklar en svag elektricitet, så svag, att den svårligen kan begagnas till sådant ändamål. Jemte nyssnämda organ finnes äfven hos rockorna, såsom doktor R. M’Donnell har visat, i närheten af hufvudet ett annat organ, hvilket icke är elektriskt, men som synes verkligen vara homologt med det elektriska batteriet hos Torpedo. Då vi slutligen icke känna något om dessa fiskars förfäder, måste det medgifvas, att vår kunskap är för ringa för att sätta oss i stånd att påstå, det inga öfvergångar äro möjliga, hvarigenom det elektriska organet har blifvit utveckladt.

Dessa samma organer synas först erbjuda en annan och vida allvarsammare svårighet, ty de förekomma i omkring ett dussin fiskarter, af hvilka flera äro vidt skilda i sina slägtförhållanden. I allmänhet om ett organ finnes hos flera medlemmar af samma klass och särskildt hos medlemmar som hafva mycket olika lefnadsvanor, kunna vi antaga dess närvaro bero på arf från en gemensam stamfader, och dess frånvaro hos några af medlemmarna på förlust genom bristande användning eller naturligt urval. Om de elektriska organerna hade gått i arf från någon uråldrig stamfader, kunde vi hafva väntat, att alla elektriska fiskar skulle hafva varit särdeles beslägtade med hvarandra, men detta är långt ifrån fallet. Geologien leder alldeles icke till den tron, att de flesta fiskar fordom egde elektriska organer, hvilka deras modifierade ättlingar nu hafva förlorat. Men om vi närmare betrakta saken, finna vi hos de olika fiskar som hafva elektriska organer, att dessa äro belägna i olika delar af kroppen, att de skilja sig i konstruktion, anordning och enligt Pacini i det sätt hvarpå elektriciteten utvecklas, och slutligen deruti att den erforderliga nervkraften ledes genom skilda nerver från vidt skilda källor, och detta är kanhända den vigtigaste af alla olikheter. Hos de blott aflägset beslägtade fiskar, hvilka ega elektriska organer, kunna derföre dessa icke betraktas såsom homologa, utan blott såsom analoga i funktion. Följaktligen finnes det intet skäl att antaga, att de hafva gått i arf från någon gemensam stamfader, ty om detta hade varit fallet, skulle de hafva betydligt liknat hvarandra i alla hänseenden. Den största svårigheten försvinner således, lemnande blott en mindre, som dock icke är obetydlig, nämligen att afgöra, genom hvilka gradvisa öfvergångar dessa organer hafva uppkommit och utvecklats hos hvarje särskild fiskgrupp.

De lysande organer, som finnas hos ett fåtal insekter af vidt skilda familjer och ordningar, och som äro belägna i skilda delar af kroppen, erbjuda en svårighet som är nästan fullkomligt lik den föregående. Andra exempel kunde anföras till exempel bland växterna; den helt egendomliga anordningen af en massa pollenkorn på en stjelk med en klibbig glandel i spetsen är tydligen densamma hos Orchis och Asclepias, slägten snart sagdt så vidt skilda som möjligt ibland fanerogamerna. I alla sådana fall, då två arter som i systemet äro vidt skilda äro försedda med likartade anomala organer, kan man göra den iakttagelsen att ehuru organets allmänna utseende och funktion är identiskt, dock alltid eller nästan alltid kan upptäckas någon väsentlig olikhet emellan dem. Jag är benägen att tro, att liksom två män ofta oberoende af hvarandra göra samma uppfinning, så har stundom det naturliga urvalet under sitt bemödande att se hvarje individ till godo och begagna analoga variationer på nästan samma sätt modifierat två organer hos två skilda organiska varelser, hvilka hafva blott obetydligt gemensamt i sin organisation såsom arf från en gemensam stamfader.

I ett nyligen utgifvet arbete har Fritz Müller undersökt ett likartadt fall för att pröfva de åsigter jag förfäktar. Flera familjer af krustaceerna omfatta ett fåtal arter, som ega organer för andning i luften, och hvilka derföre kunna lefva ofvan vattnet. I två af dessa familjer, hvilka Müller särskildt undersökte och som stå i nära slägtskap med hvarandra öfverensstämma arterna temligen fullständigt i alla vigtiga karakterer; nämligen i deras känselorganer, cirkulationssystem, i läget af de hårtofsar, med hvilka deras sammansatta magar äro beklädda och slutligen i hela bygnaden af deras gälar till de mikroskopiska hakarna. Deraf kunde man hafva väntat, att de lika vigtiga organerna för luftrespiration skulle varit lika hos de få arterna af båda familjerna som lefva på land, och detta borde framförallt de hafva väntat, som tro på särskilda skapelser; ty hvarföre skulle detta enda organ vara olika, som blifvit gifvet arterna för samma ändamål, under det alla andra vigtiga organer äro lika eller till och med identiska.

Fritz Müller påstår, att denna stora likhet i så många delar af bildningen måste i öfverensstämmelse med mina åsigter förklaras genom arf från en gemensam stamfader. Men då det stora flertalet arter i de ofvannämda familjerna, så väl som de flesta krustaceer af alla ordningar, äro vattendjur, är det i högsta grad osannolikt att deras gemensamma stamfader varit skapad att andas i luften. Müller föranleddes derföre att noggrant undersöka organerna hos arterna med luftrespiration, och hos alla fann han olikheter i vissa vigtiga punkter, såsom i mynningarnas läge, i sättet för deras öppnande och slutande och i några mindre detaljer. Sådana olikheter kunna förklaras med det antagandet, att arter af skilda familjer småningom blifvit allt mer och mer egnade att lefva ofvan vattnet och andas i luften. Ty dessa arter, som hörde till olika familjer, måste till en viss grad skilja sig från hvarandra, och i öfverensstämmelse med den grundsatsen, att hvarje variation beror på två faktorer, nämligen organismens beskaffenhet och omständigheternas natur, måste föränderligheten hos dessa krustaceer helt säkert icke hafva varit exakt densamma. Det naturliga urvalet har således haft olika material eller variationer att bearbeta för att komma till samma resultat i funktion, och de på detta sätt förvärfvade bildningarna böra nödvändigt vara skiljaktiga. Hypotesen om särskilda skapelseakter lemnar hela förhållandet oförklaradt. Denna af Fritz Müller använda bevisföring synes hafva ansenligt medverkat till att öfvertyga den utmärkta vetenskapsmannen om riktigheten af de åsigter jag i detta arbete framlagt.

I de nu anförda fallen hafva vi sett, att hos varelser af mer eller mindre aflägsen slägtskap samma ändamål vinnes och samma funktion uträttas genom organer, som till utseendet, ehuru icke i verkligheten, äro särdeles lika. Men den vanliga regeln igenom hela naturen är, att samma ändamål vinnes genom de mest olikartade medel äfven bland varelser som äro nära beslägtade med hvarandra. Huru olika äro icke fåglarnas fjädervingar och flädermössens hudvingar med alla fingrarna starkt utvecklade, och ännu mera trollsländans fyra vingar, flugans två vingar och ekoxens två vingar med dess båda täckvingar. Tvåskaliga snäckor kunna öppna och sluta sig, men efter huru många olika mönster är icke låset konstrueradt från den långa raden af väl i hvarandra gripande tänder hos en Nucula till det enkla låsbandet hos en mussla! Växternas frön spridas genom sin litenhet, — derigenom att deras kapslar förvandlas till lätta ballonglika höljen — derigenom att de äro inbäddade i en pulpa eller kött, bildadt af de mest olika delar dels närande dels med lysande färger, så att de locka till sig fåglar hvilka äta sådana frukter — derigenom att de äro försedda med hullingar af många slag och tandade agnar, så att de fastna i däggdjurens hår — och derigenom att de äro försedda med vingar eller fjun, så att de drifvas bort af minsta flägt. Jag vill anföra ett annat exempel, ty detta förhållande, att samma ändamål vinnes genom de mest olikartade medel, förtjenar väl uppmärksamhet. Några hafva framhållit den åsigten, att organiska varelser hafva blifvit skapade i många former blott för omvexlings skull, nästan liksom leksaker i en bod, men en sådan naturåskådning är förkastlig. Hos växter som hafva skilda kön och sådana hermafroditer, hos hvilka frömjölet icke frivilligt faller ned på märket, är yttre biträde nödvändigt för deras befruktning. Hos flera arter åstadkommes detta derigenom, att de lätta och osammanhängande pollenkornen af vinden drifvas bort och således af en ren slump träffa ett märke, och detta är väl det enklaste sätt som kan tänkas. Ett annat nästan lika enkelt sätt, ehuru fullkomligt olika, påträffas hos många växter, hvilka i en symmetrisk blomma utveckla några droppar nektar och följaktligen besökas af insekter och dessa öfverflytta då frömjölet från ståndarna till märket.

Från detta enkla stadium kunna vi genomgå ett outtömligt antal medel, alla för samma ändamål och tillkomna på väsentligen samma vis, men bestämmande förändringar i hvarje del af blomman. Nektarn kan samlas i olika formade förvaringsrum, med ståndare och pistiller modifierade på många vis, stundom bildande gillerlika inrättningar och stundom i stånd att utföra väl beräknade rörelser genom retlighet eller elasticitet. Från sådana bildningar kunna vi gå vidare till dess vi komma till ett sådant utomordentligt fall som Crüger nyligen har beskrifvit hos Coryanthes. Denna orchidé har en del af sin labellum eller undre läpp urhålkad till en större skål, uti hvilken nästan rent vatten oupphörligen droppar ned från tvenne horn som stå deröfver, och då skålen är till hälften fyld, rinner vattnet ut genom en ränna på ena sidan. Basaldelen af läppen står högre än skålen och är urhålkad till ett slags rum med två sidoingångar, och inuti detta rum finnas besynnerliga köttiga upphöjningar. Den mest genialiska menniska skulle icke kunna tänka ut, hvartill alla dessa delar tjena, om han icke varit vittne till hvad som försiggår. Men Crüger såg hopar af stora humlor besöka denna orchidés gigantiska blommor, icke för att suga nektar utan för att gnaga sönder de köttiga kammarna i rummet ofvanom skålen; under denna sysselsättning stötte de hvarandra ofta ned i skålen, och då deras vingar blefvo nedvätta, kunde de icke flyga bort utan måste krafla sig ut genom rännan. Crüger såg en ”oupphörlig procession” af humlor, som sålunda kraflade sig upp ur sitt ofrivilliga bad. Passagen är trång och betäckt af könpelaren, så att en humla, då den tränger sig ut, först gnider ryggen emot det klibbiga märket och sedan emot pollenmassornas klibbiga glandler. Pollenmassorna fastna således vid ryggen af den humla, som först kraflar sig ut genom gången i en utbredd blomma och föres sålunda bort. Crüger har sändt mig en blomma i sprit med en humla som han hade dödat innan den ännu hunnit krafla sig ut, och på dess rygg hade fastnat en pollenmassa. Om humlorna, så utrustade, sedan flyga till en annan blomma, eller en annan gång besöka samma blomma och af sina kamrater skuffas ned i skålen och sedan krafla sig ut genom rännan, måste pollenmassan först komma i beröring med det klibbiga märket, fastna vid den och blomman är befruktad. Nu se vi slutligen nyttan af hvarje del i blomman, af de vattenafsöndrande hornen, af skålen halffyld med vatten, som hindrar bien från att flyga bort och tvingar dem att krypa ut genom rännan och komma i beröring med de väl placerade klibbiga pollenmassorna och märket.

Blommans konstruktion hos en annan orchidé, Catasetum, är helt olika, ehuru den tjenar till samma ändamål, och den är lika märkvärdig. Bien besöka denna blomma liksom Coryanthes för att afgnaga labellum; under denna sysselsättning måste de oundvikligen vidröra en lång smal känslig tråd, eller såsom jag har kallat den, antennen. Då denna antenn vidröres meddelar den en sensation eller vibration till en viss hinna, som ögonblickligen spränges; denna frigör en fjeder, som drifver fram pollenmassan lik en pil i rät linie, så att den fastnar vid biets rygg med sin klibbiga ända. En hanblommas pollenmassa öfverflyttas sålunda till honblomman och kommer der i beröring med märket, som är klibbigt nog att bryta vissa elastiska trådar och qvarhålla pollenmassan, och på detta sätt åstadkommes befruktning.

Man kan fråga, huru man i de föregående och i oräkneliga andra exempel skall kunna förstå den graderade skala af sammansättning och de mångfaldiga medel naturen använder för att vinna samma ändamål. Svaret är utan tvifvel det redan gifna, att om två former variera, hvilka redan äfven i ringa grad skilja sig från hvarandra, föränderligheten icke är af fullkomligt samma natur och resultaten som naturliga urvalet ernår för samma ändamål följaktligen icke äro desamma. Vi böra också komma ihåg, att hvarje högt utvecklad organism har genomgått en lång serie af modifikationer och att hvarje modifierad bildning gerna går i arf, så att den icke helt och hållet går förlorad utan oupphörligen modifieras. Skapnaden af hvarje organ hos hvarje art för hvad ändamål det än begagnas är således summan af många ärfda förändringar, hvilka arten har genomgått under sin modifikation efter förändrade lefnadsvanor eller lefnadsvilkor.

Ehuru det i många fall är särdeles svårt att ana genom hvilka öfvergångar många organer hafva vunnit sitt nuvarande tillstånd, så har jag dock vid tanken på huru få de lefvande och kända formerna äro i förhållande till de utdöda och okända, blifvit förvånad öfver att finna, huru sällsynt något organ är, till hvilket inga kända öfvergångsformer föra. Det är helt säkert, att nya organer, särskildt skapade för något visst ändamål, sällan eller aldrig plötsligt uppträda inom någon klass; och detta antyder redan den gamla, ehuru något öfverdrifna grundsatsen i naturalhistorien: ”Natura non facit saltum”. Vi påträffa detta yttrande i nästan alla erfarna naturforskares skrifter; Milne Edwards har uttryckt det med orden: ”Naturen är slösande på förändringar, men njugg på nyheter”. Hvarföre skulle enligt skapelseteorien så många förändringar förekomma och så få nyheter? Hvarföre äro alla delar och organer hos så många oberoende varelser, hvar och en skapad för sin plats i naturen, så sammanlänkade genom gradvisa öfvergångar? Hvarföre skulle icke naturen taga ett hastigt steg från form till form? Med teorien om det naturliga urvalet kunna vi förstå, hvarföre hon icke gör det; ty det naturliga urvalet verkar blott genom att begagna sig af små successiva förändringar; naturen kan aldrig taga ett stort steg, utan hon måste gå framåt med små och säkra, men långsamma steg.



\section{Organer af ringa betydelse.}

Då det naturliga urvalet arbetar på lif och död — genom att beskydda den bäst utrustade och tillintetgöra den mindre väl utrustade individen — har jag stundom funnit en stor svårighet i att begripa ursprunget och bildningen af organer med ringa betydelse; denna svårighet har mången gång synts mig lika stor med den som de mest fulländade och sammansatta organer erbjuda, ehuru af helt annan beskaffenhet.

För det första känna vi för litet om hvarje organisk varelses hela ekonomi, för att bedöma, hvilka obetydliga modifikationer äro af vigt eller icke. I ett föregående kapitel har jag gifvit exempel på mycket obetydliga karakterer, såsom dunet på frukterna och färgen på fruktköttet, färgen på däggdjurens hud och hår, hvilka helt säkert röna inflytande af det naturliga urvalets verksamhet, då de stå i samband med konstitutionela olikheter eller bestämma insekternas angrepp. Girafens svans ser ut som en flugsmälla, och det synes först otroligt, att denna skulle hafva blifvit lämpad för sitt nuvarande ändamål genom successiva modifikationer, hvar och en allt mera lämplig för ett så obetydligt ändamål som att bortdrifva flugor. Dock böra vi ej allt för hastigt med bestämdhet uttala oss äfven i detta fall, ty vi veta att fördelningen och tillvaron af boskapskreatur och andra djur i Sydamerika absolut beror på deras förmåga att motstå insekternas anfall, så att individer, som på något sätt kunna försvara sig från dessa små fiender, hafva den stora fördelen att kunna utbreda sig öfver nya betesmarker. Härmed vill jag icke säga, att de större däggdjuren verkligen tillintetgöras af flugor (utom i några sällsynta fall), men de uttröttas och försvagas oupphörligt, så att de äro mera utsatta för sjukdomar och under en kommande hungersnöd med större svårighet finna sin föda eller undfly rofdjur.

Organer, som nu äro af ringa vigt hafva sannolikt i några fall varit af stor betydelse för någon uråldrig stamfader, och sedan de småningom i en tidig period blifvit fulländade, hafva de öfvergått genom arf på efterkommande arter i oförändradt skick, ehuru nu af ringa nytta, men en rent af skadlig förändring i deras struktur bör i allmänhet hafva förekommits af det naturliga urvalet. Då vi se huru vigtigt ställflyttningsorgan svansen är hos de flesta vattendjur, kunna vi måhända på detta sätt förklara dess närvaro och mångfaldiga bruk hos så många landdjur, som i sina lungor eller modifierade simblåsor förråda sitt aqvatiska ursprung. Då en väl utbildad svans blifvit formad hos ett vattendjur, har den sedan blifvit förarbetad till alla möjliga ändamål, till flugsmälla, till griporgan, eller till ett medel att vända, såsom hos hunden, ehuru detta hjelpmedel måtte vara svagt, då haren, som knappt har någon svans kan vända sig hastigt nog.

För det andra torde vi stundom med orätt tillskrifva karakterer en stor vigt, hvilka hafva uppkommit af sekundära orsaker oberoende af det naturliga urvalet. Vi böra ihågkomma, att klimat, föda etc. sannolikt hafva något, kanhända stort direkt inflytande på organisationen; att karakterer uppträda enligt lagen om återgång, att vexelverkan är ett vigtigt element vid förändringarna, och slutligen att det sexuela urvalet ofta har i hög grad modifierat de högre djurens yttre karakterer för att gifva en hanne någon fördel i striden mot andra hannar eller att tjusa honan, och karakterer, som erhållas genom sexuelt urval, kunna öfvergå på båda könen. En modifikation, som på något af ofvannämda sätt kommit till stånd har först varit utan någon direkt nytta för en art, men har sedan begagnats af dess afkomlingar under nya lifsvilkor och nya lefnadsvanor.

Om till exempel endast gröna hackspettar funnits och vi icke kände till de många svarta och brokiga som finnas, så vågar jag påstå, att vi skulle ansett denna gröna färg vara särdeles lämplig att dölja denna i träd klättrande fågel för dess fiender och att följaktligen denna karakter vore af vigt och hade förvärfvats genom det naturliga urvalet; men i sjelfva verket beror färgen sannolikt till största delen på det sexuela urvalet. En klättrande palm på Malayiska arkipelagen stiger upp ända till de högsta trädtoppar med tillhjelp af särskildt konstruerade hakar, som äro samlade i grenspetsarna, och denna inrättning är otvifvelaktigt af största vigt för växten. Men då vi på många träd, som icke klättra, se liknande hakar, hvilka sannolikt tjena till skydd mot betande däggdjur, såsom man har skäl att tro på grund af de tornklädda arterna i Afrika och Sydamerika, så kunna hakarna af denna palmart hafva bildats först för detta ändamål och sedermera begagnats af växten till andra förrättningar, då den undergått vidare modifikationer och blifvit klättrande. Gamens nakna hud å hufvudet antages i allmänhet såsom en direkt anordning för att rota i ruttna lik, och detta kan vara förhållandet, eller det kan möjligen bero på de ruttnande ämnenas direkta inverkan, men vi skola vara mycket försigtiga i att draga sådana slutsatser, då vi se huden naken på hufvudet hos kalkonen, som lefver af friska ämnen. Suturerna i kranierna af unga däggdjur hafva blifvit framhållna såsom särskildt inrättade för att underhjelpa födseln, och otvifvelaktigt underlätta de den, eller äro till och med oundgängliga derför; men då suturer förekomma äfven hos kranierna af unga fåglar och reptilier, hvilka blott behöfva bryta ett äggskal, kunna vi antaga att denna bildning har uppkommit från lagarna för utvecklingen och sedermera begagnats såsom fördelaktig vid de högre djurens födsel.

Vi veta rakt ingenting om orsakerna till hvarje obetydlig variation eller individuel afvikelse, och vi känna detta mest, då vi reflektera öfver olikheter hos raserna af våra husdjur i skilda trakter, särskildt i de mindre civiliserade trakter, der något metodiskt urval knappt kommer i fråga. De af vilda folkslag i skilda trakter hållna husdjuren måste ofta sjelfva skaffa sig sitt uppehälle och äro till en viss grad beroende af det naturliga urvalet, och individer med ringa olikheter i konstitution böra trifvas bäst under olika klimat. En god iakttagare berättar, att hos boskapen en viss färg gör dem mera utsatta för flugorna samt mottagligare för växtgifter, så att färgen äfven på detta sätt blir föremål för det naturliga urvalets verksamhet. Andra observatörer äro öfvertygade om, att ett fuktigt klimat befordrar hårväxten och att horn och hår äro ömsesidigt beroende af hvarandra. Bergraser äro öfverallt olika låglandsraser och en bergig trakt bör sannolikt inverka på bakre extremiteterna, då dessa mera tagas i anspråk, jemte möjligen bäckenet; och enligt lagen om homolog variation böra äfven de främre extremiteterna och hufvudet röna inflytande deraf. Bäckenets skapnad bör genom tryck inverka på skapnaden af vissa delar hos ungen i moderlifvet, svårigheten att andas i högre regioner bör, såsom vi hafva skäl att tro, inverka på bröstkorgens storlek och der kommer också vexelverkan med i spelet. Verkningarna på hela organismen af bristande öfning jemte öfverflöd på föda är sannolikt af än större vigt, och såsom Nathusius nyligen har visat i sitt utmärkta arbete, är detta en hufvudorsak till de stora förändringar, som svinraserna hafva undergått. Men vår kunskap är alldeles för ringa, för att vi skulle kunna anställa betraktelser öfver vigten af de kända och okända orsakerna till föränderlighet; och jag har gjort dessa anmärkningar blott för att visa, att om vi äro ur stånd att förklara våra husdjursrasers karakteristiska olikheter, hvilka likväl i allmänhet antagas hafva uppkommit genom vanlig fortplantning från en eller några få stamraser, böra vi icke lägga för mycken vigt på vår bristande kännedom om de verkliga orsakerna till analoga skilnader emellan arter. För samma ändamål kan jag anföra olikheterna mellan menniskoraserna, hvilka äro lika starkt utpräglade; något ljus kan uppenbarligen spridas öfver dessa olikheter med tillhjelp af ett särskildt slag af sexuelt urval, men så vida jag icke inläte mig i alla detaljer, skulle mitt resonnemang kunna synas frivolt.



\section{Ändamålsenlighet. Skönhet.}

Föregående anmärkningar föranleda mig att säga några ord om den protest, som några naturforskare nyligen inlagt emot ändamålsenlighetsläran, att hvarje bildning i alla enskildheter är af nytta för sin egare. De tro att många bildningar hafva blifvit skapade blott för skönhetens skull i menniskors ögon, eller såsom redan nämts blott för omvexlings skull. Om sådana läror äro sanna, skulle de gifva dödsstöten åt min teori. Jag medgifver dock fullkomligt, att många bildningar nu äro af ingen direkt nytta för deras egare, och kanske aldrig hafva varit af någon fördel för deras stamfäder. Otvifvelaktigt hafva, såsom nyligen anmärkts, den bestämda verkan af förändrade vilkor, vexelverkande variation och regress, alla lemnat sina bidrag till resultatet. Men det vigtigaste är, att hvarje lefvande djur har fått största delen af sin organisation i arf, och att följaktligen, ehuru hvar och en helt säkert är väl lämpad för sin plats i naturen, hafva många bildningar intet direkt samband med förhandenvarande lefnadsvanor. Så kunna vi näppeligen tro att höglandsgåsens eller fregattfågelns simhud är af någon nytta för dessa fåglar; vi kunna icke heller tro, att de motsvarande benen i apans arm, hästens framfot, flädermusens vinge och skälens främre extremiteter äro af någon särskild nytta för dessa djur. Vi kunna utan fara tillskrifva ärftligheten dessa bildningar. Men landgåsens och fregattfågelns stamfader hade säkerligen lika mycken nytta af sin simhud, som nu de flesta lefvande vattenfåglar. Så kunna vi tro, att skälens stamfader icke egde en fot skapad för simning, utan en fot med fem tår lämplig till gång och såsom griporgan, och vi kunna vidare våga tro, att benen i apans, hästens och flädermusens extremiteter hafva gått i arf från någon gammal stamfader, som hade mera nytta deraf än dessa djur med deras vidt skilda lefnadsvanor, och att de hafva följaktligen modifierats genom naturligt urval. Om vi fästa vederbörligt afseende på den bestämda verkan af förändrade vilkor, korrelation, regress etc, kunna vi draga den slutsatsen, att hvarje enskildhet i hvarje lefvande varelses skapnad är antingen nu eller har fordom varit af nytta, — direkt eller indirekt genom de invecklade lagarna för utvecklingen.

Beträffande den åsigt, att organiska varelser blifvit skapade sköna för menniskors nöje, — en åsigt, som framträdt med anspråk på sanning och såsom kullstörtande hela min teori, — vill jag först anmärka, att idéen om ett föremåls skönhet tydligen ligger i menniskosjälen, utan afseende på någon verklig egenskap hos det beundrade föremålet, och att idéen icke är något medfödt och oföränderligt element i menniskosjälen. Vi se detta deruti, att menniskor af olika ras hafva helt och hållet olika reglor för den skönhet de beundra hos sina qvinnor; hvarken en neger eller en kines beundrar en idealiskt skön kaukasiska. Äfven idéen om pittoresk skönhet hos scenerier har uppkommit i en nyare tid. En sådan åsigt, att sköna föremål hafva blifvit skapade för menniskans nöje, innefattar, att jordytan företedde mindre skönhet innan menniskan fans än sedan hon uppträdde på skådebanan, hvilket behöfver bevisas. De sköna volutorna och konsnäckorna från eocenperioden, alla de behagligt mejslade ammoniterna från sekundärperioden, äro de skapade för att menniskan århundraden efteråt skall beundra dem i sina samlingar? Få föremål äro vackrare, än de fina kiselskalen hos diatomaceerna; månne dessa skapades för att undersökas och beundras under mikroskop med starka förstoringar? Skönheten är i detta och många andra fall uppenbarligen helt och hållet beroende på symmetrisk utveckling. Blommor äro bland de vackraste naturalster, och de hafva blifvit vackra genom det naturliga urvalet, eller snarare granna i motsats till de gröna bladen, för att lätt observeras och besökas af insekter, hvilket underlättar deras befruktning. Jag har dragit denna slutsats deraf, att jag funnit som en allmän regel, att en blomma aldrig har krona med lysande färger, då den befruktas med vindens tillhjelp. Andra växter åter alstra i allmänhet två slags blommor, somliga öppna och färgade för att locka till sig insekter, andra slutna, färglösa och utan nektar, hvilka derföre aldrig besökas af insekter. Vi kunna deraf draga den slutsats, att om våra insekter aldrig hade funnits på jordytan, skulle växtligheten icke blifvit betäckt med sköna blommor, utan den skulle blott alstrat sådana enkla blommor, som vi se på våra furar, ekar, askar och hasselbuskar, på gräs, spenat, fräken och nässlor. Samma resonnemang gäller äfven de flesta slagen af våra sköna frukter; att ett moget smultron eller körsbär är lika behagligt för ögat som för gommen är en sak, som hvar och en vill medgifva. Men denna skönhet tjenar blott såsom vägledning för fåglar och andra djur, som äta frukterna och sprida fröen. Att detta är förhållandet sluter jag deraf, att jag hittills i hvarje fall funnit, att frön, som äro inbäddade i en frukt af något slag, i ett köttigt hölje med lysande färger eller synbart blott genom en hvit eller svart färg, alltid spridas genom frukternas förtärande af vissa djur.

Å andra sidan medgifver jag villigt, att ett stort antal husdjur såsom våra flesta dyrbara fåglar, några fiskar, några däggdjur och en mängd ståtligt färgade fjärilar och några andra insekter hafva blifvit sköna för skönhetens skull, men detta har skett icke till menniskans behag utan genom sexuelt urval, derigenom att de vackrare hannarna oupphörligen blifvit föredragna af de mindre prydda honorna. Så är äfven förhållandet med fåglarnas sång. Dertill kunna vi sluta deraf, att smaken för vackra färger och för musik går genom en stor del af djurriket. Då honan är lika vackert färgad som hannen, hvilket icke sällan är fallet bland fåglar och fjärilar, ligger orsaken helt enkelt deruti, att de färger som förvärfvats genom sexuelt urval hafva gått i arf på båda könen i stället för på hannarna allena. I vissa fall deremot har det naturliga urvalet motverkat samlandet af granna färger på honan för att förminska den fara hon då skulle löpa under den tid hon rufvar sina ägg.

Det naturliga urvalet kan omöjligen åstadkomma någon modifikation hos en art uteslutande för en annan arts nytta, men igenom hela naturen begagnar hvarje art oupphörligen till sin fördel andras skapnad. Men det naturliga urvalet kan bilda former som äro till direkt skada för andra djur och gör det ofta, såsom huggormens tänder och ichneumons äggläggningsrör, genom hvilket äggen praktiseras in i andra lefvande insekters kroppar. Om det kunde bevisas, att en del af någon arts organisation hade bildats uteslutande för en annan arts nytta, skulle min teori vara tillintetgjord, ty något sådant kunde det naturliga urvalet icke hafva åstadkommit. Ehuru många sådana uppgifter finnas i naturhistoriska arbeten, kan jag icke finna ett enda fall som synes mig vara af någon vigt. Det medgifves, att skallerormen har gifttänder till eget försvar och för att döda sitt rof, men några författare antaga, att den på samma gång har fått skallran till sin egen skada, nämligen för att varna sitt rof. Lika gerna kunde jag tro, att katten då den bereder sig till språng kröker svansen för att varna den lifdömda råttan. Men jag har icke utrymme att här inlåta mig på detta och andra likartade fall.

Det naturliga urvalet kan icke hos en varelse frambringa något som är skadligt för honom, ty det naturliga urvalet verkar blott i och för individens bästa. Intet organ bildas, såsom Paley har anmärkt, för att förorsaka sin egare någon smärta eller skada. En noggrann afvägning af den nytta och skada som hvarje del medför visar, att den på det hela taget är fördelaktig. Om under tidernas lopp och under förändrade lefnadsförhållanden någon del råkar blifva skadlig, så modifieras den eller också dör varelsen ut, såsom så många myriader varelser hafva gjort.

Det naturliga urvalet sträfvar blott att göra hvarje varelse så fullkomlig som möjligt, eller i någon mån fullkomligare än de andra inbyggarna i samma trakt, med hvilka den måste kämpa för sin tillvaro. Och detta är såsom vi se den grad af fulländning naturen uppnår. De inhemska naturprodukterna på Nya Zeeland äro till exempel fullkomliga i jemförelse med hvarandra, men de vika nu raskt tillbaka för de framträngande legionerna af från Europa införda växter och djur. Det naturliga urvalet kan icke åstadkomma absolut fullkomlighet, och så vidt vi kunna döma påträffa vi aldrig denna höga ståndpunkt i naturen. Korrektionen för ljusets aberration är enligt Müller icke ens fullkomlig hos det mest fulländade organ, menniskoögat. Om vårt förnuft leder oss till entusiastisk beundran för en mängd oefterhärmliga inrättningar i naturen, så säger oss samma förnuft, ehuru vi kunna lätt misstaga oss å båda sidor, att så många andra äro mindre fullkomliga. Kunna vi såsom fullkomlig anse getingens gadd, hvilken, då den en gång begagnats, icke kan dragas tillbaka i följd af de bakåt riktade hullingarna, och hvilkens bruk följaktligen medför insektens ovilkorliga död genom att sönderslita dess inelfvor?

Om vi antaga, att biens gadd ursprungligen har funnits hos någon aflägsen stamfader såsom ett borrnings- och sågverktyg, likt det som finnes hos så många andra medlemmar af samma stora ordning, och att det har blifvit modifieradt, men icke fullkomnadt, och att giftet, som ursprungligen var ämnadt till helt annat ändamål, såsom att frambringa galläpplen eller något dylikt, har sedermera tilltagit i skärpa, så kunna vi måhända förstå, hvarföre gaddens bruk så ofta förorsakar insektens död; ty om öfverhufvudtaget förmågan att sticka är nyttig för hela bisamhället, så uppfyller den alla det naturliga urvalets fordringar, oaktadt den medförer några individers död. Om vi förvånas öfver det i sanning underbara väderkorn, genom hvilket många insekthannar finna sina honor, kunna vi då å andra sidan beundra den ensamt för detta ändamål bestämda produktionen af tusentals drönare, hvilka icke kunna vara samhället till någon annan nytta, och hvilka till sist slagtas af sina arbetsamma och sterila systrar? Det må vara svårt, men vi måste beundra visens vilda, instinktlika hat, som drifver henne att döda sina döttrar, de unga visarna, så snart de äro födda, eller också sjelf duka under i striden, ty detta sker otvifvelaktigt till samhällets bästa; och moderskärlek eller modershat, det senare lyckligtvis sällsynt, gäller lika inför den obevekliga grundsatsen, det naturliga urvalet. Om vi beundra de sinnrika anordningar, genom hvilka orchideer och många andra växter befruktas med insekters tillhjelp, kunna vi såsom lika fullkomliga betrakta våra furar, som måste frambringa täta moln af frömjöl för att några korn skola af en gynsam vind föras fram till fröämnet?



\section[Sammanfattning]{Sammanfattning. Lagen om typens enhet och
existensvilkoren innefattas i teorien om det
naturliga urvalet.}

Vi hafva i detta kapitel behandlat några af de svårigheter och invändningar, som kunna uppställas emot min teori. Många af dem äro allvarsamma nog, men jag tror att under deras behandling något ljus har blifvit spridt öfver vissa fakta, hvilka enligt åsigten om oberoende skapelseakter äro ytterligt dunkla. Vi hafva sett, att arter icke på hvilken period som helst äro föränderliga i oändlighet, och att de icke äro hopkedjade genom en mängd öfvergångsformer, dels emedan det naturliga urvalet verkar ytterst långsamt och på en viss tid blott på ett ringa antal former, dels emedan det naturliga urvalet innefattar ett oupphörligen fortgående utrotande af de föregående formerna och övergångsformerna. Närbeslägtade arter, som nu lefva på en sammanhängande yta, måste ofta hafva bildats då ytan ej var sammanhängande och då lifsvilkoren icke märkbart förändrade sig från den ena delen till den andra. Om två varieteter bildas i två distrikt af samma område, uppkommer ofta en mellanvarietet, lämplig för en mellanliggande trakt, men af angifna skäl finnes denna intermediära varietet i allmänhet i ringare antal än de två former som den förenar; de två senare, som finnas i större antal, måste under en fortgående modifikationsprocess hafva en större fördel öfver den mindre talrika mellanvarieteten, och komma således att ersätta och utrota den.

Vi hafva i detta kapitel sett, huru försigtiga vi böra vara i det antagandet, att de mest olikartade lefnadsvanor icke kunna öfvergå i hvarandra; att en flädermus till exempel icke skulle kunnat bildas genom naturligt urval af ett djur, som ursprungligen blott kunde glida genom luften.

Vi hafva sett, att en art kan under nya lifsvilkor förändra sina vanor eller mångfaldiga dem och antaga vanor som äro olika dess närmaste samslägtingars. Deraf kunna vi förstå, om vi tillika ihågkomma, att hvarje organisk varelse försöker att lefva der han kan lefva, huru det kan finnas en höglandsgås som har simhud mellan tårna, en hackspett som lefver på marken, en trast som dyker, och en stormfågel med lommarnas lefvadsvanor.

Ehuru den tanken, att ett organ så fulländadt som ögat skulle kunna bildas genom naturligt urval, är mer än tillräcklig att göra oss vacklande, finnes likväl ingen logisk omöjlighet för att ett organ under föränderliga lefnadsvilkor genom en lång serie af olika grader, hvar och en nyttig för sin egare, slutligen kan hinna hvarje tänkbar grad af fulländning genom naturligt urval. I de fall der vi icke känna några mellanformer eller öfvergångsstadier, böra vi vara ytterst försigtiga i det antagandet, att inga sådana existerat, ty många organers och deras mellanformers homologier visa oss, hvilka underbara förändringar i funktion äro åtminstone möjliga. En simblåsa har till exempel uppenbarligen förvandlats till en lunga för luftrespiration. Öfvergångar måste ofta i hög grad hafva underlättats, om samma organ samtidigt förrättat mycket skilda funktioner och sedan blifvit helt och hållet eller till en del särskildt lämpadt för en funktion, och om af två organer, som utfört samma förrättning, det ena blifvit fulländadt med understöd af det andra.

Vi hafva sett, att ett organ, som hos två från hvarandra vidt skilda varelser tjenade till samma ändamål och till utseendet vore nästan lika, har blifvit särskildt och oberoende bildadt hos båda, men om sådana organer närmare undersökas, kunna nästan alltid väsentliga skilnader upptäckas i deras bygnad, och detta följer naturligtvis från grundsatsen om naturligt urval. Å andra sidan gäller såsom en allmän regel i hela naturen oändlig vexling i bygnad för vinnande af samma ändamål, och detta följer åter från samma stora grundsats.

I nästan alla fall äro vi allt för okunniga för att kunna påstå, att en del eller ett organ är af så ringa vigt för en arts välfärd, att modifikationer i dess bygnad icke hafva kunnat småningom ske genom naturligt urval. Men vi kunna med tillförsigt antaga, att många modifikationer som helt och hållet bero på lagarna för utvecklingen och ursprungligen icke varit af någon fördel för arten, hafva sedermera användts af de ännu mera modifierade ättlingarna af denna art. Vi kunna också antaga, att ett organ, som fordom varit af stor vigt, ofta har bibehållits (såsom svansen af ett vattendjur qvarstår hos dess afkomlingar, som äro landdjur) ehuru det har blifvit af så ringa vigt, att det icke i dess nuvarande tillstånd har kunnat förvärfvas genom det naturliga urvalet — en kraft som verkar blott genom de bäst utrustade individernas bestånd i kampen för tillvaron.

Det naturliga urvalet åstadkommer ingenting hos en art till uteslutande nytta eller skada för en annan, ehuru det väl kan frambringa delar, organer och afsöndringar, som äro i hög grad nyttiga och till och med oundgängliga eller i hög grad skadliga för en annan art, men i alla händelser äro de på samma gång nyttiga för egaren. Det naturliga urvalet måste i en tätt bebodd trakt verka hufvudsakligen genom invånarnas täflan med hvarandra och följaktligen åstadkomma fulländning eller styrka i kampen för tillvaron blott i enlighet med måttstocken för denna trakt. Invånarna på ett område gifva derföre ofta vika för invånarna i ett annat i allmänhet större område, ty på ett större område hafva flera individer kunnat existera och flera olikartade former; kampen har på den grund varit häftigare och fordringarna på fullkomlighet hafva derföre blifvit högre. Det naturliga urvalet åstadkommer icke nödvändigt absolut fullkomlighet, ej heller kan absolut fullkomlighet någonstädes påträffas, så vidt vi med vår inskränkta förmåga kunna döma.

Enligt teorien om naturligt urval kunna vi tydligt fatta fulla betydelsen af den gamla regeln i naturhistorien: ”Natura non facit saltum”. Om vi se blott på de närvarande invånarna i verlden är denna regel icke korrekt, men om vi innefatta alla de förflutna tidernas invånare, kända eller okända, måste den enligt min teori vara fullkomligt sann.

Det är allmänt erkändt, att alla organiska varelser hafva bildats enligt två stora lagar — den ena är typens enhet, den andra existensvilkoren. Med typens enhet menas den öfverensstämmelse i bygnadens grundplan, som vi se hos organiska varelser af samma klass, och hvilken är fullkomligt oberoende af deras lefnadsvanor. Enligt min teori förklaras enheten i typ genom enheten i härkomst. Uttrycket existensvilkor, som den ryktbare Cuvier så ofta använder, omfattas tillfullo af principen om naturligt urval. Ty det naturliga urvalet verkar antingen genom att nu lämpa de varierande delarna af hvarje varelse efter dess organiska och oorganiska lefnadsvilkor, eller genom en likartad process som försiggått under längesedan förflutna tidsperioder; härvid har urvalet ofta biträdts af organers användning och overksamhet, af de yttre lifsvilkorens direkta inverkan och i alla fall har det varit beroende af lagarna för utvecklingen. Lagen om existensvilkoren är derföre den högre, ty genom ärftligheten innesluter den äfven lagen om typens enhet.


