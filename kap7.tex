%SJUNDE KAPITLET.





\chapter{Instinkt.}

{\it
Instinkter jemförliga med vanor men af olika ursprung. — Grader. — Bladlöss och myror. — Föränderliga instinkter. — Tama djurs instinkter, deras ursprung. — Gökens, strutsens och parasitbiens naturliga instinkter. — Myrors instinkt att göra slafvar. — Kupbiet, dess instinkt att bygga celler. — Förändringar i instinkt och kroppsbygnad behöfva icke vara samtidiga. — Svårigheter för teorien om naturligt urval af instinkter. — Neutrer eller sterila insekter. — Sammanfattning.
}\\[0.5cm]

Instinkten har väl kunnat behandlas i de föregående kapitlen, men jag ansåg det mera passande att behandla detta ämne särskildt, isynnerhet som en instinkt, så underbar som kupbiets, hvilken drifver djuret att bygga honingskakor, sannolikt förefaller mången läsare såsom en svårighet tillräcklig att kullstörta hela teorien. Jag måste på förhand säga, att jag intet har att skaffa med ursprunget till de första själsförmögenheterna, lika litet som med lifvets första ursprung. Vi skola blott uppehålla oss vid olikheterna i instinkt och andra själsförmögenheter hos djur inom samma klass.

Jag vill icke försöka definiera ordet instinkt. Det vore lätt att visa, att denna term i allmänhet inbegriper flera särskilda själsyttringar, men hvar och en förstår hvad som menas, om jag säger, att instinkten drifver göken att flytta och lägga sina ägg i andra fåglars nästen. En handling, som vi sjelfva icke skulle kunna utföra utan erfarenhet, säges i allmänhet vara instinktmässig, om den utföres af ett djur utan erfarenhet, samt om den utföres af många individer, utan att de veta för hvad ändamål. Men jag kunde dock visa, att ingen af dessa karakterer är allmän. En liten dosis af omdöme eller förstånd, såsom Pierre Huber uttrycker sig, kommer ofta med i spelet äfven hos djur som stå lågt på naturens skala.

Frédéric Cuvier och flera äldre metafysiker hafva jemfört instinkter med vanor. Denna jemförelse gifver, såsom jag tror, en noggrann förestälning om de gränser för själsverksamheten, inom hvilka en instinktmässig handling utföres, men icke nödvändigt om dess ursprung. Vi sjelfva äro ju ofta omedvetna om våra vanor, ja stundom stå de i rak motsats till vår vilja, dock kunna de modifieras af viljan eller förnuftet. Vanor förbinda sig ofta med andra vanor eller med vissa tidsperioder eller med vissa kroppstillstånd. En gång förvärfvade qvarstå de ofta hela lifvet igenom. Flera andra likheter emellan instinkt och vana kunde uppvisas. Liksom vid upprepandet af en välkänd melodi, så följa äfven vid instinkten den ena handlingen på den andra genom ett slags rytm; om en person afbrytes i en sång eller vid framsägandet af någonting utantill, tvingas han i allmänhet att gå tillbaka för att återfinna sin vanliga tankegång. Sådant observerade P. Huber hos en löfmask, som spinner ett slags mycket sammansatt hängmatta; om han tog en sådan löfmask, som hade fullbordat sin matta till exempel till sjette portionen, och lade den i en matta, af hvilken blott tre portioner voro färdiga, så gjorde masken helt enkelt om fjerde, femte och sjette portionerna. Men om en mask togs ur en matta, förfärdigad till exempel till tredje portionen och lades i en annan, af hvilken till och med sjette portionen var färdig, så att en god del arbete redan var undangjord, så långt ifrån att inse fördelen deraf blef masken mycket förvirrad, och för att fullborda sin matta tycktes den tvingad att börja arbetet ifrån tredje portionen, der den hade slutat, och försökte på detta sätt fullborda det redan färdiggjorda arbetet.

Om vi antaga, att en genom vana antagen handling går i arf — och jag tror, att detta kan bevisas stundom vara förhållandet — så blir likheten emellan instinkten och det som ursprungligen var en vana så stor, att de knappt kunna skiljas. Om Mozart i stället för att spela piano vid tre års ålder med ovanligt liten öfning, hade spelat en melodi utan någon föregående öfning, skulle man säkerligen sagt, att han gjort det instinktmässigt. Men det vore ett stort misstag att antaga, att flertalet instinkter blifvit förvärfvade genom vana under en generation och sedan gått i arf på följande generationer. Det kan klart bevisas, att de mest underbara instinkter vi känna, nämligen kupbiets och många myrors, omöjligen kunnat förvärfvas genom vana.

Det medgifves allmänt, att instinkter äro likaså vigtiga för hvarje arts välfärd under dess nuvarande lefnadsvilkor som kroppsbygnaden. Under förändrade lefnadsförhållanden är det åtminstone möjligt, att små modifikationer i instinkterna äro af nytta för en art, och om det kan visas att instinkten kan variera aldrig så litet, kan jag icke se någon svårighet för det naturliga urvalet att skydda och småningom samla förändringar i instinkt till hvilken grad som helst, som är fördelaktig. På detta sätt hafva, såsom jag tror, alla de mest invecklade och underbara instinkter uppkommit. Likasom modifikationer i kroppsbildning härröra från och ökas genom användning eller vana och minskas eller försvinna under overksamhet, så tillgår det sannolikt äfven med instinkterna. Men jag tror att vanans verkningar äro af fullkomligt underordnad vigt i jemförelse med verkningarna af det naturliga urvalet på hvad man kan kalla instinkternas spontana variation, — det är på variationer, som åstadkommas af samma okända orsaker, som hafva till följd små afvikelser i kroppsbildning.

Ingen sammansatt instinkt kan framkallas genom naturligt urval, annat än genom en långsam och gradvis accumulation af talrika, små, men fördelaktiga variationer. Såsom förhållandet var med kroppsbildningen, så böra vi äfven här i naturen finna icke de verkliga öfvergångsstadier genom hvilka hvarje sammansatt instinkt blifvit förvärfvad — ty dessa kunna finnas blott hos hvarje arts förfäder — men vi böra hos sidolinierna af samma härstamning finna spår till sådana öfvergångar, eller vi böra åtminstone vara i stånd att visa, att öfvergångar äro möjliga, och detta kunna vi också. Oaktadt djurens instinkter hafva blifvit blott obetydligt studerade med undantag af i Europa och Nordamerika, och ehuru vi alldeles icke känna de utdöda djurarternas instinkter, har jag dock blifvit öfverraskad af att finna, huru många grader af instinkt kunna upptäckas ända till de mest sammansatta. Förändringar i instinkter kunna stundom underlättas derigenom, att samma art har olika instinkter under olika perioder af sitt lif eller på olika årstider eller under olika yttre förhållanden, i hvilket fall antingen den ena eller den andra instinkten kan bibehållas genom naturligt urval. Och vi kunna visa, att sådana exempel på olikhet i instinkt hos samma art förekomma i naturen.

Såsom förhållandet äfven var med kroppsbildning, så är äfven i enlighet med min teori instinkten hos hvarje art nyttig för densamma, men har aldrig så vidt vi kunna döma framkallats uteslutande för andras nytta. Ett af de tydligaste exempel som jag känner på djur, som uträtta något, såsom det tyckes, uteslutande för andras bästa, skulle vara bladlössen, som frivilligt öfverlemna sina söta afsöndringar åt myrorna, hvilket Huber först observerat; att de göra det frivilligt, visa följande fakta. Jag aflägsnade alla myror från en grupp af omkring ett dussin bladlöss på en växt och hindrade deras sammanträffande i flera timmar. Efter denna tids förlopp märkte jag, att bladlössen behöfde afgifva sin saft. Jag observerade dem under någon tid genom lupp, men ingen lemnade någon afsöndring; jag kitlade och strök dem med ett hårstrå, på samma sätt som myrorna bruka göra med sina antenner, men de lemnade ändå icke någon afsöndring. Derefter släpte jag till dem en myra och den tycktes genast märka hvilken rik flock den hade upptäckt; den begynte då med sina antenner stryka på abdomen af först den ena och sedan den andra af bladlössen och så snart hvar och en kände antennen, lyfte hon omedelbarligen upp sin abdomen och gaf ifrån sig en klar droppe söt saft, som myran med begärlighet uppsög. Äfven de helt unga bladlössen förhöllo sig alldeles lika, hvilket visade, att handlingen var rent instinktmässig och icke resultatet af någon erfarenhet. Det är säkert enligt Hubers iakttagelser, att bladlössen icke hysa någon obenägenhet mot myrorna; om de senare icke finnas till hands, tvingas de till slut att ge ifrån sig sin afsöndring. Men då denna är särdeles klibbig, är det otvifvelaktigt en beqvämlighet för bladlössen, att den bortskaffas; derföre är det sannolikt, att afsöndringen icke sker uteslutande till myrornas fördel. Ehuru det icke finnes något bevis på, att något djur utför någon handling uteslutande till en annan arts bästa, försöker dock hvar och en att begagna till sin fördel andras instinkter, likasom hvar och en drager nytta af andra arters svagare kroppsbygnad. Vissa instinkter kunna på detta sätt betraktas såsom mindre fullkomliga, men då denna och andra frågor icke kunna behandlas utan att ingå i detaljer, förbigår jag dem här.

Då en viss grad af variation i instinkter under naturtillståndet och sådana variationers ärftlighet äro oundgängliga för det naturliga urvalets verksamhet, så vore det af nöden, att här så många exempel som möjligt anfördes; men brist på utrymme gör det omöjligt. Jag kan blott försäkra, att instinkter variera; så varierar till exempel vandringsinstinkten både i utsträckning och riktning och kan stundom helt och hållet upphöra. Så är äfven förhållandet med fåglarnas nästen, hvilka variera delvis, beroende af de valda lägenheterna och af naturen och temperaturen i den bebodda trakten, och ofta af för oss alldeles okända orsaker. Audubon har anfört flera märkvärdiga exempel på skilnaden i samma arts nästen i de norra och södra Förenta Staterna. Man har frågat: om instinkterna äro föränderliga, hvarföre har icke biet fått förmåga att nyttja annat bygnadsmaterial, då vax fattas? Men hvad annat material kunde bien använda? Jag har sett dem arbeta med vax, som varit försatt med konsjonell eller med fett. Andrew Knight iakttog, att hans bin i stället för att med möda samla pollenkorn begagnade ett kitt af vax och terpentin, hvarmed han hade betäckt några barklösa träd. Det har nyligen blifvit visadt, att bin i stället för att söka blommor för frömjölet gerna begagna ett helt annat ämne nämligen hafremjöl. — Fruktan för en viss fiende är helt säkert en instinktmässig egenskap, såsom man kan se hos fågelungar, ehuru den stärkes af erfarenheten och genom iakttagelse af den fruktan andra djur hysa för samma fiende. Djur som bebo aflägsna öar hysa ingen fruktan för menniskan och förvärfva den blott småningom, såsom jag har visat på ett annat ställe, och derpå se vi äfven ett exempel i England deruti, att de stora fåglarna äro vida mera vilda än de små, ty de stora fåglarna hafva blifvit mest förföljda af menniskorna. Vi kunna med tillförsigt antaga de större fåglarnas vildhet bero på denna orsak, ty på en obebodd ö äro stora fåglar icke mera rädda än små, och skatan, så skygg i England, är tam i Norge liksom kråkan i Egypten.

Att själsegenskaperna hos djur af samma slag födda i naturtillståndet variera mycket, kunde visas med många fakta. Flera fall kunde också anföras på tillfälliga och sällsamma vanor hos djur, hvilka om de äro fördelaktiga för arterna kunna genom naturligt urval gifva upphof till nya instinkter. Men jag inser väl, att dessa allmänna påståenden utan detaljerade fakta göra blott svag verkan på läsaren. Jag kan blott upprepa min försäkran, att jag icke talar utan tydliga bevis.



\section[Ärfda förändringar]{Ärfda förändringar i vanor eller instinkt hos tämda djur.}

Möjligheten eller till och med sannolikheten af ärfda förändringar i instinkt i ett naturtillstånd skall framträda starkare, om vi i korthet betrakta några få fall under tama tillståndet. Vi skola på detta sätt vara i stånd att se, hvilken rol vana och urval af så kallade tillfälliga eller spontana variationer hafva spelat vid modifieringen af våra husdjurs själsegenskaper. Det är bekant huru mycket husdjuren variera till själsegenskaper. Af kattor till exempel fångar den ena endast råttor, den andra möss, och dessa egenskaper gå i arf, såsom kändt är. Enligt S:t John hemförde en katt alltid vildfågel, en annan harar eller kaniner, och en annan jagade i sumpig mark och fångade nattetid snäppor och beckasiner. En mängd egendomliga exempel kunde anföras på ärftligheten af flera olikheter i sinnesart och smak eller benägenhet för de besynnerligaste knep, i förening med vissa sinnesstämningar eller tidsperioder. Låt oss betrakta de olika hundraserna; vi kunna icke betvifla, att unga rapphönshundar (jag har sjelf sett ett slående exempel) stundom första gången de äro ute gå framför och till och med bistå andra äldre hundar; egenskapen att drifva upp vildbråd är säkerligen till en viss grad ärftlig hos raser som äro dresserade dertill, och benägenheten att springa rundt omkring fårhjorden i stället för vid sidan går äfvenledes i arf hos fårhundar. Jag kan icke finna, att dessa handlingar, utförda utan erfarenhet af de unga och nästan på samma sätt af hvarje individ, och med brinnande lust af hvarje ras och utan kännedom om ändamålet — ty den unga rapphönshunden vet lika så litet, att han genom att stå hjelper sin herre, som den hvita fjärilen vet hvarföre han lägger sina ägg på kålbladen — jag kan icke finna, att dessa handlingar skilja sig väsentligt från verkliga instinkter. Om vi såge en varg, ung och utan dressyr, så snart den vädrat sitt rof, stå orörlig som en bildstod och sedan med en egendomlig hållning sakta smyga framåt, och en annan varg springa omkring en hjord af rådjur i stället för vid sidan af dem och drifva dem till en aflägsen punkt, skulle vi helt säkert kalla dessa handlingar instinktmässiga. Tama instinkter, såsom man kunde kalla dem, äro säkerligen vida mindre fixa och oföränderliga än naturliga instinkter, men de hafva stått under inverkan af ett vida mindre strängt urval och hafva gått i arf under en ojemförligt kortare period och under mindre oförändrade lefnadsförhållanden.

Huru strängt dessa tama instinkter, vanor och böjelser gå i arf och huru egendomligt de blandas, visas väl då skilda hundraser kroaseras. Sålunda är det bekant, att en kroasering med bulldoggen har under många generationer inverkat på vindthundens mod och uthållighet, och en kroasering med en vindthund har gifvit en hel familj af fårhundar en böjelse att jaga harar. Dessa tama instinkter, på detta sätt pröfvade genom kroasering, likna naturliga instinkter, hvilka på likartadt sätt blifva egendomligt blandade och under en lång tid visa spår af båda föräldrarnas instinkter. Le Roy beskrifver till exempel en hund, hvars farfar var en varg och denna hund visade spår af sin vilda härkomst blott på ett sätt, nämligen att han icke gick i rät linie till sin herre, då denne ropade honom.

Tama instinkter betecknas stundom såsom handlingar, hvilka hafva blifvit ärfda blott från en länge fortsatt och tvungen vana, men detta är icke sant. Ingen skulle väl någonsin hafva tänkt på att lära, eller kunde hafva lärt tumletten att tumla — en handling, hvilken jag har sett utföras af unga fåglar, hvilka aldrig sett en dufva tumla. Vi kunna tänka oss, att någon dufva visade en svag böjelse för denna sällsamma vana, och att ett länge fortsatt urval af de bästa individer under på hvarandra följande generationer har gjort tumletterna till hvad de nu äro, och nära Glasgow finnas hustumletter, såsom jag har hört af Mr Brent, hvilka icke kunna flyga aderton tum utan att tumla om. Man kan betvifla, att någon skulle tänkt på att lära en hund göra stånd, om icke någon hade af naturen visat benägenhet dertill, och man känner att detta tillfälligtvis inträffar, såsom jag såg en gång hos en ren gräfsvinshund; ”ståndet” är i mångas tanke blott en förlängning af det uppehåll ett djur gör, då det bereder sig att störta öfver sitt rof. Så snart en böjelse att göra stånd en gång visat sig, fullbordades verket snart genom metodiskt urval och de ärfda verkningarna af dressyr hos hvar och en af de följande generationerna, och ett omedvetet urval fortgår ännu, derigenom att hvar och en jägare utan afsigt att förädla rasen försöker skaffa sig de hundar som jaga bäst och bäst göra stånd. Å andra sidan har i många fall vanan gjort tillfyllest. Intet djur är väl i de flesta fall svårare att tämja än den vilda kaninens ungar, och intet djur är väl tamare än den tama kaninens ungar, men jag kan näppeligen antaga, att de tama kaninerna blifvit utvalda blott för deras tamhet och derföre måste vi tro, att åtminstone största delen af den ärfda förändringen från ytterlig vildhet till ytterlig tamhet beror på vana och en långvarig sträng fångenskap.

Naturliga instinkter gå förlorade i tama tillståndet. Ett märkvärdigt exempel derpå se vi hos de hönsraser, hvilka sällan eller aldrig vilja ligga på sina ägg. Det är blott den dagliga vanan som hindrar oss att se, i huru hög grad våra husdjurs själsegenskaper beständigt blifvit modifierade. Det är knappt möjligt att betvifla, att menniskans tillgifvenhet har blifvit instinktmässig hos hunden. Alla vargar, räfvar, schakaler och arter af kattslägtet äro då de hållas tama mycket benägna att angripa höns, får och svin, och denna benägenhet har äfven visat sig obotlig hos hundar, hvilka som ungar blifvit hemförda från trakter, der vildarna icke hålla sådana djur tama, såsom Eldslandet och Australien. Å andra sidan, huru sällan behöfver man lära våra tama hundar äfven då de äro helt unga, att icke anfalla höns, får och svin! Otvifvelaktigt göra de någon gång ett sådant anfall, men då få de stryk, och om de äro oförbätterliga, så dödas de, och på detta sätt har vana och till en viss grad urval sannolikt bidragit till att genom arf civilisera våra hundar. Å andra sidan hafva kycklingar helt och hållet genom vana förlorat den rädsla för hundar och kattor som otvifvelaktigt ursprungligen varit instinkt hos dem. Kapten Hutton har berättat för mig, att de unga kycklingarna af stamarten Gallus Bankiva, som man låter kläckas af en vanlig höna, i början äro fullkomligt vilda, och så är äfven förhållandet med de unga fasaner i England, som man låtit en höna kläcka. Kycklingarna hafva visserligen icke förlorat all rädsla, blott för hundar och kattor; ty om hönan genom sitt kacklande tillkännagifver någon fara, springa de ifrån hennes skydd och dölja sig i kringliggande gräs eller snår; så göra isynnerhet kalkonerna och det sker tydligen instinktmässigt för att lemna modern tillfälle att fly, såsom vi se hos vilda fåglar, som lefva på marken. Men denna instinkt, som qvarstår hos våra kycklingar, har blifvit onyttig, då hönan i följd af bristande öfning nästan helt och hållet förlorat förmågan att flyga.

Häraf kunna vi draga den slutsatsen, att under tämjningen instinkter blifvit förvärfvade och naturliga instinkter gått förlorade dels genom vana och dels genom menniskans under flera generationer fortgående accumulativa urval af särskilda böjelser och egenskaper, hvilka ursprungligen uppträdde såsom hvad vi i vår okunnighet kalla slump. I vissa fall har en tvungen vana allena varit tillräcklig att åstadkomma ärfda förändringar i själsegenskaper, i andra fall har en tvungen vana gjort ingenting och allt är resultatet af urval, både metodiskt och omedvetet, men i de flesta fall hafva vana och urval sannolikt verkat gemensamt.



\section{Naturliga instinkter.}

Huru instinkten i naturtillståndet modifieras genom urval, kunna vi måhända bäst förstå genom att betrakta några få speciela fall. Jag vill välja blott tre af dem, som jag i ett kommande arbete ämnar behandla, nämligen den instinkt, som drifver göken att lägga sina ägg i andra fåglars bon; den instinkt, som drifver vissa myror att göra slafvar, och biens instinkt att bygga celler. Dessa två sistnämda instinkter hafva i allmänhet och det med rätta af naturforskare betraktats såsom de underbaraste kända exempel af instinkt.

Gökens instinkt. Några naturforskare antaga, att den mera omedelbara orsaken till gökens instinkt är den omständigheten, att honan lägger sina ägg icke för hvarje dag utan med två eller tre dagars mellantid, så att, om hon skulle bygga sitt eget näste och sjelf ligga på sina ägg, i samma näste skulle finnas ägg och ungar af olika ålder, så vida hon icke skulle dröja med liggningen tills alla ägg voro värpta. Om så vore förhållandet, skulle liggnings- och kläckningsprocessen bli särdeles lång, hvilket vore olämpligt, då göken flyttar temligen tidigt, och de först kläckta ungarna skulle få uppfödas af hannen allena. Men den amerikanska göken befinner sig just i denna belägenhet, ty den har sitt eget bo och har på samma gång ägg och ungar, som utkläckas efter hand. Man har både påstått och förnekat, att den amerikanska göken tillfälligtvis lägger sina ägg i andra fåglars nästen, men jag har nyligen hört af doktor Merrell, att han en gång i Illinois fann en ung gök och en ung skrika i ett bo, som tillhörde en nötskrika (Garrulus cristatus) och då de båda voro fullfjädrade, kunde något misstag svårligen begås. Jag kunde också gifva flera exempel på åtskilliga fåglar, som tillfälligtvis lägga sina ägg i andra fåglars bon. Låt oss nu antaga, att den gamla stamfadern till vår europeiska gök hade den amerikanska gökens vanor och att den tillfälligtvis lade ett ägg i en annan fågels näste. Om den gamla fågeln hade någon fördel af denna tillfälliga vana genom att kunna flytta tidigare eller af någon annan orsak, eller om den unga blef kraftigare genom att draga nytta af en annan arts missledda instinkt, än om han blifvit kläckt af sin egen moder, hvilken skulle varit besvärad af att på samma gång hafva ägg och ungar af olika ålder, så skulle antingen den gamla eller den unga vinna derpå. Analogien gifver oss anledning att tro, att de så uppfödda ungarna skulle kunna antaga moderns tillfälliga och afvikande vanor och i sin tur lägga sina ägg i andra fåglars bon och sålunda få sina ungar bättre uppfostrade. Genom en fortsatt likartad process tror jag gökens besynnerliga instinkt hafva uppkommit. Man har också nyligen påstått, att göken tillfälligtvis lägger sina ägg på bara marken, kläcker dem och uppföder sina ungar; denna sällsynta och sällsamma händelse är tydligen ett fall af återgång till den längesedan förlorade ursprungliga instinkten att bygga bon.

Man har invändt, att jag icke anmärkt andra instinkter hos göken, hvilka med orätt påstås stå i nödvändigt sammanhang härmed. Men spekulationen öfver en instinkt känd blott hos en enda art är i alla fall onyttig, ty vi hafva inga fakta till vägledning. Ända in i nyaste tider hafva blott den europeiska och den icke parasitiska amerikanska gökens instinkter varit kända, men tack vare mr E. Ramsays iakttagelser känna vi nu något om tre australiska arter, som lägga sina ägg i andra fåglars nästen. De vigtigaste punkterna äro tre: först att göken med få undantag lägger blott ett ägg i ett näste, så att den stora och glupska ungen må kunna få riklig föda; vidare att äggen äro märkvärdigt små, icke större än lärkans, som är ungefär en fjerdedel så stor som göken. För det tredje har den unga göken strax efter födelsen instinkten, styrkan och en väl skapad näbb att utkasta sina fosterbröder, hvilka då dö af köld och hunger. Man har djerft påstått, att detta är en välvillig anordning, så att den unga göken kan få tillräcklig föda och hans fosterbröder dö innan de fått för mycken känsla.

Vi skola nu vända oss till de australiska arterna; ehuru dessa fåglar i allmänhet lägga blott ett ägg i ett bo, är det icke sällsynt att finna två till tre ägg i samma näste. Bronsgökens ägg variera mycket i storlek från åtta till tio linier i längd. Om det hade varit någon fördel för denna art att hafva ännu mindre ägg för att bedraga sina fosterföräldrar eller, hvilket är mera sannolikt, att kläckas på kortare tid (ty man påstår att äggens storlek och liggningstiden stå i ett visst förhållande till hvarandra), så är det icke svårt att tro, att en ras eller art har kunnat bildas, som skulle lagt allt mindre och mindre ägg, ty dessa skulle med större säkerhet blifvit kläckta och uppfödda. Mr Ramsay anmärker, att två af de australiska gökarna, om de lägga sina ägg i ett öppet bo, visa ett afgjordt företräde åt sådana bon, som innehålla ägg af lika färg med deras egna. De europeiska arterna visa säkerligen en viss tendens till en likartad instinkt, men vika icke sällan ifrån den, och lägga sina mörka och blekt färgade ägg i löfsångarens bon med dess ljusa grönblå ägg. Hade vår gök oföränderligen visat denna instinkt, skulle den säkerligen räknats till de andra, hvilka såsom det antages äro förvärfvade; den australiska bronsgökens ägg variera enligt Ramsay utomordentligt i färg; så att i detta hänseende likasom i storlek det naturliga urvalet helt säkert har kunnat befästa någon fördelaktig variation.

Man påstår, såsom vi hafva nämt, att de unga gökarna utkasta sina fosterbröder, men angående denna sak bör anmärkas, att mr Gould, som har fästat särskild uppmärksamhet på detta ämne, är öfvertygad om att denna tro är ett misstag. Han försäkrar, att de unga fåglarna utkastas i allmänhet under de tre första dagarna, då gökungen ännu är alldeles maktlös; han påstår, att gökungen genom sitt skrik eller på något annat sätt utöfvar ett sådant välde öfver sina fosterföräldrar, att de ge honom allena någon föda, så att de andra hungra ihjäl och sedan kastas ut af de gamla fåglarna liksom äggskalen eller exkrementerna. Han medgifver likväl, att gökungen, då den blifvit äldre och starkare, kan hafva kraft och kanske instinkt att kasta ut sina fosterbröder, om de lyckas undgå hungersdöden under de första dagarna efter födseln. Mr Ramsay har kommit till samma slutsats angående de australiska arterna: han påstår, att gökungen är i början ett litet fett, hjelplöst kreatur, men ”då han växer hastigt, fyller han snart upp hela boet och dess olyckliga följeslagare, antingen qväfda af dess tyngd eller uthungrade genom dess glupskhet, utkastas af föräldrarna”. Icke destomindre finnas så många bevis både gamla och nya att göken utkastar sina fosterbröder, att det näppeligen kan betviflas. Om det nu är af stor vigt för den unga göken att få så mycken föda som möjligt strax efter födseln, kan jag icke finna någon omöjlighet för, att den genom flera generationer kan förvärfva den vana, den styrka, den skapnad, som är bäst tjenlig för fosterbrödernas utkastande; ty de gökungar, som hade denna vana och denna skapnad skulle bli bäst födda och säkrast uppfostrade. Jag kan icke finna större omöjlighet deruti än att fågelungar förvärfva instinkten att bryta äggskalet och sina temporära hårda spetsar på näbbarna för detta ändamål — eller att ormungar i sina öfverkäkar få en bortfallande skarp tand för att skära igenom det sega äggskalet, såsom Owen har anmärkt. Ty om hvarje del är underkastad individuel variation vid någon ålder och variationerna gå i arf i motsvarande ålder — satser som icke kunna bestridas — så kunna ungens instinkter och skapnad modifieras lika väl som den fullväxtes, och båda fallen måste stå och falla med hela teorien om det naturliga urvalet.

Denna vana, att tillfälligtvis lägga sina ägg i andra fåglars bon antingen af samma eller olika arter, är icke ovanlig hos hönsfåglarna och detta förklarar måhända ursprunget till en egendomlig instinkt hos en närbeslägtad grupp, strutsarna. Ty flera strutshonor förena sig och lägga först några få ägg i ett näste och sedan i ett annat, och dessa kläckas af hannarna. Denna instinkt kan sannolikt förklaras af det faktum, att honorna lägga ett stort antal ägg, men likasom göken med två eller tre dagars mellantid. Den amerikanska strutsens instinkt har likväl icke blifvit fullkomnad, ty ett stort antal ägg ligga strödda på slätterna, så att jag på en dag uppsamlade icke mindre än tjugu öfvergifna och förderfvade ägg.

Många bin äro parasitiska och lägga regelbundet sina ägg i andra bins bon. Detta är mera anmärkningsvärdt än gökens instinkt, ty dessa bin hafva icke blott sina instinkter utan äfven sin kroppsbildning modifierade i öfverensstämmelse med sina parasitiska vanor, ty de ega icke de pollensamlande organerna, hvilka varit oumbärliga, om de måst samla föda åt sina ungar. Några arter af Sphegidæ (getingliknande insekter) äro på samma sätt parasitiska på andra arter; Tachytes nigra gör i allmänhet sin egen håla och förser den med byte för sina egna larver, men M. Fabre har dock nyligen gifvit oss anledning att tro, att, om denna insekt finner en håla redan färdiggjord och provianterad af en annan geting, han begagnar prisen och blifver för tillfället parasit. I detta fall, såsom i det antagna exemplet om göken, kan jag ej se någon svårighet för det naturliga urvalet att göra en tillfällig vana permanent, om den är af nytta för arten och om insekten, hvars näste och förråd på detta nedriga vis tages i beslag, icke derigenom dör ut.

Myrans instinkt att göra slafvar. Denna märkvärdiga instinkt upptäcktes först hos Formica (Polyerges) rufescens af Pierre Huber, om möjligt skarpare observatör än fadern. Denna myra är helt och hållet beroende af sina slafvar; utan deras tillhjelp skulle arten säkerligen dö ut på ett enda år. Hannarna och de fruktsamma honorna arbeta alldeles icke, och arbetarna eller de sterila honorna, så energiska och modiga de än äro i slaffångst, göra aldrig någonting annat. De äro odugliga att bygga sina egna bon och uppföda sina egna larver. Då de gamla nästena befinnas olämpliga och de behöfva flytta, är det slafvarna som bestämma flyttningen, och de rent af bära sina herrar emellan käkarna. Så ytterligt hjelplösa äro husbönderna, att då Huber instängde trettio af dem utan någon slaf men med rikligt förråd af deras mest omtyckta födoämnen och med deras egna larver och puppor, hvilka borde egga dem till arbete, gjorde de ingenting; de kunde icke en gång föda sig sjelfva och många omkommo af hunger. Huber insläpte då en enda slaf (F. fusca), som genast grep verket an, födde och räddade de öfverlefvande, bygde några celler, vårdade larverna och stälde allt till rätta. Hvad finnes väl utomordentligare än dessa fakta? Om vi icke kände någon annan myra med samma instinkt, skulle det varit hopplöst att spekulera öfver, huru en så underbar instinkt kunnat få en sådan fulländning.

Hos en annan art, Formica sanguinea, upptäckte P. Huber likaledes samma instinkt. Denna art finnes i södra delarna af England, och dess vanor hafva blifvit studerade af mr F. Smith, vid British Museum, hvilken jag är mycken tack skyldig för upplysningar i detta och andra ämnen. Ehuru fullt litande på Hubers och Smiths försäkringar kunde jag icke frigöra mig ifrån något tvifvel då jag skulle inlåta mig på detta ämne, och det bör väl ursäktas hvar och en om han betviflar sanningen af en så utomordentlig och förhatlig instinkt som den att göra slafvar. Derföre vill jag något i detalj anföra de iakttagelser jag gjort. Jag öppnade fjorton bon tillhöriga Formica sanguinea och fann ett ringa antal slafvar i hvarje. Hannar och fruktsamma honor af slafarten (F. fusca) finnas blott i deras egna samhällen och hafva aldrig observerats i den andra artens bon. Slafvarna äro svarta och icke hälften så stora som deras röda husbönder, så att kontrasten i deras utseende är stor. Om boet svagt rubbas, komma slafvarna ut och äro likasom deras herrar mycket oroliga och försvara boet; om boet bringas i större oordning och larver och puppor komma i dagen, arbeta slafvarna ifrigt jemte sina herrar att bära dem bort till någon säker plats. Deraf är klart, att slafvarna känna sig fullkomligt såsom hemma hos sig. Under Juni och Juli månader har jag under tre år å rad i flera timmar observerat flera bon i Surrey och Sussex och såg aldrig någon slaf gå hvarken ut eller in. Då slafvarna under dessa månader äro mycket få till antal, tänkte jag, att de förhålla sig olika då de äro talrikare; men mr Smith upplyser mig, att han har observerat nästena på olika timmar under Maj, Juni och Augusti både i Surrey och Hampshire och har aldrig sett slafvar hvarken gå ut eller in i boet, oaktadt de i Augusti äro mycket talrika. Han betraktar dem derföre såsom husslafvar. Herrarna deremot ses oupphörligen bära in materialier för boet och föda af alla slag. År 1860 i Juli månad kom jag likväl till ett samhälle med en ovanligt stor mängd slafvar och jag såg några få slafvar i sällskap med sina herrar lemna boet och gå samma väg till en hög skotsk tall på 25 alnars afstånd, hvilken de gemensamt bestego sannolikt för att söka bladlöss eller sköldlöss. Enligt Huber, som hade rikligt tillfälle till iakttagelser, arbeta slafvarna i Schweitz tillsammans med husbönderna vid boens uppbyggande, och de allena öppna och stänga dörrarna morgon och afton, men såsom Huber uttryckligen anmärker, deras hufvudsakliga sysselsättning är att söka bladlöss. Denna skilnad i vanor hos husbönder och slafvar i de båda länderna beror blott derpå, att slafvarna fångas i större antal i Schweitz än i England.

En dag var jag nog lycklig att bevittna en flyttning af F. sanguinea från ett bo till ett annat, och det var ett särdeles interessant skådespel att se herrarna omsorgsfullt bära sina slafvar emellan sina käkar i stället för att bäras af dem, såsom förhållandet är med F. rufescens. En annan dag observerade jag till min förvåning ungefär ett tjog myror af F. sanguinea gå omkring på samma fläck tydligen icke för att söka föda; de närmade sig en oberoende koloni af slafmyror (F. fusca) och blefvo kraftigt tillbakadrifna; stundom hängde sig trenne af dessa myror på sina angripares ben. De sednare dödade obarmhertigt sina små motståndare och buro deras lik såsom föda hem till sitt bo på tjugunio alnars afstånd; men de lyckades icke få några puppor att uppfostra till slafvar. Jag gräfde upp några puppor af F. fusca från en annan myrstack och lade dem ned på bara marken nära stridsplatsen; dessa anammades med begärlighet och fördes bort af tyrannerna, som kanhända inbillade sig, att de slutligen blifvit segrare i sista striden.

På samma gång lade jag på samma plats några puppor af en annan art, F. flava med några få fullväxta af dessa små gula myror ännu fasthängande vid fragmenten af sitt gamla bo. Denna art begagnas stundom ehuru sällan till slafvar, såsom M:r Smith har beskrifvit. Ehuru arten är så liten, är den mycket modig och jag har sett den djerft anfalla andra myror. En gång fann jag till min förvåning en oberoende koloni af F. flava under en sten bredvid ett bo tillhörande F. sanguinea och då jag af en händelse bragte båda boen i oordning anföllo de små myrorna sina stora grannar med ett förvånande mod. Jag var nyfiken att få veta, om F. sanguinea kunde skilja pupporna af F. fusca, som de vanligen begagna till slafvar från pupporna af den lilla vilda F. flava, som de sällan taga till fånga, och det var klart, att de genast skilde dem. Ty vi hafva sett, att de begärligt och ögonblickligen grepo pupporna af F. fusca, hvaremot de blefvo mycket förskräckta, då de kommo i närheten af pupporna af F. flava eller till och med blotta jorden från deras bo, så att de genast sprungo sin väg, men efter ungefär en qvarts timma, så snart de små gula myrorna hade krupit bort, togo de mod till sig och buro bort pupporna.

En afton besökte jag en annan koloni af F. sanguinea och fann ett antal af dessa myror på hemvägen med lik af F. fusca (hvilket bevisade att det icke var någon flyttning) och en mängd puppor. Jag såg en lång rad af myror belastade med byte, hvilken räckte fyrtio alnar tillbaka till en stor tjock ljungknippa, der den sista individen af F. sanguinea kröp fram bärande en puppa, men jag var icke i stånd att finna det ödelagda nästet i ljungbusken. Nästet måste likväl funnits nära till hands, ty två eller tre individer af F. fusca sprungo omkring i den största oro och en satt orörlig på en qvist med en puppa i munnen, en bild af förtviflan öfver sitt sköflade hem.

Sådana äro de fakta, som jag kan berätta om denna underbara instinkt att göra slafvar, ehuru den icke behöft någon bekräftelse af mig. Vi böra observera, hvilken motsats i vanor F. sanguinea företer till den kontinentala F. rufescens. Den senare bygger icke sjelf sitt bo, bestämmer icke sina egna flyttningar, samlar icke föda för sig sjelf och sina ungar och kan icke ens nära sig sjelf; den är helt och hållet beroende af sina talrika slafvar. Formica sanguinea deremot eger ett mycket mindre antal slafvar och i den tidigare delen af sommaren blott få; husbönderna bestämma, när och hvar ett nytt näste skall byggas, och när de flytta bära husbönderna sina slafvar. Både i Schweitz och England tyckas slafvarna hafva uteslutande omsorg om larverna och husbönderna gå ensamma ut på expeditioner. I Schweitz arbeta herrar och slafvar tillsammans och anskaffa bygnadsmaterial; båda, men i synnerhet slafvarna, besöka och mjölka, såsom man kunde kalla det, sina bladlöss och båda samla föda för hela samhället. I England lemna herrarna allena boet för att samla bygnadsmaterial och föda åt sig sjelfva, sina slafvar och sina larver, så att i detta land slafvarna göra sina husbönder vida mindre tjenster än i Schweitz.

Jag vill icke våga en gissning på hvad vis denna instinkt hos F. sanguinea bildats. Men då myror, som icke göra slafvar, såsom jag har sett, bära in puppor af andra arter, som händelsevis ligga strödda i närheten af deras bo, är det möjligt, att sådana puppor ursprungligen insamlade till föda blifvit utvecklade, och de främmande myrorna, som på detta sätt afsigtslöst blifvit uppfödda, följa då sin egen instinkt och uträtta hvad arbete de kunna. Om deras närvaro befinnes nyttig för arten som upptagit dem — om det vore nyttigare för denna art att taga arbetare till fånga än att framföda dem — kan denna vana att insamla puppor ursprungligen såsom födoämnen genom naturligt urval förstärkas och göras permanent för ett helt annat ändamål, nämligen att uppfostra slafvar. Då instinkten en gång är förvärfvad, äfven om den är uppdrifven till ännu mindre grad än hos vår F. sanguinea, hvilken såsom vi sett har mycket mindre hjelp af sina slafvar än den schweitziska, kan det naturliga urvalet föröka och modifiera instinkten — alltid under förutsättning att hvarje modifikation är af nytta för arten — till dess en myra bildats, så föraktligt beroende af sina slafvar som F. rufescens.

Kupbiets instinkt att bygga celler. Jag vill här icke ingå i fina detaljer i detta ämne, utan blott gifva de yttre dragen af de slutsatser till hvilka jag kommit. Det måste vara en inskränkt person, som kan utan entusiastisk beundran undersöka den fina strukturen af en vaxkaka så väl lämpad för sitt ändamål. Vi höra af matematikern, att bien praktiskt löst ett svårt problem och hafva bygt sina celler af särskild form för att rymma största möjliga mängd honing med minsta möjliga åtgång på det dyrbara vaxet vid deras konstruktion. Man har anmärkt, att en skicklig arbetare med lämpliga mått och verktyg skulle finna det mycket svårt att göra vaxceller af den verkliga formen, ehuru detta utföres af en hop bin i en mörk kupa. Man må antaga hvilken instinkt som helst, synes det dock först obegripligt, huru de kunna göra alla nödvändiga vinklar och plan, eller inse när de äro riktigt gjorda. Men svårigheten är icke så stor som den först synes; allt detta vackra arbete kan bevisas följa af ett fåtal enkla instinkter.

Jag föranleddes att närmare undersöka detta ämne af mr Waterhouse, som har visat, att cellens form står i nära förhållande till närvaron af angränsande celler och följande åsigt kan måhända betraktas såsom en modifikation af hans teori. Låt oss betrakta gradationsprincipen och se, om icke naturen uppenbarar för oss sin verkningsmetod. I ena ändan af en kort serie hafva vi humlor, som begagna sina gamla kokonger till uppsamlande af honing, stundom bifogande några korta vaxrör och likaledes äfven enstaka mycket oregelbundna rundade vaxceller. I andra ändan af serien hafva vi kupbiens celler stälda i ett dubbelt lager; hvarje cell är såsom bekant ett sexsidigt prisma med baserna af de sex sidorna inpassade i en tresidig pyramid bildad af tre romber. Dessa romber hafva vissa vinklar och de tre, som bilda den pyramidala basen af en enkel cell på ena sidan af vaxkakan ingå i sammansättningen af baserna till tre närliggande celler på andra sidan. I serien emellan den ytterliga fulländningen af kupbiets celler och enkelheten i humlans hafva vi den mexikanska artens celler, Melipona domestica, som Pierre Huber noggrant beskrifvit och afbildat. Melipona sjelf står i kroppsbildning emellan kupbiet och humlan, men är mera beslägtad med den senare; den formar en nästan regelbunden vaxkaka af cylindriska celler i hvilka ungarna kläckas och dessutom några stora vaxceller till honingsreservoarer. Dessa senare äro nästan sferiska och af ungefär lika storlek och de äro samlade till en oregelbunden massa. Men den vigtigaste omständigheten är, att alla dessa celler äro lagda så nära hvarandra, att de skulle skära hvarandra eller bryta in i hvarandra, om sfererna blifvit fullbordade; men detta sker aldrig, ty bien bygga fullkomligt flata väggar emellan de sferer som på detta sätt skulle skära hvarandra. Hvarje cell består derföre af en yttre sferisk del och af två, tre eller flera plana ytor, allteftersom cellen stöter intill två, tre eller flera celler. Då en cell stöter intill tre andra celler, hvilket mycket ofta inträffar, då cellerna äro af nästan samma storlek, äro de tre plana ytorna förenade till en pyramid, och denna pyramid är såsom Huber anmärkt tydligen en imitation i groft af de tresidiga pyramidala baserna på kupbiets celler. Likasom i kupbiets celler, så ingå äfven här de tre plana ytorna i en cell nödvändigt i konstruktionen af tre närliggande celler. Det är klart att Melipona spar in vax och, hvad som är vigtigare, arbete genom detta sätt att bygga, ty de plana väggarna emellan närliggande celler äro icke dubbla utan hafva samma tjocklek som de yttre sferiska delarna och hvarje plan del bildar ändock en del af två celler.

Då jag reflekterade häröfver, föreföll det mig, att om Melipona hade gjort sina sferer på gifvet afstånd från hvarandra, gjort dem lika stora och anordnat dem symmetriskt i ett dubbelt lager, så skulle den deraf uppkomna bygnaden sannolikt blifvit likaså fulländad som kupbiets kaka. På grund deraf skref jag till professor Miller i Cambridge och denne matematiker har försäkrat mig att nedanstående teorem, hemtadt från hans meddelanden, är korrekt.

Om ett antal sins emellan lika sferer beskrifvas med sina medelpunkter i två parallela lager, hvarje sfer med sin medelpunkt på ett afstånd af radien $\times \sqrt{2}$ eller radien $\times$ 1,41421 (eller något mindre) från medelpunkterna i de sex omgifvande sfererna i samma lager, och på samma afstånd från medelpunkterna i de närliggande sfererna i det andra parallela lagret; så bilda intersektionsplanen emellan sfererna i de båda lagren en dubbel rad af sexsidiga prismer förenade med pyramidala baser af tre romber och romberna och sidorna i prismerna hafva vinklar som äro fullkomligt lika med vinklarna på bikakans celler enligt de noggrannaste mätningar. Men jag hör af professor Wyman, som gjort mycket talrika noggranna mätningar, att fulländningen i biets verk blifvit mycket öfverdrifven, och han tillägger, att hvad än den typiska formen må vara på en cell, så är den aldrig, om någonsin realiserad.

Häraf kunna vi med säkerhet sluta, att om vi kunde obetydligt modifiera de instinkter som Melipona redan eger och som hos henne icke äro så underbara, skulle äfven hon bygga en kaka lika märkvärdigt fullkomlig som kupbiets. Vi måste antaga, att Melipona har förmåga att bilda sina celler rent sferiska och lika stora, och detta vore icke så besynnerligt, då vi redan se henne bygga så till en viss grad, och då vi se de rent cylindriska hålor många insekter göra i träd, tydligen genom att vrida sig kring en punkt. Vi måste antaga, att Melipona anordnar sina celler i plana lager, såsom hon redan gör med sina cylindriska celler, och vi måste vidare antaga, detta är den största svårigheten, att hon kan på något vis med noggranhet bedöma på hvad afstånd hon står från sina medarbetare, då flera arbeta på samma gång; men hon har redan förmåga att bedöma afstånd så till vida, att hon alltid beskrifver sina sferer så att skärningsytorna bli stora och förenar då skärningspunkterna med en plan yta. Vi måste vidare antaga, och dertill är ingen svårighet alls, att, sedan sexsidiga prismer bildats genom de i samma lager liggande sferernas skärningsplan, hon kan utdraga sexsidingen till hvarje önskad längd för att tjena till honingsreservoar, på samma sätt som humlorna fästa vaxcylindrar vid de runda mynningarna af deras gamla kokonger. Genom sådana modifikationer i instinkter, som i och för sig sjelfva icke äro så märkvärdiga — knappt mera än fåglarnas instinkt att bygga bon — tror jag kupbiet har genom naturligt urval förvärfvat sin oefterhärmliga förmåga som byggmästare.

Men denna teori kan bekräftas med experimenter. Följande Tegetmejers exempel, tog jag i sär två vaxkakor och lade emellan dem en lång, tjock rektangulär vaxskifva och bien begynte straxt att gräfva små runda gropar deri; under det de gjorde dessa gropar djupare utvidgade de dem allt mera och mera, till dess de voro förvandlade till grunda skålar, som för ögat sågo ut som verkliga sfersegmenter af ungefär en cells diameter. Det var särdeles interessant för mig att se, att hvarhelst två bin hade börjat gräfva dessa urhålkningar i hvarandras närhet, hade de börjat sitt arbete på sådant afstånd från hvarandra, att då urhålkningen fått den ofvannämda vidden (ungefär af en vanlig cell) och ett djup af ungefär en sjettedel af sferens diameter, urhålkningarnas kanter måste skära hvarandra. Så snart detta inträffade, slutade bien med gräfningen och började bygga plana vaxväggar i skärningsytorna, så att hvarje sexsidigt prisma bygdes uppå den ojemna kanten af en slät urhålkning, och icke såsom i vanliga celler på de raka kanterna af en tresidig pyramid.

Derefter insatte jag i kupan i stället för ett tjockt fyrkantigt stycke en tunn och smal, blott knifbladstjock skifva färgad med konsjonell. Bien började ögonblickligen gräfva små gropar nära hvarandra på båda sidor likasom förut, men vaxskifvan var så tunn, att urhålkningarnas bottnar skulle hafva brutit in i hvarandra från de motsatta sidorna, om urhålkningen fortgått till samma djup som i föregående experiment. Bien läto likväl detta icke inträffa, utan upphörde med gräfningen i tillbörlig tid, så att groparna fingo bottnar med flata sidor, så snart de hade gått något på djupet, och dessa sidor, som bildades af en liten tunn skifva af orördt konsjonellfärgadt vax, voro så vidt ögat kunde döma belägna längs skärningsplanen emellan urhålkningarna å båda sidor af vaxstycket. På somliga ställen hade små delar af en rombisk skifva lemnats qvar emellan de motsatta urhålkningarna, på andra ställen större, men på grund af sakens onaturliga ordning hade arbetet icke blifvit prydligt utfördt. Bien måste hafva arbetat i ungefär samma proportion på båda sidor af vaxskifvan för att kunna lemna plana skifvor emellan urhålkningarna genom att sluta sitt arbete i skärningsplanen.

Om vi betänka huru böjligt tunt vax är, ser jag icke någon svårighet för bien att under arbete på båda sidor om en vaxskifva sjelfva märka, när de hafva gnagt bort vaxet till lämplig tjocklek och att då sluta sitt arbete. I vanliga kakor synas bien icke alltid lyckas arbeta med jemna steg på båda sidor, ty jag har iakttagit halffärdiga romber vid basen af en nyss börjad cell, hvilka voro svagt konkava på ena sidan, der bien antagligen gräft för hastigt, och konvexa på den andra, der bien hade arbetat mindre fort. I ett väl utprägladt fall lade jag kakan tillbaka i kupan och lät bien fortsätta arbetet en kort tid, undersökte cellen ånyo och fann den rombiska skifvan fullbordad och fullkomligt plan; den lilla skifvan var så ytterst tunn, att detta icke kunnat ske genom afgnagning på den konvexa sidan; och jag misstänker att bien i sådana fall stå på hvar sin sida och böja det smidiga och varma vaxet (hvilket går lätt för sig) till dess det blir fullkomligt plant.

Af försöket med den konsjonellfärgade vaxskifvan kunna vi se, att om de skulle bygga för sig sjelfva en tunn vaxskifva, de kunna göra sina celler af lämplig form, om de stå på vederbörligt afstånd från hvarandra, arbeta med jemna steg och bemöda sig att göra likstora sferiska hålor utan att tillåta sfererna bryta in i hvarandra. Om man undersöker kanten af en vaxkaka som är under arbete, kan man tydligen se, att bien först göra en ojemn kant rundt omkring kakan och urhålka den på båda sidor, alltjemt arbetande kretsformigt under gräfningen. De göra icke hela den tresidiga pyramiden vid basen af hvarje cell färdig på samma gång, utan blott den ena rombiska skifvan, som står på yttre kanten, eller två alltefter omständigheterna, och de fullborda icke de öfre spetsarna af romberna förr än de sexsidiga väggarna äro påbörjade. Några af dessa uppgifter öfverensstämma visserligen icke med den ryktbare François Hubers, men jag är öfvertygad om att de äro riktiga, och om utrymmet tilläte, skulle jag visa, att de stå i öfverensstämmelse med min teori.

Hubers uppgift, att den alldra första cellen urhålkas i ett litet vaxstycke med parallela sidor, är icke enligt hvad jag har sett fullkomligt riktig; första början har alltid varit en liten hufva af vax, men jag vill icke här ingå i dessa detaljer. Vi se hvilken vigtig rol urhålkningen spelar vid cellbyggandet, men det vore ett stort misstag att tro, att bien icke kunna bygga upp en ojemn vägg i det riktiga läget, längs skärningsplanet emellan två närgränsande sferer. Jag har flera prof, som tydligen bevisa detta. Till och med i den ojemna kanten omkring en vaxkaka, som håller på att tillökas, observeras stundom krökningar som i läge motsvara de rombiska basalplanen i de blifvande cellerna. Men denna vaxkant skall i alla händelser bortarbetas genom stora urhålkningar å båda sidor. Biens sätt att bygga är egendomligt; de göra alltid den första ojemna vaxväggen tio till tjugu gånger tjockare än de utomordentligt tunna cellväggarna som slutligen lemnas qvar. Vi skola förstå huru de arbeta, om vi tänka oss att murare först upplägga en bred vägg af cement och derefter börja hugga bort den å båda sidor nära marken tills en glatt mycket tunn vägg står qvar i midten, under det murarna alltjemt stapla upp det borthugna cementet jemte nytt cement på spetsen af vallen. Vi få på detta sätt en tunn vägg, som alltjemt växer uppåt, alltid krönt af en tjock list. Från alla celler, både de nyss påbörjade och de färdiggjorda, hvilka alla äro omgifna af en stark vaxlist, kunna bien skocktals krypa öfver kakan utan att skada de fina sexsidiga väggarna. Dessa variera mycket i tjocklek såsom Professor Miller benäget meddelat mig; i medeltal af tolf mätningar nära kanten af kakan äro de 1/353 tum tjocka, hvaremot de romboida skifvorna i basalpyramiden äro tjockare nästan i en proportion af tre till två; tjuguen mätningar gåfvo ett medeltal af 1/229 tum i tjocklek. Genom detta egendomliga bygnadssätt får kakan alltjemt den erforderliga styrkan med den största möjliga hushållning med vaxet.

Svårigheten att fatta huru cellerna byggas synes först ökas deraf, att en mängd bin arbeta tillsammans, i det att ett bi som en kort tid arbetat på en cell går till en annan, så att såsom Huber anmärker ett tjog individer arbeta till och med på början af den första cellen. Det har varit mig möjligt att praktiskt visa detta förhållande genom att betäcka kanterna af de sexsidiga väggarna i en enkel cell, eller yttre kanten af brädden på en vaxkaka med ett ytterst tunt lager af smält rödfärgadt vax och jag fann hvarje gång, att färgen spriddes särdeles fint af bien — så fint som någon målare kunnat göra med sin pensel — i det de togo små delar af det färgade vaxet från sitt ställe och förarbetade det på de tillväxande kanterna af cellerna rundt omkring. Bygnadsarbetet synes mig vara ett slags täflan emellan många bin, hvilka alla stå på samma afstånd från hvarandra, alla försöka att bilda lika stora sferer och derefter bygga upp eller lemna qvar orörda skärningsplanen af dessa sferer. Det var verkligen egendomligt att se i vissa svåra fall, då två stycken af en kaka stötte tillsammans i vinkel, huru ofta bien nedrefvo och återuppbygde samma cell på olika vis, stundom återtagande en form som de förut hade förkastat.

Då bien hafva en plats, på hvilken de kunna intaga lämplig position för sitt arbete — till exempel på ett trädstycke midt under en kaka som tillväxer nedåt, så att kakan måste byggas öfver en yta af trädstycket — i sådant fall kunna bien lägga grunden till väggen af en ny sexhörning noggrant på dess tillbörliga plats, så att den skjuter fram under de andra färdiga cellerna. Det är tillräckligt att bien kunna stå på sina vederbörliga afstånd från hvarandra och från väggarna af de sista fullbordade cellerna, och då kunna de bygga upp en vägg emellan två tänkta närgränsande sferer, men enligt hvad jag har sett, bortgnaga de aldrig vinklarna af en cell förr än en större del af denna och närgränsande celler är färdigbygd. Denna biens förmåga, att under vissa omständigheter anlägga en ojemn vägg i dess tillbörliga läge emellan två nyss påbörjade celler är vigtig, emedan den förklarar ett faktum, som synes först helt och hållet omstörta föregående teori, nämligen att cellerna på yttersta kanten af en vaxkaka stundom äro fullkomligt sexsidiga, men utrymmet tillåter mig icke att ingå i några detaljer häröfver. Det synes mig då icke vara någon svårighet för en enda insekt (såsom fallet är med visen) att bygga en sexsidig cell, om den arbetar omvexlande på in- och utsidan af två eller tre på samma gång påbörjade celler och dervid alltid står på det vederbörliga afståndet från de påbörjade cellerna, beskrifvande sferer omkring sig och uppbyggande väggar i skärningsplanen.

Då det naturliga urvalet verkar blott genom hopandet af små modifikationer i instinkt eller kroppsbygnad, hvar och en af nytta för individen under dess lefnadsförhållanden, kan man med skäl fråga, hvilken nytta en lång och graderad serie af modifierade bygnadsinstinkter, alla sträfvande emot den nuvarande fulländade konstruktionsplanen, kunde gifva kupbiets förfäder. Svaret tror jag icke är svårt. Celler bygda likt biens eller getingarnas vinna i styrka och inbespara arbete, utrymme och bygnadsmaterial. Det är bekant, att bien ofta äro i stor förlägenhet att få tillräcklig mängd nektar, och Tegetmeier har meddelat mig, att han med experimenter visat, att i en bikupa fjorton till femton skålpund torrt socker konsumeras för afsöndring af hvarje skålpund vax, så att en riklig mängd flytande nektar måste samlas och konsumeras af bien i kupan för att bilda det till vaxkakornas byggande nödvändiga vaxet. Dessutom måste många bin under afsöndringsprocessen förblifva sysslolösa under många dagar. Ett stort förråd honing är nödvändigt till en stor bisvärms uppehälle under vintern och kupans säkerhet beror, som kändt är, på antalet bin. Inbesparing af vax har derföre till följd besparing af honing och af tid till honingens samlande och är derföre af största vigt för en bifamiljs välbefinnande. Vanligen är en arts trefnad beroende af antalet af dess fiender, eller parasiter, eller helt och hållet andra orsaker och vore på detta sätt fullkomligt oberoende af den mängd honing bien kunna samla. Men låt oss antaga, att denna senare omständighet verkligen bestämde, såsom den ofta har gjort, huruvida en med våra humlor beslägtad biart kunde lefva i stort antal i någon trakt, och låt oss vidare antaga, att samhället lefde öfver vintern och följaktligen behöfde ett honingsförråd, så kan i detta fall intet tvifvel finnas, att det vore en fördel för vår tänkta biart, om en ringa modifikation i dess instinkter föranledde den att bygga sina vaxceller litet närmare hvarandra, så att de skuro hvarandra något litet; ty om en vägg bygdes gemensam för två celler, skulle detta alltid vara en liten inbesparing på vax och arbete. Det skulle följaktligen bli allt mer och mer fördelaktigt för vår humla, om den bygde sina celler närmare hvarandra, mera regelbundna och samlade i en massa såsom Meliponas celler; ty i sådant fall skulle en stor del af hvarje cells gränsyta kunna tjena till vägg åt andra närliggande celler och mycket vax och arbete skulle inbesparas. Af samma orsak skulle det också vara fördelaktigt för Melipona att bygga cellerna närmare hvarandra och mera regelbundna på detta sätt än hon nu gör; ty såsom vi hafva sett, skulle alla de sferiska ytorna försvinna och ersättas af plana ytor, och Melipona skulle bygga en kaka lika fullkomlig som våra bin. Utöfver detta stadium af fulländning i bygnadskonst kan icke det naturliga urvalet föra, ty kupbiets vaxkaka är så vidt vi kunna se absolut fullkomlig i besparandet af vax och arbete.

På detta sätt tror jag den underbaraste af alla instinkter, biets, kan förklaras derigenom, att det naturliga urvalet begagnat sig af talrika, successiva obetydliga modifikationer af enklare instinkter; det naturliga urvalet har genom flera grader småningom ledt bien till att i ett dubbelt lager beskrifva sferer på gifvet afstånd ifrån hvarandra och att bygga upp och urhålka vaxet längs skärningsplanen. Bien veta naturligtvis icke, att de beskrifva sina sferer på något särskildt afstånd från hvarandra, icke mera än de känna de olika vinklarna i de sexsidiga prismerna eller basalpyramiderna. Anledningen till ett naturligt urval har varit den, att cellerna böra hafva tillbörlig styrka och storlek och form för larverna, och att detta bör åstadkommas med den största besparing af vax och arbete; den svärm, som behöfver minsta arbete och minsta förlust af honing för vaxsekretionen, är således den mest gynnade och öfverlemnar i arf sin nyförvärfvade ekonomiska instinkt åt nya svärmar, hvilka i sin ordning hafva bästa utsigten att segra i kampen för tillvaron.



\section[Könlösa och sterila insekter]{Invändningar emot teorien om det naturliga urvalet i
dess tillämpning på instinkter: könlösa och sterila
insekter.}

Emot föregående betraktelser öfver instinkternas ursprung har man gjort den invändningen att ”förändringarna i kroppsbygnad och instinkter måste hafva varit samtidiga och noggrant lämpade efter hvarandra, då en modifikation i det ena utan motsvarande förändring i det andra skulle varit ofördelaktig.” Styrkan i denna invändning ligger helt och hållet i det antagandet, att förändringar i både instinkt och kroppsbildning försiggå hastigt. Såsom exempel taga vi den i förra kapitlet omnämda talgoxen (Parus major). Denna fågel håller ofta idegranens frö emellan fötterna emot en gren och hamrar derpå, till dess han kommer in till kärnan. Hvilken synnerlig svårighet finnes nu för det naturliga urvalet att bibehålla alla de obetydliga individuela variationer i näbbens skapnad, som vore allt bättre och bättre lämpliga till att bryta fröskalen, till dess en näbb vore bildad så väl lämpad derför som nötkrakans (Nucifraga), under det att på samma gång vanan eller behof eller spontan variation i smak gjorde fågeln allt mer och mer till fröätare? I detta fall antages näbben småningom modifieras genom naturligt urval efter och i öfverensstämmelse med långsamma förändringar i vana eller smak. Men låter man nu äfven talgoxens fötter variera och blifva större på grund af vexelverkan eller af någon annan okänd orsak, så är det icke osannolikt, att sådana fötter skulle föranleda fågeln att klättra mer och mer, till dess den förvärfvat nötkrakans förmåga att klättra. I detta fall antages deremot en gradvis skeende förändring i kroppsbildning leda till förändringar i instinktmässiga vanor. Vi skola taga ytterligare ett exempel. Få instinkter äro mera märkvärdiga, än svalornas på östra britiska öarna, hvilka bygga sina bon helt och hållet af förtjockad saliv. Några fåglar bygga sina nästen af dy, som man tror fuktad med saliv, och en af svalorna i Nordamerika bygger (såsom jag har sett) sitt bo af qvistar, som hopfästas med saliv och till och med med flockor af detta ämne. Är det då så osannolikt, att ett naturligt urval af svalindivider, som afsöndrade allt mer och mer saliv, skulle till slut åstadkomma en art med instinkter som ledde honom till att icke begagna annat material än intorkad saliv? Och på samma sätt i andra fall. Man måste likväl medgifva, att vi i många fall icke kunna sluta oss till, antingen instinkten eller kroppsbygnaden först varierade.

Otvifvelaktigt kunde många svårförklarliga instinkter sättas emot teorien om det naturliga urvalet — fall, i hvilka vi icke se något möjligt ursprung för en instinkt, fall i hvilka inga övergångsstadier stå att finna, exempel på instinkter af så ringa vigt, att de näppeligen kunnat komma att beröras af det naturliga urvalet, exempel på instinkter, som äro nästan identiskt lika hos djur så vidt skilda i naturens skala, att vi icke kunna förklara deras likhet genom arf från en gemensam stamfader och vi följaktligen måste tro, att de oberoende af hvarandra blifvit förvärfvade genom naturligt urval. Jag vill här icke ingå i dessa speciela fall, utan vill inskränka mig till en särskild svårighet, som först syntes mig oöfvervinnelig och i sanning tillintetgörande hela teorien. Jag menar de könlösa ofruktsamma honorna i insektkolonier; dessa könlösa skilja sig ofta i instinkter och i kroppsbildning betydligt både från hannarna och från de fruktsamma honorna, och då de äro sterila, kunna de icke fortplanta sina egenskaper.

Ämnet förtjenar väl att afhandlas i all utförlighet, men jag vill här blott hålla mig till ett enda fall, de sterila arbetsmyrorna. Huru dessa hafva blifvit sterila är visserligen svårt att förklara, men icke värre än någon annan öfverraskande modifikation i kroppsbygnaden. Man kan visa, att några insekter och andra leddjur i naturtillståndet tillfälligtvis blifva sterila, och om sådana insekter lefvat i samhällen och det för samhället varit af nytta, att ett visst antal individer årligen framfödts, hvilka voro i stånd att arbeta men i saknad af fortplantningsförmåga, så kan jag icke se någon synnerlig svårighet för det naturliga urvalet att åstadkomma detta. Men jag måste förbigå denna förberedande invändning. Den stora svårigheten ligger deruti, att arbetsmyrorna skilja sig vida ifrån både hannar och honor i kroppsskapnad, såsom i formen af thorax och i saknaden af vingar och stundom ögon samt i instinkt. Den underbara skilnaden i det senare hänseendet emellan arbetarna och de fullkomliga honorna torde visa sig bättre hos honingbien. Om en arbetsmyra eller annan könlös insekt hade varit ett vanligt djur, skulle jag utan tvekan antagit, att alla dess karakterer blifvit långsamt förvärfvade genom naturligt urval, nämligen derigenom att individer föddes med obetydliga fördelaktiga modifikationer, hvilka gingo i arf på afkomlingarna, och att dessa återigen varierade och utvaldes och så vidare. Men i arbetsmyran hafva vi en insekt som skiljer sig betydligt från sina föräldrar och dock är fullkomligt ofruktsam, så att hon icke har kunnat lemna i arf modifikationer i kroppsbildning och instinkter åt sina ättlingar. Man kan väl fråga: huru är det möjligt att förena detta fall med teorien om ett naturligt urval?

Vi skola först komma ihåg, att vi hafva oräkneliga exempel både hos våra kulturalster och i naturens produkter på alla slags olikheter i skapnad som stå i ett visst förhållande till ålder och kön. Vi hafva afvikelser, som endast förekomma hos ett kön, till och med blott under den korta period då reproduktionssystemet är verksamt, såsom i många fåglars brölloppsdrägt och i laxhannens hakformiga käk. Vi se äfven små olikheter i hornen hos vissa boskapsraser stå i sammanhang med ett genom konst åstadkommet ofullkomligt tillstånd hos hannens reproduktionsorganer. Oxarnas horn äro till storleken olika med tjurarnas och kornas horn hos vissa raser. Jag kan derföre icke inse omöjligheten af, att karaktererna kunna stå i ett visst förhållande till ofruktsamheten hos vissa medlemmar af ett insektsamhälle, svårigheten ligger blott i att förstå, huru sådana på vexelverkan beroende bildningsmodifikationer kunna långsamt förökas genom naturligt urval.

Ehuru denna svårighet synes oöfvervinnelig, förminskas den eller som jag tror försvinner den, om vi komma ihåg, att ett urval kan tillämpas på familjen lika väl som på individen och på detta sätt vinnes det önskade resultatet. De som uppföda gödkreatur önska få kött och fett väl blandade tillsammans; djuret slagtas men processen fortgår inom samma familj och lyckas. Jag sätter sådan lit till urvalets förmåga, att jag tror man med all säkerhet kan bilda en boskapsras, som alltid lemnar oxar med utomordentligt långa horn, genom att sorgfälligt utvälja sådana tjurar och kor, som lemna oxar med de längsta hornen, och dock har aldrig någon oxe fortplantat sitt slägte. Ett bättre och verkligt exempel är följande: några varieteter af ettåriga dubbla löfkojor af olika färg frambringa alltid enligt M. Verlot genom frön ett stort antal plantor som bära dubbla och fullkomligt sterila blommor, och detta är resultatet af ett långvarigt omsorgsfullt urval; om varieteten icke på samma gång lemnade andra plantor, skulle den med ens dö ut, men den lemnar alltid några enkla och fruktsamma plantor, som skilja sig från vanliga enkla varieteter blott uti sin förmåga att alstra dessa två former. De fruktsamma plantor som alstra blott enkla blommor kunna jemföras med hannarna och honorna i ett myrsamhälle och de sterila med dubbla blommor, hvilka regelbundet alstras i stor mängd, motsvara de många sterila neutrerna i samma koloni. Så tror jag äfven förhållandet har varit med de insekter som lefva i samhällen; en obetydlig modifikation i kroppsbildning eller i instinkt i förening med vissa samhällsmedlemmars ofruktsamhet har varit af nytta för samhället; de fruktsamma hannarna och honorna i samma samhälle frodades och lemnade sina fruktsamma afkomlingar i arf en sträfvan att framalstra sterila medlemmar med samma modifikation. Och jag tror, att denna process har blifvit upprepad, tilldess denna utomordentliga olikhet uppkommit emellan de fruktsamma och de sterila honorna af samma art, som vi se hos så många insekter som lefva i samhällen.

Men vi hafva ännu icke nått höjden af denna svårighet, nämligen det faktum, att de könlösa bland många myror afvika icke från de fruktsamma hannarna och honorna utan äfven från hvarandra, stundom i nästan otrolig grad och de äro på det viset delade i två eller till och med tre kaster. Dessa kaster öfvergå dessutom i allmänhet icke genom grader i hvarandra utan äro fullkomligt skilda, ja så väl skilda från hvarandra, som två arter af samma slägte eller till och med som två slägten af samma familj. Hos Eciton förekomma till exempel könlösa arbetare och stridsmän med utomordentligt afvikande käkar och instinkter; hos Cryptocerus bära arbetarna af ena slaget en besynnerlig sköld på hufvudet, hvars bruk är alldeles okändt. Hos den mexikanska Myrmecocystus lemna arbetarna af det ena slaget aldrig nästet; de underhållas af arbetarna af en annan kast och hafva en oerhördt utvidgad abdomen, som afsöndrar ett slags honing, hvilken ersätter bladlössens exkretioner, eller den tama boskap, såsom man kunde kalla det, som våra europeiska myror sköta och hålla fången.

Man kan väl tänka, att jag sätter öfvermåttan stort förtroende till grundsatsen om det naturliga urvalet, om jag icke medgifver, att sådana underbara och välgrundade fakta med ens tillintetgöra min teori. I det enklare fallet, då alla könlösa äro af ett slag och enligt min tanke blifvit afvikande från de fruktsamma hannarna och honorna genom naturligt urval, kunna vi af analogien med vanliga variationer sluta till, att de successiva, små, fördelaktiga modifikationerna icke först uppkommo hos alla neutrer i samma bo, utan blott hos några få, och att, då de samhällen egde bestånd, i hvilka honorna alstrade de flesta könlösa med fördelaktiga variationer, alla neutrer till slut erhöllo dessa karakterer. I öfverensstämmelse med denna åsigt böra vi i samma bo stundom finna neutrer som förete öfvergångsstadier i kroppsbildning och detta finna vi i sjelfva verket ofta nog, om vi betänka huru få könlösa insekter i Europa blifvit omsorgsfullt undersökta. Mr F. Smith har visat, att flera af de könlösa britiska myrorna visa öfverraskande afvikelser från hvarandra i storlek och stundom i färg; och att de yttersta formerna kunna sammanbindas genom individer som äro tagna ur samma näste. Jag sjelf har kunnat jemföra med hvarandra fullkomliga serier af detta slag. Det händer stundom, att de större eller de mindre arbetarna äro talrikast, under det de af medelstorlek äro få till antal. Formica flava har större och mindre arbetare med några få af medelstorlek, och af denna art hafva, såsom mr F. Smith iakttagit, de större arbetarna enkla ögon (ocelli) hvilka ehuru små kunna tydligt urskiljas, hvaremot de mindre arbetarna hafva sina ocelli rudimentära. Då jag har noggrant dissekerat flera exemplar af dessa arbetare, kan jag försäkra, att ögonens rudimentära tillstånd hos de mindre arbetarna icke kan förklaras blott genom deras mindre storlek, och jag tror, ehuru jag icke vågar påstå med bestämdhet, att de medelstora arbetarna hafva sina ocelli i ett mellanstadium af utveckling. Vi hafva således här två slag af sterila arbetare i samma bo, som skilja sig icke blott i storlek, utan äfven i sina synorganer och äro förenade genom ett ringa antal medlemmar i ett mellanstadium. Jag kunde gå ännu längre och tillägga, att om de mindre arbetarna hade varit mest nyttiga för samhället och de hannar och honor fortfarande blifvit utvalda, som alstrade allt flera och flera af de mindre arbetarna till dess alla voro lika, skulle vi hafva en myrart med neutrer i nästan samma tillstånd, som de könlösa af Myrmica. Ty dess arbetare hafva icke ens rudiment af ocelli, ehuru hannar och honor af detta slägte hafva väl utvecklade ocelli.

Jag vill anföra ett annat exempel. Jag var så öfvertygad om att finna öfvergångar i vigtiga delar af kroppsbildningen emellan de olika kasterna af könlösa hos samma art, att jag med glädje begagnade mig af F. Smiths anbud af talrika exemplar från samma bo af en myra, Anomma, från vestra Afrika. Läsaren torde kanhända lättast uppskatta graden af olikhet hos dessa arbetsmyror, om jag i stället för att anföra verkliga mått uppdrager en strängt noggrann jemförelse: skilnaden var lika stor, som om vi såge en mängd arbetare bygga ett hus, af hvilka många voro fem fot och fyra tum höga och andra sexton fot, men vi måste äfven tänka oss, att de större arbetarna hade hufvud som voro fyra gånger så stora som de mindres i stället för tre gånger, och käkar som voro nära fem gånger så grofva. Käkarna hos de olikstora arbetsmyrorna voro märkvärdigt olika till sin skapnad och likaså tändernas form och antal. Men ett för oss vigtigt faktum är det, att ehuru arbetarna kunna grupperas i kaster af olika storlek, de dock omärkligt öfvergå i hvarandra och detta är förhållandet äfven med den vidt skilda skapnaden af deras käkar. Jag talar med tillförsigt öfver denna sista punkt, ty sir J. Lubbock aftecknade för mig med camera lucida de käkar, som jag lösdissekerade från arbetare af olika storlek. Mr Bates har i sitt interessanta arbete ”en naturforskare på Amazonen” beskrifvit analoga fall.

Med dessa fakta framför mig tror jag, att det naturliga urvalet genom att verka på fruktsamma myror eller föräldrar kan bilda en art som måste regelbundet alstra könlösa, antingen alla mycket stora med en viss form på sina käkar eller alla af liten storlek med käkar af helt annan skapnad, eller slutligen, och detta är den största svårigheten, ett slag af arbetare af en viss storlek och skapnad och på samma gång ett annat slag af arbetare af helt olika storlek och form; en graderad serie har först bildats såsom hos Anomma och derefter hafva de yttersta formerna alstrats i allt större och större antal genom urval af de föräldrar, som födde dem, till dess alla mellanstadier helt och hållet försvunno.

En analog förklaring har Wallace gifvit på det lika invecklade förhållandet, att vissa malayiska fjärilhonor regelbundet uppträda under två eller tre skilda former; Fritz Müller har äfven på likartadt sätt förklarat det faktum, att vissa brasilianska krustaceers hannar likaledes uppträda under två helt olika former. Men detta ämne behöfver icke vidare behandlas här.

Jag har nu förklarat, huru i min tanke två bestämdt skilda kaster af ofruktsamma arbetare i samma bo hafva uppkommit, båda helt olika både hvarandra och sina föräldrar. Vi kunna se huru nyttig deras förekomst är för ett samhälle af myror enligt samma grundsats hvaraf menniskan betjenar sig, arbetets fördelning. Myror arbeta likväl med ärfda instinkter, ärfda organer eller verktyg, under det menniskan arbetar med förvärfvade kunskaper och konstgjorda instrument. Men jag måste tillstå, att jag med all min tillit till det naturliga urvalets förmåga aldrig skulle kunnat på förhand ana, att denna princip kunde vara verksam i så hög grad, om icke just förhållandet med dessa könlösa insekter hade öfvertygat mig om saken. Derföre har jag behandlat detta förhållande med något större utförlighet för att visa det naturliga urvalets förmåga och äfvenledes derföre, att detta är den allra svåraste invändning som kan uppställas emot min teori. Förhållandet är också särdeles interessant, ty det visar att hos djur likasom hos växter en viss grad af modifikation kan åstadkommas genom hopande af talrika små spontana variationer, hvilka på något sätt äro gynsamma, utan att vana eller öfning dervid varit verksamma. Ty vanor som voro egendomliga för arbetarna eller de sterila honorna, huru länge de än bibehållit sig, kunna omöjligen haft något inflytande på hannar och fruktsamma honor, hvilka allena lemna afkomlingar. Det har förvånat mig, att ännu ingen hittills framhållit detta lärorika fall emot Lamarcks välkända lära om ärfda vanor.



\section{Sammanfattning.}

I detta kapitel har jag bemödat mig att i korthet visa, att våra husdjurs själsförmögenheter variera och att variationerna gå i arf; i ännu större korthet har jag sökt visa att instinkterna variera äfven i naturtillståndet. Ingen vill bestrida att instinkterna äro af största vigt för hvarje djur. Derföre finnes icke någon verklig svårighet för det naturliga urvalet att under förändrade lefnadsförhållanden till hvad utsträckning som helst föröka små modifikationer i instinkt som på något sätt äro nyttiga. I några fall äro vana eller större och mindre öfning härvid äfven verksamma. Jag påstår icke att de fakta som detta kapitel innehåller på något sätt bestyrka min teori, men ingen af dessa invändningar tillintetgöra den. Det faktum, att instinkter icke alltid äro fullkomliga och kunna missledas, — att ingen instinkt kan visas vara till uteslutande för andra djurs nytta, ehuru djuren begagna sig af andras instinkter, — att grundsatsen i naturalhistorien ”natura non facit saltum” kan tillämpas på instinkter lika väl som på kroppsbildning och enligt föregående framstälning kan förklaras under det den på annat vis är fullkomligt oförklarlig — allt detta sträfvar å andra sidan att bestyrka teorien om det naturliga urvalet.

Denna teori bestyrkes också af några få andra fakta rörande instinkterna till exempel af det vanliga förhållandet, att beslägtade men skilda arter, som bebo skilda delar af jorden och lefva under ansenligt olika förhållanden, ofta behålla nästan samma instinkter. Enligt grundsatsen om ärftlighet kunna vi förstå, hvarföre trasten i tropiska Sydamerika bekläder sitt bo med dy på samma sätt som vår engelska trast; hvarföre hornfåglarna i Afrika och Ostindien hafva samma egendomliga instinkt att inmura honorna i ett ihåligt träd och lemna blott en liten öppning i muren, genom hvilken hannarna mata dem jemte ungarna då de äro kläckta; hvarföre den nordamerikanska kungsfågelhannen (Troglodytes) bygger ”tuppnästen” att bo i liksom hannarna af vår europeiska art, en vana, som icke förekommer hos någon annan art. Det må slutligen icke vara någon logisk deduktion, men det motsvarar vida bättre mina föreställningar att betrakta sådana instinkter, som gökens att utkasta sina fosterbröder, myrornas att göra slafvar, ichneumonidernas att lägga sina ägg i lefvande kålmaskar, icke såsom särskildt förlänade eller skapade instinkter utan såsom följder af en allmän lag, som leder till alla organiska varelsers framgång — förökning och variation ger segern åt de starkaste och låter de svagare dö ut.


