%TOLFTE KAPITLET.





\chapter[Geografisk fördelning II]{Geografisk fördelning. \\ (Fortsättning.) }

{\it
Fördelning af sötvattensalster. — Invånarna på oceanöar. — Brist på batrachier och landdäggdjur. — Relationer emellan invånare på öar och närliggande fastland. — Kolonisering från närmaste håll jemte följande modifikation. — Sammanfattning af detta och föregående kapitel.
}\\[0.5cm]

Då sjöar och flodsystem äro skilda från hvarandra genom landområden, som tjena såsom stängsel, kunde man tänka att sötvattensalster icke kunnat få någon vidsträckt spridning inom samma trakt, och då hafvet tydligen är ett ännu fruktansvärdare stängsel, att de icke skulle kunna sprida sig till aflägsna länder. Men ett rakt motsatt förhållande eger rum. Många sötvattensarter som höra till fullkomligt skilda klasser hafva icke blott en enorm utbredning, utan beslägtade arter herrska på ett anmärkningsvärdt sätt öfver hela jorden. Jag minnes mycket väl min öfverraskning att finna så mycken likhet i sötvattensinsekter, snäckor m. m., då jag först undersökte Brasiliens sötvatten, och olikheten i de deromkring boende landdjuren, jemförda med Britanniens.

Men denna förmåga hos sötvattensprodukterna att sprida sig så vidsträckt, huru oväntad den är, kan i de flesta fall förklaras genom deras förmåga att på ett för dem högeligen fördelaktigt sätt flytta kortare stycken från dam till dam eller från flod till flod och benägenheten för vidsträckt spridning följer deraf såsom en nästan nödvändig konseqvens. Vi kunna här blott fästa oss vid några få fall. Af fiskar tror jag aldrig samma art påträffas i skilda kontinenters sötvatten. Men på samma kontinent hafva samma arter ofta en vidsträckt och nästan nyckfull spridning, ty två flodsystem hafva några fiskar gemensamma och andra olika. Några få fakta synes mig tala för möjligheten af deras transport genom tillfälliga medel, såsom att fiskar i Indien ofta af hvirfvelvindar drifvas upp i luften, äfvensom vitaliteten hos deras ägg upptagna ur vatten. Men jag är mera benägen att tillskrifva sötvattensfiskarnas utbredning hufvudsakligen landets höjdförändringar under en nyare tid, som haft till följd att förut skilda floder flutit tillsammans. Exempel härpå kan också gifvas under öfversvämningar utan någon förändring af landets nivå. Den stora skiljaktigheten emellan fiskar på motsatta sidor af större bergsträckor, hvilka från en tidig period måste hafva fullkomligt förhindrat flodsystemens sammanflytande, synes också leda till detta antagande. Utan tvifvel finnas många fall af sötvattensfiskars uppträdande i mycket skilda delar af jorden som icke kunna för närvarande förklaras; men några insjöfiskar tillhöra gamla former och i sådana fall torde tiden varit tillräcklig för stora geografiska förändringar och följaktligen tid och tillfälle till mycken flyttning. I andra rummet kunna saltvattensfiskar med omsorg småningom vänjas att lefva i insjöar och enligt Valenciennes finnes knappt en enda grupp hvars alla medlemmar äro uteslutande inskränkta till sötvatten, så att en marin art af en sötvattensgrupp har kunnat flytta längs hafskusterna och sedan blifvit modifierad och lämpad för ett aflägset lands insjöar.

Några arter af sötvattensnäckor hafva mycket stor spridning och beslägtade arter, hvilka enligt vår teori härstamma från en gemensam stamfader och ur gemensam källa, äro förherskande öfver hela jorden. Deras fördelning förvånade mig i början mycket, då deras ägg icke kunna förflyttas af fåglar och omedelbarligen dödas af hafsvattnet likasom de fullväxta. Jag kunde icke ens förstå huru några naturaliserade arter hafva spridt sig hastigt utöfver samma land. Men två omständigheter, hvilka jag iakttagit — och otvifvelaktigt återstå många att observera — sprida något ljus öfver detta ämne. Då en and hastigt dyker upp ur en dam betäckt med andmat (Lemna) har jag två gånger sett dessa små växter fastna på fågelns rygg; och det har händt mig, då jag flyttade litet andmat från ett aqvarium till ett annat, att jag helt och hållet utan afsigt befolkat det ena med sötvattensnäckor från det andra. Men en annan omständighet är kanske mera verksam; jag nedhängde en andfot i ett aqvarium der en mängd ägg af sötvattensnäckor höllo på att kläckas, och jag fann att en del af de ytterst små nykläckta snäckorna kröpo upp på foten och hängde sig så fast dervid, att de icke kunde skrapas af vid upptagandet ur vattnet, ehuru vid en mera framskriden ålder de skulle frivilligt fallit af. Dessa nykläckta snäckor, ehuru af aqvatisk natur, lefde på andfoten i fuktig luft från tolf till tjugu timmar och under denna tid kan en and eller häger flyga åtminstone sex- eller sjuhundra mil, och om den af vinden drifves öfver hafvet till en oceanö eller någon annan aflägsen punkt, torde den der slå sig ned i en dam eller flod. Sir C. Lyell har meddelat mig att en Dytiscus blifvit fångad med en Ancylus (en sötvattensnäcka) fasthängande, och en vattenbagge af samma familj, en Colymbetes, flög en gång ombord på ”Beagle”, då den var på fyrtiofem mils afstånd från närmaste land: huru mycket längre den skulle kunnat drifvas af en gynsam vind, kan ingen säga.

Hvad växterna beträffar har det länge varit kändt, hvilken enorm spridning många sötvatten- och äfven många kärrarter hafva både öfver kontinenter och till de mest aflägsna oceanöar. Detta bevisas på ett slående sätt, såsom A. de Candolle anmärkt, i stora grupper af landväxter som hafva blott mycket få slägtingar som lefva i vatten, ty de senare synas omedelbarligen såsom en följd förvärfva en vidsträckt utbredning. Jag tror detta kan förklaras genom gynsamma spridningsmedel. Jag har förut omnämt att jord tillfälligtvis, ehuru sällan, i ringa mängd sitter fast vid fåglars fötter och näbbar. Vadande fåglar, som ofta besöka dyiga dammar, hafva oftast smutsiga fötter, och fåglar af denna ordning äro enligt hvad jag kan visa de största vandrare och påträffas tillfälligtvis på de mest aflägsna och kala öar midt i oceanen; de böra icke vara i stånd att slå sig ned på hafsytan, så att smutsen icke kan tvättas af deras fötter, och då de landa, skola de helt säkert flyga till sina naturliga hemvist i sötvattnen. Jag tror icke botanister observerat huru rik dyen i dammar är på frön; jag har gjort några små försök, men vill här anföra blott det anmärkningsvärdaste fallet: jag tog i Februari tre matskedar dy från tre skilda orter under vatten på botten af en liten dam; torkad vägde denna dy 6 3/4 uns; jag bevarade den täckt i min studerkammare i sex månader, plockade upp och räknade hvarje växt alltefter som den sköt upp; de voro af många slag och tillsammans 537 till antal och dock rymdes all gyttjan i en kaffekopp! I betraktande af dessa fakta anser jag, att det skulle vara en oförklarlig omständighet om vattenfåglar icke transporterade frön af samma sötvattenväxter till obebodda dammar och floder på vidt skilda punkter. Samma kraft har kunnat vara verksam för de mindre insjödjurens ägg.

Andra okända krafter hafva sannolikt också dervid spelat någon rol. Jag har funnit att insjöfiskar äta vissa slags frön, ehuru de gifva upp många andra fröslag sedan de sväljt dem: äfven små fiskar svälja frön af måttlig storlek såsom af den gula vattenliljan och Potamogeton. Några andra fåglar hafva under århundraden oafbrutet ätit fisk; de flyga derefter omkring och begifva sig till andra vatten, eller drifvas af vinden öfver hafvet: och vi hafva sett att fröen behålla sin groningsförmåga om de kastas upp eller afgå med exkrementer flera timmar efteråt. Då jag såg storleken af den vackra vattenliljans frön (Nelumbium) och ihågkom A. de Candolles beskrifning på denna växt, trodde jag att dess fördelning skulle vara fullkomligt oförklarlig; men Audubon uppgifver att han funnit frön af den stora sydliga vattenliljan (enligt Hooker sannolikt Nelumbium luteum) i magen af häger; ehuru jag icke känner det såsom faktiskt, gifver dock analogien anledning att tro, att en häger som flyger till en annan dam och der får ett duktigt mål af fisk, kastar upp ur sin mage sannolikt osmälta frön af Nelumbium, eller också har fågeln kunnat släppa fröen under ungarnas matande, på samma sätt som fåglar ofta släppa ned fiskar.

Om vi taga i betraktande alla dessa olika spridningsmedel, få vi komma ihåg, att då en dam eller flod först bildas, till exempel på en ö under höjning, den är obebodd och ett enstaka frö eller ägg har mycken utsigt att frodas. Ehuru en kamp för tillvaron alltid måste uppstå emellan invånarna i en dam, af huru få olika slag de än må vara, så måste dock, då antalet äfven i en väl befolkad dam är litet i jemförelse med antalet arter som bebo en lika stor landyta, striden sannolikt vara mindre häftig emellan vattendjur än emellan landdjur; en inkräktare från vattnen i en främmande trakt bör följaktligen hafva bättre utsigt att intaga en ny plats än förhållandet är för landkolonister. Vi böra också ihågkomma att många sötvattensalster stå lågt i naturens skala och vi hafva skäl att tro, att lägre varelser modifieras mindre hastigt än de högre, och detta bör göra flyttningstiden för hvarje vattenart större än den öfverhufvudtaget är. Vi få icke heller förglömma sannolikheten af att många arter förr hafva spridt sig så kontinuerligt som sötvattensprodukter kunna göra öfver ofantliga ytor, och sedermera dött ut i mellanliggande trakter. Men sötvattensväxternas och de lägre djurens spridning, vare sig de behålla samma form identiskt eller i någon mån modifierad, beror enligt min åsigt hufvudsakligen på den vidsträckta spridningen af deras frön och ägg genom djur, isynnerhet vattenfåglar, hvilka hafva stor förmåga att flyga och af naturen flytta från det ena vattnet till det andra.

\section{Om invånarna på oceanöar.}

Vi komma nu till den sista af de tre stora klasser af fakta, som jag har utvalt såsom erbjudande de största svårigheterna, om vi antaga den åsigten, att icke blott alla individer af samma art, hvarhelst de finnas, hafva flyttat ut från ett enda område, utan att äfven beslägtade arter, ehuru de nu bebo de mest skilda punkter hafva utgått från en enda yta, födelseorten för deras gemensamma stamfar. Jag har redan framhållit att jag icke kan erkänna Forbes åsigt om kontinenternas utsträckning, hvilken om den fullföljdes skulle leda till den åsigt att alla nu existerande öar hafva varit kontinuerligt förenade med något fastland under en nyare period. Denna åsigt skulle undanröja många svårigheter, men den skulle icke förklara alla omständigheter. I följande framställning skall jag icke inskränka mig till den blotta spridningsfrågan, utan äfven beröra några förhållanden, som stå i sammanhang med sanningen af de två teorierna, oberoende skapelser eller descendens.

Arterna af alla slag som bebo oceanöar äro få till antal jemförda med dem som finnas på kontinentalytor: A. de Candolle medgifver detta för växter och Wollaston för insekter. Nya Zeeland till exempel med dess höga berg som sträcka sig öfver 750 mils latitud jemte de utanför liggande öarna Auckland, Campbell och Chatam, innehålla 960 slag af blommande växter; om vi jemföra detta måttliga antal med arterna som svämma öfver lika ytor i sydvestra Australien eller vid Goda Hoppsudden, måste vi medgifva att någonting helt annat än blott olikhet i fysiska förhållanden har verkat en så stor skilnad i artantal. Till och med det enformiga området omkring Cambridge har 847 växter och den lilla ön Anglesea 764, men deri inbegripas dock några ormbunkar och några få införda växter, och jemförelsen är i några hänseenden icke fullt exakt. Det är fullt bevisadt att den kala ön Ascension ursprungligen egde blott ett halft dussin blommande växter, dock hafva många numera derstädes blifvit naturaliserade, såsom äfven på Nya Zeeland och hvarje annan oceanö. På S:t Helena hafva vi anledning att tro de naturaliserade växterna och djuren hafva nästan eller helt och hållet utrotat många infödda former. Den som antager åsigten om hvarje art såsom resultat af en oberoende skapelseakt måste medgifva att ett tillräckligt antal af lämpliga växter och djur icke blifvit skapade för oceanöarna, ty menniskan har utan afsigt befolkat dem vida fulltaligare och fullständigare än naturen.

Ehuru på oceanöarna arterna äro få till antal, är ofta proportionen af endemiska arter (det är sådana som icke finnas annorstädes på jorden) ytterligt stor. Om vi till exempel jemföra antalet endemiska landsnäckor på Madeira, eller endemiska fåglar på Galapagosöarna med det antal som finnes på någon kontinent, och vidare jemföra öns ytområde med kontinentens, så skola vi se sanningen häraf. Detta har man kunnat vänta a priori, ty såsom vi redan utvecklat, komma händelsevis arter efter långa mellantider in på ett nytt och isoleradt område och måste täfla med nya konkurrenter, och de böra derföre vara högeligen benägna för modifikation och skola ofta alstra grupper af modifierade afkomlingar. Men deraf följer ingalunda att emedan på en ö nästan alla arter af en klass äro egendomliga, de som tillhöra en annan klass eller en annan afdelning af samma klass också äro egendomliga, och denna olikhet synes delvis bero derpå att arter som icke äro modifierade hafva inflyttat i mängd, så att deras ömsesidiga relationer icke hafva blifvit mycket rubbade; och delvis derpå, att ofta ankommit modifierade inflyttande arter från moderlandet, med hvilka öinvånarna hafva kroaserats. Vi måste komma ihåg att afkomlingarna af sådana kroaseringar nästan alltid vinna i styrka, så att äfven en tillfällig kroasering kunde åstadkomma större verkan än på förhand kan antagas. Jag vill gifva några belysningar af föregående antydningar: på Galapagosöarna finnas 26 landfåglar, af dem äro 21 (eller kanske 23) egendomliga, hvaremot af 11 hafsfåglar blott två äro egendomliga; och det är klart att hafsfåglar hafva lättare att komma till dessa öar och oftare också komma dit än landfåglar. Bermuda å andra sidan, som ligger på ungefär samma afstånd från Nordamerika som Galapagosöarna från Sydamerika, och som har mycket egendomlig grund eger icke en enda endemisk landfågel, och vi känna af J. M. Jones utmärkta af handling om Bermuda, att rätt många nordamerikanska fåglar tillfälligtvis eller till och med ofta besöka denna ö. Nästan hvarje år enligt meddelande af E. V. Harcourt drifvas många europeiska fåglar till Madeira; denna ö är bebodd af 99 slag, af hvilka blott ett är egendomligt, ehuru mycket beslägtadt med en europeisk form, och tre eller fyra andra arter äro inskränkta till denna ö och Kanarieöarna. Öarna Bermuda och Madeira hafva alltså befolkats af fåglar från närliggande trakter, hvilka under långa tider kämpat med hvarandra och ömsesidigt lämpats efter hvarandra; då de derföre en gång rotfästat sig i sitt nya hem, hålles hvar och en af de andra i sina egna vanor och på sin egen plats, och är följaktligen föga benägen för modifikation. Hvarje tendens dertill skulle också motarbetas af kroasering med ej modifierade inflyttande individer från moderlandet. Madeira åter är bebodd af ett underbart antal egendomliga landsnäckor; hvaremot icke en enda hafssnäcka är egendomlig för dess stränder; ehuru vi icke känna hafssnäckors spridning, kunna vi dock antaga att deras ägg eller larver, möjligen fastsittande vid sjögräs eller drifved eller vid fötterna af vadarefåglar, kunna flyttas tre eller fyrahundra mil tvärt öfver öppna hafvet vida lättare än landsnäckor. De olika insektordningar som bebo Madeira förete nästan liknande fall.

Oceanöar äro stundom i saknad af djur af vissa hela klasser och deras platser intagas af andra: däggdjurens plats intages således på Galapagosöarna af reptilier och på Nya Zeeland af gigantiska fåglar utan vingar. Ehuru Nya Zeeland här upptages såsom oceanö, är det dock till en viss grad tvifvelaktigt, om den verkligen bör såsom sådan betraktas; den är särdeles stor och icke skild från Australien genom någon synnerligen djup sjö; från dess geologiska karakter och riktningen af dess bergskedja har Rev. W. B. Clarke nyligen påstått att denna ö så väl som Nya Caledonien bör betraktas såsom tillhörande Australien. Om vi fästa oss vid växterna, har Hooker visat att på Galapagosöarna de proportionela antalen af de skilda ordningarna äro mycket olika mot hvad de annars äro. Alla sådana olikheter i tal och frånvaron af vissa hela grupper af djur och växter på öar förklaras i allmänhet genom antagna olikheter i fysiska förhållanden, men denna förklaring är icke litet tvifvelaktig. Lättheten af flyttning synes hafva varit fullt så vigtig som de fysiska vilkorens beskaffenhet.

Många märkvärdigt obetydliga förhållanden kunna anföras med afseende på invånarna på oceanöar. På många öar som icke bebos af ett enda däggdjur hafva till exempel några af de endemiska växterna med fina hakar försedda frön; och få relationer äro tydligare än att med hakar försedda frön äro ämnade att förflyttas med däggdjurens pels eller ull. Men sådana frön kunna transporteras till en ö på annat vis, och då växten sedan modifieras, bildar den en endemisk art, som ännu kan bibehålla hakarna, hvilka icke böra utgöra något mindre nyttigt bihang än de skrynklade vingarna under de sammanväxta täckvingarna på många insulära skalbaggar. Öar ega vidare ofta träd eller buskar af ordningar som öfverallt annorstädes inbegripa blott örtartade växter; träd hafva nu såsom A. de Candolle har visat i allmänhet blott inskränkta utbredningsområden, hvad orsaken dertill än må vara. Träd skola derföre med föga sannolikhet uppnå aflägsna oceanöar, och en örtartad växt, som icke haft någon utsigt för en lycklig strid med de många fullt utvecklade träd som lefva på en kontinent, vinner alltid en fördel, då den stannat på en ö, genom att växa högre och högre och stiga öfver de andra örtartade växterna. I detta fall skulle det naturliga urvalet arbeta på en förstoring af växten, till hvad ordning den än hör, och derigenom förvandla den först till en buske och sedan till ett träd.



\section[Landdäggdjur på oceanöar]{Bristen på batrachier och landdäggdjur på oceanöar.}

Bory S:t Vincent har för längesedan anfört, att batrachier (grodor, ödlor) aldrig finnas på någon af de många öar, hvarmed oceanen är beströdd. Jag har bemödat mig om att bekräfta detta påstående och funnit det strängt taget sant, med undantag af Nya Zeeland, Andaman-öarna och möjligen Salomon-öarna. Denna allmänna frånvaro af grodor, paddor och ödlor på så många oceanöar kan icke förklaras af deras fysiska förhållanden; i sjelfva verket synas öar vara särskildt lämpliga för dessa djur, ty grodor hafva blifvit införda på Madeira, Azorerna och Mauritius och hafva förökat sig så att de blifvit skadliga. Men då dessa djur och deras rom omedelbarligen dödas af hafsvatten, skulle deras transport tvärt öfver hafvet vara förenad med stora svårigheter, och deraf kunna vi inse, hvarföre de icke finnas på någon oceanö. Men hvarföre de enligt skapelseteorien icke skulle blifvit skapade der, vore mycket svårt att förklara.

Däggdjur förete ett annat och likartadt fall. Jag har omsorgsfullt granskat de äldsta resebeskrifningar och hittills har jag icke funnit ett enda otvifvelaktigt fall på ett däggdjur (med undantag af infödingarnas tama djur) som bebott en ö på mer än 300 mils afstånd från en kontinent eller en kontinentalö; och många öar på mycket mindre afstånd äro lika tomma. Falklandsöarna som bebos af en varglik räf, äro närmast undantagen, men denna grupp kan icke betraktas såsom oceanisk, då den ligger på en bank som står i förening med fasta landet på ett afstånd af 280 mil; vidare flyttade isberg fordom stora stenblock till deras vestra stränder och dessa hafva förr kunnat medföra räfvar, såsom nu ofta händer i de arktiska regionerna. Dock kan man icke säga, att små öar icke kunna bära åtminstone små däggdjur, ty de finnas i många delar af jorden på små öar om de ligga nära en kontinent och svårligen kan någon ö nämnas, på hvilken icke våra små fyrfotadjur blifvit naturaliserade och rikligt förökat sig. Man kan icke enligt den vanliga åsigten om skapelse säga, att tiden icke räckt till för däggdjurens skapelse; många vulkaniska öar äro tillräckligt gamla, hvilket bevisas af den ytterliga förstöring de undergått och af deras tertiära lager: det har också varit tid för bildandet af endemiska arter af andra klasser, och på kontinenter är det kändt, att däggdjur uppträda och försvinna vida hastigare än andra och lägre djur. Ehuru landdjur icke finnas på oceanöar, finnas dock flygande djur nästan på hvarje ö. Nya Zeeland eger två flädermöss som icke finnas annorstädes på jorden; Norfolksöarna, Fidschi-arkipelagen, Boninöarna, Carolinerna och Marianerna och Mauritius ega alla sina egendomliga flädermöss. Hvarföre, kan man fråga, har den antagna skapande kraften alstrat flädermöss och inga andra däggdjur på aflägsna öar? Enligt min åsigt kan denna fråga med lätthet besvaras, ty intet landdjur kan förflyttas öfver en vidsträckt hafsyta, men flädermöss kunna flyga öfver den. Man har sett flädermöss flyga på dagen långt ute på den atlantiska oceanen; och två nordamerikanska arter besöka regelbundet eller tillfälligtvis Bermudaön på 600 mils afstånd från fastlandet. Jag hör af Tomes, som särskildt studerat denna familj, att många arter hafva en annan utbredning och finnas på kontinenter och på långt aflägsna öar. Derföre behöfva vi blott antaga, att sådana vandrande arter hafva blifvit modifierade i sina nya hem efter sin nya belägenhet och vi kunna förstå tillvaron af endemiska flädermöss på oceanöar, jemte frånvaron af alla andra landdäggdjur.

Ett annat intressant förhållande eger rum emellan djupet af det haf som skiljer öar från hvarandra eller från närmaste fastland och graden af slägtskap hos de der boende däggdjuren. Windsor Earl har gjort några observationer i detta hänseende, hvilka sedan fått en stor utsträckning genom Wallaces beundransvärda undersökningar öfver den stora Malayiska arkipelagen, som nära Celebes delas af ett område af större djup, hvilket skiljer två i hög grad olika däggdjursfaunor. På hvar sida stå öarna på en måttligt grund bank och de bebos af samma eller mycket närslägtade däggdjur. Jag har hittills icke haft tillfälle att fullfölja detta ämne i alla delar af jorden, men så långt jag har hunnit gäller denna lag. Britannien är till exempel skild från Europa genom en grund kanal och däggdjuren äro desamma på båda sidor, och samma är äfven förhållandet med alla öar nära Australiens kuster. De vestindiska öarna å andra sidan stå på en betydligt djupare bank, nära 1000 famnar djup; här finna vi amerikanska former, men arterna och till och med slägtena äro fullständigt skilda. Då graden af modifikation som djur af alla slag undergå till en del beror på tidens längd, och då det är sannolikare att öar skilda från hvarandra eller från fastlandet genom grunda kanaler hafva i en senare period hängt tillsammans än öar som äro åtskilda genom djupare kanaler, så kunna vi begripa hvarföre en relation eger rum emellan djupet af ett haf som skiljer två däggdjursfaunor och graden af deras slägtskap, en relation, som är fullkomligt oförklarlig efter teorien om oberoende skapelser.

Föregående fakta rörande invånarna på oceanöar, — nämligen fåtalet af arter med en stor proportion af endemiska former — modifikationen af medlemmar af vissa grupper och icke andra i samma klass — frånvaron af hela ordningar, såsom batrachier och landdäggdjur, trots närvaron af flädermöss — de enkla förhållandena af våra växtordningar — örtartade växters utveckling till träd m. m. synas mig bättre öfverensstämma med tron på verksamheten hos tillfälliga transportmedel, som fortgått under en lång tids förlopp, än med tron på alla oceanöars fordna förbindelse med närmaste kontinent; ty med denna senare åsigt är det sannolikt att de olika klasserna skulle hafva flyttat mera likformigt, och då arterna invandrat i massa skulle icke deras ömsesidiga relationer blifvit mycket störda och följaktligen skulle de hafva modifierats antingen alldeles icke eller på ett mera likformigt sätt.

Jag förnekar icke, att många allvarsamma svårigheter finnas för att förstå, huru många af invånarna på de mera aflägsna öarna hafva kommit till sina nuvarande hem, vare sig de nu bibehålla samma specifika form eller blifvit modifierade. Men vi få icke förbise sannolikheten deraf att öar fordom kunnat existera såsom hvilopunkter, af hvilka numera icke finnes ett spår qvar. Jag vill särskildt fästa mig vid ett sådant svårt fall. Nästan alla oceaniska öar, äfven de mest isolerade och de minsta, bebos af landsnäckor, i allmänhet endemiska arter, men stundom arter som finnas annorstädes; tydliga exempel härpå hafva gifvits af A. A. Gould för Stilla hafvet. Nu är det bekant att landsnäckor med lätthet dödas af hafsvatten, deras ägg, åtminstone sådana som jag försökt, spricka deri och dö. Det måste dock enligt vår åsigt finnas något okändt, men tillfälligt medel för deras transport. Månne icke de nykläckta ungarna stundom hänga sig fast vid fötterna på fåglar som sofva på marken och på detta sätt flyttas? Det förekom mig som om landsnäckor under vintersömn med en membranös diafragma öfver snäckans mun, kunde förflyttas i remnor af drifved tvärt öfver måttligt vidsträckta hafsytor. Och jag fann att flera arter i detta tillstånd utan skada emotstod en sänkning i vatten under sju dagar; en snäcka, Helix pomatia, sedan den på detta sätt blifvit behandlad och återkommit i sin vintersömn, lades i tjugu dagar i hafsvatten, och hemtade sig fullständigt. Under denna tids förlopp skulle snäckan kunnat drifvas af en hafsström af medelhastighet till ett afstånd af 660 geografiska mil. Då denna Helix har ett tjockt kalkartadt operculum, tog jag bort det och sedan den bildat ett nytt membranöst lock, nedsänkte jag den ånyo i hafsvatten under fjorton dagar och den hemtade sig åter och kröp bort. Baron Aucapitaine har nyligen försökt liknande experiment: han lade 100 landsnäckor af tio slägten i en med hål genomborrad låda och sänkte den i hafvet under fjorton dagars tid. Af hundra snäckor hemtade sig tjugusju. Närvaron af en operculum synes hafva varit af vigt, då af tolf specimen af Cyclostoma elegans, som är försedd dermed, lefde elfva upp igen. Det är anmärkningsvärdt, att då Helix pomatia för mig så väl motstod saltvattnets inverkan icke en af femtiofyra exemplar tillhörande fyra arter af Helix kommo till lif igen i Aucapitaines försök. Det är likväl ingalunda sannolikt, att landsnäckor på detta sätt ofta förflyttats; fåglarnas fötter är en antagligare metod.



\section[Öinvånares slägtskap]{Öinvånarnas slägtskap med inbyggare på närmaste
fastland.}

Det mest egendomliga och för oss vigtiga faktum är slägtskapen emellan de arter som bebo öar och invånarna på närmaste fastland, ehuru de icke äro identiska. Talrika exempel derpå kunde gifvas. Gralapagos-arkipelagen ligger under eqvatorn på ett afstånd af 500 och 600 mil från Sydamerikas kuster. Nästan hvarje produkt af land och vatten bär här den allmänneliga prägeln af Amerikas kontinent. Der finnas tjugusex landfåglar och tjuguen eller möjligen tjugutre af dessa upptagas såsom bestämda arter och antagas hafva blifvit skapade här; den nära slägtskapen hos de flesta af dessa fåglar med amerikanska arter i nästan hvarje karakter, i vanor, åtbörder och ljud ligger för öppen dag. Så är äfven förhållandet med de öfriga djuren och en stor del af växterna såsom Hooker visat i sin utmärkta flora från denna arkipelag. En naturforskare som betraktar invånarna på dessa vulkaniska öar i Stilla hafvet på flera hundra mils afstånd från kontinenten tycker sig ännu stå på Amerikas fasta land. Hvarpå beror då detta? Hvarföre skola dessa arter, som antagas vara skapade på Galapagosarkipelagen och icke annorstädes, bära en så uppenbar prägel af slägtskap med dem som äro skapade i Amerika? Det finnes ingenting i lefnadsvilkoren, i öarnas geologiska beskaffenhet, i deras höjd eller klimat, eller i de olika klassernas proportioner, som synnerligen liknar förhållandena på Sydamerikas kust; i sjelfva verket finnes en grad af likhet i markens vulkaniska beskaffenhet, i klimatet, höjd och öarnas storlek emellan Galapagosarkipelagen och Cap Verdöarna, men hvilken fullständig olikhet emellan deras inbyggare! Inbyggarna på Cap Verdöarna äro beslägtade med Afrikas invånare likasom Galapagosöarnas med Amerikas. Fakta sådana som dessa tillstädja ingen slags förklaring enligt den vanliga åsigten om oberoende skapelser, hvaremot i öfverensstämmelse med här framhållna åsigter det är klart att Galapagosöarna skulle lätt emottaga kolonister från Amerika antingen genom tillfälliga transportmedel eller derigenom att de fordom sammanhängt med fasta landet, likasom Cap Verdöarna från Afrika, och att sådana kolonister skulle vara benägna för modifikation, men enligt grundsatsen om ärftlighet alltid förrådande sin ursprungliga födelseort.

Många analoga fakta kunde anföras: i sjelfva verket är det en nästan allmän regel, att de endemiska alstren på öar äro beslägtade med dem på närmaste fastland eller på närmaste ö. Undantagen äro få och de flesta af dem lätt förklarliga. Så till exempel ehuru Kerguelen Land ligger närmare Afrika än Amerika, äro växterna beslägtade med Amerikas och detta mycket nära såsom vi veta af Hookers uppgifter, men enligt åsigten att denna ö har blifvit beväxt af frön som blifvit medförda af isberg ditdrifna af de förherskande strömmarna försvinner denna anomali. Nya Zeeland är i sina endemiska växter mycket mera beslägtad med Australien, det närmaste fastlandet, än med någon annan region och detta är hvad vi kunde vänta; men det är också ganska nära beslägtadt med Sydamerika, hvilket ehuru den närmaste kontinenten är så enormt aflägset, att förhållandet blir en anomali. Men denna svårighet försvinner nästan enligt den åsigten att Nya Zeeland, Sydamerika och de andra sydliga länderna hafva delvis blifvit befolkade från en nästan mellanliggande ehuru aflägsen punkt nämligen från de antarktiska öarna, då de under en varmare tertiärperiod voro beklädda med vegetation före början af den sista isperioden. Den slägtskap hvilken, ehuru svag, dock verkligen enligt Hookers försäkringar eger rum emellan floran på sydvestra hörnet af Australien och Goda Hoppsudden är vida mera anmärkningsvärd; men den är inskränkt till växterna och lärer väl otvifvelaktigt någon dag få sin förklaring.

Samma lag som styrer slägtskapen emellan invånarna på öar och på närmaste fastland är stundom utvecklad i mindre skala men på ett intressant sätt inom gränserna af samma arkipelag. Hvarje särskild ö af Galapagosarkipelagen till exempel är bebodd af skilda arter och detta är märkvärdigt; men dessa arter äro vida närmare beslägtade med hvarandra än med invånarna i någon annan del af jorden. Detta är hvad man kunde vänta, ty öar som äro belägna så nära hvarandra böra nästan nödvändigt emottaga emigranter från samma ursprungliga källa och från hvarandra. Men hvad är orsaken att många af de inflyttade hafva blifvit olika modifierade, ehuru blott i ringa grad, på öar belägna inom synhåll, som hafva samma geologiska beskaffenhet, samma höjd, klimat etc. Detta tycktes mig länge vara en stor svårighet, men den beror till hufvudsaklig del på den djupt rotade villfarelsen att betrakta de fysiska förhållandena såsom de vigtigaste; hvaremot det icke kan bestridas att beskaffenheten af de andra inbyggarna, med hvilka hvar och en har att täfla, är ett åtminstone lika vigtigt och i allmänhet vida vigtigare moment för framgång. Om vi nu betrakta de arter, som bebo Galapagosarkipelagen och likaledes finnas i andra delar af jorden, finna vi dem visa betydliga skiljaktigheter på de olika öarna. Denna skiljaktighet kunde man hafva väntat, om öarna hafva befolkats genom tillfälliga transportmedel, om ett frö till exempel af en växt hade förts till en ö och af en annan växt till en annan ö, ehuru de alla framgått ur samma källa. Derföre om under fordna tider en inflyttande art först hade slagit sig ned på en af öarna eller om den sedermera spridde sig från den ena till den andra, skulle den otvifvelaktigt hafva råkat ut för olika förhållanden på de olika öarna, ty den skulle få att täfla med olika grupper af organismer: en växt skulle till exempel finna den för honom bäst lämpliga marken intagen af något olika arter på olika öar och skulle vara utsatt för anfall af något olika fiender. Om den då varierade, skulle det naturliga urvalet sannolikt gynna olika varieteter på de olika öarna. Några arter skulle likväl sprida sig och dock behålla samma karakter öfver hela gruppen, likasom vi se några arter sprida sig vida öfver en kontinent och förblifva desamma.

Det verkligen öfverraskande förhållandet på dessa Galapagosöar och i mindre grad i några analoga fall är att hvarje ny art sedan den blifvit bildad på någon ö icke hastigt spridde sig till de andra öarna. Men ehuru öarna ligga inom synhåll från hvarandra, äro de skilda genom djupa hafsarmar, i de flesta fall vidsträcktare än engelska kanalen, och det finnes intet skäl att antaga, att de i en förfluten period varit kontinuerligt förenade. Hafsströmmarna äro hastiga och stryka tvärt igenom arkipelagen och häftiga vindar äro sällsynta, så att öarna äro vida mera skilda från hvarandra i verkligheten än kartan utvisar. Icke desto mindre äro några arter gemensamma för de olika öarna, både af dem som finnas i andra delar af jorden och af dem som äro inskränkta till arkipelagen; och vi kunna antaga från deras närvarande fördelningssätt, att de hafva spridt sig från en ö till de andra. Men vi få ofta, tror jag, en oriktig åsigt om sannolikheten af närbeslägtade arters invandring inom hvarandras område, då de sättas i förbindelse med hvarandra. Otvifvelaktigt skall en art, om den har något företräde framför en annan, på mycket kort tid helt och hållet eller till en del uttränga den; men om båda äro lika väl lämpade för sina egna platser, skola sannolikt båda behålla sina roler och hålla sig skilda för huru lång tid som helst. Om vi äro förtrogna med det förhållande, att många arter som naturaliserats genom menniskans åtgöranden hafva spridt sig med förvånande hastighet öfver vidsträckta områden, kunna vi antaga att de flesta arter skola sprida sig på samma vis; men vi böra ihågkomma att de arter som naturaliserats i nya länder i allmänhet icke äro nära beslägtade med de ursprungliga inbyggarna, utan äro väl skilda former, som i det stora flertalet fall tillhöra skilda arter, såsom A. de Candolle har bevisat. På Galapagosarkipelagen äro till och med många fåglar skilda på hvarje ö, ehuru de äro så väl lämpade att flyga från ö till ö; sålunda finnas der tre närbeslägtade arter af härmtrasten (Mimus), hvar och en begränsad till sin ö. Låt oss nu antaga, att härmtrasten från Chathamön af vinden drifves till Charlesön, som har sin egen härmtrast; hvarföre skulle den lyckas stadfästa sig der? Vi kunna med säkerhet antaga, att Charlesön är väl befolkad med sina egna arter, ty årligen läggas flera ägg och kläckas ungar i större antal än som kunna uppfödas och vi kunna antaga att den för Charlesön egendomliga härmtrasten är åtminstone lika väl lämpad för sitt hem som den för Chathamön säregna. Sir C. Lyell och Wollaston hafva meddelat mig ett anmärkningsvärdt faktum angående detta ämne, nämligen att Madeira och den närliggande ön Porto Santo ega många skilda men representerande arter af landsnäckor, hvilka lefva i bergsklyftor, och ehuru stora qvantiteter af sten årligen förflyttas från Porto Santo till Madeira har denna senare ö ännu icke blifvit koloniserad af Porto Santo-arter: icke desto mindre hafva båda öarna blifvit koloniserade af europeiska landsnäckor, hvilka otvifvelaktigt hade någon fördel öfver de infödda arterna. Efter dessa betraktelser anser jag vi icke behöfva förundra oss öfver att de endemiska och representerande arter som bebo de olika öarna af Galapagosarkipelagen icke hafva spridt sig från ö till ö. Ett föregående besittningstagande har sannolikt spelat en vigtig rol i förhindrandet af de arters sammanblandning som bebo skilda distrikt med nästan samma fysiska förhållanden på samma kontinent. Sydvestra och sydöstra hörnet af Australien hafva nästan samma fysiska förhållanden och äro dock bebodda af ett stort antal skilda däggdjur, fåglar och växter.

Samma grundsats som styrer de allmänna karaktererna af invånarna på oceanöarna, nämligen deras relation till den källa hvarifrån kolonister lättast kunna hafva kommit, jemte deras följande modifikation har den vidsträcktaste tillämpning öfver hela naturen. Vi se den på hvarje bergspets, i hvarje sjö, i hvarje träsk. Ty alpina arter äro beslägtade med de omgifvande lågländernas, med undantag af de vid glacialperioden spridda; — vi hafva sålunda i Sydamerika alpina kolibris, alpina gnagare, alpina växter etc. alla strängt tillhörande amerikanska former; och det är klart att hvarje berg, då det småningom höjdes, skulle koloniseras från de omgifvande lågländerna. Så är äfven förhållandet med invånare i sjöar och träsk, så vida icke en stor lätthet för flyttning har tillåtit många af dessa arter att vara de herskande i större afdelningar af jorden. Vi se samma grundsats i karakteren af de flesta blinda djur som bebo grottorna i Amerika och Europa. Andra analoga fakta kunde anföras. Det skall som jag tror befinnas vara allmänt sant, att hvarhelst i två regioner, de må vara aldrig så aflägsna, många beslägtade eller representerande arter finnas, der finnas likaledes några identiska arter; och hvarhelst många beslägtade arter finnas, der finnas också många former som af några naturhistoriker upptagas såsom skilda arter och af andra såsom blotta varieteter; dessa tvifvelaktiga former visa oss stegen i modifikationsprocessen.

Relationen emellan vissa arters förmåga att flytta och flyttningens utsträckning, antingen för det närvarande eller i någon förfluten period, och existensen af beslägtade arter på aflägsna punkter af jorden kan visas på ett annat och mera allmänt sätt. Gould antydde för mig för längesedan, att i de fågelslägten som sprida sig vida öfver jorden hafva många af arterna mycket stor utbredning. Jag kan svårligen betvifla att denna regel har allmän giltighet, ehuru det torde vara svårt att bevisa den. Bland däggdjur se vi den tydligt utvecklad hos flädermöss och i mindre grad hos de kattartade (Felidæ) och hundartade (Canidæ) rofdjuren. Vi se samma regel i fördelningen af fjärilar och skalbaggar; den gäller äfven för de flesta invånare i sötvatten, ty många slägten af de mest skilda klasser sprida sig öfver jorden och många af arterna hafva en enorm spridning. Härmed menar jag icke, att alla arter i slägtena med stor utbredning sjelfva hafva stor spridning utan blott några af dessa. Ej heller menar jag att arterna i sådana slägten öfverhufvudtaget hafva en mycket stor utbredning; ty detta beror till stor del på huru långt modifikationsprocessen har gått; till exempel, två varieteter af samma art bebo Amerika och Europa och således har arten stor utbredning; men om variationen har gått litet längre, skulle dessa två varieteter upptagas såsom skilda arter och utbredningen skulle blifvit betydligt reducerad. Ännu mindre menas att arter som hafva förmåga att öfverstiga stängsel och sprida sig långt nödvändigt hafva stor utbredning, såsom till exempel kraftigt bevingade fåglar; ty vi få icke förglömma att vidsträckt utbredning innefattar icke blott förmågan att passera stängsel, utan den vida vigtigare förmågan att segra i kampen för tillvaron med främmande medtäflare. Men enligt den åsigten att alla arter af samma slägte, ehuru nu fördelade på de mest aflägsna punkter af jorden hafva härstammat från en enda stamfader böra vi finna och jag tror att vi såsom allmän regel också finna, att några åtminstone af arterna hafva vidsträckt utbredning.

Vi böra ihågkomma med afseende på alla organiska varelser, att många slägten äro af mycket gammalt ursprung och arterna böra i dessa fall hafva haft lång tid till spridning och följande modifiering. Det finnes också skäl att tro af geologiska bevis, att inom hvarje stor klass de lägre organismerna förändra sig i långsammare proportion än de högre; följaktligen hafva dessa haft mera tillfälle att sprida sig vidt omkring utan att förlora samma specifika karakter. Detta förhållande jemte det att fröen och äggen af nästan alla lågt organiserade varelser äro mycket små och bättre lämpade för långa transporter förklarar sannolikt en lag som länge iakttagits och som nyligen blifvit afhandlad af A. de Candolle angående växterna, nämligen att ju lägre en grupp af organismer står, ju större utbredning har den.

Ofvan anförda förhållanden — att lägre organismer hafva större utbredning än de högre — att af slägten med stor utbredning några arter äfven hafva stora områden — sådana förhållanden som att alpina alster, sjö- och träskprodukter i allmänhet äro beslägtade med dem som lefva på de kringliggande lågländerna och de torra länderna — den öfverraskande slägtskapen emellan invånare på öar och dem på närmaste fastland — den ännu närmare slägtskapen emellan de skilda invånarna på öar af samma arkipelag — äro oförklarliga enligt den vanliga åsigten om hvarje arts oberoende skapelse, men de äro förklarliga, om vi medgifva kolonisering från den närmaste och beqvämaste källa jemte kolonisternas följande modifiering, så att de passa för sina nya hem.



\section[Sammanfattning]{Sammanfattning af detta och föregående kapitel.}

I dessa kapitel har jag bemödat mig att visa, att om vi fästa tillbörligt afseende på vår okunnighet om de fullständiga verkningarna af förändringar i landets klimat och höjd, hvilka helt säkert hafva inträffat — om vi ihågkomma huru okunniga vi äro med afseende på de många egendomliga transportmedel — om vi hålla i minnet huru ofta en art har kunnat sprida sig öfver en vid area, och derefter utrotats i mellanliggande område — finnes icke någon oöfvervinnelig svårighet att tro, att alla individer af samma art, hvarhelst de finnas, hafva utgått från samma stamfäder. Och vi komma till samma slutsats, hvilken många naturhistoriker antagit under titel enkla skapelsecentra, genom olika betraktelser, isynnerhet från betydelsen af stängsel af alla slag och från den analoga fördelningen af subgenera, slägten och familjer.

Beträffande de skilda arter som tillhöra samma slägte, hvilka enligt vår teori måste hafva utgått från en källa, så om vi fästa samma afseende som förut på vår okunnighet och komma ihåg att några lifsformer hafva förändrats mycket långsamt, samt sålunda lemna stora tidsperioder till deras flyttning äro svårigheterna långt ifrån oöfvervinneliga, ehuru de i detta fall ofta äro stora likasom i det föregående.

För att visa verkningarna af klimatförändringar på fördelningen har jag försökt att visa, hvilken vigtig rol istiden har spelat, hvilken berörde äfven eqvatorialområdet, och hvilken under köldens vexlingar i norr och söder tillät de motsatta hemisferernas produkter att blanda sig och qvarlemnade några af dem strandade på bergspetsar i alla delar af jorden. För att visa huru olika de tillfälliga transportmedlen äro har jag något längre behandlat sötvattensalstrens spridningsmedel.

Om icke oöfvervinneliga svårigheter hindra oss att medgifva, att under tidens lopp alla individer af samma art och likaledes af flera arter af samma slägte hafva utgått från samma källa, äro alla stora satser i den geografiska fördelningen förklarliga enligt teorien om flyttning med följande modifikation och förökning af nya former. Vi kunna på detta sätt förstå den stora betydelsen af stängsel vare sig af land eller vatten deruti att de icke blott skilja utan äfven tydligen bilda de olika botaniska och zoologiska provinserna. Vi kunna på detta sätt förstå samlandet af beslägtade arter inom samma områden; och orsaken hvarföre under skilda latituder, till exempel i Sydamerika, invånarna på slätter och berg, i skogar och träsk och öknar äro förenade på ett så hemlighetsfullt sätt, och likaledes äro förenade med de utdöda varelser som fordom bebodde kontinenten. Om vi komma ihåg, att den ömsesidiga relationen emellan organism och organism är af högsta vigt, kunna vi se, hvarföre två orter under nästan samma fysiska förhållanden ofta bebos af mycket olika lifsformer, ty alltefter tidens längd som förflutit sedan kolonisterna beträdde en af regionerna eller båda, alltefter beskaffenheten af den kommunikation som tillät vissa former och inga andra att inflytta, antingen i större eller mindre antal, alltefter som de inflyttande råkade komma i mer eller mindre direkt täflan med hvarandra och infödingarna eller icke, och alltefter som de inflyttande voro i stånd att variera mer eller mindre hastigt, skulle i de två eller flera områdena oberoende af deras fysiska förhållanden uppkomma oändligt invecklade lifsvilkor, — skulle der bli en nästan ändlös grad af organisk verkan och återverkan, — och vi skulle finna, såsom vi också göra, några grupper af organiska varelser högeligen och några blott obetydligt modifierade, några utvecklade till stor kraft, andra lefvande i ringa antal i de olika stora geografiska provinserna.

Enligt samma grundsatser kunna vi förstå såsom jag bemödat mig att visa, hvarföre oceaniska öar hafva blott få inbyggare men af dessa ett stort antal äro endemiska eller säregna; och hvarföre alltefter flyttningsmedlen af en grupp af varelser äfven inom samma klass alla arter äro egendomliga och af en annan grupp desamma som i andra delar af jorden. Vi kunna se hvarföre hela grupper af organismer, såsom batrachier och landdäggdjur skola saknas på oceaniska öar, under det de mest isolerade öar ega sina egendomliga luftdäggdjur eller flädermöss. Vi kunna inse, hvarföre det finnes något samband emellan närvaron på öar af däggdjur i mer eller mindre modifieradt stadium och djupet af hafvet emellan sådana öar och fastland. Vi kunna klarligen se, hvarföre alla invånare på en arkipelag, ehuru specifikt skilda på flera öar, äro nära beslägtade med hvarandra och likaledes beslägtade, men mindre nära med dem på den närmaste kontinenten eller andra källor hvarifrån inflyttningar skett. Vi kunna se, hvarföre på två ytor der mycket nära beslägtade eller representerande arter finnas, ehuru aflägsna de än äro från hvarandra, alltid måste finnas några identiska arter.

Såsom den hädangångne E. Forbes ofta framhöll finnes en öfverraskande öfverensstämmelse i lagarna för lifvet i tid och rum: de lagar som styra successionen af lifsformerna i förflutna tider äro nästan desamma som de hvilka för det närvarande reglera skiljaktigheterna på olika områden. Vi se detta i många förhållanden. Varaktigheten af hvarje art och hvarje artgrupp är kontinuerlig i tiden; ty undantagen från regeln äro så få att de utan svårighet kunna antagas bero derpå, att vi icke hittills upptäckt i ett mellanliggande lager vissa former, som saknas deruti, men som finnas derofvan och derunder: så är äfven i rummet den allmänna regeln gällande, att ett område som bebos af en enda art eller af en artgrupp är kontinuerlig och undantagen, hvilka icke äro sällsynta kunna såsom jag sökt visa förklaras genom fordna flyttningar under olika omständigheter, eller genom tillfälliga transportmedel eller derigenom att arterna dött ut i de mellanliggande områdena. Både i tid och rum hafva arter och artgrupper sina höjdpunkter af utveckling. Grupper af arter som lefva under samma tidsperiod eller inom samma yta karakteriseras ofta genom gemensamma obetydliga drag såsom skulptur eller färg. Om vi se på den långa successionen af förflutna perioder likasom de aflägsna provinserna öfver jorden finna vi att arter i vissa klasser skilja sig föga från hvarandra under det andra i en skild klass eller i en skild familj af samma ordning skilja sig betydligt från hvarandra. I både tid och rum förändras de lågt organiserade varelserna af hvarje klass i allmänhet långsammare än de högre organiserade, men i båda fallen finnas anmärkningsvärda undantag från regeln. Enligt vår teori begripas lätt alla dessa relationer så väl i rum som tid; ty antingen vi fästa oss vid de lifsformer som förändrats under successiva perioder, eller vid dem som hafva förändrats sedan de flyttat till skilda områden, i båda fallen äro formerna inom hvarje klass förenade genom samma band af vanlig generation; och i båda fallen hafva lagarna för variationen varit desamma och modifikationer hafva samlats genom samma medel, naturligt urval.


