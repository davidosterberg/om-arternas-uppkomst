%ELFTE KAPITLET.



\chapter{Geografisk fördelning.}

{\it
Den närvarande fördelningen kan icke förklaras genom olikheter i fysiska förhållanden. — Verkan af stängsel. — Slägtskap emellan samma kontinents alster. — Skapelsecentra. — Spridningssätt genom förändringar i klimat och landets nivå och tillfälligheter. — Spridningssätt under istiden. — Omvexlande istider i norr och söder.
}\\[0.5cm]

Då vi betrakta de organiska varelsernas spridning öfver jordytan öfverraskar oss först det märkvärdiga förhållandet, att hvarken likheten eller olikheten emellan invånarna i olika trakter kunna förklaras genom dessas klimatiska eller andra fysiska förhållanden. Sedan gammalt har nästan hvarje författare som sysselsatt sig med detta ämne kommit till detta resultat. Förhållandet i Amerika är ensamt nästan tillräckligt att bevisa dess sanning; ty om vi utesluta de norra delarna, der det circumpolära landet är nästan sammanhängande, öfverensstämma alla författare deruti att en af grundvalarna för den geografiska fördelningen är skilnaden emellan den gamla och den nya verlden; dock om vi resa öfver det vidsträckta amerikanska fastlandet från de centrala delarna af Förenta Staterna till dess yttersta södra punkt, påträffa vi de mest olikartade fysiska förhållanden: fuktiga områden, torra öknar, höga berg, gräsbeväxta slätter, skogar, träsk, sjöar och floder vexla nästan under hvarje temperatur. Det finnes knappt ett klimat i den gamla verlden, som icke har sin motsvarighet i den nya, åtminstone så noga som hvarje art fordrar. Utan tvifvel kunna små ytor utvisas i gamla verlden, som äro hetare än någon trakt i den nya, men dessa äro icke bebodda af en fauna skiljaktig från inbyggarna i de omgifvande distrikten, ty det är mycket sällsynt att finna en grupp af organismer inskränkt till en viss yta med i någon ringa grad olika lifsvilkor. Trots denna allmänna motsvarighet i lifsvilkoren i nya och gamla verlden, hvilken ofantlig skilnad i de lefvande alstren!

I den södra hemisferen, om vi jemföra stora landsträckor i Australien, södra Afrika och vestra sydamerika emellan 25${}^\circ$ och 30${}^\circ$ latitud, skola vi finna områden som äro ytterligt lika i alla förhållanden och dock vore det svårt att uppgifva två faunor och floror som äro mera olika. Å andra sidan må vi jemföra Sydamerikas alster söder om 35${}^\circ$ och norr om 25${}^\circ$, hvilka följaktligen äro skilda genom ett mellanrum af tio graders latitud och lefva under betydligt olika lefnadsförhållanden och dock äro de ojemförligt närmare beslägtade med hvarandra än med alstren af Afrika eller Australien under nästan samma klimat. Analoga fakta kunde gifvas rörande hafvets bebyggare.

Ett annat förhållande som öfverraskar oss i vår allmänna öfversigt är att stängsel af något slag eller hinder för fri flyttning stå i nära och vigtigt samband med olikheterna emellan alstren i olika trakter. Vi se detta i den stora skilnaden emellan nästan alla landproduktioner från nya och gamla verlden, utom i de norra delarna, der landet nästan hänger tillsammans och der under ett obetydligt olika klimat flyttningen varit fri för de norra tempererade formerna liksom nu för de strängt arktiska. Vi se samma förhållande i den stora skilnaden emellan invånarna i Australien, Afrika och Sydamerika under samma latitud; ty dessa länder äro nästan så isolerade från hvarandra som möjligt. På hvarje kontinent se vi också samma förhållande; ty på motsatta sidor af höga och sammanhängande bergsträckor, stora öknar och till och med stora floder finna vi olikartade produkter, ehuru olikheterna till graden äro mindre, då bergskedjor, öknar m. m. icke äro oöfverstigliga eller sannolikt icke hafva funnits lika länge som oceanerna hvilka skilja kontinenterna från hvarandra.

Om vi vända oss till hafvet finna vi samma lag. Inbyggarna på vestra och östra stränderna af Sydamerika äro betydligt olika och hafva ytterligt få fiskar, snäckor eller krabbor gemensamma; men Günther har nyligen visat, att på motsatta sidan af Panamanäset omkring 30 procent af fiskarna äro desamma, hvilket föranledt naturforskarna att antaga näset fordom hafva varit öppet. Vester ut från Amerikas kuster sträcker sig en vidsträckt öppen oceanyta, utan en ö såsom hållpunkt för emigranter; här hafva vi ett stängsel af annat slag och så snart detta är passeradt, möta vi på de östra öarna i Stilla hafvet en annan och helt olika fauna. Tre hafsfaunor sträcka sig således långt norrut och söderut parallelt med hvarandra på ringa afstånd under motsvarande klimat, men då de äro skilda från hvarandra genom oöfverstigliga stängsel antingen af land eller öppen sjö, äro de nästan fullständigt skilda. Å andra sidan om vi gå ännu längre vesterut från de östra öarna i de tropiska delarna af Stilla hafvet träffa vi icke på några oöfverstigliga hinder, och vi hafva oräkneliga öar till hållpunkter eller sammanhängande kuster, till dess vi efter att hafva rest öfver en hemisfer komma till Afrikas kuster, och öfver hela denna vidsträckta yta påträffa vi ingen väl markerad och bestämd hafsfauna. Ehuru så få snäckor, krabbor eller fiskar äro gemensamma för de ofvannämda tre närliggande faunor på östra och vestra Amerika och de östra öarna i Stilla hafvet, sprider sig dock mången fisk från Stilla hafvet in i Indiska oceanen och många snäckor äro gemensamma för östra öarna af Stilla hafvet och östra kusterna af Afrika på nästan motsatta longitudmeridianer.

Ett tredje förhållande som delvis innefattas i det föregående är slägtskapen emellan alstren af samma kontinent eller samma haf, ehuru arterna sjelfva äro skilda på olika punkter och stationer. Det är en lag af den vidsträcktaste allmänhet och hvarje kontinent erbjuder talrika exempel. En naturhistoriker som reser till exempel från norr till söder blir ickedestomindre öfverraskad af det sätt hvarpå successiva grupper af varelser, specifikt skilda och nära beslägtade ersätta hvarandra. Han hör närbeslägtade ehuru skilda slag af fåglar sjunga och ser deras bon nästan lika bygda, ehuru ej fullkomligt, med ägg färgade på nästan samma vis. Slätterna nära Magellans sund bebos af en art Rhea (amerikansk struts) och La Plataslätterna norrut af en annan art af samma slägte och icke af någon äkta struts (Struthio) eller Emu (Dromæus), lik dem som bebo Afrika och Australien under samma latitud. På samma slätter af La Plata se vi aguti (Dasyprocta) och bizcacha (Lagostomus), hvilka hafva nästan samma vanor som våra harar och kaniner och tillhöra samma ordning, gnagarna, men de förete en fullkomligt amerikansk bygnadstyp. Vi bestiga de höga spetsarna af Cordillererna och vi finna en alpin bizcacha (Lagidium), vi vända oss till vattnet och finna icke bäfvern och bisamråttan utan i stället coypu (Myopotamus) och capybara (Hydrochærus), gnagare af den sydamerikanska typen. Oräkneliga andra exempel kunde gifvas. Om vi se på öarna vid de amerikanska kusterna, äro invånarna, ehuru skilda arter, hufvudsakligen amerikanska, huru mycket öarna än skilja sig i geologiskt hänseende. Vi kunna se tillbaka till förflutna tider, såsom vi visat i sista kapitlet, och vi finna amerikanska typer förherskande på den amerikanska kontinenten och i amerikanska haf. Vi se i dessa fakta ett djupliggande organiskt band igenom tid och rum öfver samma ytor af land och vatten och oberoende af de fysiska förhållandena. Den naturforskare måste icke vara vetgirig som icke drifves att utforska detta band.

Detta band är helt enkelt arf, denna lag som allena så vidt vi med bestämdhet känna gör organismerna fullkomligt lika hvarandra eller, såsom vi se i varieteterna, nästan lika hvarandra. Olikheten emellan inbyggarna i olika trakter kan tillskrifvas modifikation genom naturligt urval och i underordnad grad det bestämda inflytandet af skiljaktiga fysiska förhållanden. Graden af olikhet beror på att flyttning af de mera dominerande lifsformerna från en trakt till en annan blifvit med mer eller mindre framgång förekommen i en mer eller mindre aflägsen period; — på beskaffenheten och antalet af de fordna nykomlingarna — och på invånarnas inflytande på hvarandra, som har till följd bevarandet af olika modifikationer, i det organismernas förhållande till hvarandra i kampen för tillvaron såsom jag redan ofta nämt är den vigtigaste af alla relationer. Den stora vigten af stängsel kommer således att visa sig deruti, att flyttning hindras, liksom vigten af tiden uti den långsamma modifikationsprocessen genom naturligt urval. Arter med stor utbredning, rika på individer, hvilka redan segrat öfver mången medtäflare i sina egna vidsträckta hemland, hafva största utsigterna att intaga nya platser, då de sprida sig i nya områden. I sina nya hem komma de i nya förhållanden och undergå ofta vidare modifikation och förädling, och på detta sätt blifva de fortfarande segrare och lemna efter sig grupper af modifierade ättlingar. Enligt denna grundsats om härstamning med modifikation kunna vi förstå hvarföre hela sektioner af slägten, hela slägten och till och med familjer äro inskränkta till samma ytor, hvilket såsom bekant så ofta är förhållandet.

Såsom jag i sista kapitlet nämde tror jag icke på någon lag om nödvändig utveckling. Då föränderligheten hos hvarje art är en oberoende egenskap och begagnas af det naturliga urvalet blott så vida den gagnar hvarje individ i dess invecklade kamp för tillvaron, blir graden af modifikation hos skilda arter icke någon likformig storhet. Om ett antal arter, sedan de i sina gamla hem kämpat med hvarandra, flytta i massa till ett nytt och sedermera isoleradt land, skulle de vara föga benägna för variation, ty hvarken flyttning eller isolering kan i och för sig åstadkomma någonting. Dessa grundsatser komma i verksamhet blott genom att bringa organismerna i nya relationer till hvarandra och i mindre grad till de omgifvande fysiska förhållandena. Då vi hafva sett i sista kapitlet att några former hafva bibehållit nästan samma karakter från en ofantligt aflägsen geologisk period, så hafva vissa arter flyttat öfver vidsträckta rymder och hafva blifvit föga eller alldeles icke modifierade.

Enligt denna åsigt är det klart att arter af samma slägte ehuru beboende de mest skilda verldsdelar måste hafva ursprungligen framgått ur samma källa, då de härstamma från samma stamfader. Hvad de arter beträffar, som hafva under hela geologiska perioder undergått blott ringa modifikation, är det icke mycket svårt att tro, att de hafva flyttat från samma trakt ty under de stora geografiska och klimatiska förändringar som hafva inträffat sedan uråldriga tider är nästan hvilken grad af flyttning som helst möjlig. Men i många andra fall, i hvilka vi hafva skäl att tro, att arterna af ett slägte hafva bildats inom en jemförelsevis ny tid, finnas stora svårigheter i detta hänseende. Det är också klart, att individerna af samma art, ehuru de nu bebo skilda och isolerade områden måste hafva utgått från en punkt, der deras föräldrar först bildades; ty såsom vi utvecklat i föregående kapitlet är det otroligt att identiskt lika individer skulle uppkomma från specifikt skilda föräldrar.



Enkla skapelsecentra.

Vi hafva således kommit till den fråga som blifvit så vidlyftigt behandlad af naturforskare, nämligen om arterna hafva bildats på en eller flera punkter af jordytan. Otvifvelaktigt finnas många fall, som göra det ytterst svårt att förstå, huru samma art möjligen skulle kunna flytta från en punkt till de skilda och isolerade ställen der den nu anträffas. Icke destomindre öfverväldigar oss den tanken genom sin enkelhet, att hvarje art först uppkom på ett enda område. Den som förkastar den, förkastar vera causa, vanlig fortplantning med följande flyttning och åberopar underverk. Det medgifves allmänt, att i de flesta fall det af en art bebodda området är sammanhängande, och att om en växt eller ett djur bebor två punkter så långt skilda från hvarandra eller med ett mellanrum af sådan beskaffenhet, att det icke kunde med lätthet passeras under flyttning, upptages detta förhållande såsom något märkvärdigt och såsom undantagsfall. Omöjligheten att flytta öfver ett vidsträckt haf är tydligare för landdäggdjur än för någon annan organisk varelse; och följaktligen finna vi icke något oförklarligt exempel på däggdjur som bebo skilda delar af jorden. Ingen geolog finner något besynnerligt deruti att Storbritannien eger samma fyrfotadjur som det öfriga Europa, ty de voro otvifvelaktigt en gång förenade. Men om samma art kan uppkomma på två skilda punkter, hvarföre finna vi icke ett enda däggdjur gemensamt för Europa och Australien eller Sydamerika? Lifsvilkoren äro nästan desamma, så att en mängd af europeiska djur och växter hafva blifvit naturaliserade i Amerika och Australien och några af de ursprungliga växterna äro identiskt desamma på dessa aflägsna punkter i norra och södra hemisferen? Svaret är som jag tror att däggdjuren icke kunnat flytta, hvaremot några växter i följd af deras olika spridningssätt hafva flyttat öfver det vidsträckta och afbrutna mellanrummet. Det stora och öfverraskande inflytandet af stängsel af hvarje slag kan blott begripas enligt den åsigten, att det stora flertalet arter hafva alstrats på en sida och icke hafva varit i stånd att flytta öfver på den andra. Några få familjer, många underfamiljer, ganska många slägten och ett ännu större antal sektioner äro hänvisade till ett enda område; och flera naturhistoriker hafva iakttagit, att de naturligaste slägten, eller de slägten i hvilka arterna äro närmast beslägtade med hvarandra, äro i allmänhet inskränkta till samma trakt, eller om de hafva en vidsträckt utbredning, är deras utbredningsområde sammanhängande. Hvilken sällsam anomali skulle det icke vara, om en rakt motsatt regel vore förherskande om vi gå ett steg längre ned i serien nämligen till individerna af samma art, om dessa icke hade åtminstone i början varit inskränkta till ett visst område!

Deraf synes mig, liksom mången annan naturforskare, att den mest sannolika åsigten är att hvarje art blifvit bildad på en yta allena, och sedermera flyttat från denna yta så långt som dess flyttningsförmåga och subsistensmedlen under förflutna och närvarande förhållanden tillåta. Otvifvelaktigt finnas många fall, i hvilka vi icke kunna förklara, huru en art kunde hafva passerat från en punkt till en annan. Men de geografiska och klimatiska förändringar som säkert inträffat under en sen geologisk period måste hafva söndrat många arters fordom sammanhängande områden. Vi bringas således att taga i betraktande, om undantagen från utbredningsområdenas kontinuitet äro så talrika och af så allvarsam beskaffenhet, att vi böra uppgifva den tro som allmänna betraktelser gjort sannolik, att hvarje art bildats inom ett område och flyttat derifrån så långt den kunde. Det skulle blifva för långt att afhandla alla undantagsfall af arter som bebo aflägsna och skilda punkter, ej heller vill jag ett ögonblick påstå, att någon förklaring kan lemnas på många fall. Men efter några inledande anmärkningar vill jag behandla några få de mest slående klasser af fakta, nämligen tillvaron af samma arter på spetsarna af aflägsna bergskedjor och på aflägsna punkter i de arktiska och antarktiska regionerna, och för det andra (i följande kapitel) den vidsträckta utbredningen af sötvattensprodukter, och för det tredje närvaron af samma landarter på öar och fastland skilda af hundratals mil öppen sjö. Om tillvaron af samma arter på aflägsna och isolerade punkter af jordytan i många fall kan förklaras enligt åsigten att hvarje art har flyttat från sin ursprungliga födelseort, då synes mig den åsigten, att en enkel födelseort är lag, vara den ojemförligt säkraste, i betraktande af vår okunnighet om fordna klimatiska och geografiska förändringar och de olika tillfälliga transportmedlen.

När vi behandla detta ämne, skola vi vara i tillfälle att samtidigt betrakta en för oss lika vigtig punkt, nämligen om arterna af ett slägte, hvilka enligt vår teori alla måste härstamma från en gemensam stamfader, kunna hafva flyttat från ett område och under flyttningen undergått modifikationer. Då de flesta af arterna som bebo en trakt äro olika arterna i en annan trakt, men nära beslägtade eller tillhörande samma slägte, om i alla sådana fall det kan visas, att flyttning sannolikt vid någon period försiggått från det ena området till det andra, skulle vår åsigt få ett starkt stöd; ty förklaringen är tydlig enligt teorien om härstamning med modifikation. En vulkanisk ö till exempel lyftad och bildad på ett afstånd af några få hundra mil från ett fastland, skulle sannolikt erhålla derifrån under tidernas lopp några få kolonister, och deras efterkommande, ehuru modifierade, skulle alltid genom arf vara beslägtade med inbyggarna på kontinenten. Fall af denna beskaffenhet äro vanliga och äro såsom vi hädanefter skola se oförklarliga enligt teorien om oberoende skapelser. Denna åsigt om relationen emellan arterna i flera trakter skiljer sig icke mycket från den af Wallace uttalade, hvilken kommit till den åsigten, att ”hvarje art vid sin uppkomst sammanfallit både i tid och rum med en förut lefvande beslägtad art,” och jag vet nu, att han tillskrifver detta ”sammanfallande” härstamning med modifikation.

Diskussionen om ”enkla och multipla skapelsecentra” leder icke direkt till en annan beslägtad fråga, — nämligen om alla individer af samma art äro utgångna från ett enda par, eller en hermafrodit, eller såsom några författare antaga från många samtidigt bildade individer. Bland organiska varelser som aldrig kroaseras, om sådana existera, måste hvarje art härstamma från en följd af modifierade varieteter, hvilka hafva utträngt hvarandra, men hvilka aldrig hafva blandat sig med andra individer eller varieteter af samma art, så att vid hvarje successivt stadium af modifikation och förädling alla individer af samma varietet härstamma från en enda stam. Men i det stora flertalet fall, nämligen bland alla organismer som vanligen para sig för hvarje befruktning eller tillfälligt kroaseras, hållas individerna af samma art som bebo samma område nästan likformiga genom kroasering; så att många individer alltjemt förändras samtidigt och hela graden af förädling vid hvarje stadium icke kan tillskrifvas härstamning från en enda stamfader. Jag vill förtydliga hvad jag menar: våra engelska rashästar skilja sig från hästar af hvarje annan ras; men denna olikhet och öfverlägsenhet beror icke på att de härstamma från ett enda par, utan på fortsatt omsorg vid urvalet och behandlingen af många individer under hvarje generation.

Förrän vi afhandla de tre klasser af fakta, hvilka jag har utvalt såsom företeende den största svårigheten för teorien om ”enkla skapelsecentra”, måste jag säga några få ord om spridningssätten.



\section{Spridningssätt.}

Sir C. Lyell och andra författare hafva med stor skicklighet behandlat detta ämne; här kan jag blott gifva det kortaste utdrag af de vigtigaste fakta. Förändring i klimat måste hafva haft ett mäktigt inflytande på flyttningen; en trakt som numera är omöjlig att passera kan hafva varit en allmän stråkväg för flyttning då dess klimat var olika det närvarande; jag skall derföre behandla denna afdelning af ämnet något mera i detalj. Höjdförändringar af landet måste också hafva utöfvat stort inflytande: ett smalt näs skiljer nu två hafsfaunor; sänk det eller antag att det fordom varit sänkt och de två faunorna skola nu blandas eller hafva förr varit blandade; der hafvet nu sträcker sig kan i fordna tider land hafva förenat öar eller möjligen äfven kontinenter och på detta sätt tillåtit landproduktioner att gå från den ena till den andra. Ingen geolog vill bestrida att stora nivåförändringar hafva inträffat under de nu lefvande organismernas period. Edvard Forbes framhöll, att alla öar i Atlantiska oceanen måste nyligen hafva varit förenade med Europa eller Afrika, och Europa likaledes med Amerika. Andra författare hafva på detta sätt hypotetiskt slagit bryggor öfver hvarje ocean och förenat nästan hvarje ö med något fastland. Om i sjelfva verket de af Forbes använda argumenten äro tillförlitliga, måste det medgifvas att knappt en enda ö existerar som icke nyligen varit förenad med fast land. Denna åsigt afskär den gordiska knuten, spridningen af samma art till de mest aflägsna punkter, och aflägsnar mången svårighet, men så vidt jag kan döma äro vi icke berättigade att medgifva så enorma geografiska förändringar inom de nu existerande arternas period. Det synes mig att vi hafva tillräckliga bevis på stora nivåförändringar af land och haf, men icke på så vidsträckta förändringar i våra kontinenters läge och utsträckning, som hafva förenat dem inom en nyare tid med hvarandra eller med flera mellanliggande halföar. Jag medgifver tillfullo den fordna tillvaron af många öar numera begrafda under sjön, hvilka hafva kunnat tjena till hvilpunkter för växter och många djur under flyttning. I de korallbildande oceanerna betecknas sådana sjunkna öar af korallringar eller atoller, som stå öfver dem. Då det blir allmänt antaget, hvilket otvifvelaktigt någon gång sker, att hvarje art har utgått från en enkel födelseort och då under tidernas lopp vi känna någonting bestämdt om spridningsmedlen, skola vi vara i stånd att med säkerhet spekulera öfver landets fordna utsträckning. Men jag tror icke att det någonsin kan bevisas, att inom den nyare perioden de flesta af våra kontinenter, som nu stå fullkomligt skilda, hafva varit sammanhängande, förenade med hvarandra och med öarna i oceanen. Flera fakta — såsom den stora olikheten emellan hafsfaunorna på båda sidor om nästan hvarje kontinent — den nära slägtskapen emellan de tertiära invånarna både till lands och sjös och de nuvarande inbyggarna — slägtskapen emellan däggdjur som bebo öar med dem på den närmaste kontinenten, delvis bestämd (såsom vi framdeles skola se) af den mellanliggande oceanens djup — dessa och andra sådana fakta synas mig förhindra medgifvandet af sådana rikliga geografiska revolutioner inom en nyare tid, som äro nödvändiga enligt de åsigter Forbes framstälde och som upptogos af hans många efterföljare. Naturen och de relativa proportionerna af invånarna på de oceaniska öarna synas mig likaledes förbjuda antagandet af deras fordna sammanhang med kontinenten. Ej heller gynnar den nästan allmänt vulkaniska sammansättningen af sådana öar antagandet att de äro spillrorna af sjunkna kontinenter; — om de ursprungligen hade existerat såsom sammanhängande bergskedjor skulle åtminstone några af öarna hafva bildats lika med andra bergspetsar af granit, metamorfisk skiffer, gamla fossilförande och andra klippor i stället för att bestå af blotta käglor af vulkaniska ämnen.

Jag måste nu säga några ord om hvad som kallas ”tillfälliga” spridningsmedel, men hvilka med skäl kunde kallas ”lägliga”, och inskränker mig här till växterna. I botaniska arbeten uppgifves den eller den växten vara illa lämpad för vidsträckt spridning, men för transport öfver hafvet kan den större eller mindre lättheten sägas vara fullkomligt okänd. Förr än jag med Berkeleys tillhjelp försökte några experiment, var det icke ens kändt i hvad mån frön kunde motstå den skadliga inverkan af hafsvattnet. Till min öfverraskning fann jag, att af 87 slag 64 grodde efter en indränkning af 28 dagar, och några få öfverlefde 137 dagars nedsänkning i vattnet. Det förtjenar anmärkas att vissa ordningar skadades långt mera än andra: nio Leguminoser underkastades försöket och med ett undantag motstodo de föga saltvattnets inverkan; sju arter af de beslägtade ordningarna Hydrophyllaceæ och Polemoniaceæ dödades alla genom en månads indränkning. För beqvämlighets skull använde jag hufvudsakligen små frön utan kapsel eller frukt, och då alla af dessa sjönko inom få dagar kunde de icke hafva förts öfver vidsträckta rymder af hafvet äfven om de undgått hafsvattnets skadliga inverkningar. Derefter försökte jag några större frukter, kapslar med flera, och några af dessa höllo sig en lång tid flytande. Det är väl bekant hvilken stor olikhet finnes emellan friskt och torkadt virkes förmåga att flyta, och jag tänkte mig att strömmar kunde rycka ned växter eller grenar och att dessa kunde torkas på stränderna och derefter genom en förökad hastighet i strömmen åter drifvas ut i sjön. Jag fick deraf föranledning att torka stammar och grenar af 94 växter med mogna frukter och lägga dem i hafsvattnet. Flertalet sjönk raskt, men några, som medan de voro gröna flöto blott en kort tid, höllo sig mycket längre flytande, då de voro torra; mogna hasselnötter sjönko genast, men då de voro torra höllo de sig flytande i 90 dagar och grodde sedan då de planterades; en sparrisplanta med mogna bär flöt i 23 dagar; då den var torr flöt den i 85 dagar och fröen grodde sedermera; de mogna fröen af Helosciadium sjönko på två dagar, då de voro torra höllo de sig flytande i öfver 90 dagar och bibehöllo sin groningsförmåga. Tillsammans af 94 torra växter blefvo 18 flytande i öfver 28 dagar och några af de 18 flöto mycket längre. Alltså grodde 64/87 (= 0,74) frön efter en indränkning af 28 dagar, och 18/94 (= 0,19) af de torra växterna med mogna frukter (dock till en del andra arter än de förra) flöto mer än 28 dagar; och om man får draga några slutsatser af dessa obetydliga fakta, skulle 14 procent af växterna i ett område kunna hållas flytande af hafsströmmar under 28 dagar och ändå behålla sin groningsförmåga. Efter Johnstons Physical Atlas är hastigheten i flera atlantiska strömmar 33 mil per dag (några med en medelhastighet af 60 mil per dag); efter detta medeltal skola frön af 14 procent växter i en trakt kunna flottas 921 mil öfver hafvet till ett annat land, och om de stranda och af landvindar drifvas till någon lämplig fläck, skola de gro.

Efter mig anstälde Martens liknande försök, men på mycket bättre vis, ty han nedlade frön i en låda i sjelfva hafvet, så att de omvexlande voro utsatta för väta och luft likt verkligt flytande växter. Han försökte 98 frön, de flesta af andra slag än mina, men han utvalde många stora frukter och likaledes frön af växter som lefva nära hafvet, och detta torde hafva gynnat varaktigheten af deras förmåga att flyta och att motstå saltvattnets skadliga inverkan. Å andra sidan torkade han icke växterna först eller de med frukter försedda grenarna, och detta torde hafva, såsom vi sett, varit anledning till att några af dem mycket längre höllo sig flytande. Resultatet var att 18/98 (= 0,185) af hans frön höllo sig flytande i 42 dagar och derefter ännu egde förmåga att gro. Men jag tviflar icke att växter som äro utsatta för vågorna skulle flyta kortare tid än de i våra experiment för häftig rörelse skyddade. Derföre torde det måhända vara säkrare att antaga, att fröen af omkring 10 procent af en floras växter efter torkning kunna flyta öfver en hafsyta af 900 mil och sedan gro. Det förhållande att större frukter ofta flyta längre än de små erbjuder särskildt intresse, då växter med stora frön eller frukter svårligen kunna transporteras på annat sätt, och A. de Candolle har visat att sådana växter i allmänhet hafva inskränkta gränser.

Men frön kunna tillfälligtvis transporteras på annat sätt. Drifved kastas upp på de flesta öar, äfven de som äro belägna midt i de vidsträcktaste oceaner, och infödingarna på korallöarna i Stilla hafvet skaffa sig stenar till sina redskap från rötterna af sådan drifved, och dessa stenar äro en dyrbar skatt till konungarna. Vid undersökning finner jag, att om oregelbundet formade stenar äro inbäddade i trädrötterna, små jordstycken ofta äro inneslutna i mellanrummen emellan dem och bakom dem, så fullständigt att icke den minsta del kan sopas bort under den längsta transport; från en liten jordbit, som på detta sätt var fullkomligt innesluten af veden i en omkring 50 år gammal ek grodde tre dicotyledona växter: jag är säker på noggranheten i denna iakttagelse. Jag kan vidare visa att döda fåglar då de flyta på hafvet stundom undgå att genast uppslukas och frön af många slag bibehålla i flytande fåglars kräfvor länge sin vitalitet: ärter och vicker till exempel dödas genom blott några få dagars insänkning i hafsvatten; men några sådana ifrån kräfvan af en dufva som i 30 dagar flutit på konstgjordt saltvatten visade sig till min förvåning nästan alla grobara.

Lefvande fåglar kunna näppeligen undgå att vara högeligen verksamma vid fröens förflyttning. Jag kunde gifva många fakta som visa huru ofta frön af många slag drifvas af vinden vida omkring öfver oceanen. Vi kunna med säkerhet antaga att under sådana omständigheter deras flyghastighet ofta skulle vara 35 mil i timmen och några författare hafva uppgifvit vida högre tal. Jag har aldrig sett ett exempel af närande frön som passerat igenom en fågels tarmkanal, men hårda frön gå oskadade äfven igenom en kalkons digestionsorganer. Under loppet af två månader plockade jag i min trädgård tolf slags frön ur exkrementerna af små fåglar och dessa tycktes fullkomliga och några som sattes på prof grodde. Men följande faktum är vigtigare; fåglarnas kräfva afsöndrar ingen magsaft och skadar icke det minsta, såsom jag af försök vet, fröens groningsförmåga; nu påstås med säkerhet, att då en fågel funnit och uppätit ett rikligt förråd af näring, alla frön icke komma ned i magen förr än 12—18 timmar efter måltiden. En fågel kan på denna tid lätteligen drifvas 500 mil bort, och då falkar äro kända för att uppspana trötta fåglar, så kan innehållet i deras uppslitna kräfva på detta sätt med lätthet spridas omkring. Många falkar och ugglor sluka sitt byte helt och uppkräkas efter tolf till tjugu timmar fjädrar, hvilka såsom jag vet af experiment i zoologiska trädgården innehålla grobara frön. Några frön af hafra, hvete, hirs, kanariegräs, hampa, klöfver och betor grodde efter att hafva i tolf till tjuguen timmar legat i magen af olika roffåglar och två betfrön grodde efter att hafva på detta sätt legat qvar i två dagar och fjorton timmar. Insjöfisk äter frön af många land- och vattenväxter: fiskar uppslukas vanligen af fåglar och på detta sätt kunna frön transporteras från den ena orten till den andra. Jag införde många slags frön i magarna på döda fiskar och lemnade derefter deras kroppar åt fiskörnar, storkar och pelikaner; efter en mellantid af flera timmar antingen uppkastades de tillsammans med fjädrar och hår eller öfvergingo i exkrementerna och flera af dessa frön behöllo sin groningsförmåga. Vissa frön blefvo likväl dödade under denna process.

Gräshoppor drifvas stundom långa sträckor från land; sjelf fångade jag en 370 mil från kusten af Afrika och jag har hört omtalas andra som blifvit tagna på ännu större afstånd. Sir C. Lyell har af Rev. R. T. Lowe emottagit det meddelande att i November 1844 stora svärmar af gräshoppor besökte Madeira. De voro oräkneliga till antal, så täta som snöflingor i den svåraste snöstorm och sträckte sig uppåt så långt man kunde se med kikare. Under två eller tre dagar svärmade de blott omkring långsamt i luften i en stor ellips åtminstone fem eller sex mil i diameter och om natten slogo de ned i de större träden, som voro alldeles betäckta af dem. Derefter försvunno de öfver hafvet lika hastigt som de kommit och hafva aldrig sedan besökt ön. I några delar af Natal tro några af farmers ehuru på otillräckliga skäl, att ogräsfrön införas i deras åkerfält med exkrementerna af de stora gräshoppssvärmar som ofta hemsöka detta land. För att öfvertyga mig om riktigheten häraf sände mig mr Weale i ett bref ett litet stycke torkade exkrementer i hvilka jag under mikroskopet fann flera små frön, och från dem uppdrog jag flera växter af två arter i två slägten. På detta sätt kan en gräshoppsvärm, sådan som den på Madeira, lätt blifva ett medel att införa flera växtslag på en ö som ligger långt från fastland.

Ehuru fåglarnas näbbar och fötter i allmänhet äro rena, finnes dock stundom något jord fasthängande vid dem; i ett fall fann jag sextioen frön och i ett annat tjugutvå frön i torr lerhaltig jord från foten af en rapphöna och i jorden fans en liten sten så stor som ett vickerfrö. Här är ett bättre fall: ett ben af en morkulla sändes mig af en vän med ett litet stycke torr jord vidhängande, som vägde blott nio gran, och detta innehöll ett frö af Juncus bufonius, som grodde och blommade. Många fakta kunde anföras som bevis, att marken nästan öfverallt är försedd med frön; prof. Newton sände mig till exempel benet af en rödbent rapphöna (Caccabis rufa) som blifvit sårad och icke kunde flyga, och en derpå fastsittande hård jordklimp som vägde sex och ett halft uns. Jordstycket hade bevarats i tre år, men då det sönderbröts, fuktades och sattes under en glasklocka uppspirade icke mindre än 82 växter derifrån: dessa bestodo af tolf monocotyledoner, deribland den vanliga hafran och 70 dicotyledoner, hvilka att döma af de unga bladen tillhörde åtminstone tre skilda arter. Med sådana fakta framför oss, kunna vi nu betvifla att många fåglar som af vinden årligen drifvas öfver stora sträckor af hafvet och som årligen flytta — milliontals vaktlar till exempel öfver Medelhafvet— tillfälligtvis medföra några frön som sitta fast i smuts på deras fötter eller näbbar? Men jag skall återkomma till detta ämne.

Isberg äro som bekant ofta belastade med jord och stenar och hafva äfven medfört ris, ben och bon af landfåglar, så att det svårligen kan betviflas att de tillfälligtvis, såsom Lyell framhållit, hafva förflyttat frön från en trakt till en annan i de arktiska och antarktiska regionerna, och under istiden från en del af de nu tempererade regionerna till en annan. Af det stora antal växter på Azoriska öarna som tillhöra Europa i jemförelse med arterna på andra öar i Atlantiska hafvet, som ligga närmare fastlandet, och (såsom H. C. Watson anmärkt) från deras något nordliga karakter i förhållande till latituden, misstänkte jag att dessa öar till en del blifvit beväxta af frön som under istiden med isberg blifvit transporterade dit. På min begäran skref sir C. Lyell till Hartung för att fråga, om han hade iakttagit några erratiska block på dessa öar och han svarade, att han funnit stora stycken af granit och andra klippor som annars icke finnas i arkipelagen. Deraf kunna vi med visshet sluta, att isberg fordom aflastade sin börda af klippor på stränderna af dessa öar midt i hafvet, och det är åtminstone möjligt att de kunnat medföra frön af växter från nordligare trakter.

Om vi taga i betraktande, att dessa olika transportmedel och andra medel som otvifvelaktigt ännu äro att upptäcka hafva varit i verksamhet år efter år i hundratals sekler, skulle det vara besynnerligt, om icke många växter på detta sätt blifvit förda vida omkring. Dessa transportmedel hafva stundom blifvit kallade tillfälliga, men detta är icke strängt riktigt: hafsströmmarna äro icke tillfälliga, ej heller riktningen af de förherskande vindarna. Det torde observeras, att knappt något transportmedel kan föra fröen till mycket stora afstånd, ty fröen bibehålla icke länge sin vitalitet om de en längre tid utsättas för hafsvattnets inverkan, ej heller kunna de länge stanna i fåglars kräfva eller tarmkanal. Dessa medel torde likväl räcka till för transport öfver hafssträckor af några hundra mils bredd eller från ö till ö, eller från en kontinent till en närliggande ö, men icke från en kontinent till en annan på långt afstånd. Flororna i skilda kontinenter blifva icke blandade genom sådana medel, utan förblifva lika skilda som de nu äro. På grund af strömmarnas riktning skola de icke kunna föra frön från Nordamerika till Britannien ehuru de kunna föra dem från Vestindien till våra vestra kuster, der de icke kunde uthärda vårt klimat äfven om de icke dödades af sin långa nedsänkning i hafsvattnet. Nästan hvarje år drifvas en eller två landfåglar tvärt öfver hela Atlantiska oceanen från Nordamerika till vestra kusten af England och Irland, men frön kunna dessa främlingar medföra blott på ett sätt, nämligen genom smuts som sutte fast vid deras fötter eller näbbar, hvilket i sig sjelft är en sällsynt händelse. Äfven i detta fall, huru ringa är icke utsigten för ett frö att falla i lämplig jordmån och komma till utveckling! Men det skulle vara ett stort misstag att påstå, att emedan en väl befolkad ö som Storbritannien icke har så vidt vi känna (och det skulle vara svårt att bevisa det) inom de senaste få århundradena genom tillfälliga transportmedel fått några inflyttningar från Europa eller någon annan kontinent, att derföre icke en fattigt befolkad ö, ehuru vida mera aflägsen från något fastland, skulle kunna erhålla några kolonister genom likartade medel. Af hundra frön eller djur som transporterades till en ö, äfven om denna vore mindre befolkad än Britannien, torde måhända icke mer än en vara så väl lämpad för sitt nya hem, att den kan naturaliseras. Men detta är såsom mig tyckes icke något kraftigt argument emot hvad som kunde åstadkommas genom tillfälliga transportmedel under de långa geologiska periodernas förlopp under öarnas höjning innan de blifvit fullt befolkade. På ett nästan kalt land der få eller inga förstörande insekter eller fåglar lefde skulle hvarje frö som händelsevis kom dit, om det vore lämpligt för klimatet, gro och lefva.



\section{Spridning under istiden.}

Identiteten emellan många växter och djur på bergspetsar skilda från hvarandra genom hundratals mil af lågland, der alpina arter omöjligen kunna lefva, är ett af de mest öfverraskande kända exempel på att samma art lefver på skilda punkter utan någon märkbar möjlighet för den att hafva kunnat flytta från en punkt till en annan. Det är i sjelfva verket ett märkvärdigt faktum att se så många växter af samma art lefva på de snöiga regionerna af Alperna eller Pyreneerna och i nordligaste trakterna af Europa, men det är vida mera anmärkningsvärdt att växterna på Hvita bergen i Förenta Staterna i Amerika äro desamma som på Labrador och nästan desamma (enligt Asa Gray) som på de högsta bergen i Europa. Redan för så länge sedan som 1747 ledde sådana fakta Gmelin till den slutsatsen, att samma art måste blifvit särskildt skapad på flera skilda punkter och vi skulle hafva bibehållit samma tro, om icke Agassiz ock andra hade fästat liflig uppmärksamhet på istiden, hvilken såsom vi genast skola se erbjuder en enkel förklaring på dessa förhållanden. Vi hafva tydliga bevis nästan af hvarje tänkbart slag, organiskt och oorganiskt, att inom en mycket ung geologisk period det mellersta Europa och Nordamerika kade ett arktiskt klimat. Ruinerna af ett nedbrändt hus bära icke mera vittne om sitt öde, än bergen i Skotland och Wales med deras urhålkade sidor, glatta ytor och spridda block vittna om de isfloder med hvilka dalarna förr voro uppfylda. Så mycket har klimatet i Europa vexlat, att i norra Italien gigantiska moräner, qvarlemnade af gamla glacierer, numera betäckas med vin- och maisplanteringar. Utöfver en stor sträcka af Förenta Staterna vittna erratiska block och urhålkade klippor nogsamt om en förfluten isperiod.

Istidens fordna inflytande på fördelningen af Europas invånare är enligt Edvard Forbes utveckling hufvudsakligen dock följande. Men vi skola lättare följa förändringarna om vi antaga en ny isperiod långsamt inträda och sedan försvinna liksom den förra. Då kölden kom och då hvarje mera sydlig zon blef lämplig för de nordligare invånarna, skulle dessa intaga de platser som innehades af de gamla invånarna i de tempererade zonerna. De senare måste i samma mån gå allt längre och längre söderut, så vida de icke hindras af några stängsel, då de duka under. Bergen skulle blifva betäckta med snö och is och de fordna alpina inbyggarna skulle stiga ned på slätterna. Då kölden hade nått sin höjd skulle vi hafva en arktisk fauna och flora som betäckte de centrala delarna af Europa så långt söderut som till Alperna och Pyreneerna och äfven sträckande sig in i Spanien. De nu tempererade regionerna i Förenta Staterna skulle likaledes betäckas med arktiska växter och djur och dessa skulle vara nästan desamma som i Europa; ty de nuvarande invånarna omkring polen, hvilka vi antaga öfverallt hafva flyttat söderut äro märkvärdigt likformiga rundt omkring jorden.

Då värmen återkom, återtågade de arktiska formerna norrut, tätt följda i sitt återtåg af de mera tempererade regionernas alster. Och då snön smälte från bergens fot skulle de arktiska formerna taga i besittning den blottade och uppblötta marken, alltjemt stigande uppåt allteftersom värmen tilltog och snön vidare försvann, högre och högre, under det deras bröder fortsatte sin resa norrut. Derefter, då värmen helt och hållet återkommit, skulle samma art som nyligen lefvat tillsammans i massa på Europas och Nordamerikas lågland åter finnas i den gamla och nya verldens arktiska regioner och på många isolerade bergspetsar långt från hvarandra.

Vi kunna på detta sätt förstå identiteten emellan många växter på punkter så ofantligt aflägsna som bergen i Förenta Staterna och Europa. Vi kunna på detta sätt förstå det faktum, att alpina växter i hvarje bergsträcka äro särskildt beslägtade med de arktiska former som bo rakt norr eller nästan rakt norr om dem, ty den första flyttningen då kölden kom och återflyttningen då värmen återvände måste i allmänhet hafva gått rakt från norr till söder. De alpina växterna i Skotland till exempel äro enligt H. C. Watson, och fjellväxterna på Pyreneerna enligt Ramond synnerligen beslägtade med växterna i norra Skandinavien; Förenta Staternas fjellväxter med Labradors; växterna på bergen i Sibirien med arterna i de arktiska delarna af samma land. Dessa åsigter grundade på den fullkomligt bevisade tillvaron af en istid synas mig på ett så tillfredsställande sätt förklara den nuvarande fördelningen af fjellens alster och de arktiska produkterna i Europa och Nordamerika, att om i andra trakter vi finna samma arter på aflägsna bergspetsar, vi kunna nästan utan annat bevis draga den slutsatsen, att ett kallare klimat förr tillät deras flyttning tvärt öfver mellanliggande lågland, som nu blifvit för varmt för dem.

Då de arktiska formerna först flyttade söderut och sedermera tillbaka norrut, allteftersom klimatet förändrades, böra de under sin långa flyttning icke hafva varit utsatta för någon synnerlig temperaturvexling och då de alla flyttade en masse, böra deras ömsesidiga relationer icke hafva blifvit synnerligen störda. I öfverensstämmelse med de grundsatser som denna bok innehåller böra dessa former icke hafva varit underkastade mycken modifikation. Men de alpina formerna som lemnats isolerade först vid bergens fot och sedan på deras spetsar befinna sig i något olika förhållanden; ty det är icke sannolikt att samma arktiska arter hafva lemnats på bergspetsar som stå mycket långt från hvarandra och hafva lefvat qvar sedan dess; de skola också efter all sannolikhet hafva blandats med gamla fjellarter, som måste hafva existerat på bergen före början af istiden och hvilka under dess kallaste period hafva för tillfället drifvits ned på slätterna; de måste också hafva varit utsatta för något olika klimatiska förhållanden. Deras ömsesidiga relationer måste derföre hafva blifvit i någon mån störda, och de hafva derföre blifvit benägna för modifikation; och detta finna vi hafva varit förhållandet, ty om vi jemföra de närvarande alpina växterna och djuren på Europas bergskedjor med hvarandra, så existera, ehuru många arter äro identiskt desamma, några såsom varieteter, andra såsom tvifvelaktiga arter eller underarter, andra såsom bestämdt skilda, men nära beslägtade arter, representerande hvarandra på de olika bergskedjorna.

I föregående framställning har jag antagit att vid början af vår tänkta isperiod de arktiska produktionerna voro så likformiga omkring polen som de nu äro. Men det är nödvändigt att inbegripa äfven många subarktiska och några få tempererade former, ty några af dessa äro desamma på de lägre bergsluttningarna och på slätterna af Nordamerika och Europa, och man kan fråga, huru jag förklarar denna grad af likformighet i de subarktiska och tempererade alstren af gamla och nya verlden vid början af den verkliga istiden. I närvarande tid skiljas de subarktiska och norra tempererade alstren af gamla och nya verlden genom hela Atlantiska oceanen och norra delen af Stilla hafvet. Under istiden då invånarna i gamla och nya verlden bodde längre söderut än de nu göra måste de hafva varit ännu mer skilda från hvarandra genom vida hafsytor, så att man mycket väl kan fråga huru samma art kunnat komma in i de båda kontinenterna som då voro så långt skilda åt. Förklaringen ligger, som jag tror, i klimatets beskaffenhet före begynnelsen af isperioden. Vid denna period, den yngre pliocena, voro flertalet af jordens invånare specifikt desamma som nu och vi hafva goda skäl att tro, att klimatet var varmare än nu. Deraf kunna vi antaga att de organismer som nu lefva under 60${}^\circ$ latitud under den pliocena perioden lefde ännu längre norrut under polcirkeln på 66${}^\circ$—67${}^\circ$ latitud, och att de nuvarande arktiska produktionerna då lefde i länderna ännu närmare polen. Om vi nu betrakta jordgloben, se vi att under polcirkeln landet är nästan sammanhängande från vestra Europa genom Sibirien till östra Amerika. Och detta sammanhängande polarland, med deraf följande fri flyttning under ett gynsammare klimat, bör förklara den antagna likformigheten i de subarktiska och tempererade alstren i gamla och nya verlden på en tidpunkt som föregick istiden.

Då jag af ofvan anförda skäl antager, att våra kontinenter länge förblifvit i nästan samma relativa läge, ehuru underkastade stora, partiella höjdförändringar, är jag mycket benägen att utsträcka ofvan anförda åsigt och antaga att under någon tidigare och varmare period, den äldre pliocena, ett stort antal af samma växter och djur bebodde det nästan sammanhängande polarlandet; och att dessa växter och djur både i gamla och nya verlden började långsamt flytta söderut då klimatet blef mindre varmt långt innan istidens början. Vi se nu såsom jag tror deras efterkommande mest i ett modifieradt tillstånd i de centrala delarna af Europa och Förenta Staterna. Enligt denna åsigt kunna vi förstå slägtskapen och den ändock ringa identiteten emellan alstren af Europa och Nordamerika, en slägtskap som är i hög grad anmärkningsvärd i betraktande af de två ytornas afstånd och deras särskiljande genom Atlantiska Oceanen. Vi kunna vidare fatta det sällsamma förhållande som flera iakttagare anmärkt, att Europas och Amerikas alster under den senare tertiära perioden voro mera beslägtade med hvarandra än de äro i närvarande tid; ty under dessa varmare perioder måste de norra delarna af gamla och nya verlden hafva varit nästan oafbrutet förenade af land, som tjenade till brygga för invånarnas fria flyttning, hvilken brygga sedan genom kölden blef ofarbar.

Under den långsamt aftagande värmen under den pliocena perioden så snart arterna som bebodde den nya och gamla verlden gemensamt flyttade söder om polcirkeln, måste de fullkomligt afskäras från hvarandra. Denna separation måste hafva försiggått för lång tid tillbaka åtminstone för de mera tempererade områdenas alster. Då växter och djur flyttade söderut måste de blandas i den ena stora regionen med Amerikas inhemska produkter och måste täfla med dem, och i den andra med den gamla verldens. Följaktligen hafva vi här allting gynsamt för mycken modifikation, — för vida större modifikation än bland de alpina produkterna, som inom en mycket yngre period lemnats isolerade på bergstrakterna och de arktiska länderna Europa och Nordamerika. Deraf har det kommit, att om vi jemföra de nu lefvande produktionerna i den nya och gamla verldens tempererade trakter, vi finna mycket få identiska arter (ehuru Asa Gray nyligen visat att flera växter äro identiska än man förr antog), men vi finna i hvarje stor klass många former, som några naturforskare upptaga såsom geografiska raser och andra såsom skilda arter och en massa af beslägtade eller representerande former, hvilka af alla naturforskare upptagas såsom specifikt skilda.

Liksom på land, så kan äfven i hafsvattnen en långsam flyttning söderut af en hafsfauna, hvilken under den pliocena eller en något tidigare period var nästan likformig längs de sammanhängande kusterna af polcirkeln, enligt teorien om härstamning med modifikation förklara de många beslägtade former som nu bo i fullständigt skilda hafsytor. På detta sätt tror jag vi kunna förstå närvaron af några ännu lefvande och några beslägtade former på östra och vestra stränderna af Nordamerika; och den ännu mera öfverraskande tillvaron af många beslägtade krustaceer (såsom beskrifna i Danas beundransvärda verk), af några fiskar och andra hafsdjur i Medelhafvet och i Japans haf — då dessa två haf nu äro fullständigt skilda genom bredden af en hel kontinent och en vidsträckt ocean.

Dessa fall af nära slägtskap emellan arter som antingen nu bebo eller förr bebott hafven på vestra och östra kusterna af Nordamerika, Medelhafvet och Japanska hafvet och de tempererade länderna i Nordamerika och Europa äro oförklarliga enligt skapelseteorien. Vi kunna icke fasthålla, att sådana arter blifvit skapade lika i öfverensstämmelse med de nästan lika fysiska förhållandena på de olika områdena; ty om vi till exempel jemföra vissa delar af Sydamerika med delar af södra Afrika eller Australien, se vi länder nästan lika i alla sina fysiska förhållanden men med ytterligt olika invånare.



\section{Vexlande istider i norr och söder.}

Men vi måste återvända till vårt egentliga ämne. Jag är öfvertygad att Forbes åsigt kan få ännu större utsträckning. I Europa påträfta vi de tydligaste bevis på en isperiod från vestra kusterna af Britannien till Uralbergen i öster och Pyreneerna i söder. Från de infrusna däggdjuren och beskaffenheten af bergvegetationen kunna vi sluta till att Sibirien likaledes berördes deraf. På Libanon betäckte enligt Hooker evig snö fordom den centrala axeln och alstrade glacierer som rullade 4000 fot ned i dalgångarna. Längs Himalaya på 900 mil aflägsna punkter hafva glacierer lemnat märken efter sitt fordna nedstigande, och i Sikkim såg Hooker mais växa på gigantiska gamla moräner. Söder om den asiatiska kontinenten på motsatta sidan om eqvatorn veta vi nu af J. Haasts och Hectors utmärkta undersökningar att ofantliga glacierer fordom stego ned ganska lågt på Nya Zeeland; och samma växter som Hooker fann på vidt skilda bergspetsar på denna ö berätta samma historia om en förfluten isperiod. Från några fakta som blifvit mig meddelade af W. B. Clarke tyckes också på bergen i sydöstra hörnet af Australien finnas spår af isens verkningar i en förfluten period.

Vi skola nu vända oss till Amerika; i norra hälften hafva isburna klippstycken funnits på östra sidan af kontinenten så långt ned som 36${}^\circ$—37${}^\circ$ lat. och på kusterna af Stilla hafvet, der klimatet nu är så olika, ned till 46${}^\circ$ latitud. Erratiska block hafva också iakttagits på Klippiga Bergen. På Cordillererna i södra Amerika nästan under eqvatorn sträckte sig glaciererna en gång under deras nuvarande höjd. I Chili undersökte jag en vidsträckt grushög med stora block, sträckande sig midt öfver Portillodalen, hvilken otvifvelaktigt fordom bildade en ofantlig morän, och D. Forbes har meddelat mig, att han i flera delar af Cordillererna från 13${}^\circ$ till 30${}^\circ$ sydlig latitud på ungefär 12,000 fots höjd funnit djupt fårade klippor, liknande dem han så väl kände från Norge, och likaledes stora grusmassor, som inneslöto fårade kiselstycken. Längs hela denna sträcka på Cordillererna existera nu icke några glacierer, icke ens på mycket ansenligare höjder. Längre söderut på båda sidor om kontinenten från 41${}^\circ$ lat. till den sydligaste ändan hafva vi det tydligaste bevis på isens verkningar i oräkneliga block som förflyttats långt från sitt ursprung.

Af alla dessa fakta — utsträckningen af isens verkningar rundtomkring de norra och södra hemisfererna — periodens unga datum i båda hemisfererna — dess varaktighet i båda under en lång tid, hvilket vi kunna sluta af dess resultat — och slutligen glacierernas låga nedstigande längs hela Cordillerernas linie tycktes mig förr att vi icke kunde undgå den slutsatsen att temperaturen på hela jorden samtidigt varit sänkt under istiden. Men Croll har nu i en serie beundransvärda arbeten försökt att visa, att ett isklimat är resultatet af flera fysiska orsaker, som kommit i verksamhet genom en förökning i jordens excentricitet. Alla dessa orsaker verka till samma mål; men den mäktigaste synes vara excentricitetens inverkan på hafsströmmarna. Af Crolls undersökningar följer, att kalla perioder reguliert återkomma efter tio à femtontusen år; men att på mycket längre mellantider på grund af vissa omständigheter kölden är ytterligt stark, och räcker under längre tider. Croll tror att den sista stora isperioden inträffade för ungefär 240,000 år sedan och räckte med obetydliga klimatförändringar i 160,000 år. Angående äldre istider hysa flera geologer med stöd af direkta bevis den öfvertygelsen att sådana inträffat under Miocen- och Eocenperioden för att icke tala om äldre formationer. Men för vårt närvarande ämne är det vigtigaste resultat till hvilket Croll kommit, att då den norra hemisferen passerar en kall period, är temperaturen i den södra hemisferen deremot höjd med mycket mildare vintrar, hufvudsakligen beroende på förändringar i hafsströmmarna. Tvärtom är förhållandet med den norra hemisferen då den södra har sin isperiod. Dessa slutsatser äro som vi skola se mycket vigtiga för den geografiska fördelningen; men jag vill först anföra fakta som tarfva en förklaring.

I Sydamerika kar Hooker visat att utom många beslägtade arter emellan fyrtio och femtio af de blommande växterna på Eldslandet, som utgöra en icke obetydlig del af dess fattiga flora, äfven tillhöra Europa och Nordamerika, så ofantligt aflägsna dessa äro från hvarandra i motsatta hemisferen. På de höga bergen i det mellersta Amerika påträffas en mängd särskilda arter tillhörande europeiska slägten. På Organbergen i Brasilien funnos af Gardener några få europeiska, några antarktiska och några andiska slägten, hvilka icke finnas i de låga mellanliggande heta länderna. På Silla de Caraccas fann den ryktbare Humboldt för längesedan arter af slägten som äro karakteristiska för Cordillererna.

I Afrika påträffas på Abyssiniens berg flera för Europa karakteristiska former och några få representanter af floran på Goda Hoppsudden. På sistnämda ställe har man funnit några få europeiska arter som tros icke vara införda af menniskan och på bergen flera representerande europeiska former som icke finnas i de emellan vändkretsarna liggande delarna af Afrika. Hooker har också nyligen visat, att flera af de växter som bo på de öfre delarna af den höga ön Fernando Po och på de närliggande Cameroonbergen i Guineabugten äro nära beslägtade med dem som finnas på bergen i Abyssinien och likaledes med dem i det tempererade Europa. Det tyckes nu också såsom jag hör af Hooker, att några af dessa tempererade växter hafva af Rev. R. T. Lowe upptäckts på bergen å Cap Verdöarna. Denna utsträckning af samma tempererade former nästan under eqvatorn tvärt öfver hela Afrikas kontinent och till bergen i Cap Verd-arkipelagen är ett af de mest märkvärdiga förhållanden i växternas fördelning.

På Himalaya och på de isolerade bergsträckorna på indiska halfön, på Ceylons höjder, och på Javas vulkaniska käglor påträffas många växter antingen identiskt desamma eller representerande hvarandra och på samma gång representerande växter i Europa som icke blifvit funna på de mellanliggande heta låglanden. En förteckning öfver slägten funna på Javas höga spetsar liknar en samling på en höjd i Europa. Ännu mera öfverraskande är det förhållande att särskilda sydliga australiska former representeras af vissa växter som växa på spetsarna af Borneos berg. Några af dessa australiska former sträcka sig såsom jag hör af Hooker längs höjderna på Malakkahalfön och äro sparsamt spridda å ena sidan öfver Indien och å den andra så långt norrut som till Japan.

På de södra bergen i Australien har F. Müller upptäckt flera europeiska arter, andra, som icke införts af menniskor, finnas på lågländerna, och enligt Hookers uppgift kan en lång lista upprättas på europeiska slägten som blifvit funna i Australien men icke i de mellanliggande heta zonerna. I sitt utmärkta arbete ”Introduction to the flora of New Zealand” meddelar Hooker analoga anmärkningsvärda förhållanden rörande växternas fördelning på denna stora ö. Deraf se vi att vissa växter som lefva på de mera höga bergen i tropikerna i alla delar af jorden och på de tempererade slätterna i norr och söder äro antingen samma identiska arter eller varieteter af samma art. Det torde likväl observeras att dessa växter icke äro strängt arktiska former, ty såsom H. C. Watson har anmärkt ”om vi gå från polen till de eqvatoriela latituderna bli alpflororna mindre och mindre arktiska”. Utom dessa beslägtade eller identiska former finnas många arter som bebo samma vidt skilda områden, tillhörande slägten som nu icke finnas i de tropiska mellanliggande låglanden.

Dessa korta antydningar hafva blott tillämpning på växter, men några få analoga fakta kunde gifvas som röra landdjur. Bland hafvets alster inträffa liknande förhållanden; såsom ett exempel vill jag anföra ett yttrande af en stor auktoritet prof. Dana, ”att det helt säkert är ett underbart faktum att nya Zeeland i sina krustaceer mera liknar Storbritannien, dess antipod, än någon annan del af jorden”. J. Richardson talar också om uppträdandet af nordliga fiskformer på kusterna af Nya Zeeland, Tasmanien m. m. Hooker har meddelat mig att tjugufem algarter äro gemensamma för Nya Zeeland och Europa utan att hafva blifvit funna i mellanliggande haf.

Af föregående förhållanden nämligen närvaron af tempererade former på högländerna öfver hela det tropiska Afrika och längs Indiska halföarna till Ceylon och Malayiska arkipelagen och i mindre utpräglad grad tvärt öfver det vidsträckta tropiska Sydamerika, synes det nästan säkert att på någon förfluten period otvifvelaktigt under den strängaste delen af istiden lågländerna i dessa stora kontinenter öfverallt under eqvatorn innehades af ett ansenligt antal tempererade former. På denna period var eqvatorialklimatet på hafsytan sannolikt ungefär detsamma som nu på en höjd af från fem till sex tusen fot under samma latitud eller kanske snarare kallare. Under den kallaste perioden måste lågländerna under eqvatorn hafva varit beklädda med en blandad tropisk och tempererad vegetation, lik den Hooker beskrifvit såsom växande yppigt på höjden af fyra till fem tusen fot på de lägre sluttningarna af Himalaya, men måhända med ännu större öfvervigt af tempererade former. Vidare på den bergiga ön Fernando Po i Guineabugten fann Mann tempererade europeiska former som började visa sig vid höjden af omkring femtusen fot. På bergen i Panama på en höjd af blott tvåtusen fot, fann Seeman vegetationen lik Mexikos ”med former från de heta zonerna harmoniskt blandade med tempererade”.

Låt oss nu se, om Crolls slutsatser att den norra hemisferen lydde under den stora isperiodens ytterliga köld, under det den södra var varmare, sprider något ljus öfver den närvarande som det tyckes oförklarliga fördelningen af åtskilliga organismer i de tempererade delarna af båda hemisfererna och på tropikernas berg. Beräknad efter år måste istiden hafva varit mycket lång, och om vi komma ihåg öfver hvilka vidsträckta ytor några naturaliserade växter och djur hafva spridt sig inom några få århundraden, måste denna period hafva varit tillräcklig för hvarje grad af flyttning. Då kölden blef mer och mer häftig, veta vi att arktiska former bröto in i de tempererade regionerna, och från de nyss anförda fakta kan näppeligen något tvifvel återstå att några af de kraftigare, de dominerande och mest spridda tempererade formerna verkligen bröto in i de eqvatoriala lågländerna. Inbyggarna i dessa lågland skulle på samma gång flytta in i de tropiska och subtropiska regionerna i söder, ty den södra hemisferen var under denna period varmare. Vid istidens aftagande, då båda hemisfererna småningom återtogo sina fordna temperaturer, måste de norra tempererade formerna som bebo lågländerna under eqvatorn drifvas till sina förra hem eller duka under och ersättas af de eqvatoriala formerna som återvände från södern. Några af de norra tempererade formerna måste nästan med säkerhet bestiga något närbeläget högland, hvarest de, om det vore tillräckligt högt, länge skulle förblifva vid lif, liksom de arktiska formerna på Europas berg. De kunna ega bestånd äfven om klimatet icke varit fullkomligt lämpadt för dem, ty temperaturförändringen måste hafva varit mycket långsam och växter ega otvifvelaktigt en viss förmåga af acklimatisering, såsom bevisas deraf att de lemna sina afkomlingar olika konstitutionel förmåga att motstå hetta och köld.

Under händelsernas reguliera förlopp måste den södra hemisferen derefter inträda i en isperiod under det de norra hemisferernas klimat blef varmare och de södra tempererade formerna skulle då i sin ordning invandra på de eqvatoriala lågländerna. De norra formerna som förut lemnats qvar på bergen stiga nu ned och blanda sig med de södra formerna. Då värmen återvände, skulle dessa senare återgå till sina förra hem, lemnande några få arter på bergen och förande med sig söderut några af de förra tempererade formerna som nedstigit från sina bergiga hemland. Vi få således några arter identiskt desamma i de norra och södra tempererade zonerna och på bergen i de mellanliggande tropiska regionerna. Men de arter som under en lång tid lemnats qvar på dessa berg eller i motsatta hemisferer måste råka i strid med många nya former och skulle komma under något olika fysiska förhållanden, derföre skulle de vara särdeles benägna för modifikation och böra i allmänhet nu existera såsom varieteter eller representerande arter, hvilket också är förhållandet. Vi måste också fasthålla i minnet de förflutna isperioderna i båda hemisfererna, ty dessa skola förklara enligt samma grundsatser att många fullständigt skilda arter bebo samma vidt skilda områden och tillhöra slägten som nu icke finnas i de mellanliggande heta zonerna.

Det är ett anmärkningsvärdt faktum som Hooker strängt framhåller angående Amerika och de Candolle för Australien, att många fler identiska eller nu obetydligt modifierade arter hafva flyttat från norr till söder än i motsatt riktning. Vi se likväl några få sydliga former på bergen på Borneo och i Abyssinien. Jag antager, att denna öfvervägande flyttning från norr till söder beror på landets större utsträckning i norr och derpå att de nordliga formerna i sina hemland hafva funnits i större antal och följaktligen genom naturligt urval och konkurrens uppnått en högre grad af fulländning eller dominerande kraft än de södra formerna. Och således om de två serierna blandas i de eqvatoriala trakterna under de vexlande isperioderna, skulle de norra formerna vara de kraftigare och kunna behålla sina platser på bergen och sedermera flytta söderut med de sydliga formerna; de södra formerna äro deremot alltid underlägsna de norra. På samma sätt se vi för det närvarande, att ganska många europeiska alster bekläda marken i La Plata, Nya Zeeland och till mindre grad Australien, efter att hafva besegrat infödingarna; hvaremot ytterligt få sydliga former hafva blifvit naturaliserade i någon del af den norra hemisferen, ehuru hudar och ull och andra ämnen som kunnat medföra frön hafva i betydlig mängd införts till Europa från La Plata och under de senare trettio eller fyrtio åren från Australien. Neilgherrie-bergen i Indien göra likväl ett särskildt undantag; ty såsom jag hör af Hooker finnas der flera sjelfsådda australiska former som naturaliserats. Före den sista stora istiden voro otvifvelaktigt de tropiska bergen befolkade med infödda alpina former, men dessa hafva nästan öfverallt vikit undan för de mera dominerande formerna, som bildats på de större ytorna i nordens större verkstäder. På många öar hafva de naturaliserade produkterna i antal nästan uppnått eller till och med öfvergått de inhemska och detta är det första steget till deras utrotning. Berg äro likasom öar på land, och deras invånare hafva vikit undan och vika fortfarande undan för de genom menniskans medverkan naturaliserade kontinentala formerna.

Samma grundsatser låta tillämpa sig på fördelningen af landdjur och hafsproduktioner i de norra och södra tempererade zonerna och på de tropiska bergen. Om under istiden hafsströmmarna voro helt olika mot hvad de nu äro, kunna några af de tempererade hafvens invånare hafva nått eqvatorn, af dessa skulle några få möjligen med ens vara i stånd att flytta söderut genom att hålla sig i de kallare hafsströmmarna, under det andra hafva stannat och lefvat qvar i de kallare djupen, till dess den södra hemisferen i sin ordning utsattes för ett isklimat som tillät deras vidare flyttning; nästan på samma sätt som enligt Forbes isolerade områden bebodda af arktiska alster ännu i dag existera i de djupare delarna af de tempererade hafven.

Jag tror ingalunda att alla svårigheter i fördelningen och slägtskapen emellan de arter som nu lefva så vidt skilda i norr och söder och stundom på mellanliggande bergskedjor med ofvanstående åsigter äro undanröjda. Flyttningslinierna kunna icke med noggranhet utstakas. Vi kunna icke säga hvarföre dessa arter hafva flyttat och icke andra; hvarföre vissa arter hafva modifierats och lemnat ursprung till andra former under det andra förblifvit oförändrade. Vi kunna icke hoppas att förklara sådana fakta förr än vi kunna säga hvarföre en art och icke en annan blifvit naturaliserad genom menniskans åtgöranden i främmande land; hvarföre en art har två eller tre gånger så stor spridning och är två eller tre gånger så allmän som en annan art inom deras egna områden.

Flera speciela svårigheter återstå således att lösas; till exempel tillvaron enligt Hooker af samma växter på punkter så enormt aflägsna som Kerguelen-Land, Nya Zeeland, och Eldslandet; men såsom Lyell förmodar kunna isberg hafva medverkat till deras spridning. Förekomsten af arter, hvilka ehuru skilda tillhöra uteslutande sydliga slägten, på sådana aflägsna punkter af södra hemisferen är ett mera anmärkningsvärdt förhållande. Några af dessa arter äro så väl skilda att vi icke kunna antaga att sedan sista isperiodens början tiden har varit tillräcklig för deras flyttning och följande modifikation i nödvändig grad. Dessa fakta synas mig angifva, att skilda arter af samma slägten hafva flyttat i radierande linier från en gemensam medelpunkt, och jag är böjd att söka så väl i den södra som i den norra hemisferen en tidigare och varmare period som föregick istidens början, då de antarktiska länderna som nu äro betäckta med is buro en i hög grad egendomlig och isolerad flora. Man kan antaga, att förrän denna flora under den sista istiden dog ut, några få former redan blifvit vida kringspridda till olika delar af den södra hemisferen genom tillfälliga transportmedel och med biträde af nu sjunkna öar såsom hvilopunkter. De södra stränderna af Amerika, Australien och Nya Zeeland hafva på detta sätt kunnat få en lika färgning af samma egendomliga lifsformer.

Sir C. Lyell har i ett med mitt resonnemang nästan identiskt uttryckssätt framstält förmodanden öfver verkningarna af stora klimatvexlingar öfver hela jorden på den geografiska fördelningen af de lefvande varelserna. Och vi hafva nu sett att Crolls slutsats att successiva isperioder i den ena hemisferen sammanföllo med varmare perioder i den motsatta hemisferen, tillsammans med antagandet af en ringa modifikation af arterna, förklarar en mängd fakta i fördelningen af samma och beslägtade lifsformer i alla delar af jordklotet. De lefvande vattnen hafva under vissa perioder flutit från norr och sedan från söder och hafva i båda fallen nått eqvatorn, men lifsströmmen har gått med större kraft från norr än i motsatt riktning och har följaktligen friare öfversvämmat södern. Likasom tidvattnet lemnar sina spår i horisontala linier som sträcka sig längre upp på stränder der floden är högre, så hafva de lefvande vattnen lemnat sin lefvande drift på våra bergspetsar i en linie som sakta stiger från de arktiska lågländerna till en stor höjd under eqvatorn. De olika varelser som på detta sätt lemnats strandade kunna jemföras med vilda menniskoraser uppdrifna och qvarlefvande i bergfästen i nästan hvarje land, hvilka utgöra en för oss intressant qvarlefva af de kringliggande lågländernas fordna inbyggare.
