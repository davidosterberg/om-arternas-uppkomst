\chapter*{Historisk öfverblick.}




Hos flertalet naturforskare gäller den åsigten, att arterna äro oföränderliga alster, hvar och en skapad för sig, och denna åsigt har af många författare blifvit försvarad med mycken talang. Blott några få deremot antaga, att arter äro i stånd att undergå förändringar, och att de nu lefvande formerna genom verklig fortplantning framgått ur andra tidigare former. Om vi icke fästa oss vid några antydningar hos den klassiska forntidens författare (Aristoteles) som kunna hafva afseende härpå, var Buffon den förste i nyare tider, som på vetenskapligt sätt behandlat denna fråga. Lamarck var den förste, hvilkens åsigter i detta ämne väckte stor uppmärksamhet. Denna med rätta prisade vetenskapsman offentliggjorde sina åsigter först 1801 och derefter betydligt utvidgade 1809 i sin Zoologie philosophique, samt 1815 i sin inledning till de ryggradslösa djurens naturalhistoria; i alla dessa arbeten uppstälde han den läran, att alla arter, menniskan inbegripen, härstamma ifrån andra arter. Han har den stora förtjensten att först hafva riktat uppmärksamheten på sannolikheten, att alla förändringar i den organiska så väl som den oorganiska verlden äro följder af naturlagar och icke af märkvärdiga tillfälligheter. Lamarck synes hafva kommit till antagandet om arternas gradvisa förändring hufvudsakligen genom svårigheten att skilja arter och varieteter från hvarandra, genom den nästan oafbrutet löpande serien af former i många grupper af organismer och genom analogien med alstren af vår domesticering. Beträffande medlen hvarigenom arternas förvandling försiggick, tillskref han något de yttre lefnadsförhållandena, en del kroasering af de redan bestående formerna, och det mesta härledde han från organernas bruk och bristande användning, alltså vanans verkan. Denna sista kraft synes han tillskrifva alla de sköna ändamålsenligheterna i naturen, såsom till exempel girafens långa hals, som sätter honom i stånd att afbeta grenarna på höga träd. Han antog tillika en lag om fortgående utveckling och då följaktligen alla lifsformer sökte att gå framåt, antog han för att förklara tillvaron af de enklaste naturprodukter äfven i våra dagar en generatio spontanea för sådana former.

Etienne Geoffroy S:t Hilaire antog redan 1795, såsom hans son Isidore berättar i hans lefnadsbeskrifning, att våra så kallade arter blott äro afarter af en och samma typ. Först år 1828 publicerade han sin öfvertygelse att samma former icke hafva bibehållit sig oförändrade sedan tidernas början. Geoffroy synes hafva sökt orsaken till förändringarna hufvudsakligen i lefnadsförhållandena eller »le monde ambiant», men han var försigtig i att draga slutsatser och trodde icke att nu bestående arter äro underkastade några förändringar; hans son säger: C’est donc un problème à réserver entierement à l’avenir, supposé que l’avenir doive avoir prise sur lui.»

År 1813 föredrog doktor W. C. Wells inför Royal Society en »berättelse om en fru af hvit ras, hvars hud till en del liknar en negers»; men uppsatsen offentliggjordes icke förr än hans båda berömda Essays »öfver daggen» och »det enkla seendet» hade utkommit år 1818. I denna uppsats har han tydligen uppfattat principen om det naturliga urvalet och detta är den första påvisade framställning deraf. Men han använde den blott på menniskoraserna och blott på vissa karakterer. Efter ett yttrande, att neger och mulatt ega immunitet emot vissa tropiska sjukdomar, anmärker han för det första, att alla djur till en viss grad äro benägna för variation, och för det andra, att landthushållare förädla sina husdjur genom urval. Han tillägger vidare: hvad som i sista fallet »sker genom konst, synes ske af naturen med samma verksamhet, ehuru långsammare, vid bildandet af menniskoslägtets varieteter, som äro bygda »på ett sätt, som är lämpligt för de trakter de bebo. Bland de tillfälliga menniskovarieteter, som uppträda bland de få och spridda invånarna i Afrikas mellersta trakter, skola några bättre än andra vara i stånd att uthärda landets sjukdomar. I följd deraf skola raserna förökas, under det andra aftaga, både derföre att de icke kunna uthärda sjukdomarna och emedan de ej äro i stånd att konkurrera med sina kraftfullare grannar. Efter hvad som redan blifvit sagdt antager jag för afgjordt, att färgen hos denna kraftfullare ras måste vara mörk. Men då benägenheten för variation ännu qvarstår, så bilda sig under tidernas lopp allt mörkare och mörkare raser och då den mörkaste bäst passar för klimatet, så blir denna i det land der den uppstått, om icke den enda, åtminstone den förherskande rasen». Han utsträcker samma betraktelser vidare på de hvita invånarna i de kallare klimaten. Jag är mr Bruce i Förenta Staterna högeligen förbunden för hänvisningen på det anförda stället i dr Wells uppsats.

I fjerde bandet af Horticultural Transactions 1822, och i sitt arbete öfver Amaryllidaceæ (1837, s. 19, 339) förklarade W. Herbert, sedermera decanus i Manchester, att det vore »genom horticulturförsök fullkomligt afgjordt, att växtarter blott äro en högre och beständigare grad af varieteter.» Han utsträcker samma åsigt äfven till djurriket och tror att af hvarje slägte ursprungligen skapats enstaka arter med en högre grad af bildbarhet, och att dessa sedermera hufvudsakligen genom kroasering, men äfven genom variation hafva alstrat alla våra nuvarande arter.

År 1826 uttalade professor Grant i slutparagrafen till sin bekanta afhandling om Spongilla (Einburgh Philos. Journ. XIV, p. 283) sin mening helt och hållet vara, att arter uppstått ur andra arter och förädlas genom fortfarande förändringar. Samma åsigt har han äfven upprepat år 1834 i Lancet i sin femtiofemte föreläsning.

Vidare utvecklade Patrick Matthew i sin bok: Naval Timber and arboriculture sin öfvertygelse öfver arternas uppkomst helt och hållet i öfverensstämmelse med den af Wallace och mig i Linnean Journal gifna och i detta verk utarbetade framställning. Olyckligtvis skref dock Matthew sina åsigter mycket kort i spridda drag i ett bihang till ett arbete öfver ett helt annat ämne, så att de blefvo alldeles okända, till dess Matthew sjelf fäste uppmärksamheten derpå i Gardeners Chronicle den 7 April 1860. Afvikelserna i hans teori ifrån min äro icke af stor betydenhet: han tyckes anse, att jorden vid vissa perioder varit nästan alldeles obebodd och sedan blifvit befolkad, och han framställer såsom ett alternativ, att nya former kunna frambringas »utan närvaron af någon modell eller frö af tidigare aggregater.» Jag är icke fullt säker om jag riktigt förstår några ställen, men han tyckes tillskrifva lefnadsvilkorens direkta inverkan ett stort inflytande. Deremot uppfattade han fullt klart den fulla kraften af det naturliga urvalets princip.

Den utmärkte geologen von Buch uttalar i sin utmärkta Description Physique des Isles Canaries (1836) tydligt sin åsigt, att varieteter långsamt öfvergå till permanenta arter, hvilka icke längre kunna kroaseras.

Rafinesque skrifver år 1836 följande i sin New Flora of North America: »Alla arter kunna en gång hafva varit varieteter och många varieteter hafva småningom blifvit arter genom att antaga konstanta och egendomliga karakterer;» men längre fram tillägger han »med undantag af originaltyperna eller slägtenas stamfäder.

År 1843 och 1844 har professor Haldeman (i Boston Journal of Natur. Hist.) sammanstält skälen för och emot hypotesen om arternas utveckling och modifiering; hans åsigt synes öfverensstämma med denna teori.

Vestiges of Creation utkom år 1844. I den tionde förbättrade upplagan (1853) säger den anonyme författaren: »den på mogen öfverlägning grundade åsigten är, att de olika grupperna af lefvande varelser från de enklaste och äldsta upp till de högsta och nyaste äro under Guds försyn resultaterna af, primo en lifsformerna meddelad impuls, som på bestämda tider genom generationer fört dem framåt genom olika grader af organisation ända upp till de högsta dicotyledoner och vertebrater; dessa grader äro få till antal och vanligen skilda från hvarandra genom luckor i de organiska karaktererna, hvilka åstadkomma svårigheter vid bestämmandet af slägtskap — för det andra af en annan impuls som är förenad med de vitala krafterna och under loppet af generationer sträfvar att modifiera organiska bildningar i öfverensstämmelse med yttre omständigheter, såsom föda, boningsort och klimat, detta är naturalteologernas ändamålsenligheter.» Författaren tror tydligen, att organisationen fullkomnas genom plötsliga språng, men att verkningarna af lefnadsvilkoren äro gradvisa. Med mycken kraft bevisar han på allmänna grunder, att arter icke äro några oföränderliga produkter. Men jag kan ej se huru de båda antagna impulserna i vetenskaplig mening kunna förklara de talrika och sköna ändamålsenligheter, som vi se öfverallt i naturen; jag kan ej finna att vi på detta sätt få någon förklaring öfver huru till exempel en hackspett har blifvit så väl skapad för sina särskilda lefnadsvanor. Ehuru arbetet i de tidigare upplagorna röjde föga djupa kunskaper och en stor brist på vetenskaplig försigtighet, så vann det dock en vidsträckt spridning på grund af dess glänsande och hänförande stil. I min tanke har det gjort en stor tjenst derigenom att det i vårt land har riktat uppmärksamheten på detta ämne, undanröjt fördomarna och på detta sätt röjt väg för liknande åsigter.

År 1846 offentliggjorde veteranen ibland geologerna M. J. d’Omalius d’Halloy i en utmärkt ehuru kort afhandling sin åsigt, att nya arter med mera sannolikhet kunna antagas hafva uppkommit genom härstamning med modifiering än genom en särskild skapelseakt: denna åsigt hade författaren uppstält först år 1831.

Professor Owen skref år 1849 (i Nature of Limbs) följande: »Idéen om en grundtyp hade uppenbarat sig i vår planets djurverld under flera sådana modifikationer långt före tillvaron af de djur, som i verkligheten förtydliga den. Hvilka naturliga lagar eller sekundära orsaker hafva åstadkommit den regelbundna successionen och utvecklingen af sådana organiska företeelser, derom äro vi hittills okunniga.» I sin Adress to British Association år 1858 talar han om »den skapande kraftens fortsatta verksamhet eller lefvande väsendens ordnade vardande,» såsom ett axiom. Längre fram, då han uppehåller sig vid den geografiska fördelningen tillägger han: »dessa företeelser rubba vår lit till antagandet, att Apteryx på Nya Zeeland och den bruna hjerpen (Tetrao scoticus) i England äro skilda skapelser, hvar och en för sin ö. Vi böra äfven alltid ihågkomma, att zoologen med ordet skapelse menar en process, som han icke närmare känner.» Han utförer vidare denna åsigt genom att tillägga: »om zoologen upptager sådana fall som hjerpen, såsom bevis på fågelns särskilda skapelse i och för sådana öar, så uttrycker han dermed blott, att han icke vet huru denne kommit dit och blott dit, och genom att på detta sätt uttrycka sin okunnighet tillkännagifver han också sin tro, att både fågeln och ön hafva för sin uppkomst att tacka en första skapande kraft.» Om vi tolka dessa yttranden med hvarandra, tyckes denna framstående forskare år 1858 hafva känt sig rubbad i sin tro, att Apteryx och hjerpen först uppträdde i sina respektiva hem på ett obekant sätt.

Denne adress afgafs sedan Wallace’s och min afhandling om arternas uppkomst, hvartill vi genast komma, blifvit lästa för Linnean Society. Då första upplagan af detta arbete utkom, var jag såväl som många andra missledd af sådana uttryck som »den skapande kraftens fortsatta verksamhet,» att jag inbegrep prof. Owen ibland andra palæontologer, som voro fast öfvertygade om arternas oföränderlighet, men detta tyckes (Anatomy of Vertebrates) vara ett sällsamt misstag å min sida. I sista upplagan antog jag och såsom mig ännu synes med full rätt, att prof. Owen medgaf, att det naturliga urvalet väl kan hafva gjort någonting för bildandet af nya arter, men att lagarna derför ännu äro okända och obevisade. Jag anförde äfven några utdrag af en korrespondens emellan prof. Owen och utgifvaren af London Review, hvaraf det syntes klart så väl för denne som för mig, att prof. Owen gjorde anspråk på att hafva förr än jag offentliggjort teorien om naturligt urval, men så vidt man kan döma af vissa nyligen framstälda yttranden har jag åter delvis eller helt och hållet tagit fel. Det är hugnande för mig att finna, att andra finna det lika svårt som jag att förstå prof. Owens stridiga yttranden och att förena dem med hvarandra. Hvad angår sjelfva framställningen af teorien om naturligt urval är det fullkomligt oväsentligt om prof. Owen föregick mig eller icke, ty vi voro båda enligt det nyss anförda förekomna för lång tid tillbaka af doktor Wells och Matthew.

Isidore Geoffroy S:t Hilaire framstälde i sina föreläsningar år 1850 i korthet sina skäl för den åsigten, att »artkarakterer äro oföränderliga för hvarje art, så länge den förblifver under samma omständigheter: de modifieras, om de omgifvande förhållandena vexla.» »I korthet, iakttagelsen öfver de vilda djuren visar redan arternas begränsade föränderlighet. Försöken med vilda djur som blifvit tämda och med husdjur som förvildats, visa den ännu tydligare. Samma försök visa vidare, att de derigenom åstadkomna skiljaktigheterna kunna få betydelse af slägtkarakterer.» I sin Histoire naturelle générale (1859) utvecklar han analoga åsigter.

Af en nyligen utkommen skrift synes framgå, att, att doktor Freke år 1851 (Dublin Medical press) framstälde den läran, att alla organiska varelser härstamma från en urform. Hans skäl och behandling af ämnet skiljer sig vida från mina, men då doktor Freke nu utgifvit sin Origin of species by means of organic affinity, torde det svåra försöket att gifva någon föreställning om hans åsigter vara öfverflödigt.

Herbert Spencer (i Leader 1852 och sedermera i Spencers Essays) har med utmärkt skicklighet sammanstält teorierna om skapelsen och de organiska varelsernas utveckling. Från analogien med kulturalster från förändringar i många arters foster, från svårigheten att skilja arter och varieteter, från grundsatserna om allmänna öfvergångar sluter han, att arterna förändrats, och tillskrifver dessa förändringar vexlande yttre förhållanden. Författaren har också behandlat psykologien enligt grundsatsen om nödvändigheten af själsförmögenheternas gradvisa förvärfvande.

År 1852 har Naudin i en afhandling om arternas uppkomst (Revue horticole, delvis återgifven i Nouvelles archives du Muséum) uttalat sin åsigt om arternas bildning på analogt sätt med varieteters bildning i kulturtillståndet, och den senare tillskrifver han menniskans valförmåga. Men han visar icke hur urvalet verkar i naturtillståndet. Han tror liksom Herbert, att arter vid deras uppkomst voro mera plastiska än nu och lägger mycken vigt på hvad han kallar finalitet, »en hemlighetsfull, obestämd kraft, för några liktydig med blind förutbestämning, för andra med en förutseende vilja, hvars oupphörliga verkan på de lefvande varelserna bestämmer i alla verldsåldrar formen, omfånget och varaktigheten af hvar och en af dem allt efter dess bestämmelse i den sakernas ordning den tillhör. Det är denna makt som bringar hvarje lem i harmoni med det hela, lämpande den efter den förrättning han bör uppfylla i naturens allmänna organism, en förrättning som är för honom grunden till hans tillvaro».

År 1853 har en berömd geolog grefve Keyserling (i Bulletin de la Société Géologique) framstält den meningen, att likasom nya sjukdomar förorsakade genom någon miasma antagas hafva uppkommit och spridt sig öfver jorden, så hafva vid vissa tider fröen till existerande arter blifvit kemiskt afficierade af omgifvande molekyler af egendomlig beskaffenhet och på detta sätt gifvit upphof till nya former.

Samma år 1853 lemnade doktor Schaaffhausen en utmärkt uppsats i Verhandlungen des Naturhist. Vereins der preuss. Rheinlande, i hvilken han framhåller de organiska formernas progressiva utveckling på jorden. Han antager att många arter under långa tidrymder hållit sig oförändrade, hvaremot några få blifvit modifierade. Arternas begränsning förklarar han genom mellanstående formers undergång. »Lefvande växter och djur skiljas icke från de utdöda genom nya skapelser, utan böra betraktas såsom deras afkomlingar genom oafbruten fortplantning.»

En väl känd fransk botanist Lecoq skrifver år 1854 i sina Etudes sur la géographie botanique: »Man ser att våra forskningar öfver arternas beständighet eller föränderlighet leder rakt till de åsigter som hystes af två med rätta ryktbara män Geoffroy Saint-Hilaire och Goethe.» Några andra spridda ställen i hans arbete göra dock något tvifvelaktigt, huru långt han utsträcker sina åsigter om arternas förändringar.

»Skapelsens filosofi» har blifvit på ett utmärkt sätt behandlad af Baden Powell i hans Essays on the Unity of worlds 1855. Ingenting kan vara mera öfverraskande än det sätt hvarpå han visar att införandet af nya arter är »en regelbunden, icke en tillfällig företelse», eller såsom John Herschel uttrycker sig »en naturlig process i motsats till ett underverk.»

Tredje bandet af Journal of the Linnean Society innehåller afhandlingar inlemnade den 1 Juli 1858 af Wallace och mig, hvaruti teorien om naturligt urval af Wallace framställes med beundransvärd skärpa och klarhet.

Von Baer, för hvilken alla zoologer hysa stor aktning, uttalade omkring år 1859 sin öfvertygelse hufvudsakligen grundad på lagarna om geografisk fördelning, att former som nu äro fullkomligt skilda hafva utgått från en enda stamform (Rud. Wagner, Zoologisch-anthropologische Untersuchungen, 1861).

I Juni 1859 höll prof. Huxley inför Royal Institution ett föredrag om de ständiga typerna i djurriket, och anför deri följande: »Det är svårt att fatta betydelsen af sådana fakta, om vi antaga, att hvarje djur- och växtart, eller hvarje stor organisationstyp bildades och sattes på jordytan med långa mellantider af en särskild akt af den skapande kraften, och man måste icke förglömma, att ett sådant antagande saknar stöd af tradition och uppenbarelse lika väl som den strider emot den allmänna analogien i naturen. Betrakta vi å andra sidan de persistenta typerna i deras förhållande till den åsigten att arter som lefva på någon tid äro resultatet af förut existerande arters gradvisa modifikation — en åsigt som, ehuru obevisad och illa behandlad af några af dess anhängare, är den enda fysiologien lånar något understöd — synes deras tillvaro visa, att graden af modifikation som lefvande varelser hafva undergått under den geologiska tiden är mycket ringa i jemförelse med hela serien af skedda förändringar.

I December 1859 utgaf doktor Hooker sin Introduction to the Australian Flora. I första delen af detta stora arbete medgifver han sanningen af arternas härstamning med modifikation och stöder denna åsigt med många egna iakttagelser.

Första upplagan af detta arbete utkom i November 1859, den andra i Januari 1860, den tredje i April 1861, den fjerde i Juni 1866 och den femte på hösten 1869.


