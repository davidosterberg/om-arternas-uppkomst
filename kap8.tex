%ÅTTONDE KAPITLET.





\chapter{Bastardbildning.}

{\it
Skilnad emellan ofruktsamhet vid första kroaseringen och hos bastarder. — Ofruktsamheten till graden föränderlig, icke allmän, förökad genom kroasering emellan nära slägtingar, förminskad genom domesticering. — Lagar för bastarders ofruktsamhet. — Ofruktsamhet är ingen särskild egenskap utan sammanfaller med andra afvikelser. — Förökas icke genom naturligt urval. — Orsaker till ofruktsamhet vid första kroaseringen och hos bastarder. — Jemförelse emellan verkningarna af förändrade lefnadsförhållanden och af kroasering. — Dimorfism och trimorfism. — Fruktsamhet vid varieteters kroasering och hos mestiser icke allmän. — Bastarder och mestiser jemförda utan afseende på deras fruktsamhet. — Sammanfattning.
}\\[0.5cm]

Den ibland naturforskare allmänt gängse meningen är att arter vid kroasering blifvit begåfvade med ofruktsamhet för att förekomma deras sammanblandning. Denna åsigt tyckes visserligen först vara i hög grad sannolik, ty arter inom samma trakt kunde svårligen hållas skilda, om fri kroasering vore möjlig. Detta ämne är på många sätt af vigt för oss, isynnerhet som arternas ofruktsamhet då de först kroaseras och ofruktsamheten hos deras bastarder icke kan hafva blifvit förvärfvad genom ett fortsatt bibehållande af successiva gynsamma grader af ofruktsamhet. Den sammanfaller, såsom jag hoppas kunna visa, med afvikelser i reproduktionssystemet hos stamarten, och är hvarken en särskild förvärfvad eller förlänad egenskap.

Vid behandlingen af detta ämne hafva två klasser af fakta i allmänhet blifvit sammanblandade, hvilka i grund och botten äro vidt skilda från hvarandra, nämligen ofruktsamheten vid två arters första kroasering och ofruktsamheten hos deras bastarder.

Rena arter hafva naturligtvis sina reproduktionsorganer i ett fullkomligt tillstånd och dock lemna de vid kroasering ringa eller ingen afföda. Bastarder å andra sidan hafva sina reproduktionsorganer i ett tillstånd, som gör dem ur stånd att förrätta sina funktioner, såsom tydligen kan ses hos det manliga elementet både hos växter och djur, ehuru organerna sjelfva äro fullkomliga till sin skapnad såsom de mikroskopiska undersökningarna gifva vid handen. I det första fallet äro de två sexuela elementen som skola bilda embryot båda fullkomliga, i det senare äro de antingen alldeles icke eller ofullständigt utvecklade. Denna skilnad är vigtig, när orsaken till ofruktsamheten, som är gemensam för båda fallen, skall tagas i betraktande. Denna skilnad har sannolikt blifvit förbisedd, då ofruktsamheten i båda fallen betraktats såsom en särskild egendomlighet utöfver området för vår reflexionsförmåga.

Fruktsamheten hos varieteter, det är former som med visshet eller sannolikhet antagas härstamma från gemensamma föräldrar, vid kroasering samt fruktsamheten hos deras afkomlingar, mestiserna, är med afseende på min teori af lika stor vigt som arternas ofruktsamhet, ty den synes göra en stor och tydlig skilnad emellan arter och varieteter.



\section{Grader af ofruktsamhet.}

Vi skola först vända oss till ofruktsamheten vid arters kroasering och hos deras bastarder. Det är omöjligt att studera två samvetsgranna och utmärkta observatörers memoarer och arbeten, Kölreuter och Gärtner, hvilka egnat detta ämne nästan hela sitt lif, utan att känna sig djupt öfvertygad om allmänheten af en viss grad af ofruktsamhet. Kölreuter gör regeln allmän, men dervid hugger han af knuten, ty så snart han finner två former, som af de flesta vetenskapsmän betraktas som skilda arter men som äro fullkomligt fruktsamma vid kroasering, upptager han dem såsom varieteter. Gärtner gör likaledes regeln allmän, och bestrider den fullkomliga fruktsamheten i Kölreuters tio fall. Men Gärtner är dock i dessa och i många andra fall tvungen att noggrant räkna fröen för att bevisa, att det finnes någon grad af sterilitet. Han jemför alltid största antalet frön, som bildas vid två arters första kroasering, och största antalet frön, som deras bastard kan sätta, med medeltalet frön af de båda rena stamarterna i naturtillståndet. Men en svår källa till misstag synes mig här vara införd. För att blifva hybridiserad måste en växt kastreras, och hvad som är mera vigtigt, den måste afstängas för att hindra insekter ditföra frömjöl från andra växter. Nästan alla växter, med hvilka Gärtner experimenterade voro planterade i krukor och höllos i ett rum i hans hus. Att sådant fortfarande bör inverka på en växts fruktsamhet kan icke betviflas, ty Gärtner anför i sin tabell ungefär tjugu fall af växter, hvilka han kastrerade och artificielt befruktade med deras eget frömjöl, och hälften af dessa tjugu växter hade fått sin fruktsamhet till en viss grad försvagad; häruti inbegreps dock icke sådana växter som Leguminoser, hos hvilka manipulationerna äro ytterst svåra. Dessutom då Gärtner upprepade gånger kroaserade några former såsom den vanliga röda och den blåa arfven (Anagallis arvensis och coerulea), som de bästa botanister upptaga såsom varieteter, och fann dem absolut sterila, kunna vi betvifla, att många arter verkligen äro så sterila vid kroasering, som han trodde.

Å ena sidan är det säkert, att åtskilliga arters ofruktsamhet vid kroaseringen är så olika i styrka och erbjuder så många olika grader, och å andra sidan, att rena arters fruktsamhet så lätt rubbas af hvarjehanda omständigheter, att det för praktiska ändamål är särdeles svårt att säga, hvar den fullkomliga fruktsamheten slutar och ofruktsamheten begynner. Jag tror icke något bättre bevis härför kan behöfvas, än att de två mest erfarna observatörer som någonsin lefvat, Kölreuter och Gärtner, hafva kommit till diametralt motsatta åsigter med afseende på samma art. Det är också mycket lärorikt att jemföra de bevis våra bästa botanister framhålla i frågan, om vissa tvifvelaktiga former böra antagas för arter eller varieteter, med de bevis vissa experimentatörer hemta från fruktsamheten, eller bevis som samma författare hemtar från sina experiment under olika år. På detta sätt kan visas, att hvarken fruktsamhet eller ofruktsamhet lemnar någon tydlig skilnad emellan art och varietet, men att de bevis som hemtas från denna källa småningom försvinna och blifva lika tvifvelaktiga som de bevis, som hemtas från afvikelser i konstitution och skapnad.

Med afseende på ofruktsamhet hos bastarder under successiva generationer, var Gärtner visserligen i stånd att uppdraga några bastarder från en kroasering emellan rena arter genom omsorgsfull vård under sex eller sju och i ett fall tio generationer, men han försäkrar dock att deras fruktsamhet aldrig tilltog, utan i allmänhet aftog betydligt och hastigt. Här bör först anmärkas, att om någon afvikelse i bildning eller konstitution är gemensam för båda föräldrarna, öfvergår denna ofta förstorad på afkomman, och båda de sexuela elementen äro hos bastardväxter redan till en viss grad förändrade. Men jag tror, att i alla dessa fall fruktsamheten blifvit minskad genom en helt och hållet oberoende orsak nämligen kroasering emellan för nära slägtingar. Jag har samlat en stor mängd fakta som visa, å ena sidan att en tillfällig kroasering med en distinkt individ eller varietet förökar afkommans styrka och fruktsamhet, och å andra sidan att kroasering emellan nära slägtingar förminskar deras styrka och fruktsamhet, så att jag måste medgifva riktigheten af denna nästan allmänna tro. Bastarder uppfödas sällan af experimentatörer i stort antal, och då stamarterna eller andra beslägtade bastarder i allmänhet växa i samma trädgård, måste man omsorgsfullt förekomma besök af insekter under blomningstiden; bastarder befruktas derföre i allmänhet med sitt eget frömjöl under hvarje generation och detta är sannolikt skadligt för deras fruktsamhet, som redan är försvagad genom deras hybrida ursprung. Jag bestyrkes i denna öfvertygelse af ett påstående, som Gärtner ofta upprepar, att äfven de mindre fruktsamma bastarderna, om de artificielt befruktas med frömjöl af en annan bastard af samma slag, stundom bli betydligt mera fruktsamma trots de vanliga skadliga verkningarna af manipulationen, och att denna förökning i fruktsamheten alltjemt tilltar. Vid artificiel befruktning tages händelsevis (såsom jag vet af mina egna experiment) frömjöl från ståndarna i en annan blomma lika ofta som ifrån ståndarna i den blomma som skall befruktas, så att en kroasering emellan två blommor, ehuru båda på samma stånd, på detta sätt åstadkommes. En så noggrann iakttagare som Gärtner har väl under sina invecklade experiment dessutom kastrerat sina bastarder och detta har i hvarje generation gjort en kroasering med frömjöl från skilda blommor nödvändig, antingen från samma stånd eller ett annat af samma hybrida beskaffenhet. Och det sällsamma faktum, att fruktsamheten tilltager hos successiva generationer af bastarder som befruktas artificielt tvärt emot förhållandet med bastarder som befrukta sig sjelfva, kan såsom jag tror förklaras genom förhindrandet af kroasering emellan för nära slägtingar.

Vi skola nu vända oss till de resultat, som en tredje erfaren experimentator vunnit, Rev. W. Herbert. Med lika stort eftertryck försäkrar han, att några bastarder äro fullkomligt fruktsamma — lika fruktsamma som de rena stamarterna — som Kölreuter och Gärtner, att en viss grad af ofruktsamhet emellan skilda arter är en allmän naturlag. Han arbetade med några arter, som äfven Gärtner hade användt till sina experiment. Skilnaden i deras resultat kan måhända förklaras genom Herberts större skicklighet i trädgårdsskötsel och derigenom att han hade drifhus till sin disposition. Af hans många vigtiga uppgifter vill jag här blott anföra en såsom exempel, nämligen att ”hvarje frö i en frökapsel af Crinum capense, som var befruktad med frömjöl från Crinum revolutum, alstrade en planta, hvilket han aldrig såg inträffa vid naturlig befruktning.” Vi hafva således här ett exempel på fullkomlig eller till och med mer än fullkomlig fruktsamhet vid den första kroaseringen emellan två skilda arter.

Detta exempel gifver mig anledning att omnämna ett egendomligt faktum, nämligen att individer af vissa växtarter af Lobelia, Verbascum och Passiflora kunna med lätthet befruktas med frömjöl från en skild art, men icke med frömjöl från samma stånd, oaktadt detta frömjöl kan vara fullkomligt friskt, hvilket kan visas genom att dermed befrukta andra stånd eller arter. I slägtet Hippeastrum, i Corydalis, såsom Prof. Hildebrand har visat, och hos vissa orchideer enligt Scott och Fritz Müller äro alla individer så beskaffade. Hos vissa arter kunna således vissa abnorma individer och af andra arter alla individer verkligen hydridiseras vida lättare, än de befruktas med frömjöl från samma individ. Vi skola anföra ett exempel. En lök af Hippeastrum aulicum satte fyra blommor; Herbert befruktade tre med deras eget frömjöl och den fjerde befruktades sedermera med frömjöl från en sammansatt bastard som härstammade från tre arter; resultatet var att ”fruktämnena hos de tre första blommorna snart upphörde att växa och vissnade efter få dagar, hvaremot den kapsel, som blifvit befruktad med frömjöl från bastarden, tillväxte raskt och mognade och lemnade goda frön som frodades väl.” Herbert upprepade dessa försök under flera år och alltid med samma påföljd. Dessa växter, af hvilka blott en del individer icke kunna befruktas af sitt eget frömjöl, ehuru de tyckas vara fullkomligt friska och ehuru både frömjöl och fröämnen äro fullkomligt goda vid befruktning med andra arter, måste dock vara i något onaturligt tillstånd, och dessa fall tjena till att visa, på hvilka obetydliga och hemlighetsfulla orsaker en arts större och mindre fruktsamhet stundom beror.

Trädgårdsodlares praktiska försök förtjena att uppmärksammas, om de också icke äro utförda med vetenskaplig noggranhet. Det är väl bekant, på hvilket inveckladt sätt arterna af Pelargonium, Fuchsia, Calceolaria, Petunia, Rhododendron m. fl. hafva blifvit kroaserade och dock sätta många af dessa bastarder utan svårighet frö. Herbert försäkrar till exempel, att en bastard af Calceolaria integrifolia och plantaginea, arter som i sin allmänna habitus äro betydligt olika, ”fortplantar sig sjelf lika fullkomligt, som om den varit en naturlig art från Chilis berg.” Jag har nedlagt mycken möda på att öfvertyga mig om graden af fruktsamhet vid några sammansatta kroaseringar af Rhododendron och jag har kommit till det resultat, att många af dem äro fullkomligt fruktsamma. Mr C. Noble meddelar mig till exempel, att han till ympning uppdragit stammar af en bastard emellan Rhododendron ponticum och Catawbiense och att denna sätter frön ”så lätt man kan tänka sig.” Om bastarder vid riktig behandling alltjemt tilltagit i fruktsamhet i hvarje ny generation, såsom Gärtner trodde, skulle detta varit väl bekant för våra trädgårdsegare. Trädgårdsodlare uppdraga hela sängar af samma bastarder och dessa allena underkastas lämplig behandling, ty de olika individerna af samma hybrida varietet kunna med insekters tillhjelp fritt kroasera sig med hvarandra och det skadliga inflytandet af kroasering med nära slägtingar är derigenom förekommet. Hvar och en kan lätt öfvertyga sig om verksamheten af insekternas biträde genom att undersöka blommorna hos de mera sterila Rhododendronbastarderna, hvilka icke lemna något frömjöl, och han skall då finna på deras märken massor af frömjöl, som insekter fört med sig dit från andra blommor.

Hvad djuren beträffar har ett vida mindre antal noggranna försök med dem blifvit anstälda än med växter. Om våra systematiska uppstälningar förtjena förtroende, det vill säga om djurslägtena äro lika väl skilda från hvarandra som växtslägtena, kunna vi påstå, att djur som äro mera skilda i naturens skala med vida större lätthet kunna kroaseras än växter, men bastarderna äro som jag tror mera sterila. Jag betviflar, att något fall af en fullkomligt fruktsam djurbastard kan bevisas såsom fullt autentiskt. Vi få likväl komma ihåg, att då så få djur kroasera sig fritt i fångenskapen, endast ett ringa antal experiment blifvit anstälda. Kanariefågeln har till exempel blifvit kroaserad med nio andra finkar, men då ingen af dessa nio arter kroaseras fritt i fångenskap, hafva vi icke rätt att vänta, att de första kroaseringarna emellan dem och kanariefågeln skulle blifva fruktsamma, ej heller deras bastarder. Hvad åter beträffar fruktsamheten hos nya generationer af de mera fruktsamma bastarderna, känner jag knappt ett enda exempel på, att två familjer af samma bastard samtidigt blifvit uppfödda från olika föräldrar, så att man kunnat förekomma olägenheten af kroasering emellan slägtingar. Tvärtom hafva bröder och systrar vanligen kroaserats i hvarje ny generation, tvärtemot alla fackmäns upprepade varningar. Och i detta fall är det alldeles icke förvånande, att den ärfda ofruktsamheten hos hybriderna alltjemt tilltagit. Om vi med afsigt sammanparade bröder och systrar af någon ren djurart, som af någon orsak hade den minsta benägenhet för ofruktsamhet, skulle rasen helt säkert slockna ut inom få generationer.

Ehuru jag icke känner något fullt trovärdigt fall af fullkomligt fruktsamma djurbastarder, har jag dock anledning att tro, att bastarder af Cervulus vaginalis och Reevesii samt af Phasianus colchicus och P. torquatus äro fullkomligt fruktsamma. Man har nyligen påstått, att två så skilda arter som hare och kanin, om de bringas till parning alstra en nästan fullkomligt fruktsam afkomma, men detta påstående är ännu tvifvel underkastadt. Bastarder af den vanliga och den kinesiska gåsen (Anser cygnoides), arter som äro så skilda, att de i allmänhet upptagas i olika slägten hafva ofta lemnat ungar i detta land med den rena stamarten, och i ett enda fall hafva de varit fruktsamma sinsemellan. Detta har Mr Eyton åstadkommit, som uppfödde två bastarder af samma föräldrar, men af olika kull och af dessa två fåglar fick han icke mindre än åtta bastarder (barnbarn af den rena gåsen) i ett bo. I Indien måste likväl dessa genom kroasering vunna gäss vara vida mera fruktsamma, ty två utmärkta män, Mr Blyth och kapten Hutton hafva försäkrat mig, att hela flockar af dessa kroaserade gäss hållas i olika delar af landet, och då de hållas för nyttan der ingendera af stamarterna finnes måste de helt säkert vara i hög grad fruktsamma.

De olika raserna af alla slags husdjur äro fullkomligt fruktsamma då de kroaseras och dock härstamma de i många fall från två eller flera vilda arter. Från detta faktum måste vi draga den slutsatsen, att antingen de ursprungliga stamarterna alstrade fullkomligt fruktsamma bastarder, eller att de bastarder som uppföddes i tamt tillstånd sedermera blefvo fruktsamma. Det senare alternativet, som Pallas först framstälde, synes mest sannolikt och kan knappt betviflas. Det är till exempel nästan säkert, att våra hundar härstamma från flera vilda arter och dock äro alla, med undantag måhända af vissa infödda hushundar i Sydamerika fullkomligt fruktsamma sinsemellan, och analogien gifver mig mycken anledning att betvifla, att de ursprungliga arterna först kroaserades fritt med hvarandra och lemnade fruktsamma bastarder. Jag har nyligen fått afgörande bevis på att hybriderna af den indiska puckeloxen (Zebu) och den vanliga nötboskapen äro sinsemellan fullkomligt fruktsamma, och på grund af Rütimeyers iakttagelser öfver deras vigtiga osteologiska olikheter, äfvensom Blyths observationer öfver olikheterna i vanor, läte, konstitution etc. måste dessa två former betraktas såsom så väl skilda arter som några andra på jorden. Enligt denna åsigt om många husdjurs ursprung måste vi antingen uppgifva tron på den nästan allmänna ofruktsamheten vid kroasering emellan skilda djurarter, eller måste vi betrakta ofruktsamheten icke såsom en outplånlig karakter, utan såsom en egenskap som kan aflägsnas genom tämjning.

Om vi slutligen taga i betraktande alla de faststälda fakta rörande kroasering af djur och växter, måste vi draga den slutsatsen, att en viss grad af ofruktsamhet är en ytterst vanlig företeelse både vid första kroaseringen och hos bastarderna, men att den efter hvad vi nu känna icke kan vara ovilkorligt allmän.



\section[Lagar för ofruktsamhet]{Lagar för ofruktsamheten vid första kroaseringen
och hos bastarder.}

Vi skola nu litet mera i detalj betrakta de omständigheter och reglor, som styra ofruktsamheten vid den första kroaseringen och hos bastarderna. Vår förnämsta uppgift skall vara att se, huruvida dessa reglor tillkännagifva, att arterna hafva blifvit begåfvade med denna egenskap för att förekomma deras kroasering och sammanblandning till en ytterlig förvirring. De här följande reglorna och slutföljderna äro hufvudsakligen hemtade från Gärtners utmärkta arbete om bastardbildning bland växter. Jag har gjort mig mycken möda att erfara, i hvad mån dessa reglor hafva tillämpning inom djurriket, och i betraktande af huru ringa vår kännedom är om djurbastarder, har jag blifvit öfverraskad att finna samma reglors allmänna tillämpning i både växt- och djurriket.

Det har redan blifvit anmärkt, att fruktsamheten både vid första kroaseringen och hos bastarderna gradvis öfvergår från noll till fullkomlig fruktsamhet. Det är märkvärdigt, på huru många egendomliga vägar detta kan visas, men jag kan här blott i enkla drag anföra några fakta. Om frömjöl från en växt i en familj förflyttas till märket på en växt af helt annan familj, utöfvar det icke mera inflytande derpå än oorganiskt dam. Från denna absoluta nollpunkt finnes en fullkomlig serie ända upp till fullkomlig fruktsamhet, om pollenkornen af olika arter inom samma slägte appliceras på märket af någon af dem, och vi hafva sett, att i vissa abnorma fall denna fruktsamhet stiger till en utomordentlig grad, utöfver den som växtens eget frömjöl kan åstadkomma. Bland sjelfva bastarderna finnas några, hvilka aldrig hafva satt och sannolikt aldrig skola sätta ett enda fruktsamt frö, icke en gång med frömjöl från föräldrar af ren art, men i några af dessa fall kan ett första spår till fruktsamhet upptäckas deruti, att frömjölet af en ren stamart bringar bastardens blomma att vissna hastigare än den annars skulle gjort, och en blommas tidiga vissnande är som bekant ett tecken till börjande befruktning. Från denna yttersta grad af ofruktsamhet hafva vi sjelfbefruktande bastarder, som sätta allt större och större antal frö, ända upp till den fullkomligaste fruktsamhet.

Bastarder af två arter, som äro mycket svåra att kroasera och sällan lemna någon afföda, äro i allmänhet mycket sterila, men jemförelsen emellan svårigheten att åstadkomma första kroaseringen och de deraf uppkomna bastardernas ofruktsamhet — två klasser af fakta, som i allmänhet förblandas med hvarandra — är på inga vilkor sträng. Det finnes många fall, i hvilka två rena arter kunna förenas med ovanlig lätthet, såsom i slägtet Verbascum, och alstra talrika hybrida afkomlingar, men dessa äro ytterst ofruktsamma. Å andra sidan finnas arter, som mycket sällan eller med yttersta svårighet kunna kroaseras, men om en gång bastarder bildas, äro de mycket fruktsamma. Dessa två motsatta fall inträffa stundom inom samma slägte, till exempel Dianthus.

Fruktsamheten vid första kroaseringen och hos bastarderna rubbas vida lättare af ogynsamma förhållanden än fruktsamheten hos rena arter. Men graden är dessutom i sig sjelf föränderlig, ty den är icke alltid densamma, om samma två arter kroaseras under samma förhållanden, utan beror till en del på de individers konstitution, hvilka komma att väljas till experimentet. Så är äfven förhållandet med bastarderna, ty deras fruktsamhetsgrad skiljer sig ofta betydligt hos olika individer uppdragna från frön ur samma fröhus, oaktadt de lefva under samma förhållanden.

Med uttrycket systematisk affinitet menas likheten emellan arter i skapnad, konstitution, särdeles i de delar, som äro af hög fysiologisk betydelse och visa blott få afvikelser hos närslägtade arter. Fruktsamheten vid första kroaseringen emellan arter och hos de bastarder som deraf uppkomma beror i hög grad på deras systematiska affinitet. Detta bevisas deraf, att hybrider aldrig hafva uppkommit emellan arter, som af systematiker ställas i skilda familjer, och å andra sidan deraf, att mycket närbeslägtade arter i allmänhet förena sig med lätthet. Men öfverensstämmelsen emellan systematisk affinitet och lätthet för kroasering är icke sträng. En mängd exempel kunde gifvas på närslägtade arter, hvilka icke eller blott med yttersta svårighet kunna bringas till parning, och å andra sidan på mycket skilda arter, som med lätthet låta kroasera sig. I samma familj kan finnas ett slägte, som Dianthus, af hvilket många arter utan svårighet kunna kroaseras, och ett annat slägte, Silene, i hvilket de mest ihärdiga ansträngningar icke hafva lyckats att emellan ytterst beslägtade arter åstadkomma en enda bastard. Till och med inom samma slägte träffa vi på samma olikhet; de många arterna af Nicotiana (tobak) hafva till exempel blifvit i vida större skala kroaserade än arterna af något annat slägte, men Gärtner har funnit, att N. acuminata, som icke är någon särdeles afvikande art, envist emotstod alla befruktningsförsök, så att den icke kunde befrukta, ej heller befruktas af åtta andra arter af slägtet Nicotiana. Många analoga fakta kunde anföras.

Ännu har ingen kunnat bestämma, hvad slags eller hvilken grad af olikhet i någon upptäckbar karakter är tillräcklig att förhindra två arters kroasering. Man kan visa, att arter kunna kroaseras, hvilka äro helt olika i sitt lefnadssätt och allmänna utseende, och som hafva starkt utpräglade skiljaktigheter i hvarje blomdel, till och med i frömjölet, i frukten och i hjertbladen. Ettåriga och fleråriga örter, träd med affallande blad och träd som äro gröna hela året om, växter som bebo skilda orter och äro tjenliga för helt olika klimat kunna ofta med lätthet kroaseras.

Med ömsesidig kroasering emellan två arter menar jag sådana fall, som då till exempel en hingst först kroaseras med en åsnehona och sedan en åsnehanne med ett sto: dessa två arter kunna då sägas blifvit ömsesidigt kroaserade. I lättheten att åstadkomma ömsesidig kroasering finnes ofta den största möjliga olikhet. Sådana fall äro särdeles vigtiga, ty de bevisa, att förmågan af kroasering hos två arter ofta är fullkomligt oberoende af deras systematiska slägtskap eller af hvarje olikhet i hela deras organisation med undantag af reproduktionsorganerna. Olikheten i resultat vid ömsesidig kroasering emellan två arter var för längesedan observerad af Kölreuter. Vi skola anföra ett exempel; Mirabilis jalapa kan med lätthet befruktas med frömjöl från M. longiflora och de på sådant sätt uppkomna bastarderna äro ganska fruktsamma; men Kölreuter försökte mer än tvåhundra gånger under åtta års tid att befrukta M. longiflora med frömjöl af M. jalapa, men utan resultat. Några andra lika slående exempel kunde gifvas och Thuret har observerat samma förhållande med några sjöväxter. Gärtner fann vidare, att denna olikhet i lättheten att åstadkomma ömsesidig kroasering i mindre grad är ytterst vanlig. Han har iakttagit den äfven emellan mycket beslägtade former, såsom Matthiola annua och glabra, hvilka många botanister upptaga blott såsom varieteter. Det är också ett anmärkningsvärdt förhållande, att fruktsamheten hos bastarder som uppkommit genom ömsesidig kroasering i allmänhet är olika i ringa grad och stundom betydligt, ehuru de sällan afvika i yttre karakterer, och likväl äro de naturligtvis bildade af samma två arter på det sätt, att den ena arten först blifvit begagnad som fader och sedan som moder.

Flera andra egendomliga reglor kunde hemtas från Gärtner; några arter hafva till exempel en anmärkningsvärd förmåga att kroasera sig med andra arter; några arter af samma slägte visa en särdeles stor benägenhet att förläna sina hybrida afkomlingar sitt utseende, men dessa två egenskaper äro icke nödvändigt förenade. Det finnes vissa bastarder, hvilka i stället för att hafva som vanligt karakterer, som stå midt emellan föräldrarnas, nästan fullkomligt likna den ena, och ehuru de till det yttre äro så lika sina föräldrar af ren art, äro de dock med få undantag ytterst sterila. Så förekomma äfven ibland bastarder, hvilka i allmänhet i skapnad stå midt emellan föräldrarna, undantagsvis abnorma individer, hvilka utomordentligt likna den ena af föräldrarna; och dessa bastarder äro nästan alltid ytterst sterila, äfven om de andra bastarderna uppdragna af frön från samma fröhus hafva en ansenlig grad af fruktsamhet. Dessa fakta visa att fruktsamheten hos bastarderna är fullkomligt oberoende af deras yttre likhet med föräldrarna af ren art.

Om vi betrakta de nu gifna reglorna för fruktsamheten vid första kroaseringen och hos bastarder, se vi, att, om former förenas, hvilka måste anses såsom goda och skilda arter, deras fruktsamhet genomgår alla grader från noll till fullkomlig fruktsamhet eller till och med under vissa vilkor derutöfver. Deras fruktsamhet är äfven i sig sjelf föränderlig, jemte det att den är ytterligt känslig för gynsamma och ogynsamma förhållanden. Den är ingalunda alltid till graden lika vid första kroaseringen och hos de deraf uppkomna bastarderna, och dessas fruktsamhet står icke i något förhållande till graden af likhet med någondera af föräldrarna i yttre utseende. Och slutligen är lättheten för första kroaseringen emellan två arter icke alltid beroende af deras systematiska slägskap eller graden af likhet emellan dem. Den sista satsen bevisas klart af de skiljaktiga resultaten af ömsesidig kroasering emellan samma två arter, ty allt efter som den ena eller andra arten användes såsom fader eller moder, finnes i allmänhet någon olikhet, och stundom den största möjliga, i lättheten att åstadkomma en kroasering. Dessutom skilja sig äfven bastarderna af ömsesidiga kroaseringar ofta i fruktsamhet.

Månne nu dessa invecklade och egendomliga reglor gifva vid handen, att arterna blifvit begåfvade med ofruktsamhet vid kroasering endast af det skäl, att de icke skola sammanblandas i naturtillståndet? Jag tror det icke. Ty hvarföre skulle ofruktsamheten vara af så olika grad vid arternas kroasering, då vi måste antaga det vara af lika vigt för alla arter att skyddas för sammanblandning? Hvarföre skulle graden af ofruktsamhet i sig sjelf vara föränderlig hos individer af samma art? Hvarföre skulle en art kunna kroaseras med lätthet och dock lemna blott sterila bastarder, och hvarföre skulle andra arter, som blott med största svårighet kunna kroaseras, lemna fullt fruktsamma bastarder? Hvarföre skulle ofta finnas så stor olikhet i resultaten af ömsesidig kroasering emellan två arter? Hvarföre, kan man äfven fråga, har bastardbildning någonsin blifvit möjlig? Att gifva arterna den särskilda förmågan att alstra bastarder och sedan sätta en gräns för deras vidare fortplantning genom olika grader af ofruktsamhet, som icke stå i något närmare sammanhang med lättheten af den första kroaseringen, det synes mig i sanning vara en sällsam anordning.

Ofvanstående reglor och fakta synas mig deremot tydligt visa, att ofruktsamheten både vid första kroaseringen och hos bastarder är helt enkelt tillfällig eller beroende på okända afvikelser i reproduktionsorganerna; olikheterna äro af så egendomlig och begränsad natur, att vid ömsesidig kroasering det manliga elementet af den ena arten fritt inverkar på det qvinliga hos den andra, men icke tvärtom. Det är kanske lämpligt att litet mera fullständigt med ett exempel förklara hvad jag menar med att ofruktsamheten är beroende på andra skiljaktigheter och icke en särskild förlänad egenskap. Då en växts förmåga att kunna inympas på en annan eller icke är helt och hållet likgiltig för dess välfärd i naturtillståndet, antager jag, att ingen vill anse denna förmåga såsom en särskild förlänad egenskap, utan att hvar och en medgifver att den sammanfaller med skiljaktigheter i lagarna för de båda växternas utveckling. Vi kunna stundom se skälet, hvarföre ett träd icke går på ett annat, uti olikheter i deras tillväxt, i hårdheten af deras ved, i safvens beskaffenhet och dylikt, men i en mängd fall kunna vi icke angifva något skäl alls. Till och med stora olikheter, såsom om växternas storlek är betydligt olika, om den ena är trädartad, den andra en ört, eller den ena alltid grön, den andra blott om sommaren, och om de äro lämpliga för olika klimat, äro icke alltid tillräckliga hinder för ympningen. Såsom vid bastardbildning, så är äfven vid ympning möjligheten begränsad af systematisk affinitet, ty ingen har varit i stånd att ympa ihop träd som höra till olika familjer, och å andra sidan kunna beslägtade arter och varieteter af samma art i allmänhet men icke ovilkorligen ympas i hvarandra med lätthet. Men denna förmåga är såsom vid bastardbildning ingalunda absolut beroende af systematisk slägtskap. Ehuru många skilda arter inom samma familj hafva blifvit ympade tillsammans, lyckas ympningen i andra fall icke med arter af samma slägte. Päronet kan vida lättare inympas på qvittenträd, som upptages såsom särskildt slägte, än på äppelträd, som tillhör samma slägte. Till och med olika varieteter af päronet slå an med olika grad af lätthet på qvittenträd och på samma sätt förhålla sig olika varieteter af aprikos och persika emot vissa plommonvarieteter.

Gärtner fann stundom en medfödd olikhet hos individerna af två arter vid kroasering och samma förhållande tror Sageret äfven ega rum med olika individer af samma två arter, då de ympas tillsammans. Vid ömsesidig kroasering är lättheten att åstadkomma förening ofta långt ifrån lika och så är stundom äfven förhållandet vid ympning; de vanliga krusbären till exempel kunna icke inympas på vinbär, hvaremot vinbären, ehuru med någon svårighet, slå an på krusbärsbuskar.

Vi hafva sett, att bastarders ofruktsamhet, hvilka hafva sina reproduktionsorganer i ett ofullkomligt tillstånd, är ett fall som bör skiljas ifrån svårigheten att sammanpara två rena arter med fullkomliga reproduktionsorganer, men dessa två skilda fall gå till en viss grad parallelt. Något analogt förekommer äfven vid ympning. Thouin fann, att tre arter af Robinia, hvilka på sina egna rötter lätt satte frö och som utan synnerlig svårighet läto ympa sig på en annan art, genom ympningen blefvo ofruktsamma. Vissa arter af Sorbus lemnade deremot efter inympning på andra arter dubbelt så mycket frukt som på sina egna rötter. Detta senare faktum påminner oss om de utomordentliga Hippeastrum, Passiflora m. fl. hvilka sätta vida mera frön då de befruktas med frömjöl af en skild art, än om de befruktas med frömjöl från samma stånd.

Vi se sålunda, att, ehuru det finnes en stor och tydlig skilnad emellan inympade stammars blotta adhesion och föreningen emellan hanliga och honliga element vid reproduktionsakten, dock en viss grad af parallelism eger rum emellan resultaten vid ympning och kroasering af skilda arter. Och då vi måste betrakta de egendomliga och invecklade lagarna för den olika lätthet med hvilken träd kunna inympas på hvarandra såsom sammanfallande med okända skiljaktigheter i deras vegetativa system, så tror jag att de ännu mera invecklade lagarna för lättheten af en första kroasering sammanfalla med okända skiljaktigheter i deras reproduktiva system. Dessa skiljaktigheter följa i båda fallen till en viss grad, såsom man kunnat vänta, den systematiska slägtskapen, genom hvilket uttryck hvarje slags likhet och olikhet emellan organiska varelser betecknas. Fakta synas mig på intet vis gifva vid handen, att den större eller mindre svårigheten af att ympa eller kroasera olika arter är en särskild förlänad egenskap, ehuru med afseende på kroasering denna svårighet är likaså vigtig för arternas varaktighet och bestånd, som den vid ympning är likgiltig för deras välfärd.



\section[Orsaker till ofruktsamhet]{Orsaker till ofruktsamhet vid första kroaseringen och
hos bastarder.}

En tid syntes det mig sannolikt, såsom äfven för andra, att ofruktsamheten vid den första kroaseringen och hos bastarder blifvit småningom förvärfvad genom naturligt urval af obetydligt förminskade grader af fruktsamhet, som frivilligt uppträdde liksom hvarje annan variation hos vissa individer af en varietet vid kroasering med en annan varietet. Ty det skulle tydligen vara fördelaktigt för två varieteter eller begynnande arter, om de kunde hindras att blanda sig med hvarandra, enligt samma grundsats, som menniskan följer, då hon vid utväljande af två varieteter måste hålla dem skilda. I första rummet må det anmärkas, att skilda trakter ofta bebos af artgrupper och af enstaka arter, som om de bringas tillsammans och kroaseras befinnas vara mer eller mindre ofruktsamma; nu kan det tydligen icke hafva varit af någon fördel för sådana åtskilda arter att hafva blifvit sinsemellan sterila och detta kan således icke hafva åstadkommits af det naturliga urvalet; men man kan måhända påstå, att om en art blefve ofruktsam med någon landsman, ofruktsamhet med andra arter skulle blifva en nödvändig följd. Vidare står det nästan lika mycket i strid med teorien om det naturliga urvalet som med den särskilda skapelsen, att vid ömsesidig kroasering det manliga elementet af en art skulle blifvit ytterst overksamt emot en annan art, under det på samma gång denna andra arts manliga element är i stånd att fritt befrukta den första arten; ty detta besynnerliga tillstånd hos reproduktionssystemet kunde icke vara fördelaktigt för någondera arten.

Om vi betrakta sannolikheten för det naturliga urvalets verksamhet att göra arterna sinsemellan fruktsamma, befinnes en stor svårighet ligga i tillvaron af många gradvisa öfvergångar från obetydligt förminskad fruktsamhet till absolut ofruktsamhet. Det må medgifvas, att enligt ofvan utvecklade grundsats det skulle gagna en art under bildning att vara till en viss ringa grad ofruktsam vid kroasering med dess stamform eller med någon annan varietet; ty ett mindre antal hybridiserade och försämrade afkomlingar skulle då kunna blanda sitt blod med de nya under bildning varande arterna. Men den som vill göra sig möda att reflektera öfver de stadier genom hvilka denna första grad af ofruktsamhet skulle förökas genom naturligt urval till den höga grad, som är allmän för arter, hvilkas olikheter blifvit så stora, att de upptagas i skilda slägten eller familjer, han skall finna ämnet ytterligt inveckladt. Efter mogen öfverlägning synes det mig, att detta icke kan vara en följd af naturligt urval, ty det kunde icke hafva varit af någon direkt nytta för ett djur att med svårighet kroaseras med en annan individ af olika varietet och på detta sätt lemna få afkomlingar; sådana individer kunde följaktligen icke hafva blifvit utvalda eller bibehållna. Eller låt oss taga två arter, hvilka i sitt nuvarande tillstånd vid kroasering lemna några få ofruktsamma afkomlingar; hvad finnes här som skulle gynna de individer, hvilka råkade att vara begåfvade med en obetydligt högre grad af ofruktsamhet, hvilka på detta sätt stodo ett steg närmare absolut ofruktsamhet? Ett framåtgående af detta slag måste, om teorien för det naturliga urvalet här kan användas, oupphörligen hafva inträffat med många arter, ty en mängd äro sinsemellan fullkomligt ofruktsamma. Vi hafva visserligen skäl att tro, hvad de sterila, könlösa insekterna beträffar, att modifikationer i deras kroppsbildning och fruktsamhet blifvit långsamt samlade af det naturliga urvalet, då samhället som de tillhörde derigenom fick en viss fördel öfver andra samhällen af samma art, men om en djurindivid, som icke hör till något samhälle, blefve ofruktsam vid kroasering med någon annan varietet, skulle denna icke sjelf vinna någon fördel deraf eller indirekt gifva andra individer af samma varietet någon fördel och på detta sätt föranleda deras bibehållande. Från dessa betraktelser drager jag den slutsatsen, att hvad djuren beträffar de olika grader af fruktsamhet, som arter förete vid kroasering, icke kunna hafva långsamt förökas genom naturligt urval.

Ibland växter är det möjligt att förhållandet kan vara något annorlunda. Hos många föra insekter beständigt frömjöl från närstående växter till märkena i hvarje blomma och hos några arter åstadkommes detta af vinden. Om nu frömjölet af en varietet, då det förflyttades på märket af samma varietet, skulle genom spontan variation blifva i aldrig så ringa grad mäktigare än andra varieteters frömjöl, skulle detta vara en fördel för varieteten, ty dess eget frömjöl skulle då tillintetgöra verkningarna af andra varieteters frömjöl och på detta sätt förhindra karakterens försämring. Och ju mera öfvervigt varietetens eget frömjöl kunde få genom naturligt urval, ju större skulle fördelarna vara. Från Gärtners undersökningar känna vi, att hos arter som äro sinsemellan ofruktsamma hvar och ens frömjöl alltid är verksammare på sitt eget märke än andra arters; men vi veta icke om denna öfvervigt är en följd af ofruktsamheten eller om ofruktsamheten är en följd af denna öfvervigt. Om den senare åsigten är riktig, så måste, då denna öfvervigt blir allt större och större genom naturligt urval, emedan den är fördelaktig för en art under bildning, äfven ofruktsamheten som följer af denna öfvervigt på samma gång ökas, och resultatet blir olika grader af ofruktsamhet, såsom vi påträffa hos färdiga arter. Denna åsigt kunde utsträckas till djuren, om honan före hvarje börd emottog flera hannar, så att det sexuela elementet af hennes egen varietet genom sin öfvervigt tillintetgjorde verkningarna af de andra hannarnas, som tillhörde andra varieteter, men vi hafva intet skäl att tro, att detta är förhållandet åtminstone bland landdjuren, ty de flesta hannar och honor para sig för hvarje befruktning och några förena sig för hela lifvet.

Vi kunna således antaga, att hvad djuren beträffar ofruktsamheten vid arters kroasering icke blifvit långsamt ökad genom naturligt urval, och då denna ofruktsamhet följer samma allmänna lagar i växt- som i djurriket, är det osannolikt ehuru skenbart möjligt, att bland växterna kroaserade arter skulle blifvit ofruktsamma genom en olika process. På grund af dessa betraktelser och ihågkommande, att sådana arter i allmänhet äro sterila vid kroasering, hvilka aldrig lefvat tillsammans i samma trakt och som derföre icke kunnat hafva någon fördel af att blifva sinsemellan ofruktsamma, samt med tanken på att vid ömsesidig kroasering emellan två arter stundom finnes den största olikhet i deras ofruktsamhet, måste vi öfvergifva den tron, att det naturliga urvalet härvid varit verksamt. Vi ledas således till vår förra sats, att ofruktsamheten vid första kroaseringen och hos bastarder är helt enkelt sammanfallande med okända skiljaktigheter i reproduktionssystemet hos stamarterna.

Vi skola nu försöka att litet närmare betrakta den sannolika beskaffenheten af dessa olikheter, hvilka föranleda ofruktsamhet vid den första kroaseringen och hos bastarder. Rena arter och bastarder skilja sig såsom redan är anmärkt i sina reproduktionsorganers tillstånd, men från det nedan anförda om ömsesidigt dimorfa och trimorfa växter vill det synas som om någon okänd lag funnes, som har till följd att ungarna af en icke fullkomligt fruktsam förening sjelfva blifva mer eller mindre ofruktsamma.

Vid den första kroaseringen emellan rena arter beror den större eller mindre svårigheten att åstadkomma en förening och erhålla en afföda på flera skilda orsaker. Det måste stundom vara en fysisk omöjlighet för det hanliga elementet att nå ägget, såsom i de fall då en växt har en pistill så lång, att pollenrören icke kunna nå ned till fröämnena. Man har också iakttagit, att om frömjöl af en art förflyttas till märket af en annan icke närbeslägtad art, pollenrören visserligen skjuta fram men icke kunna genomtränga märkets yta. Det hanliga elementet kan å andra sidan verkligen uppnå det honliga, men utan att bringa embryot till utveckling, såsom förhållandet tyckes hafva varit med några af Thurets experimenter med Fuci. Någon förklaring kan icke gifvas på dessa fakta, lika litet som man kan förklara, hvarföre vissa träd icke kunna ympas på hvarandra. Slutligen kan ett embryo utvecklas men omkomma i en mycket tidig period. Detta senare alternativ har icke varit föremål för tillräcklig uppmärksamhet, men på grund af iakttagelser som blifvit mig meddelade af mr Hewitt, hvilken hade stor erfarenhet i bastardbildning af fasaner och höns, tror jag, att embryots död är en mycket vanlig orsak till ofruktsamhet vid en första kroasering. Mr Salter har nyligen framlagt resultaten af en undersökning på omkring femhundra ägg, som han fått genom olika kroaseringar emellan tre arter af slägtet Gallus och deras hybrider. Största delen af dessa ägg hade blifvit befruktade, och i större delen af de befruktade äggen hade embryonerna antingen blifvit delvis utvecklade och sedan aborterat eller hade de blifvit nära fullt utvecklade, men de unga kycklingarna hade icke varit i stånd att bryta skalen. Af de kycklingar som föddes dogo mer än fyra femtedelar inom de första dagarna eller åtminstone veckorna ”utan någon märkbar orsak, blott som det tycktes derföre att de icke kunde lefva;” så att af femhundra ägg blott tolf kycklingar uppföddes. Hybrida embryoners tidiga död inträffar sannolikt på samma sätt hos växterna; åtminstone är det kändt, att bastarder som uppdragas af skilda arter stundom äro svaga och dvärgartade och dö i tidig ålder, hvarpå Max Wichura nyligen gifvit några slående exempel bland hybrida pilar. Det kan här förtjena anmärkas, att i några fall af partenogenesis embryoner från obefruktade silkesmaskägg passerade sina tidigare utvecklingsstadier och derefter omkommo liksom embryoner, som bildats genom kroasering emellan två skilda arter. Förrän jag erhöll kännedom om dessa fakta, ville jag icke gerna tro, att förtidig död så ofta förekom hos hybrida embryoner, ty då bastarderna en gång äro framfödda, äro de i allmänhet friska och långlifvade såsom vi se förhållandet vara med den vanliga mulåsnan. Bastarder äro likväl utsatta för olika omständigheter före och efter födseln: då de en gång äro födda och lefva i en trakt der deras båda föräldrar lefva, äro de i allmänhet placerade under gynsamma lefnadsförhållanden. Men en bastard är blott till hälften delaktig af moderns natur och konstitution, och derföre så länge den före födseln näres i moderlifvet eller i ägg och frön frambragta af modern, kan den vara utsatt för förhållanden som till en viss grad äro ogynsamma, och följaktligen kan den lättare omkomma i en tidig period, isynnerhet som alla unga varelser äro synnerligen känsliga för skadliga eller onaturliga lefnadsförhållanden. Men det är dock mera sannolikt, att orsaken ligger i en viss ofullkomlighet i sjelfva impregneringsakten, som gör att embryot blir ofullständigt utveckladt, än att den ligger i de omständigheter för hvilka embryot sedermera utsättes.

Förhållandet är helt olika med bastarder, hos hvilka de sexuela elementen äro ofullständigt utvecklade. Jag har mer än en gång antydt, att jag samlat en mängd fakta som visa, att om växter och djur förflyttas från deras naturliga förhållanden, de äro ytterligt blottstälda för allvarsamma rubbningar i reproduktionsorganerna. Detta är i sjelfva verket ett stort hinder för djurens tämjning. Emellan den på detta vis uppkomna ofruktsamheten och bastardernas sterilitet finnas många likheter. I båda fallen är ofruktsamheten oberoende af allmänna helsotillståndet och är ofta åtföljd af omåttlig storlek och yppighet. I båda fallen förekommer ofruktsamheten i flera grader; i båda fallen är det manliga elementet mest utsatt för rubbningar, men stundom det honliga mera än hannens. I båda fallen håller fruktsamheten till en viss grad jemna steg med systematisk slägtskap, ty hela grupper af växter och djur blifva ofruktsamma under samma naturliga förhållanden och hela grupper af arter sträfva att alstra sterila bastarder. Å andra sidan emotstår stundom en art i en grupp stora förändringar i yttre förhållanden med oförsvagad fruktsamhet och vissa arter i en grupp lemna ovanligt fruktsamma bastarder. Ingen kan på förhand säga, om ett visst djur skall fortplanta sig i fångenskapen eller om en utländsk växt skall sätta frö vid odling, och icke heller kan han på förhand säga, om två arter af ett slägte skola lemna mer eller mindre ofruktsamma bastarder. Slutligen om organiska varelser under flera generationer försättas under omständigheter som för dem icke äro naturliga, äro de ytterst benägna för att variera, hvilket till en del synes bero derpå, att deras reproduktionsorganer hafva lidit någon rubbning ehuru svagare än då ofruktsamhet följer. Så är äfven förhållandet med bastarder, ty deras afkomlingar i följande generationer äro särdeles benägna att variera, såsom hvarje experimentator har iakttagit.

Vi se således, att om organiska varelser försättas under nya och onaturliga förhållanden, och om bastarder bildas genom onaturlig kroasering af två arter, det reproduktiva systemet oberoende af allmänna helsotillståndet på likartadt sätt rubbas i sin verksamhet. I ena fallet hafva lefnadsförhållandena förändrats, ehuru ofta i så ringa grad att vi ej kunna märka det, i det andra fallet hafva de yttre vilkoren förblifvit desamma, men organisationen har blifvit bragt i oordning derigenom, att två kroppsbildningar och konstitutioner blifvit blandade till en. Ty det är näppeligen möjligt att två organisationer kunna förenas till en utan någon rubbning i utvecklingen eller olika delars och organers periodiska verkan och ömsesidiga förhållanden. Om bastarder äro i stånd att fortplanta sig med hvarandra, lemna de i arf åt sina afkomlingar från generation till generation denna sammansatta organisation, och derföre böra vi icke förvånas öfver, att deras ofruktsamhet, ehuru till en viss grad föränderlig, dock icke aftager; den är till och med benägen att tilltaga, och detta är vanligen resultatet af parning emellan för nära slägtingar, såsom vi förut förklarat. Ofvanstående åsigt, att bastarders ofruktsamhet beror på tvänne organisationers sammanblandning till en, har nyligen blifvit starkt framhållen af Max Wichura, men jag måste tillstå, att ofruktsamheten hos afkomlingarna af dimorfa och trimorfa växter, om individer af samma form paras, gör denna åsigt något tvifvelaktig. Vi måste likväl komma ihåg, att dessa växters ofruktsamhet blifvit förvärfvad för ett visst ändamål och kan skilja sig i ursprung från bastardernas.

Vi måste tillstå, att vi hvarken med denna eller någon annan åsigt kunna begripa flera fakta som röra bastarders ofruktsamhet, till exempel den olika fruktsamheten hos bastarder af ömsesidig kroasering, eller den förökade ofruktsamheten hos de bastarder, hvilka tillfälligtvis och undantagsvis nära likna sina föräldrar. Ej heller vill jag påstå att föregående anmärkningar gå till botten af saken; ty vi hafva ingen förklaring deröfver, att en organism blir ofruktsam då den försättes under onaturliga förhållanden. Allt hvad jag har försökt att visa är, att i två fall, i vissa hänseenden beslägtade, ofruktsamhet är det vanliga resultatet, i det ena fallet deraf att lefnadsförhållandena blifvit ändrade, i det andra fallet deraf att organisation eller konstitution blifvit rubbade genom tvänne organisationers sammanblandning till en.

En likartad öfverensstämmelse förefinnes äfven i en beslägtad, men mycket olika klass af fakta. Det är en gammal och nästan allmän tro, grundad på en mängd bevis, att små förändringar i lefnadsvilkoren äro välgörande för alla lefvande varelser. Vi se derföre landhushållare och trädgårdsmästare ofta flytta sina frön och rotknölar från en mark och ett klimat till ett annat och tillbaka igen. Under djurens tillfrisknande se vi ofta stora fördelar af nästan hvarje förändring i lefnadsvanor. Både ibland växter och djur finnes åter en mängd bevis, att kroasering emellan individer af samma art, som till en viss grad afvika från hvarandra, gifver afkomman styrka och fruktsamhet och att parning emellan nära slägtingar fortsatt under flera generationer nästan alltid medförer svaghet och ofruktsamhet, isynnerhet om de hållas under samma lefnadsförhållanden.

Häraf synes mig, att å ena sidan små förändringar i lifsvilkoren äro välgörande för alla varelser, och å andra sidan, att svaga kroaseringar, det vill säga kroaseringar emellan hannar och honor af samma art, hvilka hafva varierat och blifvit obetydligt afvikande, gifva afkomlingarna styrka och fruktsamhet. Men vi hafva sett, att stora förändringar i lefnadsförhållandena, eller förändringar af särskild beskaffenhet ofta göra organiska varelser till en viss grad ofruktsamma och att större kroaseringar, det är kroaseringar emellan hannar och honor som blifvit betydligt eller specifikt skilda, lemna bastarder som i allmänhet äro till en viss grad ofruktsamma. Jag kan icke föreställa mig att denna parallelism är en tillfällighet eller inbillning. Båda serierna af fakta synas vara sammanbundna af något okändt gemensamt band, som står i nära sammanhang med lifsprincipen; denna princip är, såsom Herbert Spencer anmärkt, att lifvet beror på eller består uti en oupphörlig verkan och återverkan emellan åtskilliga krafter, hvilka såsom öfverallt i naturen sträfva att hålla hvarandra i jemvigt; om denna jemvigt i ringa grad rubbas af någon förändring, vinna tydligen de vitala krafterna i makt.

\section{Dimorfism och trimorfism.}

Detta ämne bör här i korthet behandlas och vi skola finna att det sprider något ljus öfver bastardbildningen. Flera växter, som höra till skilda ordningar, förete två former, hvilka finnas i nästan lika antal och skilja sig i ingenting annat än i sina reproduktionsorganer; den ena formen har en lång pistill med korta ståndare, den andra en kort pistill med långa ståndare och bådas pollenkorn äro af olika storlek. Af de trimorfa växterna finnas tre former, som likaledes skilja sig i pistillernas och ståndarnas längd, i pollenkornens storlek och färg och i några andra hänseenden; och då hvar och en af de tre formerna hafva två slags ståndare, finnas inalles sex slags ståndare och tre slags pistiller. Dessa organers längd stå i sådant förhållande till hvarandra, att hos två af formerna hälften af ståndarna hafva samma höjd som märket i den tredje. Nu har jag visat, och resultatet har bekräftats af andra iakttagare, att för att fullkomlig fruktsamhet skall inträffa det är nödvändigt, att märket af den ena formen befruktas med frömjöl från ståndarna af motsvarande höjd i den andra formen. I dimorfa arter äro derföre två parningar, som kunna kallas legitima, fullt fruktsamma, och två, som kunna kallas illegitima, äro mer eller mindre ofruktsamma. I de trimorfa arterna äro sex parningar legitima eller fullt fruktsamma och tolf äro illegitima eller mer eller mindre ofruktsamma.

Den ofruktsamhet, som kan iakttagas hos flera dimorfa och trimorfa växter vid illegitim befruktning, det är med frömjöl från ståndare, som i höjd icke motsvara pistillen, är mycket olika till graden ända upp till absolut och ytterlig ofruktsamhet; på samma sätt som förhållandet är vid kroasering af skilda arter. Liksom graden af ofruktsamhet i det senare fallet till betydlig del beror på mer eller mindre gynsamma lefnadsförhållanden, så har, såsom jag har funnit, dessa äfven inflytande på ofruktsamheten vid illegitima befruktningar. Det är väl bekant, att om frömjöl af en skild art försättes på en blommas märke och dennas eget frömjöl sedermera äfven efter en ansenlig tids förlopp bringas i beröring med samma märke, det senares verkan har en sådan öfvervigt öfver det förras, att det i allmänhet tillintetgör dess verkningar; förhållandet är detsamma med frömjölet i de olika formerna af samma art, ty det legitima frömjölet har stor öfvervigt öfver det illegitima, om båda bringas i beröring med samma märke. Jag öfvertygade mig härom genom att befrukta flera blommor först illegitimt och tjugufyra timmar derefter legitimt med frömjöl från en egendomligt färgad varietet, och alla de nya plantorna antogo dess färg. Detta visar att det legitima frömjölet som applicerades tjugufyra timmar efteråt hade helt och hållet tillintetgjort eller förhindrat verkan af det förut använda illegitima frömjölet. Vid ömsesidig kroasering emellan två arter är resultatet ofta mycket olika och samma förhållande inträffar med de trimorfa växterna. En form af Lythrum salicaria med medelstor pistill befruktades illegitimt utan synnerlig svårighet med frömjöl från de långa ståndarna i en annan form med kort pistill, men den senare gaf icke ett enda frö vid befruktning med frömjöl från de långa ståndarna i den förra formen.

I alla dessa hänseenden och i andra som kunde tilläggas förhålla sig formerna af samma otvifvelaktiga art vid illegitim sammanparning fullkomligt på samma sätt som två skilda arter vid kroasering. Detta har gifvit mig anledning att under fyra år noggrant undersöka en mängd plantor uppdragna från flera illegitima sammanparningar. Det förnämsta resultatet är att dessa illegitima plantor, såsom de kunna kallas, icke äro fullkomligt fruktsamma. Det är möjligt att af dimorfa arter uppdraga båda och af trimorfa alla tre illegitima formerna, och dessa kunna sammanparas legitimt. Dervid finnes intet synbart skäl hvarföre de icke skulle lemna lika många frön, som deras föräldrar vid legitim befruktning. Men detta är icke fallet, de äro alla ofruktsamma men i olika grad, några äro så ytterligt och ohjelpligt sterila, att de under fyra år icke gåfvo ett enda frö eller ens ett fröhus. Ofruktsamheten hos dessa illegitima växter vid legitim sammanparning emellan dem kan fullkomligt jemföras med bastardernas vid kroasering sinsemellan. Om å andra sidan en bastard kroaseras med endera af de rena stamarterna, förminskas i allmänhet ofruktsamheten betydligt, och detsamma inträffar om en illegitim växt befruktas af en legitim. På samma sätt som bastardernas ofruktsamhet icke alltid motsvarar svårigheten att åstadkomma en första kroasering emellan de två stamarterna, så var ofruktsamheten hos vissa illegitima plantor ovanligt stor, under det den parning ur hvilken de uppkommit alldeles icke var ofruktsam. Hos bastarder, som fås ur samma fröhus, är graden af ofruktsamhet i sig sjelf föränderlig och så är äfven förhållandet med illegitima växter. Många bastarder blomma rikligt och länge, under det andra mera ofruktsamma sätta blott få blommor och svaga, ömkliga dvärgar; fullkomligt likartade fall inträffa med de illegitima ättlingarna af flera dimorfa och trimorfa arter.

Med ett ord, den närmaste identitet i karakter och förhållande råder emellan illegitima växter och bastarder. Det är knappt någon öfverdrift att påstå, att de förra äro bastarder men alstrade inom gränserna af samma art genom olämplig förening af vissa former, under det vanliga bastarder uppkomma af en olämplig förening emellan så kallade skilda arter. Vi hafva också redan sett, att det är den största likhet i alla hänseenden emellan den första illegitima parningen och den första kroaseringen emellan skilda arter. Detta torde måhända blifva fullt tydligt genom ett exempel: vi kunna antaga, att en botanist finner två väl markerade varieteter (och sådant händer) af den form utaf Lythrum salicaria, som har lång pistill och att han genom kroaseringsförsök ämnar bestämma, om de äro skilda arter. Han skall då finna, att de gifva blott omkring en femtedel af det tillbörliga antalet frön och att de i alla ofvan anförda hänseenden förhålla sig som om de vore skilda arter. Men till yttermera visso uppdrager han plantor af sina antagna bastardfrön och han finner, att plantorna äro ömkliga dvärgar och ytterligt ofruktsamma och att de i alla hänseenden förhålla sig som verkliga bastarder. Han torde då påstå, att han verkligen bevisat i enlighet med den allmänna åsigten, att hans två varieteter voro så väl skilda arter som några andra på jorden, men han har dock fullkomligt misstagit sig.

De fakta som nu blifvit framstälda angående dimorfa och trimorfa växter äro vigtiga, först emedan de visa oss, att det fysiologiska beviset, som hemtas från en förminskad fruktsamhet både vid första kroaseringen och hos bastarder, icke är något säkert kriterium på artskilnad; för det andra, emedan vi kunna draga den slutsatsen, att det finnes något okändt föreningsband emellan ofruktsamheten vid en illegitim parning och hos deras illegitima ättlingar och vi kunna utsträcka denna åsigt till kroasering och bastarder; för det tredje, emedan vi finna, och detta tyckes mig vara af synnerlig vigt, att två eller tre former af samma art kunna finnas utan någon olikhet i något hänseende med undantag af deras reproduktionsorganer, hvilka vid vissa sammanparningar äro ofruktsamma. Bland de dimorfa växterna äro blott föreningarna emellan två skilda former fullt fruktsamma och lemna fullt fruktsamma afkomlingar, under det föreningar emellan individer af samma form äro mer eller mindre sterila, så att resultatet är raka motsatsen af hvad som inträffar med skilda arter. Hos dimorfa växter är denna ofruktsamhet fullkomligt oberoende af hvarje afvikelse i struktur eller konstitution, ty den inträffar vid förening emellan individer, som höra icke blott till samma art utan till samma form. Den måste derföre bero på beskaffenheten hos de sexuela elementen, hvilka äro så inrättade, att de hanliga och honliga element som finnas på samma form icke passa för hvarandra, under det de som finnas på olika former ömsesidigt inverka på hvarandra. Af dessa betraktelser synes det sannolikt, att ofruktsamheten vid skilda arters kroasering och hos deras bastarder uteslutande beror på beskaffenheten hos deras sexuela element och icke på någon allmän afvikelse i struktur eller konstitution. Vi komma i sjelfva verket till samma slutsats, om vi betrakta de ömsesidiga kroaseringar, i hvilka hannen af den ena arten icke eller blott med största svårighet kan sammanparas med honan af den andra, under det en omvänd kroasering med största lätthet går för sig; ty denna skilnad i lättheten att åstadkomma ömsesidig kroasering och i fruktsamheten hos afkomlingarna måste bero på, att antingen det hanliga eller det honliga elementet hos den första arten blifvit förändradt i sitt förhållande till den andra artens i högre grad än tvärtom. Den utmärkte iakttagaren Gärtner kom likaledes till samma slutsats, nämligen att arternas ofruktsamhet vid kroasering beror på olikheter blott inom det reproduktiva organsystemet.



\section[Fruktsamhet vid variateters kroasering]{Fruktsamhet vid varieteters kroasering och hos
deras mestiser.}

Man kan framhålla såsom ett öfverväldigande argument, att det måste vara någon väsentlig skilnad emellan arter och varieteter, då de senare, huru mycket de än skilja sig från hvarandra i yttre utseende, kroaseras med fullkomlig lätthet och lemna fullt fruktsamma afkomlingar. Med få undantag, som straxt skola anföras, är detta regel. Men ämnet är fullt af svårigheter, ty så snart två naturliga former, som hittills ansetts som varieteter, befinnas vara i någon mån ofruktsamma vid kroasering, upptagas de med ens af de flesta naturforskare såsom arter. Den blåa och den röda arfven (Anagallis), hvilka de flesta botanister betrakta såsom varieteter, äro enligt Gärtner icke fullkomligt fruktsamma vid kroasering och på den grund upptager han dem såsom bestämda arter. Om vi på detta sätt röra oss i cirkel, måste vi helt säkert medgifva alla naturliga varieteters fruktsamhet.

Om vi nu vända oss till de varieteter som uppkommit eller antagas hafva uppkommit under domesticering, se vi oss ännu invecklade i tvifvel. Ty om det till exempel står fast, att den tyska spetsen (Canis familiaris pomeranus) lättare kroaseras med räfven än andra hundar, eller att vissa Sydamerikanska infödda hushundar ogerna para sig med europeiska hundar, torde den förklaringen ligga närmast till hands, och sannolikt är den riktig, att dessa hundar härstamma från ursprungligen skilda arter. Den fullkomliga fruktsamheten hos så många domesticerade varieteter, som till utseende äro vidt skilda från hvarandra, till exempel dufvor, eller kål, är ickedestomindre ett märkvärdigt förhållande, isynnerhet om vi besinna, huru många arter finnas, hvilka vid kroasering äro ytterst ofruktsamma, ehuru de synnerligen likna hvarandra. Flera betraktelser göra likväl fruktsamheten hos domesticerade varieteter mindre besynnerlig. I första rummet böra vi observera, att graden af yttre olikhet emellan två arter icke är något säkert tecken till inbördes ofruktsamhet, så att likartade skiljaktigheter emellan varieteter icke heller bevisa något; det är nästan säkert, att hos arterna orsaken ligger uteslutande i deras sexuela konstitution. De omständigheter under hvilka domesticerade djur och växter lefvat hafva haft så liten förmåga att modifiera deras reproduktionsorganer på ett sätt som för till inbördes ofruktsamhet, att vi hafva goda skäl att antaga Pallas’ rakt motsatta åsigt, att sådana förhållanden i allmänhet undanröja denna sträfvan, så att de domesticerade ättlingarna af arter, hvilka i naturtillståndet varit till en viss grad sterila vid kroasering, numera blifvit fullt fruktsamma sinsemellan. Hvad växter beträffar, så har odlingen alldeles icke åstadkommit någon ofruktsamhet emellan skilda arter, utan tvärtom hafva i flera redan antydda fall vissa växter blifvit förändrade i motsatt riktning, så att de blifvit oförmögna till sjelfbefruktning men ännu bibehållit förmågan att befruktas af och sjelfva befrukta andra arter. Om denna Pallas’ lära antages om ofruktsamhetens eliminering genom länge fortsatt domesticering, och den kan svårligen vederläggas, blir det i högsta grad osannolikt, att liknande omständigheter skulle både förorsaka och motarbeta samma tendens; ehuru i vissa fall bland arter som hafva en egendomlig konstitution ofruktsamhet tillfälligtvis kan blifva följden. På detta sätt kunna vi såsom jag tror förklara, hvarföre bland våra husdjur inga varieteter uppkommit som äro sinsemellan sterila, och hvarföre bland växter blott ett ringa antal sådana fall, som straxt skola anföras, blifvit iakttagna.

Den verkliga svårigheten i ifrågavarande ämne är icke, såsom mig synes, den att domesticerade varieteter icke blifvit sinsemellan ofruktsamma vid kroasering, utan att detta så allmänt händt med naturliga varieteter, så snart de blifvit modifierade i tillräcklig grad att kunna upptagas såsom permanenta arter. Vi äro långt ifrån att noga känna orsakerna dertill, dock bör detta icke förvåna oss, då vi se huru djupt okunniga vi äro med afseende på den normala och abnorma verkan af reproduktionsorganerna. Men vi kunna se, att arterna under sin kamp för tillvaron med talrika medtäflare måste lefvat under mera likformiga förhållanden under långa perioder än våra domesticerade varieteter, och detta kan väl göra en stor skilnad i resultatet. Ty vi känna huru ofta vilda djur och växter blifva ofruktsamma, om de tagas ifrån sina naturliga lefnadsförhållanden och bringas i fångenskap, och reproduktionsfunktionerna hos organiska varelser, som alltid lefvat och blifvit långsamt modifierade under naturliga förhållanden, torde sannolikt på samma sätt vara särdeles känsliga för inflytandet af en onaturlig kroasering. Domesticerade alster å andra sidan, hvilka ursprungligen icke voro så särdeles känsliga för förändringar i deras lefnadsförhållanden och hvilka nu i allmänhet med oförminskad fruktsamhet kunna uthärda upprepade förändringar i sina förhållanden, böra frambringa varieteter som blott obetydligt skadas i sina reproduktionsorganer genom kroasering med andra varieteter, som uppkommit på samma sätt.

Jag har hittills talat såsom om varieteter af samma art vore oföränderligen fruktsamma vid kroasering. Men det är omöjligt att förneka tillvaron af en viss grad af ofruktsamhet i de få följande fallen, hvilka jag i korthet vill anföra. Bevisen äro åtminstone lika så goda som de skäl vi hafva att tro på ofruktsamheten hos en mängd arter, och äro dessutom hemtade från mina motståndares vittnesbörd, hvilka i alla andra fall betrakta fruktsamhet och ofruktsamhet såsom säkra kriterier på artskilnad. Gärtner lät under flera år ett dvärgartad slag af mais med gula frön och en stor varietet med röda frön växa nära hvarandra i sin trädgård, och ehuru dessa växter hafva skilda kön, kroaserades de aldrig af sig sjelfva. Derefter befruktade han tretton blomax på den ena med frömjöl från den andra, men blott ett enda hufvud satte någon frukt och detta enda hufvud hade alstrat blott fem frön. Behandlingssättet har i detta fall icke kunnat vara skadligt, enär växterna hafva skilda kön. Ingen har som jag tror misstänkt, att dessa maisvarieteter äro skilda arter, och det är af vigt att anmärka, att de uppdragna hybrida växterna voro fullkomligt fruktsamma, så att icke ens Gärtner vågade anse de båda varieteterna såsom specifikt skilda.

Girou de Buzareingues kroaserade tre gurkvarieteter, hvilka liksom maisen hafva skilda kön, och han försäkrar att deras ömsesidiga befruktning är så mycket svårare ju större deras skiljaktigheter äro. Jag vet icke, huru stor lit man kan sätta till dessa experiment, men de former, hvarmed försöken anstäldes, upptagas af Sageret såsom varieteter, som grundar sin klassifikation hufvudsakligen på ofruktsamheten, och Naudin har kommit till samma slutsats.

Följande fall är mera anmärkningsvärdt och synes först alldeles otroligt, men det är resultatet af en förvånande mängd försök, som under många år gjorts med nio arter af Verbascum af en så god iakttagare som Gärtner, motståndare till mina åsigter, nämligen att de gula och hvita varieteterna vid kroasering lemna mindre frön, än de lika färgade varieteterna af samma art. Han försäkrar vidare, att om gula och hvita varieteter af en art kroaseras med gula och hvita varieteter af en annan skild art, flera frön uppkomma vid kroasering emellan likfärgade varieteter än emellan olikfärgade. Mr Scott har också arbetat med Verbascumarter och varieteter och ehuru han icke var i stånd att bestyrka Gärtners resultat vid kroasering af skilda arter, fann han att de olika färgade varieteterna af samma art lemnade mindre antal frön än de lika färgade i en proportion af 86 till 100. Dessa varieteter skilja sig dock i intet hänseende från hvarandra mer än i blommornas färg och den ena varieteten kan ofta uppkomma ur den andras frön.

Kölreuter, hvilkens noggranhet blifvit bekräftad af alla följande observatörer, har visat det märkvärdiga faktum, att en särskild varietet af den vanliga tobaken var mera fruktsam än andra varieteter vid kroasering med en vidt skild art. Han arbetade med fem former som man i allmänhet anser för varieteter, hvilket han äfven bevisade med det strängaste prof, ömsesidig kroasering, och alla mestiserna voro fullkomligt fruktsamma. Men en af dessa varieteter då den begagnades antingen såsom fader eller moder och kroaserades med Nicotiana glutinosa, lemnade alltid bastarder som icke voro så ofruktsamma som de bastarder, hvilka erhöllos genom kroasering af de fyra andra varieteterna med N. glutinosa. Det reproduktiva systemet hos denna varietet måste derföre blifvit på något sätt och i någon grad modifieradt.

Med dessa fakta för ögonen kan ingen påstå, att varieteter vid kroasering oföränderligen äro fullkomligt fruktsamma. Då svårigheterna äro så stora att bevisa varieteters ofruktsamhet i naturtillståndet, ty om en antagen varietet bevisas vara till en viss grad ofruktsam, upptages den nästan allmänt såsom art; — då menniskan hos sina domesticerade varieteter fäster afseende blott vid de yttre karaktererna och då sådana varieteter icke under en lång period lefvat under likformiga lefnadsförhållanden, så kunna vi häraf draga den slutsatsen, att fruktsamhet icke utgör en väsentlig skilnad emellan varieteter och arter vid kroasering. Den så allmänt förekommande ofruktsamheten vid arters kroasering kan helt säkert betraktas icke såsom en särskild förvärfvad eller förlänad egenskap, utan sammanfallande med okända förändringar i deras sexuela element.



\section[Bastarder och mestiser]{Jemförelse emellan bastarder och mestiser utan
afseende på deras fruktsamhet.}

Oberoende af fruktsamheten kunna bastarder och mestiser jemföras med hvarandra i flera andra hänseenden. Gärtner, hvars ifriga önskan var att kunna draga en bestämd gränslinie emellan arter och varieteter, kunde finna blott få och såsom mig synes fullkomligt ovigtiga olikheter emellan de så kallade bastarderna af arter och de så kallade mestiserna af varieteter. Och å andra sidan visa de i många vigtiga hänseenden stor öfverensstämmelse.

Jag skall här behandla detta ämne i största korthet. Den vigtigaste skilnaden är att i första generationen mestiser äro mera föränderliga än bastarder, men Gärtner medgifver, att bastarder af arter som länge blifvit odlade ofta äro mycket föränderliga i den första generationen, och jag har sjelf sett slående exempel derpå. Gärtner medgifver vidare att bastarder emellan mycket nära beslägtade arter äro mera föränderliga än bastarder som härstamma från skilda arter, och detta visar att olikheten i variation gradvis försvinner. Om mestiser och de mera fruktsamma bastarderna fortplantas i flera generationer, är en ytterlig grad af föränderlighet i båda fallen väl känd, men några få exempel kunna gifvas på både bastarder och mestiser, som länge bibehållit samma karakter. Föränderligheten i de följande generationerna är dock måhända större hos mestiser än hos bastarder.

Denna större föränderlighet hos mestiser synes alls icke öfverraskande. Ty föräldrarna till mestiserna äro varieteter, och oftast domesticerade varieteter (mycket få försök hafva blifvit gjorda med naturliga varieteter) och detta innefattar, att der nyligen funnits föränderlighet, som ofta fortfar och förökar den föränderlighet som är en följd af kroaseringen. Den ringa föränderligheten hos bastarderna i första generationen i motsats till de följande generationerna är ett egendomligt förhållande, som förtjenar att uppmärksammas, ty det understöder den åsigt jag har bildat mig om en af orsakerna till vanlig föränderlighet, nämligen att det reproduktiva systemet, som är ytterligt känsligt för förändringar i lefnadsomständigheterna, under dessa förhållanden icke kan fullgöra sin egentliga funktion, att alstra ättlingar i alla hänseenden identiska med stamformen. Bastarder i första generationen härstamma nu från arter (med undantag af de länge odlade) hvilkas reproduktionsorganer icke på något sätt rubbats, och de äro derföre icke föränderliga, men bastarderna sjelfva hafva sina reproduktionsorganer betydligt angripna och deras afkomlingar äro i hög grad föränderliga.

Men låt oss återgå till vår jemförelse emellan mestiser och bastarder. Gärtner påstår, att mestiser mera än bastarder sträfva att återgå till endera stamformen, men om detta är sant, är det säkerligen blott en skilnad i grad. Gärtner framhåller dessutom med eftertryck, att bastarderna af länge odlade växter äro mera benägna för återgång än bastarder af arter i deras naturtillstånd, och detta förklarar sannolikt den egendomliga skilnaden i de resultat till hvilka olika forskare kommit. Max Wichura betviflar, att bastarder någonsin återgå till sin stamform och han arbetade med icke odlade arter af pil, under det Naudin å andra sidan i de starkaste uttryck framhåller den hos bastarder nästan allmänna återgången, och han arbetade hufvudsakligen med odlade växter. Gärtner påstår vidare, att om två med hvarandra närbeslägtade arter kroaseras med en tredje, bastarderna äro vidt skilda från hvarandra, hvaremot om två väl skilda varieteter af en art kroaseras med en annan art, bastarderna icke äro mycket olika. Men så vidt jag kan finna, är detta påstående grundadt på ett enda försök, och synes rakt stridande emot resultaten af flera försök som Kölreuter anstält.

Dessa äro de enda, vigtiga skilnader som Gärtner lyckats uppställa emellan bastarder och mestiser ibland växter. Graderna och arten af likhet med de respektive föräldrarna hos mestiser och bastarder, särskildt hos bastarder af nära beslägtade arter, följa enligt Gärtner samma lagar. Om två arter kroaseras, har den ena stundom så stor öfvervigt, att bastarden blir mest lik honom, och så tror jag förhållandet äfven är med växtvarieteter. Bland djuren har den ena varieteten helt säkert denna öfvervigt öfver en annan varietet. Hybrida växter som uppkommit genom ömsesidig kroasering likna hvarandra i allmänhet betydligt och så är äfven förhållandet med mestiser af en ömsesidig kroasering. Både bastarder och mestiser kunna återföras till endera af de rena stamformerna genom upprepade kroaseringar i följande generationer med någon af dem.

Alla dessa anmärkningar äro tydligen användbara på djur, men ämnet är här mycket inveckladt till en del i följd af tillvaron af sekundära sexualkarakterer, men i synnerhet i följd af den vanligen hos ett kön öfvervägande förmågan att på afkomman inprägla sin bild, både vid kroasering emellan arter och varieteter. Jag tror till exempel, att de författare hafva rätt som påstå, att åsnan har en sådan öfvervigt öfver hästen, så att både mulan och mulåsnan mera likna åsnan än hästen, men att denna öfvervigt är starkare hos åsnehannen än honan, så att mulan som uppkommer af en åsnehanne och sto, mera liknar åsnan än mulåsnan, som uppkommer af åsnehona och hingst.

Mycken vigt hafva några författare lagt på det faktum, att det blott är mestiserna, som i karakter icke stå midt emellan sina föräldrar utan mera likna en af dem, men detta händer stundom äfven bastarderna, ehuru, jag medgifver det, vida mindre ofta än mestiserna. Om man betraktar de fall som jag har samlat af genom kroasering bildade djur, hvilka visade största likhet med en af föräldrarna, synes likheten hufvudsakligen inskränkt till karakterer, som till sin beskaffenhet äro nästan monströsa och hafva uppstått plötsligt — såsom albinism, melanism, brist på svans eller horn, eller öfverloppstår, — och icke hafva något sammanhang med karakterer som blifvit småningom förvärfvade genom urval. Plötsliga återgångar till den fulländade karakter som endera af föräldrarna egde böra följaktligen med större sannolikhet inträffa med mestiserna hvilka härstamma från varieteter, som ofta plötsligt bildats och voro halft monströsa till karakter, under det bastarderna uppkommit af långsamt och naturligt bildade arter. På det hela taget öfverensstämmer jag helt och hållet med dr Prosper Lucas, hvilken efter att hafva pröfvat ett ofantligt antal fall hemtade från djuren kommer till den slutsatsen, att lagarna för barnets likhet med föräldrarna äro desamma, vare sig föräldrarna skilja sig mycket eller litet från hvarandra, om de tillhöra samma varieteter eller olika varieteter eller olika arter.

Oberoende af frågan om fruktsamhet eller ofruktsamhet, synes i alla hänseenden finnas en allmän och nära likhet emellan afkomlingarna af kroaserade arter och kroaserade varieteter. Om vi betrakta arterna såsom särskildt skapade och varieteterna såsom uppkomna genom sekundära lagar, skulle denna likhet vara ett oförklarligt faktum, men det öfverensstämmer fullkomligt med den åsigten, att det icke finnes någon väsentlig skilnad emellan arter och varieteter.



\section{Sammanfattning.}

Första kroaseringarna emellan former tillräckligt skilda att upptagas såsom skilda arter äro oftast likasom dessas bastarder ofruktsamma, dock icke alltid. Ofruktsamheten är af alla grader och är ofta så obetydlig, att de mest noggranna experimentatörer kommit till diametralt motsatta åsigter vid formernas bestämmande enligt detta prof. Ofruktsamheten är i sig sjelf föränderlig hos individer af samma art och är ytterst mottaglig för verkan af gynsamma och ogynsamma förhållanden. Graden af ofruktsamhet följer icke strängt den systematiska slägtskapen utan styres af flera egendomliga och invecklade lagar. Den är i allmänhet olika och stundom betydligt olika vid ömsesidig kroasering emellan samma två arter, och den är icke alltid till graden lika vid den första kroaseringen och hos de derigenom uppkomna bastarderna.

På samma sätt som vid träds ympning en arts eller en varietets förmåga att slå an på en annan sammanfaller med olikheter, vanligen af okänd natur, i deras vegetativa system, så sammanfaller äfven vid kroasering den större eller mindre lättheten för en art att förena sig med en annan med okända afvikelser i deras reproduktionssystem. Den tron, att arter blifvit särskildt begåfvade med olika grader af fruktsamhet för att förekomma deras kroasering och sammanblandning i naturen, har icke mera skäl för sig än den tron, att träden blifvit särskildt begåfvade med olika och något analoga grader af svårighet vid hopympning för att hindra dem i våra skogar att växa tillsammans.

Ofruktsamheten vid första kroaseringen och hos hybriderna har icke så vidt vi kunna döma blifvit förvärfvad genom naturligt urval. Vid den första kroaseringen synes den bero på flera omständigheter, i några fall till en väsentlig del på embryots tidiga död. Hos bastarderna beror den måhända på att hela deras organisation blifvit bragt i oordning derigenom att den är sammansatt af två skilda former, och den är nära beslägtad med den ofruktsamhet som ofta uppkommer hos rena arter, då de utsättas för onaturliga lefnadsförhållanden. Denna åsigt understödes af en jemförelse af annat slag, nämligen för det första, att kroasering af former, som äro blott obetydligt olika, förökar kraften och fruktsamheten hos afkomman, under det parning emellan slägtingar är skadlig; och för det andra att små förändringar i lefnadsförhållandena tydligen föröka alla organiska varelsers kraft och fruktsamhet, under det stora förändringar äro skadliga. Men de fakta som anförts om ofruktsamheten vid illegitim parning af dimorfa och trimorfa växter och hos deras illegitima afkomlingar göra det sannolikt, att något okändt föreningsband i alla fallen sammanbinder graden af fruktsamhet vid den första föreningen med fruktsamheten hos afkomlingarna. Betraktandet af dessa fakta rörande dimorfism jemte resultaten af ömsesidig kroasering leder tydligen till den slutsats, att den förnämsta orsaken till ofruktsamhet är begränsad till olikheter i de sexuela elementen. Men hvarföre hos arterna de sexuela elementen skulle så allmänt blifvit mer eller mindre modifierade, ända till inbördes ofruktsamhet, det känna vi icke.

Det är ingenting öfverraskande, att svårigheten att kroasera två arter och ofruktsamheten hos deras hybrida afkomlingar i de flesta fall motsvara hvarandra, äfven om de härröra af skilda orsaker, ty båda bero på graden af olikhet emellan de kroaserade arterna. Ej heller är det någonting öfverraskande, att lättheten att åstadkomma en första kroasering och fruktsamheten hos de deraf uppkomna bastarderna och förmågan att ympas — ehuru denna senare förmåga tydligen beror på helt olika omständigheter — alla till en viss grad löpa parallelt med den systematiska slägtskapen af de former som äro valda för experimentet, ty systematisk affinitet innefattar likheter af alla slag.

Första kroaseringen emellan två former som antagas för varieteter eller äro tillräckligt lika för att anses såsom sådana äfvensom deras mestiser äro visserligen i allmänhet fruktsamma, dock icke såsom ofta påstås ovilkorligen. Denna nästan allmänna och fullständiga fruktsamhet, är icke heller någonting öfverraskande om vi ihågkomma, huru gerna vi argumentera i cirkel med afseende på varieteterna i naturtillståndet, och om vi komma ihåg, att större antalet varieteter hafva bildats under domesticering genom urval af blott yttre skiljaktigheter, och att de icke under längre tid lefvat under likformiga lefnadsförhållanden. Vi måste också särdeles lägga på minnet, att länge fortsatt domesticering sträfvar att bortskaffa ofruktsamheten och kan derföre sannolikt icke förläna denna egenskap. Oberoende af frågan om fruktsamhet finnes i alla andra hänseenden den närmaste likhet emellan bastarder och mestiser, i deras föränderlighet, och i deras egenskap att antaga karakterer från båda föräldrarna. Slutligen ehuru vi äro alldeles okunniga om den egentliga orsaken till ofruktsamhet vid första kroaseringen och hos bastarder, synas mig de fakta som anförts i detta kapitel icke strida emot den åsigten, att varieteter och arter icke äro väsentligen skilda.


