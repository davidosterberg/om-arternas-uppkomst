%Baserat på
%http://sv.wikisource.org/w/index.php?title=Om_arternas_uppkomst_genom_naturligt_urval_eller_de_b%C3%A4st_utrustade_rasernas_best%C3%A5nd_i_kampen_f%C3%B6r_tillvaron/Kapitel_2&oldid=99293

%ANDRA KAPITLET.





\chapter[Naturtillståndet]{Arternas förändring i naturtillståndet.}

{\it
Föränderlighet. — Individuela olikheter. — Tvifvelaktiga arter. — De allmänna och mest utbredda arterna variera mest. — Arterna af de större slägtena i hvarje land variera oftare än de mindre slägtenas arter. — Många arter af de större slägtena likna varieteter deri, att de äro nära beslägtade med hvarandra men på olika sätt, och att de bebo ett mycket inskränkt område.
}


\section{Föränderlighet.}

Innan vi på de organiska varelserna i naturtillståndet tillämpa de grundsatser, hvilka vi behandlat i föregående kapitel, måste vi först undersöka i huru hög grad de äro underkastade förändringar i naturtillståndet. För att fullständigt behandla detta ämne fordrades att uppsätta en lång lista af torra fakta, hvilket jag dock spar till ett kommande arbete. Jag vill icke heller uppehålla mig vid de olika definitioner, som gifvits på begreppet art, ty ingen af dem har ännu tillfredsställt alla naturforskare och vanligen innesluter definitionen ett obekant element, en särskild skapelseakt. Uttrycket varietet är lika svårt att definiera, gemensamt ursprung är emellertid i allmänhet deri inbegripet ehuru sällan möjligt att uppvisa. Monstrositet är efter min uppfattning en ansenlig afvikelse i bildning, som är antingen direkt skadlig eller åtminstone icke nyttig för arten. Några författare begagna äfven uttrycket variation såsom term för att beteckna afvikelser som framkallats af de yttre lefnadsvilkorens direkta inflytande och variationer i denna mening anses för icke ärftliga. Hvem kan dock med visshet säga, att den dvärgartade beskaffenheten af några snäckor i Östersjöns flodmynningar eller af växterna på alpernas höjder eller ett djurs tätare pels i de högre breddgraderna, att icke dessa afvikelser i vissa fall åtminstone under några generationer äro ärftliga, och i detta fall borde man i min tanke kalla formen varietet.

Det kan måhända vara tvifvelaktigt om plötsliga och stora bildningsafvikelser, sådana som vi ofta se hos våra husdjur och kulturväxter, isynnerhet de sednare, i naturtillståndet kunna fortplanta sig. Hvarje del af en organism står i ett så underbart samband med de invecklade lefnadsförhållandena, att det synes lika osannolikt, att någon del på en gång uppträdt i sin fullkomlighet, som att en menniska skulle hafva omedelbart uppfunnit en sammansatt maskin i dess fulländade form. Under domesticering uppträda ofta monstrositeter, som kunna jemföras med normala bildningar. Så födas ofta svin med ett slags snabel liknande tapirens eller elefantens. Om nu någon vild art af svinslägtet hade en sådan snabel, så kunde man tro, att denna art hade plötsligt uppträdt såsom monstrositet. Men trots ifrigt sökande har det icke lyckats mig att finna i naturtillståndet goda exempel på arter, hvilkas variationer motsvara monstrositeterna hos deras slägtingar i tama tillståndet. Uppträda monströsa former af detta slag i naturtillståndet och fortplantas de (hvilket icke alltid är fallet), så måste de kroaseras med den vanliga formen, då de blott förekomma enstaka, och fortplantas derföre i modifieradt tillstånd. Blifva de efter en sådan kroasering beständiga, så får deras bestånd tillskrifvas den omständigheten, att modifikationen på något sätt är för djuret nyttig under de förhandenvarande lefnadsvilkoren, så att i detta fall ett naturligt urval kommer i fråga.



\section{Individuela olikheter.}

De många obetydliga olikheter, som ofta förekomma emellan afkomlingar af likartade föräldrar eller emellan sådana individer, för hvilka man kan antaga ett likartadt ursprung, kan man kalla individuela olikheter, emedan de ofta förekomma hos individer af samma art, som på ett inskränkt område bo nära tillsammans. Ingen tror väl, att alla individer äro noga bildade efter samma modell. Dessa individuela olikheter äro af stor vigt för oss, emedan de ofta gå i arf, såsom hvar och en säkerligen haft tillfälle att iakttaga, och härigenom gifva de tillfälle till ett naturligt accumulativt urval, på samma sätt som menniskan använder sin accumulativa valförmåga på sina domesticerade raser. Dessa individuela olikheter beträffa dock i allmänhet endast delar som i naturforskarens ögon äro oväsentliga, men jag kunde dock äfven af en lång lista af fakta visa, att hos individer af en art äfven sådana delar variera, som både från klassifikatorisk och fysiologisk synpunkt måste anses för väsentliga. Jag är öfvertygad, att den mest erfarna naturforskare skulle förvånas öfver den mängd af möjliga afvikelser till och med i vigtiga delar, som han kunde samla, såsom jag har gjort, under årens lopp från goda källor. Men man måste komma ihåg, att systematikern icke med glädje upptäcker föränderligheter i vigtiga karakterer, och att det icke gifves mången, som finner nöje i att sorgfälligt undersöka inre organer och jemföra dem med hvarandra hos många exemplar af samma art. Så skulle jag aldrig hafva väntat, att nervstammarna från det stora centralganglion hos en insekt kunde väsentligt variera i sina förgreningar, och dock har sir John Lubbock för kort tid sedan hos slägtet Coccus visat variationer af dessa nerver, som påminna om en trädstams oregelbundna förgrening. Samma utmärkta naturforskare har äfven visat, att musklerna hos vissa insektlarver äro långt ifrån likformiga. Författarna röra sig ofta i en cirkel med sina slutsatser, då de påstå, att vigtiga organer aldrig variera, ty i praxis räkna dessa författare till de vigtigare organerna just sådana, som icke variera, och under denna förutsättning kan icke något exempel anföras på ett varierande vigtigt organ, men från en annan synpunkt torde man kunna uppräkna flera sådana exempel.

I sammanhang med dessa individuela olikheter står en annan sak, som synes mig svår att förklara, nämligen de slägten, som stundom kallas proteiska eller polymorfa, emedan deras arter visa en hög grad af föränderlighet, så att knappt två naturforskare kunna vara ense om, hvilka former böra betraktas som arter eller varieteter. Bland växterna vill jag anföra såsom exempel slägtena Rubus, Rosa, Hieracium, bland foglarna Brushanen (Machetes pugnax) och dessutom flera insekter och brachiopoder. Några arter i dessa polymorfa slägten hafva fasta och bestämda karakterer. De slägten som i en trakt visa sig polymorfa äro det äfven i andra trakter med få undantag och att döma efter brachiopoderna hafva de äfven varit det under förgångna tider. Dessa omständigheter äro så till vida i stånd att åstadkomma förvirring, som detta slags föränderlighet är oberoende af lefnadsvilkoren. Jag är böjd att antaga, att vi i dessa polymorfa slägten träffa på afvikelser blott i sådana delar af deras bildning, som för arten äro hvarken nyttiga eller skadliga och följaktligen icke komma med i räkningen vid det naturliga urvalet.

Individer af samma art erbjuda ofta olikheter i byggnad, som icke direkt sammanhänga med föränderligheten, såsom olikheter emellan de båda könen af flera djur, de två eller tre formerna af sterila honor eller arbetare bland insekterna, och i många af de lägre djurens outvecklade eller larvtillstånd. Det gifves dessutom äfven fall, som lätt förvexlas med föränderlighet och som ofta äfven blifvit förvexlade dermed, ehuru de äro vidt skilda derifrån, nämligen fall af dimorfism och trimorfism. Jag hänvisar till de tre skilda former, hvilka vissa djur och vissa hermafroditiska växter förete. Wallace, som nyligen riktat uppmärksamheten på detta ämne, har visat, att några fjärilshonor på Malayiska arkipelagen regelbundet uppträda under två eller tre tydligt skilda former, hvilka icke sammanbindas af några mellan dem stående varieteter. Många skinnbaggars uppträdande i bevingadt eller vinglöst tillstånd bör sannolikt betraktas såsom ett fall af dimorfism och icke af blott föränderlighet. Fritz Müller har nyligen beskrifvit analoga, ehuru ännu besynnerligare exempel härpå från hannarna af vissa brasilianska krustaceer; hannen af en Tanais förekommer regelbundet under två vidt från hvarandra skilda former utan öfvergångsstadier, den ena har mycket starkare och olika formade klor för att gripa honan, den andra har i stället sina känseltrådar rikligare försedda med lukthår för att hafva mera utsigter att finna sin hona. Hannarna af en annan krustace, en Orchestia-art, förekomma under två skilda former, hvilkas klor till skapnad afvika från hvarandra mer än klorna hos de flesta arter af samma slägte. Hvad växterna beträffar har jag nyligen visat, att arter af flera vidt skilda ordningar förete två eller tre former, hvilka skilja sig i flera vigtiga punkter, såsom pollenkornens storlek och färg; och ehuru alla dessa former äro tvåkönade, afvika de i alstringsförmåga från hvarandra i så hög grad, att de måste befrukta hvarandra ömsesidigt för att vinna full fruktsamhet. Men ehuru nu de få dimorfa och trimorfa djur och växtformer, som hittills äro undersökta, icke sammanhänga genom några öfvergångsformer, så är det dock sannolikt, att sådana finnas i andra fall. Wallace observerade en fjäril, som på en och samma ö visade en lång serie af varieteter förenade genom mellanformer, och de yttersta lederna i denna serie liknade de båda formerna af en närbeslägtad dimorf art, som förekom på en annan del af Malayiska arkipelagen. Detsamma gäller om myror; de vanliga arbetsmyrorna förekomma vanligen i mycket olika former, men i många fall förenas dessa, såsom vi skola se, genom flera varieteter. Det tyckes likväl vara ett högst märkvärdigt förhållande, att samma fjärilhona är i stånd att på en gång framföda tre feminina och en masculin form, att en krustacéhanne skulle afla en masculin och en feminin form, alla vidt skilda från hvarandra, och att en tvåkönad växt skulle i samma frökapsel innesluta tre olika tvåkönade former, och dock äro dessa fall blott de tydligaste bevis för den allmänna lag, att hvarje hona framföder hannar och honor, som i några fall äro på ett förvånande sätt olika hvarandra.



\section{Tvifvelaktiga arter.}

De former som visserligen i ansenlig grad hafva karakteren af arter, men som äro så lika andra former eller så nära förbundna med dem genom mellanstående former, att naturforskarna ogerna upptaga dem såsom skilda arter, äro i flera hänseenden de vigtigaste för oss. Vi hafva allt skäl att tro, att många af dessa tvifvelaktiga och beslägtade former under en längre tid hafva bibehållit sina karakterer i sin hemort, tillräckligt länge för att betraktas såsom goda och äkta arter. Då en naturforskare kan sammanbinda tvänne former genom andra, som hafva intermediära karakterer, behandlar han i praxis den ena såsom en varietet af den andra och betraktar den ovanligare eller den först beskrifna såsom art. Vid afgörandet af en sådan fråga, om en form bör upptagas såsom varietet af en annan, möta ofta stora svårigheter, hvilka jag här icke vill uppräkna, äfven om båda äro förenade genom mellanformer, och det vanliga antagandet att dessa mellanformer äro bastarder undanröjer icke alltid dessa svårigheter. I många fall blir dock en sådan form betraktad som varietet af en annan, icke emedan mellanformerna verkligen blifvit funna, utan emedan analogien föranleder iakttagaren att antaga, antingen att dessa mellanformer någonstädes äro att finna, eller att de hafva fordom existerat, och här är stort utrymme för tvifvel och gissningar.

Vid bedömandet af en sådan forms egenskap af art eller varietet hafva vi endast en ledning, naturforskarnas åsigt, om den är grundad på sundt omdöme och rik erfarenhet. I många fall få vi dock taga vår tillflykt till majoriteten, ty få tydliga och kända varieteter gifvas väl, hvilka icke af en eller annan kompetent domare blifvit upptagna som arter.

Att varieteter af sådan tvifvelaktig natur äro långt ifrån sällsynta, kan ej bestridas. Om man jemför de af olika botanister uppstälda flororna för Storbritannien, Frankrike och Förenta Staterna, finner man snart, att ett stort antal former hafva blifvit upptagna af den ena såsom verkliga arter, under det den andra betraktar dem såsom blott varieteter. H. C. Watson, till hvilken jag står i förbindelse för ett välvilligt biträde i flera fall, har uppgifvit för mig 182 engelska växter, som vanligen anses som varieteter men dock af några botanister förklarats för arter, men härvid har han förbigått många obetydliga varieteter, hvilka icke desto mindre af en och annan upptagits såsom arter, och han har helt och hållet åsidosatt flera polymorfa slägten. Bland slägten, som innefatta de mest polymorfa former, räknar Babington 251 arter, Bentham blott 112, en skilnad af 139 tvifvelaktiga former! Bland djuren, som förena sig blott för hvarje parning och ofta byta vistelseort, kunna sådana former, som af zoologer räknas än för arter än för varieteter, sällan förekomma i samma trakt, men på skilda områden äro de ej ovanliga. Huru många af dessa foglar och insekter i Nordamerika och Europa, hvilka så obetydligt afvika från hvarandra, hafva icke af utmärkta vetenskapsmän ansetts för otvifvelaktiga arter och af andra för varieteter eller så kallade geografiska raser! I flera värderika uppsatser, som Wallace nyligen publicerat öfver åtskilliga djurgrupper, isynnerhet öfver Malayiska arkipelagens fjärilar, antyder han en indelning i variabla former, lokala former, geografiska raser eller underarter och äkta representerande arter. De variabla formerna visa betydliga olikheter inom området af samma ö; de lokala formerna äro på hvarje särskild ö temligen konstanta och bestämda, men jemför man alla sådana från de olika öarna, så blifva skilnaderna så små, så talrika och gradvisa, att det blir omöjligt att bestämma eller beskrifva många af dessa former, ehuru de extrema formerna äro tillräckligt skarpt bestämda. De geografiska raserna eller underarterna äro fullständigt fixa och isolerade landformer, men då de icke afvika från hvarandra genom starkt markerade och vigtiga kännetecken, ”så kan intet bevis, utan blott den individuela åsigten bestämma, hvad som bör kallas för art eller varietet”. Representerande arter intaga i naturens hushållning på hvarje ö samma plats som de lokala formerna och underarterna, men då olikheterna emellan dem ehuru ej bestämda äro större än emellan de lokala formerna och underarterna, så tagas de af naturforskarna för goda arter. Icke destomindre saknas ett bestämdt kriterium, efter hvilket man kan särskilja variabla former, lokala former, underarter och representerande arter.

Då jag för många år sedan jemförde foglar från de särskilda Galapagos-öarna både med hvarandra och med foglar från Amerikas fastland, blef jag mycket öfverraskad af osäkerheten och den fullkomliga godtyckligheten i bestämmandet af arter och varieteter. På öarna i den lilla Madeiragruppen finnas många insekter, hvilka i Wollastons arbete betecknas som varieteter, men med all säkerhet skulle af mången entomolog förklaras för skilda arter. Äfven Irland har några få djur, som några zoologer kalla arter, ehuru de i allmänhet anses för varieteter. Några erfarna ornitologer betrakta vår ripa (Lagopus) blott såsom en markerad ras af den norska arten, ehuru de flesta förklara den vara en särskild för England egendomlig art. Ett stort afstånd emellan tvänne tvifvelaktiga formers hemland bestämmer många naturforskare att förklara dem för arter, men då frågas, huru stort afstånd är dertill nödvändigt? Om man kallar afståndet emellan Europa och Amerika för stort, skall då äfven afståndet mellan Europa och Azorerna vara tillräckligt, eller Madeira, eller Canarieöarna, eller emellan de olika öarna af denna lilla överld.

En utmärkt entomolog i Nordamerikas Förenta Stater, B. D. Walsh, har nyligen riktat uppmärksamheten på några förhållanden, som äro analoga med dessa lokalformer och geografiska raser, men dock i visst hänseende olika. Han beskrifver dessa fall utförligt under benämningarna phytophaga varieteter och phytophaga arter. De flesta växtätande insekter lefva af en växtart eller en växtgrupp, men några begagna till föda utan åtskilnad flera vidt skilda arter, utan att deraf variera. Walsh har emellertid observerat andra fall, då detta framkallade små men konstanta olikheter i färg, storlek eller beskaffenheten af deras secretioner, antingen hos larven eller hos den utbildade insekten allena eller hos båda. I ett fall åstadkom en olika näring små bildningsafvikelser blott hos den utbildade hannen, i andra fall åter rönte både hanne och hona inflytande deraf. Om olikheterna äro något markerade, och om båda könen och alla åldrar äro modifierade antagas formerna af alla entomologer för särskilda arter. Ingen kan här för andra uppgifva en gräns, äfven om han kan göra det för sig sjelf, och med säkerhet bestämma, hvilka af de phytophaga formerna böra kallas arter och hvilka af dem äro varieteter. Walsh räknar för varieteter de former, hvilka kunna antagas hafva bibehållit sin förmåga att otvunget kroaseras, för arter åter dem som förlorat denna förmåga. Då afvikelserna i alla dessa fall bero derpå, att insekterna under en längre tid hafva lefvat af fullkomligt olika växter, så kan man icke vänta att finna mellanlänkar emellan dessa på sådant sätt bildade former, dock måste sådana former förr hafva funnits och förenat de nu divergerande formerna med deras gemensamma upphof. Naturhistorikerna förlora derigenom sin bästa ledtråd vid bestämmandet, om sådana former äro att anse för arter eller varieteter. Detta är äfven förhållandet med nära beslägtade organismer af tvifvelaktig rang, som bebo skilda kontinenter eller aflägsna öar. Men om ett djur eller en växt har en vidsträckt utbredning öfver en och samma kontinent, eller om den bebor flera öar af samma arkipelag, och om den på olika områden visar olika former, så har man alltid en god utsigt att finna mellanformer, som förena de extrema formerna med hvarandra; dessa nedsjunka då till rangen af varieteter.

Några vetenskapsmän förneka en varietetsbildning hos djuren och gifva derföre de obetydligaste afvikelser värde af specifika karakterer, och till och med om samma form identiskt förekom mer i två skilda regioner eller skilda geologiska perioder, så antaga de, att två arter uppträda med samma utstyrsel. Uttrycket art blir derigenom en onyttig abstraktion, hvarunder man förstår en antagen särskild skapelseakt. Man kan emellertid icke bestrida, att många former som af kompetenta domare anses för varieteter hafva så fullständiga artkarakterer, att de af likaså kompetenta män anses för äkta arter. Men att inlåta sig på frågan, med hvad rätt de kallas arter eller varieteter, så länge icke någon definition af dessa termer blifvit allmänt antagen, det vore att kämpa mot väderqvarnar.

Många af de väl utpräglade varieteterna förtjena att noga tagas i betraktande, emedan man har hemtat många interessanta argumenter från deras geografiska utbredning, analoga variationer, bastardbildning m. m. för att få stödjepunkter för deras tillbörliga rang. Utrymmet tillåter mig dock icke att här inlåta mig derpå. Sorgfällig undersökning skall i de flesta fall bringa naturforskarna till öfverensstämmelse i sådana frågor, huru dylika tvifvelaktiga former böra anses. Vi måste dock bekänna, att de flesta tvifvelaktiga former finnas i de bäst bekanta länderna. Jag har blifvit förvånad öfver det faktum, att af sådana växter eller djur som i sitt naturtillstånd varit för menniskan af stor nytta eller af någon annan orsak ådragit sig hennes synnerliga uppmärksamhet hafva nästan öfverallt varieteter blifvit upptecknade, och dessa varieteter hafva stundom af en eller annan författare upptagits som arter. Huru sorgfälligt är icke den vanliga eken studerad: och nu uppställer en tysk författare öfver ett dussin arter af former som hittills alltid ansetts för varieteter och i England kunde man anföra de största botaniska auktoriteter, af hvilka några betrakta de båda formerna pedunculata och sessiliflora för väl skilda arter, andra deremot för varieteter.

Jag vill här anföra ett nyligen utkommet arbete af A. de Candolle öfver ekarna på hela jorden. Aldrig har någon haft större material till att urskilja arterna och ej heller skulle detta ämne kunnat bearbetas af någon med mera ifver och skarpsinne. Han gifver först en detaljerad skildring öfver alla punkter i hvilka arternas bygnad varierar och uppskattar numeriskt afvikelsernas förekomst. Han upptager öfver ett dussin kännetecken hvilka man finner variera på en och samma gren, stundom efter ålder och utveckling, stundom utan märkbar orsak. Dylika kännetecken hafva naturligtvis intet specifikt värde, men såsom Asa Gray anmärker i sitt referat af denna afhandling, de äro sådana som vanligen upptagas i artbestämningarna. De Candolle säger vidare, att han som arter betraktar de former, hvilka afvika från hvarandra i kännetecken, som aldrig variera på ett och samma träd och aldrig äro förenade med mellanformer. Efter denna uppgift, resultatet af så mycket arbete tillägger han uttryckligen: ”De begå ett stort misstag som alltid upprepa, att flertalet af våra arter äro tydligt begränsade och att de tvifvelaktiga arterna utgöra en minoritet. Detta antogs vara sant, så länge man blott ofullkomligt kände ett slägte och dess arter grundades på några få exemplar, det vill säga voro provisoriska. Men så snart vi kommit till en närmare kännedom om dem, framträdde mellanformerna och tvifvel höjdes mot arternas begränsning”. Han tillägger ytterligare, att det är just de bäst kända arterna som erbjuda det största antalet af sjelfständiga varieteter och undervarieteter. Så har den vanliga eken, Quercus Robur, tjugoåtta varieteter, hvilka med undantag af sex kunna hänföras till tre grupper eller underarter, Q. pedunculata, sessiliflora och pubescens. De former, som sammanbinda dessa tre underarter med hvarandra äro relativt sällsynta, och om dessa nu sällsynta öfvergångsformer skulle dö ut, så skulle, såsom Asa Gray anmärker, de tre underarterna förhålla sig till hvarandra på samma sätt som de fyra eller fem provisoriskt antagna arterna, hvilka gruppera sig nära omkring den typiska Quercus Robur. Han säger slutligen, att af de i inledningen uppräknade trehundra till ekfamiljen hörande arterna åtminstone två tredjedelar äro provisoriska, det vill säga otillräckligt kända för att passa in i den ofvan gifna definitionen af begreppet art. Jag bör äfven nämna, att de Candolle icke längre anser arterna för oföränderliga skapelser, utan han har nu kommit till det resultat, att teorien om arternas härstammande från hvarandra är den naturligaste, och att den äfven ”bäst öfverensstämmer med de kända fakta ur palæontologi, växt- och djurgeografi, anatomi och klassifikation”. Dock, tillägger han, ett direkt bevis fattas ännu.

Då en ung naturforskare börjar studera en för honom okänd grupp af organismer, så förvirras han först vid afgörandet af den frågan, hvilka kännetecken bestämma arterna och hvilka tillhöra varieteterna, ty han känner ännu intet om beskaffenheten och storleken af de förändringar som gruppen kan erbjuda, och detta bevisar åtminstone, huru allmänt någon föränderlighet förekommer. Men om han inskränker sina iakttagelser till en klass inom ett visst område, så skall han snart vara på det klara i denna fråga. Han skall i allmänhet vara benägen att göra många arter, emedan han anser de af honom observerade formernas olikheter vara betydliga och emedan han ännu har för ringa kännedom om analoga olikheter i andra grupper och andra länder för att kunna beriktiga detta första intryck. Utvidgar han nu kretsen för sina iakttagelser, så skall han stöta på flera svårigheter, han skall träffa på ett stort antal närbeslägtade former. Vidgas hans erfarenhet ännu mer, så skall han slutligen få klart för sig, hvad han bör kalla art eller varietet, men detta mål skall han blott ernå, om han medgifver en hög grad af föränderlighet, och han skall ofta se riktigheten af sina bestämningar dragas i tvifvel af andra naturforskare. Om han nu får tillfälle att studera beslägtade former från andra ej numera sammanhängande länder, i hvilket fall han knappt torde hoppas att finna mellanstadierna mellan sina tvifvelaktiga former, så har han ingenting annat än analogien att stödja sig vid och svårigheterna nå nu sin höjd.

En bestämd gränslinie har säkerligen hittills ej blifvit dragen hvarken mellan arter eller underarter, det vill säga sådana former, som enligt några vetenskapsmän nära ehuru ej fullt uppnå rangen af arter, ej heller mellan underarter och utpräglade varieteter och slutligen ej heller emellan de ringare varieteterna och individuela olikheter. Dessa olikheter gripa omärkligt in i hvarandra i en nästan oafbruten serie, och en serie gör i allmänhet intrycket af en verklig öfvergång.

De individuela olikheterna, som för systematikern hafva föga värde, anser jag för oss hafva den största betydelse, emedan de bilda första steget till sådana obetydliga varieteter, som i naturhistoriska arbeten anses värda att omnämnas. De varieteter, som äro något betydligare och beständigare, betraktar jag såsom trappsteg, som föra oss till de mera påtagliga och permanenta varieteterna, såsom dessa åter föra oss till underarterna och slutligen till arterna. Öfvergången från ett af dessa trappsteg till ett annat närmast högre kan i några fall härröra blott från en långvarig inverkan af olika yttre förhållanden i skilda länder; dock har jag icke mycket förtroende till denna åsigt och en varietets öfvergång från en blott obetydligt afvikande form till en annan som mera skiljer sig från moderväxten tillskrifver jag verkan af det naturliga urvalet, som hopar individuela afvikelser i en viss riktning, såsom framdeles närmare skall utvecklas. Jag tror derföre, att en väl utpräglad varietet med skäl kan kallas för en begynnande art, men om denna tro låter rättfärdiga sig, det måste framgå af den allmänna betydelsen af de i detta verk framstälda fakta och åsigter.

Man behöfver icke antaga, att alla varieteter eller begynnande arter verkligen stiga till rangen af art, de kunna slockna ut i detta primitiva stadium eller kunna bestå såsom varieteter under långa perioder; detta har Wollaston visat vara förhållandet med varieteterna af vissa fossila landsnäckor på Madeira. Om en varietet frodades, så att den i antal öfverträffade moderväxten, så skulle varieteten uppstiga till rangen af art och arten nedsjunka till rangen af varietet; eller också kunde varieteten undantränga och utrota moderväxten, eller slutligen kunde båda fortfarande hafva bestånd jemte hvarandra såsom oberoende arter. Till detta återkomma vi framdeles.

Af denna framställning är tydligt, att jag anser uttrycket art såsom en term, godtyckligt och för beqvämlighets skull använd på en serie af hvarandra liknande individer, och att den icke väsentligt skiljer sig från termen varietet, som användes på mindre afvikande och vacklande former. Termen varietet, såsom skilnad från individuela afvikelser, är likaså godtycklig och begagnas äfven blott för beqvämlighets skull.



\section[Vidt utbredda arter variera mest]{De allmänna och vidt utbredda arterna variera mest.}

På grund af teoretiska betraktelser trodde jag, att några interessanta resultater angående de mest varierande arternas natur och förhållanden skulle vinnas genom att tabellariskt sammanställa alla varieteter i flera väl utarbetade floror. I början tycktes mig detta vara en enkel sak, men H. C. Watson, för hvars värderika tjenster och biträde i denna fråga jag är särdeles tacksam, öfvertygade mig snart, att det skulle vara förknippadt med många svårigheter, hvilket Hooker sedermera visade ännu bestämdare. Behandlandet af dessa svårigheter spar jag derföre till ett kommande arbete jemte tabellerna öfver de varierande arternas numeriska förhållanden. Med Hookers tillåtelse torde jag få nämna, att han anser de följande satserna för fullkomligt välgrundade, sedan han sorgfälligt genomläst mina manuskript och granskat mina tabeller. Men hela detta ämne, som här naturligtvis måste behandlas i korthet, är temligen inveckladt, och hänsyftningar på ”kampen för tillvaron” och ”karakterernas divergens” och andra kapitel äro oundvikliga.

Alphonse de Candolle och andra botanister hafva visat, att sådana växter som hafva en vidsträckt utbredning i allmänhet erbjuda varieteter, och detta kunde vi äfven vänta, då de äro utsatta för olika fysikaliska inflytelser och komma i beröring med andra grupper af organismer, hvilket, såsom vi framdeles få se, är af ännu större vigt. Men mina tabeller visa vidare, att äfven inom ett bestämdt begränsadt område de allmännaste arter, det vill säga, de som förete största antalet individer och som inom detta område äro mest spridda (hvilket icke får förblandas med ”vidsträckt utbredning” och icke heller med ”allmänhet”), ofta gifva upphof till varieteter tillräckligt utpräglade för att uppräknas i ett botaniskt arbete. Det är derföre också de frodigaste eller, såsom man kan kalla dem, de dominerande arterna — nämligen de, som hafva den visträcktaste utbredningen på jordens yta och inom sitt område äro mest spridda och på individer rikast — som oftast lemna väl markerade varieteter eller, såsom jag betraktar dem, begynnande arter. Detta skulle vi kunna tänka oss på förhand, ty då varieteterna måste kämpa med andra invånare i samma trakt för att i någon mån vinna beständighet, så skola de redan dominerande arterna vara mest i stånd att efterlemna afkomlingar, hvilka med några lätta modifikationer ärfva de fördelar, som satt föräldrarna i stånd att få öfvermakten öfver sina grannar. Vid dessa antydningar om öfvermakten är dock att märka, att de blott hafva afseende på de former som kunna komma i täflan med hvarandra eller med andra medlemmar af samma klass eller slägte med lika lefnadssätt. Med afseende på en arts allmännelighet och individantal sträcker sig derföre jemförelsen blott till medlemmar af samma grupp. Man kan kalla en växt dominerande, om den är rikare på individer och mera spridd än andra växter, som lefva under ungefär samma yttre omständigheter på samma område. En sådan växt blir icke derföre i den här använda mening mindre herskande, att en vattenconferva eller en parasitsvamp är oändligt rikare på individer och ännu mera spridd än den. Om åter en conferva eller en parasitsvamp i nämda hänseenden öfverträffar sina anförvandter, då är den dominerande ibland växter af sin klass.



\section[Större slägtena variera oftare]{Arterna af de större slägtena i ett land variera oftare än arterna af de små.}

Om man delar de växter, som bebo ett område och äro beskrifna i en flora, i två delar, af hvilka den ena innehåller arterna af de större slägtena och den andra arterna af de mindre, så skall man finna ett större antal af de allmänna och spridda eller dominerande arterna i den grupp, som innefattar de större slägtena. Detta kunde vi äfven tänka oss a priori, ty redan det enkla faktum, att många arter af samma slägte bebo ett land, visar, att landets organiska eller oorganiska förhållanden innehålla något för slägtet gynsamt, och vi kunde följaktligen hafva väntat att af de större slägtena, det vill säga de som hafva ett större antal arter, finna ett proportionsvis större antal af dominerande arter. Men det är så många orsaker, som sträfva att fördunkla detta förhållande, att jag är öfvarraskad att i mina tabeller finna en så liten majoritet på de större slägtenas sida. Jag vill här blott anföra två af dessa orsaker. Sötvattensväxter och saltväxter hafva vanligen en vidsträckt utbredning och äro mycket spridda, men detta tyckes stå i samband med naturen af deras vistelseort och har föga eller intet sammanhang med storleken af slägtena till hvilka de höra. Växter som stå på en lägre organisationsgrad äro vanligen mycket mera spridda än högre organiserade växter och här finnes heller icke något intimare förhållande till slägtenas storlek. Orsakerna till detta sista fenomen skola behandlas i kapitlet om arternas geografiska utbredning.

Min åsigt, att arterna blott vore starkt utpräglade och begränsade varieteter, ledde mig till den förutsättningen, att arterna af de större slägtena i ett land skulle oftare förete varieteter än arterna af de mindre slägtena, ty hvarhelst många närbeslägtade arter hafva bildats, böra äfven nu i allmänhet bilda sig många varieteter eller begynnande arter; der många stora träd växa, der vänta vi äfven att finna unga telningar. Der många arter af ett slägte hafva bildats genom variation, der hafva omständigheterna varit gynsamma för variation, och då kunna vi äfven vänta, att omständigheterna fortfarande skola vara dertill gynsamma. Å andra sidan, om vi betrakta hvarje art såsom resultatet af en särskild skapelseakt, så finnes intet rimligt skäl, hvarföre flera varieteter skulle anträffas i en grupp som har många arter, än i en grupp med få arter.

För att pröfva detta antagandes riktighet har jag ordnat i två lika delar växterna i tolf olika länder och skalbaggarna i två distrikter, arterna af de större slägtena i den ena gruppen, arterna af de mindre i den andra och det har oföränderligen visat sig, att de större slägtena förete varieteter af ett proportionsvis större antal arter än de mindre slägtena. Bland de större slägtena hafva dessutom de arter som variera ett vida större antal varieteter, än de arter af de små slägtena, som förete några varieteter. Till samma resultat kommer man äfven med en annan indelning och om man ur tabellerna utesluter alla slägten med en till fyra arter. Dessa fakta äro lätta att förklara om man utgår från den åsigten, att arterna blott äro starkt utpräglade och permanenta varieteter, ty hvarhelst många arter af samma slägte hafva bildats eller, om uttrycket tillåtes mig, hvarhelst artfabrikationen har drifvits i större skala, kunna vi vänta oss att finna fabrikationen ännu i full verksamhet, i synnerhet som vi hafva anledning att tro, att tillverkningen af nya arter går långsamt för sig. Och detta är säkerligen fallet, om varieteter betraktas såsom begynnande arter, ty mina tabeller visa tydligt såsom en allmän regel, att hvarhelst många arter af ett slägte hafva bildats, visa dessa arter ett antal varieteter eller begynnande arter, som öfverstiger medeltalet af varieteter. Härmed vill jag icke säga, att alla stora slägten nu variera betydligt och fortfarande föröka sitt antal, eller att de små slägtena alls icke bilda några varieteter och nya arter; detta skulle vara mycket olyckligt för min teori, och geologien bevisar klart, att små slägten under tidernas lopp hafva betydligt förökats och att de stora slägtena, sedan de nått sitt maximum, hafva sjunkit tillbaka och försvunnit. Allt hvad vi här ville visa, är att hvarhelst många arter af ett slägte bildats, der äro många fortfarande under bildning och detta är säkerligen riktigt.



\section[Arter ur större slägten likna varieteter]{Många arter af de större slägtena likna varieteter 
deri, att de äro mycket nära men på olika sätt
beslägtade med hvarandra och hafva ett
begränsadt utbredningsområde.}

Det finnes ännu andra anmärkningsvärda förhållanden emellan arterna af stora slägten och deras upptecknade varieteter. Vi hafva sett, att det icke finnes någon osviklig skilnad emellan arter och starkt utpräglade varieteter, och i de fall då mellanstadier emellan de tvifvelaktiga formerna ännu icke blifvit funna, äro naturforskarna nödgade att vid bestämningen taga sin tillflykt till graden af olikhet emellan två former, i det de af analogi bedöma, om olikheterna äro tillräckliga för att höja blott den ena eller båda till rangen af arter. Graden af olikhet är således ett vigtigt kriterium vid bestämmandet, om två former skola gälla för arter eller varieteter. Men nu hafva Fries och Westwood gjort den iakttagelsen, den förra på växter, den senare på insekter, att inom stora slägten graden af olikhet emellan arterna ofta är utomordentligt liten. Jag har sökt pröfva detta med siffror och så vidt mina ofullständiga resultat räcka, har jag funnit det bekräftadt. Jag har derföre äfven rådfrågat mig med några noggranna och erfarna iakttagare, och efter öfverläggning hafva vi i denna sak varit ense. I detta hänseende hafva de stora slägtenas arter en större likhet med varieteter än de mindres. Detta förhållande kan äfven uttryckas med andra ord, nämligen att i de större slägtena, hvarest ett proportionsvis större antal varieteter eller begynnande arter ännu fabriceras, många af de redan färdiggjorda arterna till en viss grad likna varieteterna, ty de skiljas från hvarandra genom obetydligare olikheter än vanligt.

Arterna af de stora slägtena äro vidare beslägtade med hvarandra på samma sätt som varieteterna af en art. Ingen naturhistoriker påstår, att alla arter af ett slägte äro lika väl skilda från hvarandra; de kunna vanligen delas i subgenera, sektioner eller ännu mera underordnade grupper. Såsom Fries riktigt anmärker, äro dessa artgrupper vanligen samlade såsom drabanter omkring vissa andra arter. Och hvad äro varieteter annat än formgrupper af olika ömsesidig slägtskap, samlade omkring vissa former, stamarter? Otvifvelaktigt finnes en ytterst vigtig skilnad emellan arter och varieteter; nämligen att graden af olikhet emellan varieteterna, om man jemför dem med hvarandra eller med stamarten, är vida mindre än emellan arterna af samma slägte. Men när vi komma att afhandla grundsatsen om ”karakterens divergens”, såsom jag kallar den, så skola vi se, huru detta bör förklaras, och huru de obetydliga olikheterna emellan varieteterna kunna växa ut till de större olikheterna emellan arterna.

Det finnes ännu en annan punkt som förtjenar att tagas i betraktande. Varieteterna äro i allmänhet utbredda öfver ett inskränkt område, hvilket är lätt begripligt; ty hade en varietet större utbredning än stamarten, så skulle benämningen omvändas. Men det finnes skäl för det antagandet, att de arter som äro mycket nära beslägtade med andra och i detta hänseende likna varieteter ofta äro spridda inom mycket trånga gränser. Så har H. C. Watson påpekat för mig i sin med omsorg uppstälda växtkatalog öfver London 63 växter som deri äro upptagna såsom arter, ehuru han anser dem så nära beslägtade med andra arter, att deras rang är tvifvelaktig. Dessa sextiotre tvifvelaktiga arter äro spridda öfver omkring 6,9 af de provinser, i hvilka Watson indelar Storbritanien. I samma katalog uppräknas femtiotre erkända varieteter och dessa sträcka sig öfver 7,7 af provinserna, under det att arterna, till hvilka dessa varieteter höra, hafva en utsträckning öfver 14,3 provinser. De erkända varieteterna hafva således öfverhufvud taget en nästan lika så inskränkt utbredning som de nära beslägtade former, hvilka Watson har betecknat såsom tvifvelaktiga arter, men hvilka af andra engelska vetenskapsmän betraktas såsom goda och äkta arter.



\section{Sammanfattning.}

Varieteter kunna således icke skiljas från arter annat än genom följande igenkänningstecken: för det första genom upptäckande af mellanformer; för det andra genom en viss obestämd grad af olikhet, ty två former som blott obetydligt afvika från hvarandra anses allmänt blott för varieteter, äfven om sammanbindande mellanformer icke kunnat påträffas; men denna grad af olikhet, som anses nödvändig för upphöjandet af tvänne former till rang af art, den är fullkomligt obestämd. I slägten, som hafva ett artantal som öfverstiger medeltalet hafva arterna också ett antal varieteter som öfverstiger medeltalet. I stora slägten äro arterna beslägtade men olika nära och ordna sig gruppvis omkring vissa arter. Sådana arter som äro mycket nära beslägtade med andra hafva en inskränkt utbredning. I alla dessa hänseenden förete arterna af de större slägtena en stor analogi med varieteter, och dessa analogier äro lätta att förstå, om vi antaga att arterna en gång varit varieteter och hafva uppstått ur dessa; de äro deremot fullkomligt oförklarliga, om arter äro oberoende skapelser.

Vi hafva nu äfven sett, att de frodigaste dominerande arterna af de större slägtena i hvarje klass i medeltal lemna största antalet varieteter, och såsom vi framdeles skola se, hafva varieteterna benägenhet att öfvergå i nya och bestämda arter. Derigenom få också de större slägtena en benägenhet att förökas, och i hela naturen se vi hos de nu dominerande lefvande formerna en sträfvan att bibehålla och utvidga sitt välde genom att lemna efter sig i mängd modifierade och herrskande afkomlingar. Men å andra sidan äro äfven de större slägtena, såsom vi framdeles få se, benägna att upplösa sig i små slägten, och så afdelas organismerna på hela jorden i allt mer och mer underordnade grupper.


