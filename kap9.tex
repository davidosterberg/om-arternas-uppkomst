%NIONDE KAPITLET.


\chapter[Om urkundernas ofullständighet]{Om de geologiska urkundernas ofullständighet.}

{\it
Om saknaden af öfvergångsvarieteter emellan nu existerande former. — Beskaffenheten af utdöda öfvergångsvarieteter, deras antal. — Tidperiodernas längd beräknad efter aflagringar. — Periodernas längd beräknad i år. — Bristfälligheten i våra palæontologiska samlingar. — Granitytors blottande. — Afbrott i geologiska formationer. — Frånvaron af öfvergångsvarieteter i alla formationer. — Artgruppers plötsliga uppträdande. — Deras plötsliga uppträdande i de lägsta kända fossilförande lagren. — Den beboeliga jordens ålder.
}\\[0.5cm]

I sjette kapitlet uppräknade jag de förnämsta invändningar, som kunna med skäl uppställas emot de åsigter, som framhållas i denna bok. De flesta af dem hafva nu behandlats. En är synnerligen svår, nämligen det förhållande att artformerna äro väl bestämda och icke hopblandade genom otaliga öfvergångsformer. Jag angaf några skäl hvarföre sådana former för det närvarande i allmänhet icke förekomma under de förhållanden som tyckas mest gynsamma för deras närvaro, en vidsträckt och sammanhängande yta med små öfvergångar i de fysiska förhållandena. Jag bemödade mig att visa, att hvarje arts lif i vida högre grad beror på närvaron af andra redan bestämda organiska former än på klimatet, och att derföre de styrande lifsvilkoren icke liksom värme och fuktighet omärkligt förändras. Jag bemödade mig också att visa, att öfvergångsvarieteter, som finnas i mindre antal än de former de förena i allmänhet undanträngas och tillintetgöras under de öfrigas fortgående modifikation och förädling. Den egentliga orsaken till att föreningslänkar icke numera påträffas i naturen beror på det naturliga urvalet, genom hvilket nya varieteter ersätta och undantränga sina stamformer. Men då denna process fortgått i enorm skala, så måste antalet af fordom existerande mellanformer vara i sanning ofantligt. Hvarföre är då icke hvarje geologisk formation, hvarje lager fullt af sådana föreningslänkar? Geologien skall helt säkert icke upptäcka någon sådan väl graderad organisk kedja, och detta är måhända den närmaste och svåraste invändning som kan framställas emot min teori. Förklaringen ligger, såsom jag tror, i den ytterliga ofullständigheten af de geologiska urkunderna.

Först måste vi ihågkomma, hvad slags mellanformer enligt min teori måste fordom hafva existerat. Jag har funnit det svårt vid betraktande af tvänne arter att låta bli att för mig sjelf i tanken utrita former, som stå direkt emellan dem. Men detta är en fullkomligt falsk åsigt; vi skola alltid söka former som stå emellan hvarje art och en gemensam men okänd stamfader, och stamfadern lär i allmänhet hafva i något hänseende skilt sig från alla sina modifierade afkomlingar. Vi må anföra ett enkelt exempel. Påfågeldufvan och kroppdufvan härstamma båda från klippdufvan; om vi egde alla öfvergångsvarieteter som funnits, skulle vi hafva en ytterst tät serie emellan båda och klippdufvan, men vi skulle icke hafva några direkta öfvergångsvarieteter emellan påfågeldufvan och kroppdufvan, ingen till exempel, som på samma gång egde en något utbredd stjert och en något utvidgad kräfva, de karakteristiska dragen för dessa två raser. Dessa två raser hafva dessutom blifvit så mycket modifierade, att om vi icke hade något historiskt eller indirekt bevis för deras ursprung det skulle varit omöjligt att af blotta jemförelsen med klippdufvans, Columba livia, kroppsbildning afgöra om de härstammade från denna eller från någon beslägtad art såsom skogsdufvan, Columba oenas.

Så är äfven förhållandet med naturliga arter; om vi betrakta väl skilda former till exempel hästen och tapiren, hafva vi intet skäl att tro, att direkta öfvergångsformer emellan dem någonsin funnits, utan emellan hvar och en af dem och en okänd gemensam stamfader. Den gemensamma stamfadern bör i sin hela organisation haft mycken allmän likhet med hästen och tapiren, men i vissa delar af sin kroppsbildning bör han hafva skilt sig betydligt från båda, kanhända mera än de skilja sig ifrån hvarandra. I alla sådana fall torde vi vara ur stånd att upptäcka stamfadern till en eller flera arter, äfven om vi noggrant jemförde dess kroppsbildning med hans modifierade afkomlingar, så vida icke vi på samma gång hade en nästan fullständig kedja af föreningslänkar.

Det är i alla fall möjligt enligt min teori, att den ena af två lefvande former härstammar från den andra, till exempel en häst från en tapir, och i detta fall böra direkta föreningslänkar emellan dem finnas. Men ett sådant fall skulle innefatta, att en form under en lång period förblifvit oförändrad, under det att dess afkomlingar undergått en modifikation; och principen om täflan emellan organism och organism, emellan barn och fader torde göra ett sådant fall mycket sällsynt, ty i alla händelser sträfva de nya och förädlade formerna att undantränga de gamla icke förädlade.

Enligt teorien om det naturliga urvalet hafva alla lefvande arter varit förenade med stamarten till hvarje slägte, och skiljaktigheterna hafva icke varit större än vi se i närvarande tid emellan varieteterna af samma art, och dessa stamarter, som nu i allmänhet äro utdöda, hafva i sin tur varit på samma sätt förenade med ännu äldre arter, och så vidare bakåt, alltid konvergerande emot den gemensamma stamfadern till hvarje stor klass. Antalet föreningslänkar och öfvergångsformer emellan alla lefvande och utdöda arter måste derföre hafva varit ofantligt stort. Men om denna teori är riktig hafva de helt säkert lefvat på jorden.



\section[Om tidsperiodernas längd]{Om tidperiodernas längd, beräknad efter aflagring
och denudation.}

Oberoende af möjligheten att finna fossila qvarlefvor af en så oändlig mängd föreningslänkar kan man invända, att tiden icke kunnat vara tillräcklig för en så hög grad af organisk förändring, då alla förändringar försiggått mycket långsamt. Det är knappt möjligt för mig att framställa för läsaren, om han icke är praktisk geolog, alla de fakta, som i någon mån sätta oss i stånd att fatta tidens längd. Den som kan läsa sir Charles Lyells stora arbete ”the Principles of Geology”, om hvilket framtida historieskrifvare skola erkänna, att det åstadkommit en hel revolution i naturvetenskapen, och dock icke inse den omätliga längden af tidperioderna, han må gerna strax slå igen boken\footnote{C. Lyells arbete, ”the Principles of Geology”, finnes i svensk bearbetning af G. Lindström, under titel Geologiens grunder. Ö. a.}. Det är icke tillräckligt att läsa the Principles of Geology eller att läsa specialafhandlingar af olika iakttagare öfver särskilda formationer, af hvilka hvar och en söker gifva ett otillräckligt begrepp om tidrymden för hvarje formation eller hvarje lager. Vi kunna bäst få någon föreställning om förflutna tider genom att lära känna de krafter som varit i verksamhet, och genom att lära känna huru mycken landyta blifvit blottad och huru mycket sediment som aflagrats. Såsom Lyell har väl anmärkt äro utsträckningen och tjockleken af våra sedimentära formationer resultatet af och måttet på den denudation, som jordens yta har undergått. Derföre skulle hvar och en undersöka för sig sjelf den stora stapeln af på hvarandra liggande lager och betrakta, huru strömmarna drifva ned sand och huru vågorna nöta bort klipporna, för att kunna fatta i någon mån längden af de förflutna perioderna, hvaraf vi se minnesmärken rundt omkring oss.

Det är nyttigt att vandra längs en strand, bildad af måttligt hårda klippor och iakttaga förstörelseprocessen. Flodvattnet når i de flesta fall klipporna blott en kort tid två gånger om dagen, och vågorna äta sig in i dem blott då de äro betäckta med sand och grus, ty det är klart, att rent vatten icke har någon verkan på en fast klippa. Till slut är basen af klippan underminerad, väldiga stycken falla ned och de som ligga fast måste nötas bort bit för bit, atom för atom, till dess de blifvit så reducerade i storlek, att de kunna rullas bort af vågorna och söndersmulas till grus och sand eller dy. Men huru ofta se vi icke vid klippornas fot rundade stycken, tätt beväxta med hafvets alster, hvilket visar, huru föga de afnötas af vågorna, huru sällan vågorna förmå att rulla dem bort. Dessutom om vi flera mil följa en klippig strand, som är inbegripen i en sådan förstörelseprocess, finna vi den blott här och der längs en kort sträcka eller omkring en udde ännu undergå någon nötning. Ytans utseende och vegetationen visar, att öfverallt år hafva förflutit sedan vattnet sköljde deras fot.

Vi hafva likväl nyligen af Ramsays observationer i spetsen för utmärkta observatörer, Jukes, Geikie, Croll och andra sett att luftens förstörande inverkan är vida vigtigare än hafvets och vågornas makt. Hela landytan är utsatt för luftens och regnvattnets kemiska inverkan med dess upplösta kolsyra och i kallare trakter för frosten; den söndersmulade massan föres bort äfven utför sakta sluttningar genom tungt regn och i vida större utsträckning än man kan tro af vinden isynnerhet i torra områden; den föres derefter vidare af strömmar och floder hvilka om de äro häftiga fördjupa sina bäddar och söndersmula styckena. På en regnig dag se vi äfven i obetydligt ojemna trakter verkningarna af atmosferens förstörande förmåga i de dyiga bäckar, som flyta ned utför hvarje sluttning. Ramsay och Whitaker hafva visat, och iakttagelsen är verkligen öfvertygande, att de branta sluttningarna i Englands Wealdentrakter och de, som sträcka sig tvärs öfver landet och som förut allmänt ansågos såsom fordna hafskuster icke hafva kunnat bildas på detta sätt, ty hvar och en är bildad af en och samma formation under det våra nuvarande hafsklippor äro bildade af flera genomskurna formationer. Då detta är förhållandet tvingas vi att medgifva, att sluttningarnas uppkomst beror derpå, att de klippor hvaraf de äro sammansatta hafva motstått atmosferens inflytande bättre än den omgifvande ytan, att denna yta följaktligen blifvit småningom lägre och lägre och de hårdare klipporna blifvit orörda. Ingenting kan bättre inprägla i minnet längden af tidperioderna enligt vårt begrepp om tid än den öfvertygelsen, att atmosferiska inflytelser åstadkommit så stora resultat, hvilka tydligen hafva så ringa förmåga och tyckas arbeta så långsamt.

Om vi på detta sätt blifvit öfvertygade om huru långsamt landet nötes bort af atmosferiska och hafvets inflytelser, böra vi för att uppskatta längden af förflutna tidperioder taga i öfvervägande å ena sidan de klippmassor som blifvit bortförda från vidsträckta ytor och å andra sidan tjockleken af våra sedimentära formationer. Jag har med stor förvåning sett vulkaniska öar, som blifvit afnötta af vågorna och skalade rundt omkring till lodräta klippor af tvåtusen fots höjd, ty de sakta sluttande lavaströmmarna, som fordom varit i flytande tillstånd, visade tydligt, huru långt de hårda klippbäddarna en gång sträckt sig ut i hafvet. Samma historia förtälja ännu tydligare dessa stora remnor, längs hvilka lagren blifvit upplyftade å ena sidan eller nedtryckta å den andra till en höjd eller ett djup af tusentals fot; ty sedan jordskorpan remnade, och det blir ingen skilnad om höjningen skedde hastigt eller såsom de flesta geologer nu tro mycket långsamt och stötvis, har ytan blifvit så fullkomligt utjemnad, att intet spår af dessa vidsträckta förkastningar är utvändigt synligt. Cravenremnan till exempel sträcker sig trettio mil uppåt och längs denna linie äro lagren förflyttade i vertikal riktning från 600 till 3,000 fot. Professor Ramsay har offentliggjort en berättelse öfver en sådan remna på Anglesea, der förkastningen var 2,300 fot, och han har meddelat mig att i Merionethshire finnes en af 12,000 fot, och dock finnes i dessa fall ingenting på landytan, som visar sådana ofantliga förkastningar; klipporna å ömse sidor af remnan hafva blifvit bortsopade.

Å andra sidan äro i alla delar af jorden staplarna af sedimentära lager af en underbar tjocklek. I Cordillererna uppskattade jag en konglomeratmassa till tiotusen fot, och ehuru konglomeraten samlats på kortare tid än finare sediment, så tjena de dock att visa, huru långsamt massorna måste hafva samlats, då de bestå af afnötta, rundade kiselstenar, som alla bära på sig tidens stämpel. Professor Ramsay har uppgifvit för mig maximaltjockleken af följande formationer i olika delar af Stor-Britannien, mestadels från verkliga mätningar:

\nopagebreak
\begin{tabular}{lrl}
Palæozoiska lager & 57,154 & fot. \\
Sekundära lager   & 13,190 &  „   \\
Tertiära lager    & 2,240  &  „
\end{tabular}

\noindent
tillsammans 72,584 fot, nära tretton och tre qvarts engelska mil. Några af formationerna, som i England representeras af tunna lager, uppnå på kontinenten tusentals fot i tjocklek. Enligt de flesta geologers åsigt hafva vi dessutom emellan hvarje successiv formation ofantligt långa perioder utan några afsättningar, så att de höga staplarna af sedimentära klippor i England lemna blott en oegentlig föreställning om den tid som förflutit under deras bildning. Betraktelsen af alla dessa fakta gör nästan samma intryck som de fåfänga försöken att fatta evigheten.

Icke desto mindre är detta intryck falskt. Mr Croll anmärker i en särdeles intressant uppsats, att vi icke misstaga oss ”i att göra längden af de geologiska perioderna för stor” utan vid deras uppskattning i år. Om geologer betrakta de stora och invecklade fenomenen och derefter figurer som representera flera millioner år, göra båda helt olika intryck och figurerna förklaras med ens såsom för små. Men med afseende på landsträckors denudation visar Croll genom att beräkna den kända mängden af sediment som årligen nedföres af vissa strömmar, att tusen fot af en klippa, söndersmulad genom atmosferens inflytande, på detta sätt kan förflyttas från hela ytans medelhöjd under loppet af sex millioner år. Detta tyckes vara ett förvånande resultat, och vissa betraktelser leda till den misstanken att det är mycket för högt, men äfven om vi antaga hälften eller fjerdedelen, så är det ännu öfverraskande. Få af oss veta likväl hvad en million verkligen betyder. Mr Croll gifver följande illustration. Man tager en smal pappersremsa af 83 fot och 4 tums längd och sträcker ut den utefter väggen af en stor sal och vid ena ändan utmärker man tiondedelen af en tum; om då denna tiondedel motsvarar hundra år, så motsvarar hela remsan en million år. Men vi böra komma ihåg hvad hundra år innebära för den fråga som är föremålet för detta arbete. Många utmärkta landhushållare hafva under sin lifstid allena i så hög grad modifierat några af de högre djuren, som fortplanta sitt slägte vida långsammare än de flesta af de lägre djuren, att deras produkter väl förtjena namnet underraser. Få personer hafva med tillbörlig omsorg sysselsatt sig med en ras i mer än ett halft århundrade, så att etthundra år motsvara frukten af tvänne mäns ansträngningar. Vi få dock icke antaga, att arter i naturtillståndet förändra sig lika hastigt som husdjuren under ledningen af ett metodiskt urval. Jemförelsen blefve riktigare med de resultat som följa af omedvetet urval, det är bibehållandet af de nyttigaste eller vackraste djuren utan någon afsigt att modifiera rasen, men genom denna process af omedvetet urval hafva flera raser blifvit märkbart förändrade under loppet af två eller tre århundraden.

Arternas förändringar försiggå likväl mycket långsammare och inom samma trakt blott hos få på en gång. Denna långsamhet beror derpå, att alla invånare i samma trakt redan äro så väl afpassade efter hvarandra, att några nya platser i naturens hushållning icke finnas förr än efter långa mellantider, då förändringar af något slag i de fysiska förhållandena eller genom inflyttning hafva inträffat, och individuela olikheter eller variationer af rätt beskaffenhet att bättre fylla de nya platserna under förändrade förhållanden uppträda icke på en gång. I år kunna vi icke på något sätt uppskatta huru lång tid en art behöfver att modifieras. Från graden af solens hetta och från data hemtade från sista isperioden antager mr Croll, att blott sextio millioner år förflutit sedan aflagringen af första Cambriska formationen. Detta tyckes vara en mycket kort period för så många och så stora förändringar i lifsformer, som säkerligen sedan dess inträffat. Det måste medgifvas, att vissa i beräkningen ingående moment äro mer eller mindre tvifvelaktiga och Sir W. Thomson gifver en vidsträckt gräns för den möjliga åldern af jordens beboelighet. Men som vi hafva sett, kunna vi icke fatta hvad siffrorna 60,000,000 verkligen betyda och under denna eller kanhända ännu längre tid hafva land och vatten öfverallt alstrat lefvande kreatur, alla inbegripna i kampen för tillvaron och underkastade förändringar.



\section[De palæontologiska samlingarna]{Om torftigheten af de palæontologiska samlingarna.}

Låtom oss nu vända oss till våra rikaste palæontologiska museer och hvilken eländig utställning blifva vi varse! Att våra samlingar äro ytterst ofullständiga medgifves af hvar och en. Den utmärkte palæontologen Edward Forbes’ yttrande får icke förglömmas, att en mängd af våra fossila arter äro kända och benämda efter enstaka, ofta sönderbrutna exemplar eller från några få specimen samlade på en fläck. Blott en ringa del af jordytan har blifvit geologiskt undersökt och ingen del med tillräcklig omsorg, såsom de vigtiga upptäckter bevisa, som hvarje år göras i Europa. Ingen fullkomligt mjuk organism kan bevaras. Snäckor och ben skola förderfvas och försvinna, om de ligga på botten af sjön, der intet sediment samlas. Jag tror vi ofta taga fel, om vi i tysthet medgifva, att sediment afsättes öfver nästan hela sjöbotten tillräckligt hastigt att inbädda och skydda fossila qvarlemningar. Utöfver en stor del af oceanen tillkännagifver dess klara blåa färg vattnets renhet. De många berättelser om en formation likformigt betäckt af en annan senare formation efter en ofantlig tids förlopp utan att den underliggande bädden bär spår till någon nötning eller förstöring, synas mig icke kunna förklaras på något annat sätt, än att sjöbotten icke sällan i långa tider förblifver i oförändradt skick. De qvarlefvor som inbäddas vare sig i sand eller grus skola då lagren höja sig i allmänhet upplösas af kolsyran i det genomträngande regnvattnet. Några af de djurslag, som lefva på området emellan det högsta och lägsta tidvattnet, tyckas sällan bevaras. De olika arterna af Chthamalinæ (en underfamilj af de sessila cirripederna) bekläda hafsklipporna öfver hela verlden i oändligt antal; de hålla sig noggrant vid stränderna med undantag af en enda art i medelhafvet, som bor på djupt vatten, och denna har funnits fossil på Sicilien, hvaremot ingen annan art hittills blifvit funnen i någon tertiär formation; dock känner man nu att slägtet Chthamalus fans under kritaperioden. Många stora aflagringar, som behöfva ofantlig tid för att samlas äro helt och hållet i saknad af organiska qvarlemningar utan att vi äro i stånd att dertill angifva något skäl. Ett af de mest slående exempel härpå är Flyschformationen, som består af skiffer och sandsten ända till sextusen fot i tjocklek i en utsträckning af 300 mil från Wien till Schweitz, och ehuru denna stora massa blifvit ytterst noggrant genomforskad hafva inga fossilier blifvit funna med undantag af några få vegetativa qvarlefvor.

Med afseende på de landalster, som lefde under den sekundära och palæozoiska perioden är det öfverflödigt att framhålla, att den kännedom som hemtas från fossila qvarlefvor är ytterst ringa. Ingen enda landsnäcka har förr än helt nyligen varit känd inom dessa vidsträckta perioder med undantag af en art som Lyell och Dawson upptäckte i Nordamerikas stenkolslager, af hvilken snäcka nu öfver hundra exemplar äro samlade. Beträffande qvarlemningar af däggdjur torde en enda blick på en historisk tabell i Lyells handbok vida bättre än hela sidor af detaljer visa huru tillfälligt och sällsynt deras bevarande är. Ej heller böra vi förvåna oss öfver deras sällsynthet, om vi komma ihåg huru stor del af de tertiära formationernas fossila djurben blifvit upptäckta antingen i grottor eller i sötvattens-aflagringar och att icke en enda grotta eller verkligt sötvattenslager är kändt tillhörande de sekundära och palæozoiska perioderna.

Men ofullständigheten i de geologiska urkunderna härrör i hög grad från en annan vigtigare orsak än någon af de föregående, nämligen deraf, att de olika formationerna äro skilda från hvarandra af långa tidrymder. Denna sats har blifvit med eftertryck framhållen af många geologer och palæontologer, hvilka liksom E. Forbes alldeles icke trodde på arternas förvandling. Om vi betrakta formationerna såsom de äro ordnade i tabeller i vetenskapliga arbeten eller följa dem i naturen, så kunna vi svårligen tro annat än att de följa omedelbart på hvarandra. Men vi veta till exempel af Sir R. Murchisons stora arbete öfver Ryssland, hvilka stora luckor finnas emellan de olika efter hvarandra följande formationerna i detta land; så äfven i Nordamerika och i många andra delar af jorden. Den skickligaste geolog skulle, om hans uppmärksamhet varit inskränkt uteslutande till dessa stora områden, aldrig hafva anat, att under de perioder, hvilka i hans land voro tomma och kala, stora massor af sediment med nya och egendomliga lifsformer hopats på andra ställen. Och om i hvarje särskildt område svårligen en föreställning kan bildas om den tidrymd som förflutit emellan aflagringarna, så kunna vi antaga att detta ingenstädes kan ske. De talrika och stora förändringarna i den mineralogiska sammansättningen af följande formationer, som i allmänhet innefatta stora förändringar i det omgifvande landets geografi, hvarifrån sedimentet kommit, öfverensstämma med den åsigten, att vidsträckta tidrymder hafva förflutit emellan hvarje formation.

Men vi kunna såsom jag tror se hvarföre de geologiska formationerna i hvarje region äro nästan oföränderligen intermittenta, det vill säga icke hafva följt hvarandra utan afbrott. Knappast något öfverraskade mig mera, då jag undersökte många hundra mil af de sydamerikanska kusterna, hvilka inom en ny period hafva höjts flera hundra fot, än frånvaron af alla nya aflagringar af tillräcklig utsträckning att gälla för äfven en kort geologisk period. Längs hela vestkusten, som bebos af en egendomlig hafsfauna, äro tertiära lager så ofullständigt utvecklade att sannolikt intet minnesmärke af flera successiva och egendomliga hafsfaunor skola bevaras till en aflägsen tid. En liten reflexion skall förklara, hvarföre längs de stigande kusterna af vestra Sydamerika inga vidsträcktare formationer med nya eller tertiära qvarlefvor kunna finnas, ehuru sedimentaflagringen fortgått i stor skala i långa tider, såsom man kan se af den enorma förstöringen af kustklipporna och från de dyiga strömmar som flyta ned till hafvet. Förklaringen ligger otvifvelaktigt deruti, att de littorala och sublittorala aflagringarna oupphörligen sopas bort, så snart de genom landets långsamma höjning kommit inom området för hafsvågornas förstörande verkningar.

Vi kunna antaga, att sedimentet måste samlas i ytterligt tjocka, solida eller vidsträckta massor för att kunna motstå vågornas oupphörliga verkningar vid första höjningen och under successiva nivåförändringar jemte de följande verkningarna af atmosferen. Sådana tjocka och vidsträckta sedimentsamlingar kunna bildas på två sätt, såsom vid stort djup, i hvilket fall botten icke är bebodd af så många och vexlande former som i de mera grunda hafven; och då denna massa höjes, skall den gifva blott en ofullständig föreställning om de organismer, som lefde på jorden vid tiden för dess bildning; eller också kunna sediment aflagras till en viss tjocklek och utsträckning på en grund botten, om den är inbegripen i långsam sänkning. I senare fallet så länge sänkningen går i samma proportion som sedimentaflagringen, förblifver sjön grund och gynsam för många vexlande former och på detta sätt bildas en rik fossilförande formation tjock nog att vid höjning motstå hvarje grad af förstöring.

Jag är öfvertygad, att nästan alla våra gamla formationer, som alltigenom större delen af sin tjocklek äro rika på fossilier, hafva på detta sätt bildats vid fortgående sänkning. Sedan jag 1845 offentliggjorde mina åsigter i detta ämne har jag följt med geologiens framsteg och blifvit öfverraskad af att finna, huru den ena författaren efter den andra vid behandling af den eller den formationen kommit till den slutsatsen, att den blifvit bildad under sänkning. Jag kan tillägga, att de gamla tertiära formationerna på vestra kusten af Sydamerika, som blifvit tjocka nog att emotstå en sådan förstöring som hittills, men som knappt äro i stånd att räcka till aflägsnare geologiska perioder, blifvit afsatta under bottens sänkning och på detta sätt nått en ansenlig tjocklek.

Alla geologiska fakta förtälja oss tydligt, att hvarje yta har undergått talrika långsamma nivåförändringar och dessa hafva antagligen varit af stor utsträckning. Fossilrika formationer, tillräckligt tjocka och vidsträckta att motstå förstörelsen hafva följaktligen kunnat bildas öfver vidsträckta rymder under sänkningsperioder men blott på sådana ställen, der aflagringen var tillräcklig att hålla sjön grund och inbädda och skydda qvarlefvor innan de hunnit förstöras. Å andra sidan, så länge sjöbotten förblef stillastående kunde icke tjocka lager hafva samlats i de grunda delarna, som äro de för lifvet mest gynsamma. Ännu mindre kunde detta hafva händt under omvexlande höjnings- och sänkningsperioder, eller för att tala noggrannare, de lager, som under sänkningsperioden hade afsatt sig, skulle blifvit förstörda, sedan de genom höjningen kommit inom området för vågornas verkningar.

Dessa anmärkningar röra hufvudsakligen de littorala och sublittorala aflagringarna. I ett vidsträckt och grundt haf, såsom inom en stor del af malayiska arkipelagen, der djupet varierar från 30 à 40 till 60 famnar, har en vidsträckt formation kunnat bildas under en höjningsprocess utan att ännu lida synnerligen under den långsamma höjningen, men formationens tjocklek kunde icke vara stor, då den på grund af höjningen måste vara mindre än det djup der den bildats; ej heller skulle aflagringen vara mycket fast eller betäckt af öfverliggande formation, så att den skulle lätt kunna nötas bort af atmosferens förstörande inverkan och af hafvets verkningar under följande nivåförändringar. Hopkins har emellertid påstått, att om en del af ytan efter höjningen åter sänkt sig innan den blifvit blottad, det under höjningen bildade lagret, om det också icke är tjockt, skulle kunna skyddas af friska ansamlingar och på detta sätt bevaras under en lång period.

Hopkins uttalar vidare sin åsigt, att aflagringar af ansenlig utsträckning i horisontalriktningen sällan blifvit fullständigt förstörda. Men alla geologer, med undantag af de få som tro, att våra nuvarande metamorfiska skiffer och plutoniska klippor en gång bildade jordklotets ursprungliga kärna, skola medgifva att de senare klipporna blifvit till ofantlig grad blottade. Ty det är knappt möjligt, att sådana klippor kunnat blifva härdade och kristalliniska i obetäckt tillstånd; men om den metamorfoserande verksamheten yttrat sig på stort djup af oceanen, så skulle den fordna skyddande manteln icke behöft vara mycket tjock. Om vi då medgifva, att gneissen, glimmerskiffern, graniten, dioriten och så vidare en gång nödvändigt måste varit betäckta, huru kunna vi förklara de visträckta nakna ytorna af sådana klippor i många delar af jorden, så vida vi icke antaga, att de efteråt blifvit fullkomligt befriade från alla ofvanliggande lager? Att sådana vidsträckta ytor finnas kan icke betviflas; granitregionen på Parime beskrifves af Humboldt såsom åtminstone nitton gånger så stor som Schweitz. Söder om Amazonfloden visar Boués karta en af sådana klippor sammansatt yta, som är nästan så stor som Spanien, Frankrike, Italien, en del af Tyskland och Britiska öarna tillsammanstagna. Denna region har icke blifvit nog omsorgsfullt undersökt, men enligt resandes öfversstämmande intyg är granitytan mycket vidsträckt; von Eschwege gifver en detaljerad beskrifning af dessa klippor, som sträcka sig från Rio de Janeiro 260 geografiska mil inåt landet, och jag reste 150 mil i en annan riktning och såg ingenting annat än granitklippor. Talrika specimen samlade längs hela kusten från närheten af Rio Janeiro till mynningen af La Plata, ett afstånd af 1100 geografiska mil, undersökte jag och fann dem alla höra till samma klass. Inåt landet längs hela norra stranden af La Plata såg jag jemte yngre tertiära lager blott en liten fläck af obetydligt metamorfoserade klippor, som endast kunde hafva utgjort en del af granitklippornas ursprungliga betäckning. Vi skola nu vända oss till en väl bekant trakt, Förenta Staterna och Canada. Ur professor Rogers vackra karta har jag utklipt och vägt papperet och på detta vis bestämt, att de metamorfiska (undantagande de semi-metamorfiska) och granitiska klipporna i en proportion af 190 till 125 öfvervägde alla de nyare palæozoiska formationerna. I många trakter skulle de metamorfiska och granitklipporna synas i en vida större utsträckning om alla sedimentära lager aflägsnades, som hvila olikformigt på dem och icke hafva kunnat utgöra en del af den ursprungliga mantel, under hvilken de antagit sitt kristalliniska tillstånd. Derföre är det sannolikt, att i några delar af jorden hela formationer blifvit fullständigt blottade utan att bibehålla ett spår af sin fordna betäckning.

En anmärkning förtjenar att här anföras. Under höjningsperioden böra landytan och de närliggande grunda hafspartierna hafva förökats och nya boningsplatser uppkommit, — allt omständigheter, hvilka äro gynsamma för bildandet af nya varieteter och arter; men under sådana perioder blir i allmänhet en lucka i de geologiska urkunderna. Under sänkningen minskas å andra sidan den bebodda ytan och antalet invånare (med undantag af kustinvånarna på ett fastland, då det först afdelas i en mängd öar) och följaktligen skola under sänkningen många former dö ut, men få nya arter eller varieteter bildas, och det är under dessa sänkningsperioder som de på fossilier rikaste lagren blifvit afsatta.



\section[Om bristen på öfvergångsvarieteter]{Om bristen på öfvergångsvarieteter i hvarje särskild
formation.}

Efter alla dessa betraktelser kunna vi icke betvifla, att de geologiska urkunderna betraktade såsom ett helt äro ytterligt ofullständiga, men om vi inskränka vår uppmärksamhet till en särskild formation, blir det mycket svårare att förstå, hvarföre vi icke deri finna öfvergångsvarieteter emellan de beslägtade arterna som lefde vid dess början och vid dess slut. Flera exempel äro anförda på arter, som förete varieteter i öfre och nedre delarna af samma formation; Trautschold uppgifver ett antal exempel bland ammoniterna, och Hilgendorf har beskrifvit ett särdeles egendomligt fall af tio öfvergångsformer af Planorbis multiformis i de successiva lagren utaf Schweitz’ sötvattensformationer. Ehuru hvarje formation ovilkorligen behöft ett ofantligt antal år för sin bildning, kunna flera skäl anföras, hvarföre icke alla i allmänhet innesluta en graderad serie af föreningslänkar emellan de arter, som lefde vid dess början och vid dess slut; dock kan jag knappt gifva tillbörlig vigt åt följande betraktelser.

Ehuru hvarje formation betecknar en mycket lång följd af år, är hvar och en sannolikt kort i jemförelse med den tid som behöfves för att förvandla en art till en annan. Jag vet väl, att två palæontologer, hvilkas åsigter äro värda mycken uppmärksamhet, nämligen Bronn och Woodward hafva antagit, att hvarje formations varaktighet är två eller tre gånger så lång som artformernas. Men såsom mig synes hindra oss oöfvervinneliga svårigheter att komma till någon visshet i detta hänseende. Om vi se en art först uppträda midt i en formation, torde den slutsatsen vara förhastad, att den icke förut någorstädes existerat. Om vi å andra sidan se en art försvinna, förrän de sista lagren blifvit afsatta, torde det vara lika förhastadt att antaga, att den då dött ut. Vi glömma, huru liten Europas yta är i jemförelse med hela jordytan, och ej heller hafva de olika lagren af samma formation i hela Europa blifvit med fullkomlig noggranhet jemförda.

Bland hafsdjur af alla slag kunna vi med säkerhet antaga att i följd af klimatiska och andra förändringar vidsträckta flyttningar egt rum, och om vi se en art först uppträda i en ny formation, är det sannolikt att den då först inflyttade på ett nytt område. Det är väl bekant till exempel, att flera arter uppträdde något tidigare i de palæozoiska lagren i Nordamerika än i Europa; tydligen har en viss tid åtgått för deras flyttning från Amerikas till Europas haf. Vid undersökningen af de yngsta aflagringarna i flera delar af jorden har öfverallt anmärkts, att några få ännu existerande arter äro allmänna i aflagringen, men hafva dött ut i de kringliggande hafven, eller tvärtom, att några nu äro talrika i de närliggande hafven, men äro sällsynta eller saknas i dessa lager. Det är särdeles lärorikt att reflektera öfver den bekanta flyttningen bland Europas invånare under istiden, som bildar blott en del af en hel geologisk period; äfvensom att reflektera öfver nivåförändringarna, den ytterliga klimatförändringen, och den omätliga tidrymden, hvilket allt innefattas i samma isperiod. Det kan dock betviflas att i någon verldsdel sedimentära aflagringar med fossila qvarlefvor på samma område afsatt sig under hela denna period. Det är till exempel icke sannolikt att sediment afsattes under hela isperioden nära Missisippis mynning inom gränsen för det djup på hvilket hafsdjur bäst frodas, ty vi känna att stora geografiska förändringar inträffade vid denna tid i andra delar af Nordamerika. Om sådana lager som afsattes i grundt vatten nära Missisippis mynning under någon del af istiden höjas, skola organiska qvarlefvor sannolikt först uppträda och försvinna på olika höjd, på grund af arternas flyttningar och geografiska förändringar. Och en geolog som i en aflägsen framtid undersöker dessa lager skulle frestas att antaga, att de inbäddade fossiliernas lefnad öfverhufvudtaget hade varit af kortare varaktighet än istidens längd, då den i stället är vida längre, nämligen från tiden före isperioden ända till närvarande dag.

För att en fullkomlig gradering emellan två former i den öfre och lägre delen af samma formation skulle kunna finnas, borde aflagringen hafva oupphörligen fortgått under en mycket lång period, så att tillräcklig tid funnes för den långsamma modifikationsprocessen; derföre måste aflagringen vara mycket tjock och de i förändring inbegripna arterna måste lefva på samma område hela denna tid. Men vi hafva sett, att en mäktig formation, fossilförande genom hela sin tjocklek, kan bildas blott under en sänkningsperiod, och på det att djupet under hela tiden måtte vara i det närmaste oförändradt, hvilket är nödvändigt om samma hafsarter skola lefva på samma ställe, måste aflagringen hålla jemna steg med sänkningen. Men samma sänkningsrörelse bör försätta under vatten den yta hvarifrån sedimentet härstammar och på detta sätt förminska tillgången under det sänkningen fortgår. I sjelfva verket är denna nästan noggranna jemvigt emellan tillgången på sediment och sänkningen sannolikt något som sällan inträffar, ty mer än en palæontolog har iakttagit, att mycket tjocka aflagringar vanligen äro fattiga på organiska qvarlefvor med undantag af deras öfre eller nedre gränser.

Det vill synas som hvarje särskild formation likasom hela stapeln af formationer på ett visst område i allmänhet bildats med afbrott. Om vi se en formation sammansatt af lager af olika mineralogisk sammansättning, hvilket ofta är fallet, kunna vi med skäl misstänka, att aflagringsprocessen varit mycket afbruten, då en förändring i hafsströmmarna och tillförsel af sediment af olika beskaffenhet i allmänhet berott på geografiska förändringar, som fordra mycken tid. Ej heller skall den noggrannaste undersökning af en formation gifva någon föreställning om den tid som behöfts för dess afsättning. Många exempel kunde gifvas på lager, som äro blott några få fot i tjocklek, hvilka motsvara formationer som på andra ställen hafva tusentals fots mäktighet och som derföre måste behöft en ofantlig period för sin bildning; dock skulle ingen som icke kände detta förhållande ana hvilken ofantlig tid förflutit under det så tunna lagrets bildning. Många exempel kunde gifvas på att de lägre lagren i en formation blifvit höjda, blottade, åter sänkta och så betäckta med de öfre lagren af samma formation, fakta som visa hvilka vidsträckta, så lätt förbisedda mellantider som inträffat under formationens bildning. I andra fall hafva vi i de stora fossila träd, som ännu stå upprätt såsom de växte, de klaraste bevis på många långa mellantider och höjdförändringar under aflagringsprocessen, hvilka ingen någonsin skulle anat, såvida icke träden lyckats bibehålla sig. Så funno Sir C. Lyell och doktor Dawson i ett 1,400 fot mäktigt kolförande lager i Nya Skotland gamla med trädrötter genomdragna lager det ena öfver det andra på icke mindre än 68 olika höjder. Om samma arter påträffas i botten, på midten och på ytan af en formation, är det sannolikt att de icke lefvat på samma fläck under aflagringsperioden, utan hafva försvunnit och åter uppträdt kanhända många gånger under samma period. Om derföre sådana arter under någon geologisk period skulle undergå någon ansenlig grad af modifikation, skulle ett tvärsnitt icke innefatta alla de fina övergångsstadier, som enligt vår teori måste hafva funnits emellan dem, utan plötsliga ehuru kanhända små formförändringar.

Det är af största vigt att komma ihåg, att naturforskare icke hafva någon gyldene regel att åtskilja arter och varieteter; de medgifva en liten grad af föränderlighet hos arterna, men så snart de påträffa en större grad af skilnad emellan två former, upptaga de båda såsom arter, så vida de icke äro i stånd att sammanbinda dem genom fina öfvergångsstadier. Och på grund af nyss angifna skäl kunna vi sällan hoppas att åstadkomma detta i någon geologisk formation. Låt oss antaga B och C vara tvänne arter och att en tredje A finnes i ett äldre underliggande lager; om då A i karakter stode midt emellan B och C, skulle den helt enkelt upptagas såsom en tredje skild art, så vida den icke på samma gång vore nära förenad med endera af dem eller med begge formerna genom öfvergångsvarieteter. Vi få icke heller förglömma, att A kan vara den gemensamma stamfadern till både B och C utan att nödvändigt behöfva stå midt emellan dem i alla hänseenden. Vi kunna derföre finna stamarterna och deras modifierade ättlingar från de lägre och öfre lagren af samma formation och såvida vi icke påträffade talrika öfvergångsformer, skulle vi icke märka deras nära förvandtskap och vi skulle deraf föranledas att antaga dem för skilda arter.

Det är väl bekant på hvilka ytterligt små skiljaktigheter många palæontologer grunda sina arter, och de göra detta så mycket heldre, om exemplaren komma från skilda lager i samma formation. Många erfarna palæontologer hafva nu sänkt flera af D’Orbignys och andras fina arter till rangen af blotta varieteter och deri finna vi ett slags bevis på det förändringssätt, som vi enligt teorien böra finna. Må vi åter betrakta de senare tertiära aflagringarna, hvilka innesluta många snäckor, som flertalet naturforskare antaga vara identiska med nu lefvande arter, under det andra utmärkta vetenskapsmän såsom Agassiz och Pictet påstå dem vara specifikt skilda, ehuru skilnaden medgifves vara mycket ringa. Så vida vi icke tro, att dessa utmärkta naturforskare blifvit vilseledda af sin lifligare inbillningskraft och att dessa senare tertiära arter verkligen alldeles icke visa någon skilnad från deras nu lefvande representanter, eller så vida vi icke antaga den stora majoriteten af vetenskapsmän hafva orätt och att de tertiära arterna äro alla specifikt skilda från de nya, så hafva vi här ett bevis på, huru allmänt obetydliga modifikationer förekomma af önskadt slag. Om vi betrakta de större mellantider emellan skilda men på hvarandra följande lager af samma stora formation, finna vi att de inbäddade fossilierna, ehuru de nästan allmänt upptagas såsom specifikt skilda, äro vida mera beslägtade än arter som finnas i till tiden mera skilda formationer; så att vi här åter hafva otvifvelaktiga bevis för förändringar i den af teorien önskade riktningen, men till detta senare ämne skall jag återkomma i följande kapitel.

Såsom vi förut sett, hafva vi skäl att tro, att varieteterna af växter och djur som fortplanta sig hastigt och icke vandra mycket i början äro lokala, och att sådana lokala varieteter icke sprida sig vida omkring och uttränga sina stamarter förr än de blifvit modifierade och fulländade i någon ansenlig grad. Enligt denna åsigt är utsigten liten att i en formation i någon trakt upptäcka alla tidiga öfvergångsstadier emellan två former, ty de successiva förändringarna antagas hafva varit lokala eller inskränkta till någon liten fläck. De flesta hafsdjur hafva en vidsträckt utbredning, och vi hafva sett, att bland växter de som hafva vidsträcktaste utbredningen oftast förete varieteter; derföre är det sannolikt, att bland snäckor och andra hafsdjur de som haft största utbredningen, vida öfverskridande gränserna för de kända geologiska formationerna i Europa, oftast gifvit upphof först till lokala varieteter och sedan till nya arter, och detta bör åter i hög grad minska utsigten för oss att kunna uppspåra öfvergångsformerna i någon geologisk formation.

Det är en vida vigtigare omständighet, som leder till samma resultat, såsom doktor Falconer nyligen framhållit, att nämligen den period under hvilken hvarje art undergått modifikation, ehuru lång vid uppskattning i år, dock är kort i jemförelse med den tid under hvilken den förblef oförändrad.

Vi få icke glömma att i närvarande tid då vi hafva fullkomliga exemplar till undersökning två former sällan kunna förenas genom mellanstående varieteter och på detta sätt bevisas tillhöra samma art, förrän ett större antal exemplar samlats från många håll, och med de fossila arterna kan detta sällan åstadkommas af palæontologerna. Huru föga vi äro i stånd att sammanbinda arter med talrika fina fossila föreningslänkar, skola vi måhända bäst förstå, om vi till oss sjelfva ställa den frågan, huruvida geologer i en långt aflägsen framtida period skola vara i stånd att bevisa, att våra olika raser af nötkreatur, får, hästar och hundar härstamma från en enda eller från flera ursprungliga stammar, eller om vissa på kusterna af Nordamerika boende hafssnäckor, hvilka af några conchologer antagas såsom arter skilda från deras europeiska representanter och af andra såsom blotta varieteter, om dessa snäckor verkligen äro varieteter, eller såsom det kallas specifikt skilda. Geologen skulle i en framtid kunna bevisa detta endast genom att i fossilt tillstånd upptäcka talrika öfvergångsformer, och att detta skulle lyckas är i högsta grad osannolikt.

Författare som tro på arternas oföränderlighet hafva upprepade gånger försäkrat, att geologien icke lemnar några sammanbindande former. Detta är ett fullkomligt misstag. Sir J. Lubbock har sagt: ”hvarje art är en länk emellan andra beslägtade former.” Detta se vi tydligen, om vi taga ett slägte med ett tjog nya och utdöda former och tillintetgöra fyra femtedelar af dem, ty i detta fall betviflar ingen, att de återstående blifva mycket mera skilda från hvarandra. Om de yttersta arterna i slägtet på detta sätt förstöras, står slägtet sjelft i de flesta fall mera skildt från andra beslägtade genera. Kamelen och svinet, eller hästen och tapiren äro nu tydligen mycket skilda former, men om vi tillägga de fossila däggdjur som redan blifvit upptäckta af den familj som omfattar kamelen och svinet, så blifva dessa former förenade genom länkar, som icke hafva så stora luckor emellan sig. Kedjan af föreningslänkar går likväl icke i dessa eller något annat fall rakt från den ena lefvande formen till den andra, utan tager en omväg genom de former som lefvat under längesedan förflutna tider. Hvad de geologiska undersökningarna icke hafva uppenbarat är den fordna tillvaron af oändligt talrika öfvergångsformer, så fina som de nu existerande varieteterna, hvilka skulle sammanbinda de nu lefvande arterna med de utdöda. Detta böra vi icke vänta och det har dock upprepade gånger framstälts såsom en särdeles skarp invändning emot min teori.

Det kan vara behöfligt att sammanfatta föregående anmärkningar öfver orsakerna till de geologiska urkundernas ofullständighet uti ett tänkt exempel. Malayiska arkipelagen är ungefär af Europas storlek från Nord-Cap till Medelhafvet, och från Britannien till Ryssland, och motsvarar derföre utsträckningen af alla de geologiska formationer, hvilka med någon noggranhet blifvit undersökta, om vi undantaga Nordamerikas Förenta Stater. Jag öfverensstämmer fullkomligt med Mr Godwin-Austen, att den malayiska arkipelagen med dess talrika stora öar åtskilda af vidsträckta och grunda haf motsvarar Europas fordna tillstånd, under det de flesta af våra formationer bildades. Den malayiska arkipelagen är en af de på organiska varelser rikaste trakter på hela jorden, men om alla de arter hopsamlades, som hafva lefvat der, huru ofullständigt skulle de representera hela jordens naturalhistoria.

Men vi hafva allt skäl att tro, att arkipelagens landprodukter på ett ofullständigt sätt skola bibehållas i de formationer, som vi antaga vara under bildning. Icke många af de verkliga hafsdjuren eller af dem som lefva på nakna underhafsklippor, skulle blifva inbäddade, och de som inbäddades i sand eller grus skulle icke bibehålla sig till en aflägsen tid. Öfverallt der intet sediment afsatte sig på hafsbotten eller der sedimentet icke afsatte sig nog hastigt att skydda de organiska kropparna från förstörelse kunna icke heller några organiska qvarlefvor bibehållas.

Formationer rika på fossilier af många slag och af tillräcklig mäktighet för att ega bestånd in i en framtid så långt aflägsen som de sekundära formationerna ligga i det förflutna, skulle i allmänhet bildas i arkipelagen blott under sänkningsperioderna. Dessa sänkningsperioder skulle vara skilda från hvarandra genom omätliga mellantider, under hvilka ytan antingen höjde sig eller vore stillastående; under höjningen skulle de fossilförande lagren på de brantare stränderna förstöras nästan lika fort som de bildades genom vågornas oupphörliga inverkan, såsom vi nu se på kusterna af Sydamerika. Äfven på de vidsträcktare och grundare hafspartierna inom arkipelagen kunde sedimentära aflagringar näppeligen samlas till större mäktighet under höjningsperioderna eller betäckas och skyddas af följande aflagringar, så att de skulle kunna bibehålla sig in i en aflägsen framtid. Under sänkningsperioder, skulle sannolikt många lifsformer förstöras, under höjningsperioder skulle många varieteter bildas, men de geologiska minnesmärkena skulle vara föga fullständiga.

Det kan betviflas, att längden af någon större sänkningsperiod öfver hela eller en del af arkipelagen jemte en samtidig sedimentaflagring skulle öfverstiga de specifika formernas varaktighet i allmänhet, och detta vilkor vore oundgängligt för bevarandet af alla öfvergångsformer emellan två eller flera arter. Om alla sådana grader icke fullständigt vore bibehållna, skulle öfvergångsvarieteterna blott förefalla såsom lika många nya och bestämda arter. Det är också sannolikt, att hvarje större sänkningsperiod skulle afbrytas af nivåförändringar och att små klimatiska förändringar skulle försiggå under så långvariga perioder, och i dessa fall skulle arkipelagens invånare flytta, och intet fullständigt vittnesbörd om deras modifikationer kunde nedläggas i någon formation.

Många af arkipelagens hafsinvånare hafva nu en utbredning af flera tusen mil utöfver dess gränser, och analogien leder oss till den tron, att det hufvudsakligen skulle vara dessa arter med vidsträckt utbredning som oftast skulle alstra nya varieteter, och varieteterna skulle i allmänhet först vara lokala eller inskränkta till ett rum, men om de egde något afgjordt företräde eller om de vidare modifierades och förädlades skulle de långsamt sprida sig och uttränga sina stamformer. Om sådana varieteter återvända till sina fordna hem, skulle de afvika från sitt fordna utseende i nästan likformig ehuru kanhända ytterst ringa grad, och då de skulle finnas inbäddade i obetydligt afvikande underafdelningar af samma formation skulle de enligt de grundsatser många palæontologer följa upptagas såsom nya och skilda arter.

Om det finnes någon grad af sanning i dessa anmärkningar kunna vi icke med rätta vänta att i våra geologiska formationer finna ett oändligt antal af dessa fina öfvergångsformer, som enligt vår teori hafva förenat alla de utdöda och nu lefvande arterna af samma grupp i en lång, förgrenad lifskedja. Vi behöfva blott betrakta några få länkar och sådana finna vi helt säkert, några mera aflägset, andra helt nära beslägtade med hvarandra; och om dessa länkar, de må vara aldrig så nära beslägtade, finnas i olika afdelningar af samma formation, upptagas de af många palæontologer såsom arter. Men jag påstår icke, att jag någonsin skulle anat, huru torftiga de underrättelser äro, som kunna hemtas ur de bäst bibehållna geologiska formationerna, om icke frånvaron af oräkneliga öfvergångslänkar emellan de arter, som lefde vid formationens början och slut, bragt min teori i så stort trångmål.



\section[Beslägtade arter]{Om det plötsliga uppträdandet af hela grupper
beslägtade arter.}

Det plötsliga uppträdandet af hela artgrupper i vissa formationer har af flera palæontologer — till exempel Agassiz, Pictet och Sedgewick, — blifvit framhållet såsom en svår invändning emot åsigten om arternas förvandling. Om talrika arter som höra till samma slägten eller familjer verkligen hafva på en gång kommit till lif, skulle detta faktum vara tillintetgörande för teorien om härstamning med modifikation genom naturligt urval. Ty utvecklingen af en formgrupp, som helt och hållet härstammar från en stamfader, måste hafva varit en ytterst långsam process och stamfadern måste hafva lefvat långa tidrymder före sina modifierade ättlingar. Men vi öfverskatta oupphörligen fullständigheten af de geologiska urkunderna och draga den falska slutsatsen, att vissa slägten och familjer icke hafva funnits till före en viss afdelning af en formation, emedan man icke observerat dem förut. Ett positivt palæontologiskt bevis kan alltid mottagas med tillförsigt, men ett negativt är värdelöst, såsom erfarenheten så ofta visat. Vi glömma alltjemt huru stor jorden är i jemförelse med den yta, på hvilken våra geologiska formationer blifvit noggrant undersökta, vi glömma, att grupper af arter hafva kunnat finnas annorstädes, och långsamt förökat sig, innan de invandrade i Europas och Förenta Staternas gamla arkipelager. Vi lägga icke tillbörlig vigt på de enorma tidrymder som förflutit emellan våra formationer — längre kanhända i många fall än den tid, som behöfts för bildandet af hvarje formation. Dessa mellantider hafva gifvit tillfälle till arternas förökande från en eller några få stamformer, och i de följande formationerna synas sådana grupper såsom helt plötsligt skapade.

Jag vill här återkalla i minnet en förut gjord anmärkning, att det måtte behöfvas en lång följd af år för att göra en organism passande för ett nytt och egendomligt lefnadssätt till exempel att flyga genom luften; och att följaktligen öfvergångsformerna ofta varit inskränkta till en enda trakt, men att då några få arter på detta sätt förvärfvat en stor fördel öfver andra organismer, en jemförelsevis kort tid vore nödvändig att åstadkomma många divergenta former, som skulle sprida sig hastigt och vida utöfver jorden. Professor Pictet i sitt utmärkta referat af detta arbete vid omnämnandet af tidiga öfvergångsformer då han tager fåglar till exempel, kan icke se huru de successiva modifikationerna af en antagen prototyps främre extremiteter möjligen skulle kunnat vara af någon fördel. Men låt oss betrakta söderhafvets pinguiner; hafva icke dessa fåglar sina främre extremiteter i ett fullkomligt mellanstadium emellan ”hvarken verkliga armar eller verkliga vingar?” Dessa fåglar blifva dock segervinnare i kampen för tillvaron, ty de finnas i oändlig mängd och af många slag. Jag antager icke, att vi här se verkliga öfvergångsgrader som fågelvingarna genomgått; men hvad är det för synnerlig svårighet att tro, att det kunde gagna pinguinens modifierade ättlingar att först få förmåga att flaxa längs ytan af oceanen liksom Anas brachyptera och slutligen stiga från dess yta och sväfva genom luften?

Jag vill här anföra några få exempel för att upplysa föregående anmärkningar och visa huru benägna vi äro för villfarelse i det antagandet, att hela grupper af arter plötsligen framträdt. Äfven i en så kort period som emellan första och andra upplagorna af Pictets stora verk i Palæontologien, den ena utkommen 1844—46, den andra 1853—57, hafva åsigterna om flera djurgruppers första uppträdande och försvinnande blifvit betydligt modifierade, och en tredje upplaga skall sannolikt ytterligare behöfva ändras. Jag vill påminna om det välkända faktum, att i geologiska arbeten som utkommo för icke så många år sedan däggdjuren alltid ansågos hafva plötsligen uppträdt i början af tertiära perioden. Och en af de rikaste bland kända ansamlingar af fossila däggdjur hör nu till midten af den sekundära perioden, och verkliga däggdjur hafva blifvit upptäckta i den nya röda sandstenen nära början af denna stora formation. Cuvier plägade påstå, att ingen apa fans i något tertiärt lager, men numera hafva utdöda arter upptäckts i Indien, Sydamerika, och i Europa ända bort i miocenlagret. Om icke fotspår händelsevis bibehållits i den nya röda sandstenen i Förenta Staterna, hvem skulle vågat antaga, att jemte reptilierna icke mindre än åtminstone trettio fågelarter, några af jättestorlek, existerat under denna period? Intet spår till ben har blifvit upptäckt i dessa lager. Oaktadt antalet leder som de fossila fotspåren visa öfverensstämma med antalet leder på nu lefvande fåglars tår, betvifla dock några författare, att de djur, som lemnat dessa intryck voro verkliga fåglar. Ända till helt nyligen hafva dessa författare kunnat påstå och några hafva gjort det, att hela fågelklassen plötsligen uppträdde under eocenperioden, men vi känna nu på professor Owens auktoritet, att en fågel helt säkert lefde under den öfre grönsandens afsättning; och ännu senare hafva vi lärt känna en sällsam fogel, Archeopteryx, med en lång ödlelik svans med ett par fjädrar vid hvarje led, och vingarna försedda med två fina klor; denna fågel har blifvit upptäckt i Solenhofens oolitiska skiffer. Svårligen kan någon ny upptäckt kraftigare än denna visa huru litet vi hittills känna af jordens fordna inbyggare.

Jag kan anföra ett annat exempel, som mycket förvånat mig, då det försiggått under mina egna ögon, och hvilket jag redan behandlat i en uppsats öfver fossila cirripeder. Det stora antalet lefvande och utdöda tertiära arter, den utomordentliga rikedomen på individer af många arter öfver hela jorden från de arktiska regionerna till eqvatorn, som bebo olika djup ända till 50 famnar från flodens högsta höjd, det fullständiga bibehållandet af exemplar i de äldsta tertiära lager, lättheten att igenkänna äfven stycken af en valvel, alla dessa omständigheter ledde mig till den slutsatsen, att om sessila cirripeder funnits under den sekundära perioden, de helt säkert skulle bibehållits och upptäckts, och då icke en enda art blifvit upptäckt i lager af denna period, antog jag att denna stora grupp plötsligen uppträdt vid början af de tertiära formationerna. Detta var en stor förlägenhet för mig, då det såsom jag trodde var ett exempel till på stora artgruppers plötsliga uppträdande. Men mitt arbete hade knappt blifvit utgifvet förrän en utmärkt palæontolog, Bosquet, sände mig en ritning af ett fullständigt exemplar af en omisskännelig sessil cirriped, som han sjelf funnit i kritan i Belgien. Och för att göra fallet så slående som möjligt var den sessila cirripeden en Chthalamus, ett mycket allmänt, stort och vidt utbredt slägte, af hvilket icke ett enda exemplar funnits hittills ens i tertiära lager. Deraf veta vi bestämdt, att sessila cirripeder existerade under den sekundära perioden, och dessa cirripeder kunna hafva varit stamfäder till våra många tertiära och lefvande arter.

Det af palæontologerna oftast åberopade exempel på en hel artgrupps plötsliga uppträdande är de s. k. benfiskarna (Teleostei), som förekomma långt ned i kritaperioden. Denna grupp innefattar det stora flertalet nu existerande arter. Professor Pictet har nyligen förflyttat deras förekomst ännu ett steg längre tillbaka, och några palæontologer tro, att vissa af de ännu äldre fiskarna, hvilkas slägtskapsförhållanden äro ofullständigt kända, äro verkliga benfiskar. Om vi antaga, att hela denna grupp uppträdde, såsom Agassiz påstår, vid början af kritaformationen, vore detta förhållande visserligen högst märkvärdigt; men jag kan ej finna, att det vore ett oöfvervinneligt inkast emot dessa åsigter, med mindre det kunde bevisas, att arterna af denna grupp likaledes uppträdde hastigt och samtidigt öfver hela jorden på samma period. Det är nästan öfverflödigt att anmärka, att knappt en enda fossil fisk är känd från trakten söder om eqvatorn, och om vi genomgå Pictets palæontologi, skola vi finna, att blott få arter äro kända från flera formationer i Europa. Några få fiskfamiljer bebo nu ett inskränkt område; benfiskarna kunna hafva haft lika inskränkt utbredningsområde och hafva spridt sig vida omkring, sedan de i ett mindre haf vunnit sin utveckling. Vi hafva icke rätt till det antagandet, att hafven alltid hafva varit så öppna från norr till söder som nu. Just i närvarande tid, om den malayiska arkipelagen förvandlades till land, skulle de tropiska delarna af indiska oceanen bilda en vidsträckt och fullständigt omsluten bassin, i hvilken mången stor grupp af hafsdjur kunde förökas, och der skulle de vara instängda till dess några af arterna blefve lämpade för ett kallare klimat och kunde passera de södra uddarna af Afrika och Australien och på detta sätt uppnå andra aflägsna sjöar.

Af dessa betraktelser, af vår bristfälliga kännedom om andra länders geologi utom Europas och Förenta Staternas och af den omstörtning i våra palæontologiska åsigter som de senare tio årens upptäckter hafva verkat synes det mig lika djerft att uppsätta dogmer öfver de organiska varelsernas uppträdande på jorden, som för en naturhistoriker att efter fem minuters landstigning på en kal trakt i Australien afhandla antalet och utbredningen af landets naturalster.



\section[Beslägtade arter i de lägsta lagren]{Om det plötsliga uppträdandet af grupper af beslägtade
arter i de lägsta kända fossilförande lagren.}

Det finnes ännu en med den förra sammanhängande svårighet som är ännu mera allvarsam. Jag menar det sätt, hvarpå många arter i flera af de stora hufvudafdelningarna i djurriket plötsligt uppträda i de lägsta fossilförande lagren. De flesta af de argument, som öfvertygat mig att alla nu lefvande arter af samma grupp härstamma från en gemensam stamfader, låta nästan med samma kraft tillämpa sig på de äldsta kända arter. Det kan till exempel icke betviflas, att alla siluriska trilobiter härstamma från samma krustacé, som måste hafva lefvat långt före den siluriska perioden, och som sannolikt föga liknade något kändt djur. Några af de äldsta siluriska djur såsom Nautilus, Lingula m. fl. skilja sig obetydligt från de nu lefvande formerna, och enligt vår teori kan icke antagas, att dessa gamla arter voro stamfäder till alla arter af samma grupper som sedermera framträdt, ty de äro icke i någon mån mellanformer i sina karakterer.

Följaktligen, om teorien är sann, är det obestridligt, att långa tider förflöto förrän de äldsta siluriska eller cambriska lagren afsattes, lika länge, eller sannolikt längre än hela tiden emellan den cambriska formationen och närvarande tid, och att under hela denna långa period jorden hvimlade af lefvande varelser. Här påträffa vi en fruktansvärd invändning, ty det synes tvifvelaktigt att jorden under så lång tid varit i stånd att bära lefvande varelser. W. Thompson antager, att jordskorpans stelnande svårligen kan hafva inträffat för mindre än 20 eller mer än 400 millioner år tillbaka, men antagligen emellan 98 och 200 millioner år. Dessa vidsträckta gränser visa huru tvifvelaktiga data äro och andra moment torde böra intagas i problemet. Croll beräknar att ungefär 60 millioner år förflutit sedan den cambriska perioden, men att döma af de obetydliga förändringar inom organismernas verld sedan istiden synes denna tid vara allt för kort för de många och stora förändringar af lefvande varelser som helt säkert försiggått sedan den cambriska perioden; och de förut förflutna 140 millioner år kunna näppeligen betraktas såsom tillräckliga för utbildandet af de olika lefvande former, som helt säkert existerade emot slutet af cambriska perioderna.

Man kan fråga, hvarföre vi icke finna rika fossilförande aflagringar i dessa antagna tidigare perioder, men härpå kan jag icke gifva något tillfredsställande svar. Några utmärkta geologer med Murchison i spetsen hyste ända intill senare tider den öfvertygelsen, att vi i de organiska qvarlefvorna från de lägsta siluriska lagren se lifvets första gryning. Andra lika kompetenta domare såsom Lyell och E. Forbes hafva bestridt detta antagande. Vi få icke förglömma, att blott en ringa del af jorden är känd med noggranhet. För icke länge sedan har Barrande till det gamla siluriska systemet lagt ett annat ännu lägre, rikt på nya och egendomliga arter. Qvarlefvor af flera former hafva också upptäckts under Barrandes så kallade primordialzon i Longmyndgruppen, som nu delas i två lager och utgör det undre cambriska systemet. Närvaron af fosfathaltiga stycken och bituminösa ämnen i några af de lägsta azoiska klipporna tillkännagifver sannolikt lif vid denna period. Nu har på sista tiden den stora upptäckten af Eozoon skett i den Laurentiska formationen i Canada, ty efter Carpenters beskrifning på detta fossil kan man ej gerna tvifla på dess organiska natur. Under det siluriska systemet finnas i Canada stora serier af lager och i det lägsta af dem har Eozoon blifvit funnen; Sir W. Logan försäkrar att ”deras förenade tjocklek kan vida öfverstiga alla de följande klippornas tjocklek från nedersta delen af den palæozoiska serien intill närvarande tid. Vi föras sålunda tillbaka till en så aflägsen period, att uppträdandet af den så kallade primordialfaunan (Barrande’s) kan betraktas som jemförelsevis ung.” Eozoon hör till den lägst organiserade af alla djurklasser, men står i sin klass på en hög organisationsgrad; den lefde i oräknelig mängd och som Dawson har anmärkt, helt säkert af andra mindre organiska varelser, som måste hafva lefvat i stor mängd. De ord jag skref år 1859 om de vidsträckta perioder som sannolikt förlupit före det cambriska systemets bildande äro nästan desamma som Sir W. Logan sedan begagnat. Icke destomindre är det svårt att angifva något antagligt skäl för frånvaron af fossilrika lager under de öfre cambriska formationerna. Det synes icke sannolikt att de äldsta lagren skulle blifvit fullständigt afnötta genom denudation, eller att deras fossilier skulle blifvit fullkomligt tillintetgjorda genom metamorfosering, ty om detta hade varit fallet, skulle vi hafva funnit blott små spår af formationerna som följde närmast efter dem i tiden, och dessa skulle alltid hafva existerat i delvis metamorfoseradt tillstånd. Men de beskrifningar vi ega öfver de siluriska aflagringarna i omätliga områden i Ryssland och Nordamerika gifva icke något stöd för den åsigten, att ju äldre en formation är, desto mer måste den hafva lidit genom denudation och metamorfism.

Detta förhållande måste tillsvidare stå oförklaradt, och skall med rätta utgöra ett kraftigt argument emot de åsigter som här framhållas. För att visa att det kan framdeles erhålla sin förklaring vill jag dock uppställa följande hypotes. Från beskaffenheten af de organiska qvarlefvor som icke synas hafva bebott stora djup i de olika formationerna i Europa och Nordamerika, och från mängden af aflagringar, ofta milstjocka, af hvilka formationerna äro bildade, kunna vi draga den slutsatsen att från de första till de sista stora öar eller landområden, hvarifrån sedimentet härstammade, funnos i trakten af de nu existerande fastlanden Europa och Nordamerika. Men vi känna icke hvad sakernas ställning var under tiden emellan de olika följande formationerna; antingen Europa och Förenta Staterna under dessa mellantider existerade såsom torra landet eller såsom en submarin yta nära land, på hvilken intet sediment afsattes, eller såsom botten i ett öppet och omätligt haf.

Om vi betrakta de nuvarande oceanerna, som äro tre gånger så vidsträckta som landet, finna vi dem besådda med många öar; men icke en enda verklig hafsö (med undantag af Nya Zeeland, om den kan kallas en verklig hafsö) företer, så vidt man hittills känner, något spår af en palæozoisk eller sekundär formation. Deraf kunna vi möjligen antaga, att under den palæozoiska och sekundära perioden hvarken kontinenter eller kontinentala öar funnos der våra oceaner nu sträcka sig, ty om sådana hade funnits skulle med all sannolikhet palæozoiska och sekundära formationer bildats af sediment som uppkommit genom deras afnötning, och dessa skulle åtminstone delvis blifvit upplyftade under de nivåförändringar som måste hafva inträffat under dessa ofantliga perioder. Om vi kunna draga några slutsatser af dessa förhållanden, så skola vi antaga, att der våra oceaner nu finnas, der hafva oceaner funnits från de aflägsnaste perioder, om hvilka vi hafva någon kännedom och å andra sidan, att der kontinenter nu finnas, der hafva alltid vidsträckta landområden funnits alltsedan den tidigaste siluriska perioden, ehuru otvifvelaktigt underkastade stora nivåförändringar. Den färglagda karta som medföljer mitt arbete om korallrefven har gifvit anledning till min åsigt, att de stora oceanerna ännu äro stora sänkningsytor, de stora arkipelagerna föränderliga områden och kontinenterna höjningsytor. Men vi hafva intet skäl att antaga, att förhållandet varit detsamma från verldens begynnelse. Våra fastland synas hafva bildats genom den höjande kraftens öfvervigt under många nivåförändringar, men kunna icke dessa ytor med öfvervägande höjning hafva förändrats under tidernas lopp? Vid en period som föregick den siluriska tiden hafva kontinenter kunnat existera der oceaner nu utbreda sig; och fria och öppna oceaner kunna hafva funnits der våra kontinenter nu stå. Och dock vore man icke berättigad att antaga, att, om till exempel botten af Stilla hafvet nu förvandlades till land, vi der skulle finna sedimentära formationer i ett tillstånd som tillkännagaf deras de siluriska lagrens öfverstigande ålder, förutsatt att sådana blifvit afsatta; ty det kan väl hända, att lager som sänkt sig några mil närmare jordens medelpunkt och som varit underkastade trycket af det derofvan liggande vattnets enorma vigt, torde hafva undergått vida större metamorfosering än lager som alltid stannat närmare ytan. De ofantliga ytorna i några delar af jorden, till exempel i Sydamerika, af nakna metamorfiska klippor, som måste blifvit upphettade under starkt tryck, hafva alltid synts mig behöfva en särskild förklaring, och vi kunna måhända tänka oss, att vi i dessa vidsträckta ytor se de många formationer som föregingo den siluriska perioden i ett blottadt och metamorfiskt tillstånd.

De svårigheter som här blifvit behandlade — nämligen att ehuru vi i våra geologiska formationer finna många länkar emellan de arter som nu finnas och som fordom funnits, vi dock icke finna oändligt fina öfvergångsformer som sammanbinda dem med hvarandra; — det plötsliga uppträdandet af flera hela grupper af arter i våra Europeiska formationer; — den nästan fullkomliga frånvaron, så vidt vi hittills känna, af fossilrika formationer under de cambriska lagren — äro alla otvifvelaktigt af den allvarligaste beskaffenhet. Vi se detta i den omständigheten, att de utmärktaste palæontologer, Cuvier, Agassiz, Barrande, Pictet, Falconer, E. Forbes m. fl. och alla våra största geologer såsom Lyell, Murchison, Sedgewick m. fl. hafva enhälligt ofta med häftighet fasthållit vid arternas oföränderlighet. Men Sir Charles Lyell lemnar nu stödet af sin stora auktoritet åt motsatta sidan, och de flesta andra geologer och palæontologer hafva blifvit betydligt rubbade i sin tro. De som antaga, att de geologiska urkunderna äro till någon grad fullständiga, skola otvifvelaktigt med ens tillbakavisa teorien. Jag för min del betraktar (för att följa Lyells bildlika uttryck) den geologiska urkunden, såsom en ofullständig verldshistoria, skrifven på vexlande dialekter; af denna historia ega vi blott den sista volymen, innefattande blott två eller tre länder. Af denna volym har blott här och der ett enstaka kapitel blifvit bibehållet och af hvarje sida blott här och der ett par rader. Hvarje ord af det långsamt vexlande språket, mer eller mindre olika i de följande kapitlen, kan föreställa de lifsformer som äro inbäddade i våra formationer och hvilka falskeligen synas oss blifvit införda med afbrott. Enligt denna åsigt förminskas i hög grad eller försvinna helt och hållet de ofvan afhandlade svårigheterna.


