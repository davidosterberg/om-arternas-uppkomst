\cleardoublepage
\phantomsection
\addcontentsline{toc}{chapter}{Inledning}
\chapter*{Inledning.} 

Då jag såsom naturforskare åtföljde engelska skeppet Beagle och landsteg i Sydamerika, blef jag i hög grad öfverraskad af vissa egendomligheter vid fördelningen af de organiska varelser som bebo landet och förhållandet emellan de utdöda och de nu lefvande organiska generationerna. Såsom läsaren får se af sista kapitlen af detta arbete, syntes mig dessa förhållanden sprida något ljus öfver arternas uppkomst, detta mysteriernas mysterium, såsom en af våra största filosofer kallat det. Efter min hemkomst år 1837 kom jag på den tanken, att denna fråga skulle kunna bringas ett steg framåt genom ett ihärdigt hopsamlande och öfvervägande af alla de fakta, som möjligen egde något sammanhang med frågan. Först sedan jag i fem års tid varit sysselsatt dermed, trodde jag mig kunna gå djupare i saken och skref några kortare meddelanden deröfver, hvilka jag vidare utarbetade under loppet af år 1844, och till utkastet bifogade jag några reflexioner, hvilka syntes mig temligen pålitliga. Från denna tid har jag ända till nu varit sysselsatt med oafbrutna forskningar öfver detta ämne. Jag hoppas man har öfverseende med dessa detaljer som röra min person; afsigten med deras anförande är blott att visa, att jag icke förhastat mig i att bringa frågan till afgörande.

Mitt verk är nu nära fulländadt; men då det ännu behöfs två eller tre år för att komplettera detsamma och min helsa är opålitlig, så har man nödgat mig att offentliggöra detta utkast. Jag såg mig så mycket mera föranlåten dertill, som Wallace vid studiet af den malayiska arkipelagens naturhistoria kommit till ungefär samma resultat som jag angående arternas bildning. År 1858 sände han mig en afhandling deröfver med begäran att tillställa den sir Charles Lyell, som sände den till Linnean society, i hvars journal den nu finnes införd. Sir Charles Lyell, äfvensom doktor Hooker, hvilka båda kände till mina arbeten (den sednare hade läst mitt utkast af år 1844), gåfvo mig det rådet att offentliggöra ett kort utdrag ur mina manuskript på samma gång som Wallaces afhandling.

Detta utdrag, som jag härmed framlägger för läsaren, måste nödvändigt vara ofullkomligt. Jag kan icke anföra några auktoriteter för mina särskilda uppgifter, och jag måste bedja läsaren sätta sin lit till min noggrannhet. Tvifvelsutan hafva äfven misstag insmugit sig, dock tror jag mig allestädes hafva åberopat blott tillförlitliga källor. Jag kan här öfverallt blott anföra de allmänna slutsatser till hvilka jag kommit, bifogande endast få upplysande fakta, men dessa skola, som jag hoppas, i de flesta fall göra tillfyllest. Ingen kan bättre än jag inse nödvändigheten af att med alla deras detaljer meddela alla fakta, på hvilka mina resultat stödja sig, och jag hoppas få göra det framdeles i ett annat arbete\footnote{År 1868 utkom i två volymer Darwins stora arbete, ”On the variations of animals and plants under domestication”, och i inledningen dertill lofvar författaren att framdeles utförligt behandla hvarje kapitel i detta verk.
Ö. a.}. Ty jag vet väl, att i denna bok kommer att afhandlas knappt en enda punkt, vid hvilken man icke kunde anföra fakta som synas leda till rakt motsatta åsigter. Men ett riktigt resultat vinnes blott genom att sammanställa alla grunder och fakta, som tala för och mot hvarje enskild fråga, och noga afväga dem emot hvarandra, men detta kan icke rätt väl ske här.

Då en naturforskare reflekterar öfver arternas uppkomst och der\-vid tager i betraktande organismernas ömsesidiga slägtskap, deras embryonala förhållanden, deras geografiska utbredning, deras uppträdande i olika geologiska perioder och andra dylika omständigheter, så är det lätt begripligt, att han kommer på den tanken, att arterna icke äro skapade oberoende af hvarandra, utan härstamma från andra arter, på samma sätt som varieteter. Det oaktadt torde denna åsigt, till och med om den vore välgrundad, icke hafva full giltighet, så länge det icke vore möjligt att visa, hvilka förändringar de talrika arter som nu bebo vår jord hafva undergått, så att de kunnat vinna den fullkomliga och efter deras lefnadsförhållanden så väl lämpade byggnad, hvilken med rätta ådrager sig vår beundran. Naturforskare hänvisa ständigt på yttre omständigheter, klimat, näringsämnen o. s. v. såsom de enda möjliga orsakerna till arternas variationer. I mycket inskränkt mening kan detta vara sant, såsom vi framdeles få se. Men man skulle gå för långt, om man ville i dessa yttre omständigheter se orsaken t. ex. till hackspettens organisation, bildningen af hans fot, hans stjert, hans näbb och hans tunga, hvilket allt sätter honom i stånd att plocka fram insekter under trädens bark. Samma vore förhållandet med misteln, en växt, som hemtar sin näring från vissa träd, och hvars frön måste spridas af vissa fåglar, och hvars blommor, som äro skildkönade, behöfva vissa insekters biträde för att öfverflytta hanblommornas frömjöl på honblommorna, — det vore äfven att gå för långt att betrakta denna parasitväxts organisation, med dess beroende af de nämda helt olika organiska varelserna, såsom en verkan af yttre omständigheter eller sjelfva växtens vilja eller vana.

Derföre är det af största vigt att vinna en klar insigt uti de medel, genom hvilka sådana förändringar kunna åstadkommas. Då jag började mina iakttagelser syntes det mig sannolikt, att ett sorgfälligt studium af husdjuren och kulturväxterna skulle lemna den bästa utväg till att lösa denna svåra uppgift. Och jag har icke misstagit mig, utan i detta, som i alla invecklade fall, har jag funnit, att studiet af de förändringar djuren och växterna undergå i kulturtillståndet har lemnat den bästa och säkraste ledtråd. Jag kan icke nog starkt framhålla vigten af sådana studier, hvilka dock af naturforskare i allmänhet mycket försummas.

Af denna orsak egnar jag då första kapitlet af detta utkast åt förändringarna i kulturtillståndet. Vi skola deraf se, att ärftliga förändringar i stor utsträckning äro möjliga, och något som icke är mindre vigtigt, menniskans förmåga att genom urval till afvel (”selection”) på afkomman småningom samla en massa små förändringar. Derefter skall jag öfvergå till arternas föränderlighet i naturtillståndet, men jag är olyckligtvis nödgad att behandla detta ämne i korthet, då det blott genom att meddela hela listor af fakta kan något fullständigare behandlas. Vi skola dock vara i tillfälle att framhålla de omständigheter, som mest gynna variationerna. I nästa afdelning afhandlas kampen för tillvaron emellan de organiska varelserna, hvilken ovederläggligt framgår ur förökningens fortgång i geometrisk progression. Det är Malthus’ lära tillämpad på hela växt- och djurriket. Då det stora antal individer, som framfödas af hvarje art, icke kunna fortfarande blifva vid lif och i följd deraf kampen för tillvaron oupphörligen förnyas, så följer, att en individ, som helt obetydligt afviker från de öfriga, men som kan hafva direkt fördel af denna lilla olikhet, alltid har mera utsigt att bibehålla sin tillvaro under mångfaldiga och föränderliga lefnadsvilkor, och derföre blir af naturen utvald. Enligt den stränga ärftlighetslagen sträfvar alltid en sådan af naturen utvald varietet att fortplanta sin nya och afvikande form.

Detta naturliga urval är hufvudföremålet för fjerde kapitlet och skall der utförligt behandlas; och vi skola då finna, att det naturliga urvalet är anledningen till de mindre väl utrustade organismernas undergång och på detta sätt åstadkommes hvad jag kallat karakterens divergens. I den derpå följande afdelningen komma att behandlas de invecklade och föga kända lagarna för föränderligheten. I de fyra följande kapitlen skall jag söka angifva de mest betydande svårigheterna för vår teori och först och främst svårigheten att förklara öfvergångarna, eller huru man skall förstå, att ett enkelt väsen eller organ kan ombildas till ett högre utveckladt väsen eller ett högre utbildadt organ; vidare djurens instinkt eller själsförmögenheter, för det tredje bastardbildning eller kroaserade arters ofruktsamhet och kroaserade varieteters fruktsamhet, och för det fjerde våra geologiska kunskapers ofullständighet. I nästa kapitel skall jag uppehålla mig vid organismernas successiva uppträdande i olika geologiska perioder, i elfte och tolfte kapitlet deras geografiska utbredning, i det trettonde deras klassifikation eller ömsesidiga slägtskapsförhållanden i utbildadt och i embryonalt tillstånd. I sista kapitlet skall jag slutligen gifva en kort sammanfattning af hela arbetets innehåll med några slutanmärkningar.

Ingen skall förundra sig öfver, att ännu så mycket är oförklaradt angående arternas och varieteternas ursprung, om han tager i betraktande vår ringa kännedom om alla de omkring oss lefvande varel\-sernas ömsesidiga förhållanden. Hvem kan förklara, hvarföre en art förekommer i stort antal och öfver ett stort område, under det en närstående art är sällsynt och inskränkt till ett litet rum. Och dock äro dessa förhållanden af största vigt så till vida, att de betinga såsom jag tror de nu lefvande varelsernas välfärd och deras efterkommandes framtida trefnad. Men ännu mindre känna vi förhållandet mellan denna jords otaliga invånare under dess bildningshistorias talrika perioder. Om derföre mycket ännu är dunkelt och ännu länge skall förblifva dunkelt, så hafva dock de noggrannaste studier och det mest fördomsfria omdöme hvaraf jag är mäktig för mig undanröjt alla tvifvel om, att den åsigt är oriktig, hvilken de flesta naturforskare hysa, och hvilken äfven jag en lång tid hyst, att nämligen hvarje art är skapad oberoende af de öfriga. Det är min fulla öfvertygelse, att arterna icke äro oföränderliga; att alla de till ett så kalladt slägte hörande arterna i rätt nedstigande linie härstamma från en vanligen utdöd art, på samma sätt som varieteterna af en art härstamma från denna. Slutligen är jag öfvertygad om, att det naturliga urvalet har varit det hufvudsakligaste om icke det enda medlet till organismernas förändring.


