% Baserat på 
% http://sv.wikisource.org/w/index.php?title=Om_arternas_uppkomst_genom_naturligt_urval_eller_de_b%C3%A4st_utrustade_rasernas_best%C3%A5nd_i_kampen_f%C3%B6r_tillvaron/Kapitel_1&oldid=98985


%FÖRSTA KAPITLET.

\chapter[Kulturtillståndet]{Arternas förändring i kulturtillståndet.}

{\it
Orsaker till föränderlighet. — Vanans verkningar. — Utvecklingens vexelverkan. — Ärftlighet. — Domesticerade varieteters karakter. — Svårighet att skilja mellan arter och varieteter. — Varieteters uppkomst af en eller flera arter. — Tama dufvor, deras olikheter och ursprung. — Grundsatser som förr blifvit följda vid urval till afvel och följderna deraf. — Planmässigt och omedvetet urval. — Våra domesticerade rasers ursprung. — Omständigheter, som gynna menniskans valförmåga.
}\\[0.5cm]


\section{Orsaker till föränderlighet.}

Om vi betrakta flera individer af en varietet eller undervarietet af våra gamla kulturväxter och husdjur, så faller oss först i ögonen, att de i allmänhet afvika från hvarandra mera än individerna af en art eller varietet i naturtillståndet. Betänka vi nu, huru mångfaldiga kulturväxter och husdjur hafva i alla tider lefvat under de mest olika klimat och undergått den mest olika behandling, så tvingar sig den tanken på oss, tror jag, att våra kulturalsters stora föränderlighet är verkan af mindre enformiga lefnadsförhållanden, som afvika från stamarternas i vilda tillståndet. Någon sannolikhet finnes äfven för Andrew Knights mening, att denna föränderlighet till en del står i sammanhang med en rikligare tillgång på föda. Det synes äfven vara klart, att organismerna igenom flera generationer måste lefva under dessa nya förhållanden, innan någon märklig grad af variation kan framträda hos dem, och att föränderligheten fortgår under flera generationer, om organisationen en gång har börjat variera. Man känner intet exempel på, att en föränderlig organism i kulturtillståndet förlorat sin föränderlighet. Våra äldsta kulturväxter såsom hvetet gifva ofta ännu i dag nya varieteter, och våra äldsta husdjur äro alltjemnt mäktiga af förändring och förädling.
Så vidt jag är i stånd att döma efter ett långvarigt studium af ämnet, synas lefnadsförhållandena verka på två vägar, — direkt på hela organisationen eller blott på vissa delar, och indirekt genom att förorsaka rubbningar i det reproduktiva systemet. Med afseende på den direkta verkan måste vi komma ihåg, att i hvarje fall äro två faktorer verksamma, organismens beskaffenhet och lifsvilkorens beskaffenhet, såsom Prof. Weismann nyligen förfäktat och jag tillfälligtvis visat i mitt arbete ”Variation under domestication”. Den förra synes vara den vigtigaste, ty nästan lika variationer uppkomma under så vidt vi kunna döma olika förhållanden, och å andra sidan uppstå olika variationer under vilkor som synas vara fullkomligt likformiga. Verkningarna på afkomman äro antingen bestämda eller obestämda. De kunna anses såsom bestämda, då alla eller nästan alla afkomlingar under flera generationer modifieras på samma sätt. Det är ytterst svårt att komma till någon slutsats om härledningen af de förändringar som på detta sätt blifvit bestämdt inledda; föga tvifvel kan dock vara om många små variationer, såsom att storleken beror på graden af föda, färgen på födans beskaffenhet, hudens och pelsens tjocklek på klimatet etc. Hvar och en af de ändlösa variationer som vi se i våra hönsfåglars fjäderbeklädnad måste hafva någon verkande orsak, och om samma orsak skulle verka likformigt under en lång serie af generationer på många individer, skulle sannolikt alla modifieras på samma vis. Sådana fakta som den invecklade och utomordentliga utväxten, som oföränderligen följer på indrypandet af en den ringaste droppa gift från en galläpplebildande insekt, visa oss, hvilka egendomliga modifikationer kunna härröra från en kemisk förändring i växternas saft.

Obestämd föränderlighet är en mycket mera vanlig följd af förändrade förhållanden än bestämd föränderlighet och har sannolikt spelat en mera vigtig rol i bildandet af våra domesticerade raser. Obestämd föränderlighet se vi i de ändlösa små egendomligheter som skilja individerna af samma art och som icke kunna skrifvas på ärftlighetens räkning, vare sig från föräldrarna eller från någon mera aflägsen stamfader. Äfven starkt markerade skilnader uppträda ofta emellan ungarna af samma kull och i frön från samma frökapsel. På långa mellantider uppkomma bland millioner individer uppfödda i samma trakt och på nästan samma födoämnen afvikelser i bildning så starkt utpräglade att de förtjena att kallas monstrositeter, men monstrositeter kunna icke med någon skarp gräns skiljas från mindre varieteter. Alla sådana förändringar i struktur antingen ytterligt små eller starkt utpräglade, hvilka uppträda ibland många individer som lefva tillsammans, kunna betraktas såsom lifsvilkorens obestämda verkan på hvarje individuel organism, nästan på samma sätt som kölden verkar på olika men på ett obestämdt vis alltefter individens konstitution, hosta eller snufva, reumatism eller inflammation i olika organer.

Beträffande det som jag har kallat den indirekta verkan af förändrade förhållanden, nämligen deras inverkan på det reproduktiva systemet, kunna vi antaga att föränderlighet åstadkommes på detta sätt, dels på grund af det förhållande, att detta system är i hög grad känsligt för hvarje förändring i vilkoren, och dels på grund af den likhet, hvilken enligt Kölreuters och andras iakttagelser finnes emellan den föränderlighet som är en följd af skilda arters kroasering och den som kan observeras hos alla djur och växter, då de uppfödas under nya och onaturliga förhållanden. Många fakta visa klart och tydligt huru ytterst känsligt det reproduktiva systemet är för mycket små förändringar i de omgifvande förhållandena. Ingenting är lättare än att tämja ett djur, men få saker erbjuda så många svårigheter som att i fångenskapen bringa det till frivillig fortplantning till och med i de fall, då man kan förmå hanne och hona till parning. Huru många djur finnas icke, hvilka icke vilja fortplanta sig, oaktadt de lefva i en föga sträng fångenskap i sin egen hembygd. Detta tillskrifver man vanligen ehuru oriktigt en urartad instinkt, men vi se ju många kulturväxter vegetera och blomstra i full utveckling utan att sätta frö. I några få sådana fall har man upptäckt, att högst obetydliga förhållanden kunna hafva en afgörande inverkan på fröbildningen, såsom litet vatten mer eller mindre på en viss tid af utbildningen. Här är icke stället att ingå i den mängd detaljer, som jag har samlat öfver denna märkvärdiga fråga; men för att visa, huru egendomliga de lagar äro, som betinga djurens fortplantning i fångenskapen, vill jag blott anföra, att rofdjur, till och med från tropikerna, temligen lätt fortplanta sig hos oss i fångenskap (med undantag dock af de s. k. plantigrada, eller björnfamiljen, hvilka blott sällan föda ungar), hvaremot köttätande foglar blott i sällsynta fall eller nästan aldrig lägga fruktsamma ägg. Många utländska växter hafva fullkomligt odugligt frömjöl under samma förhållanden som de mest ofruktsamma bastardväxter. Om vi å ena sidan se husdjur och kulturväxter ofta till och med i svagt och sjukligt tillstånd ordentligen fortplanta sig i fångenskapen, under det å andra sidan fångna individer, fullkomligt tämda, könsmogna och kraftfulla (hvarpå jag kan anföra talrika exempel) af omärkliga orsaker så djupt angripas i sina reproduktionsorganer, att desamma ej kunna utöfva sin verksamhet, så torde det ej förundra oss, att dessa organer, om de verkligen bringas att fungera, icke utföra sin förrättning på alldeles regelmässigt vis och alstra en afkomma som icke är alldeles lik föräldrarna. Jag kan tillägga, att några djur och växter uthärda tämjning och odling utan att variera mera än i vilda tillståndet, på samma sätt som några organismer (såsom kaniner och vesslor) fortplanta sig under de onaturligaste förhållanden, en sak som blott bevisar, att deras reproduktionsorganer häraf icke angripas.

Några naturforskare hafva påstått, att all variation är förenad med könsfortplantning, men detta är säkerligen ett misstag, ty jag har i ett annat arbete uppsatt en lång lista på ”sporting plants” såsom trädgårdsmästarna kalla dem; — det är växter, som plötsligt frambringa ett enstaka skott med en ny och från växtens öfriga skott ofta betydligt afvikande karakter. Dessa skottvariationer, såsom man kan kalla dem, kan man fortplanta genom ympning och på flera andra sätt, någon gång äfven med frön. De äro i naturen ytterligt sällsynta, men ej ovanliga hos odlade växter. Då ett enkelt skott ibland de många tusende, som produceras år efter år under likformiga förhållanden på samma träd, helt säkert kan plötsligen antaga en ny karakter; och då skott på skilda träd, som växa under olika vilkor stundom lemna nästan samma varietet — skott på persikoträd frambringa till exempel nektariner och skott på vanliga rosor gifva mossrosor — så se vi tydligen, att lefnadsförhållandenas beskaffenhet är af fullkomligt underordnad vigt i jemförelse med organismens beskaffenhet vid bestämmandet af hvarje särskild variationsform; — icke af mera vigt än naturen af den gnista, genom hvilken en brännbar massa antändes, har för bestämmandet af lågans natur.



\section[Vanans verkningar]{Vanans verkningar. Utvecklingens vexelverkan. 
Ärftlighet.}

Vanor äro ärftliga och hafva ett afgjordt inflytande, så att förflyttning af en växt från ett klimat till ett annat kan förändra dess blomstringstid. Hos djur visar detta sig ännu tydligare; hos tama änder har jag funnit, att vingbenen äro lättare och benen i bakre extremiteterna tyngre i förhållande till hela skelettet än hos vilda änder, och jag tror att man utan fara kan tillskrifva detta den omständigheten, att tama anden flyger mindre och går mer, än änderna i vildt tillstånd pläga göra. Jufrets starkare utveckling, som går i arf hos kor och getter i sådana trakter, der de regelbundet mjölkas, är ett annat bevis på vanans verkningar. Det finnes icke någon art af husdjur, som icke i någon trakt har hängande öron, och mig synes derföre den åsigt hafva mycket skäl för sig, att dessa hängande öron bero derpå, att öronmusklerna ej begagnas, då djuren i tama tillståndet så sällan oroas af hotande faror.

Det gifves nu många lagar som reglera föränderligheten, af hvilka några få äro kända, ehuru ej så noga, och jag skall sednare i korthet anföra dem. Här vill jag blott omnämna hvad man kallar utvecklingens vexelverkan. En förändring i embryot eller larven skall sannolikt äfven åstadkomma förändringar hos det fullväxta djuret. Vid monstrositeter är de olika kroppsdelarnas ömsesidiga beroende mycket märkvärdigt och Isidor Geoffroy S:t Hilaire anför många exempel derpå i sitt stora verk. Stuteriegare tro, att långa ben vanligen åtföljas af förlängdt hufvud. Några fall af correlation synas helt sällsamma; så äro till exempel helhvita kattor med blå ögon vanligen döfva. Färg och egendomlighet i konstitution stå i sammanhang med hvarandra, hvarpå många märkvärdiga exempel kunna anföras både bland växter och djur. Ur de af Heusinger samlade fakta framgår, att vissa växter inverka skadligt på hvita får och svin, under det de mörkare icke angripas. Professor Wyman har nyligen meddelat mig ett mycket lärorikt exempel af detta slag. Han frågade några farmers i Florida, hvaraf det kom, att alla deras svin voro svarta, och fick derpå till svar, att svinen förtärde en färgrot (Lachnanthes) och detta hade till följd att deras ben färgades rosenröda och att deras hofvar föllo af; detta var dock icke förhållandet med de svarta svinen. En af ”crackers” tillade: ”Vi utvälja de svarta ungarna af en kull till uppfostran, emedan de allena hafva någon utsigt att blifva vid lif.” Nakna hundar hafva ofullständig tandbyggnad och man påstår, att lång- eller grofhåriga idislare lätt få långa horn eller fler än två; dufvor med fjädrade fötter hafva en hinna mellan de yttre tårna, dufvor med kort näbb hafva små fötter och de med lång näbb hafva stora fötter. Om man derföre utväljer lämpliga individer af växter och djur till afvel och på detta vis gör en egendomlighet i bygnaden mera framträdande, så skall man säkert äfven i andra delar framkalla förändringar enligt denna hemlighetsfulla lag om utvecklingens vexelverkan.

Resultatet af de många antingen alldeles obekanta eller blott dunkelt framträdande lagarna för föränderligheten är utomordentligt sammansatt och skiftande. Det är väl mödan värdt, att studera olika afhandlingar öfver våra gamla kulturväxter såsom hyacinter, potates t. o. m. dahlier o. s. v. och det är verkligen öfverraskande att se, i huru tallös mängd olikheter förekomma i bildning och sammansättning, hvarigenom dessa varieteter och undervarieteter obetydligt afvika från hvarandra. Hela deras organisation synes hafva blifvit plastisk för att afvika från föräldrarnas typ än i det ena än i det andra hänseendet.

Förändringar som icke gå i arf hafva för oss ingen betydelse, men redan antalet af ärftliga afvikelser i kroppens bygnad, vare sig af större eller mindre fysiologisk vigt, är oändligt. Dr Prosper Lucas’ afhandling i två starka band är det bästa och fullständigaste vi hafva häröfver. Hvarje stuteriegare hyser intet tvifvel, att böjelsen till arf är mycket stor: ”lika föder lika”, är deras grundaxiom och blott teoretiserande skriftställare hafva derom hyst något tvifvel. Om en afvikelse ofta förekommer och vi se den hos både far och son, så kunna vi icke säga, om den icke härrör från samma orsaker, som hafva utöfvat sitt inflytande på begge. Men om ibland flera individer af samma art, hvilka synbarligen lefva under samma vilkor, hos en individ förekommer en sällsynt afvikelse — hos en ibland millioner — till följd af något egendomligt sammanträffande af omständigheter, och om samma afvikelse sedan åter uppträder hos barnet, så nödgar oss redan sannolikhetsprincipen att antaga dess återuppträdande hafva skett genom arf. Hvar och en har ju redan hört omtalas fall, då så sällsynta företeelser, som albinism, tagghud, hårighet m. m. hafva förekommit hos flera medlemmar af samma familj. Men om så sällsynta och sällsamma afvikelser verkligen gå i arf, så måste mindre sällsamma och mera vanliga afvikelser med så mycket större skäl anses vara ärftliga. Ja, det vore kanhända riktigast att betrakta hvarje karakter såsom ärftlig, och motsatsen som ett undantag.

De lagar som reglera karakterernas ärftlighet äro fullkomligt okända och ingen kan säga orsaken, hvarföre samma egendomlighet i vissa fall är ärftlig hos åtskilliga individer af samma art och hos individer af olika arter, men i andra fall deremot icke går i arf; hvarföre barnet stundom återgår till vissa karakterer hos farfadern eller farmodern eller ännu äldre förfäder; hvarföre en egendomlighet ofta öfvergår från ett kön till båda, eller i andra fall bibehåller sig blott inom ett och samma kön. Ett faktum, som har någon betydelse för oss, är det, att egendomligheter, som uppträda hos hannarna af våra husdjur, antingen uteslutande eller åtminstone företrädesvis öfvergå på den manliga afkomman. Det är en ännu vigtigare och som jag tror tillförlitlig regel att, i hvilken period af lifvet en ärftlig egendomlighet än må visa sig, i motsvarande ålder uppträder den äfven hos afkomlingen ehuru stundom tidigare. I många fall är detta det enda möjliga, emedan de ärftliga egendomligheterna, till exempel i hornboskapens horn, först kunna framträda då afkomlingarna nått mogen ålder, och likaså gifves det som bekant egendomligheter hos silkesmasken, som visa sig under larv- eller pupptillståndet. Men ärftliga sjukdomar och några andra förhållanden gifva mig anledning att tro, att denna regel har en allmännare giltighet, och att, äfven der ingen uppenbar grund finnes för en afvikelses uppträdande i en viss ålder, denna afvikelse dock har stor benägenhet att hos afkomlingen visa sig i samma lefnadsperiod, i hvilken den uppträdde hos fadern. Jag tror att denna regel är af största vigt för att förklara embryologiens lagar. Dessa anmärkningar hafva för öfrigt afseende på egendomlighetens första synbara framträdande och icke på dess första anledning, som kanske ligger redan i det manliga eller qvinliga fortplantningselementet, på samma sätt som afkomman af korthornig ko med en långhornig tjur först sent i lifvet kan visa sina långa horn, oaktadt första orsaken dertill ligger redan uti faderns sperma.

Då jag nu har kommit in på frågan om organismernas sträfvan att återvända till förfädrens bildning, så vill jag här anföra ett påstående, som ofta yttras af naturforskare, att nämligen alla våra husdjur, om de förvildades, skulle antaga sina vilda stamfäders karakter, visserligen småningom, men dock säkert, och på grund häraf har man påstått, att från tama raser kunna inga slutsatser dragas med afseende på arterna i naturtillståndet. Jag har dock förgäfves bemödat mig att få reda på de bevis, på hvilka detta så ofta och så bestämdt uttalade påstående stöder sig. Det torde vara svårt att bevisa dess riktighet, ty vi kunna säga med säkerhet, att ganska många af de mest utpräglade tama varieteter icke kunna lefva i vildt tillstånd. I många fall känna vi icke en gång stammen och kunna då lika litet öfvertyga oss, om en fullständig regress har inträdt eller ej. Likaledes skulle det vara nödvändigt att försätta blott en enda varietet i frihet för att undvika kroasering. Men då våra varieteter helt säkert i vissa kännemärken antaga urformernas karakter, så synes det mig dock icke osannolikt, att de olika afarterna af vår vanliga kål till exempel skulle nästan helt och hållet antaga sin vilda urform, om man under många generationer fortsatte att odla dem i en fattig jordmån, i hvilket fall en del af följden dock måste tillskrifvas jordmånens omedelbara inverkan. Om nu försöket lyckas eller ej, det är för oss af ringa betydelse, emedan genom sjelfva försöket lefnadsvilkoren ändras. Om det läte bevisa sig, att våra domesticerade raser visa en stark benägenhet till återgång, det vill säga till att aflägga de antagna karaktererna, så länge de hållas tillsamman i stora massor och under oförändrade lifsvilkor, så att den här möjliga fria parningen kan förebygga några små afvikelser i bildning genom att sammanblanda dem, i sådant fall skulle jag medgifva, att resultat från tama varieteter ej gälla med afseende på arterna. Men det finnes icke en skymt till bevis för en sådan mening. Ett påstående att våra vagns- och ridhästar, våra lång- och korthornade nötkreatur, våra mångfaldiga fjäderfä och köksväxter icke kunna fortplanta sig genom ett ändlöst antal af generationer, ett sådant påstående vore stridande mot all erfarenhet.



\section[Domensticerade varieteter]{Domesticerade varieteters kännemärken. Svårighet
att skilja emellan varieteter och arter. Varieteters
uppkomst af en eller flera arter.}

Om vi betrakta de ärftliga varieteterna af våra husdjur och kulturväxter och jemföra dem med närstående arter, så finna vi oftast en mindre öfverensstämmelse i kännetecken hos hvarje sådan varietet än hos äkta arter. Tama raser hafva ofta äfven en något monströs karakter, det vill säga, att om de också i flera ovigtiga punkter skilja sig från hvarandra och från öfriga arter af samma slägte, så förete de dock ofta i någon enskild del afvikelser i yttersta grad så väl från de andra varieteterna, som isynnerhet från de närstående arterna i naturtillståndet. Med undantag af dessa fall (och kroaserade varieteters fullkomliga fruktsamhet, hvarom mera framdeles) afvika de domesticerade varieteterna af en och samma art från hvarandra på lika sätt som de hvarandra närmast stående arterna af samma slägte i naturtillståndet, ehuru i ringare grad. Jag tror, att man måste medgifva detta, om man betänker, att det knappt gifves några domesticerade raser bland djur eller växter, hvilka ej af kompetenta domare förklarats vara blott varieteter, och af andra lika kompetenta domare ansetts härstamma från ursprungligen skilda arter. Funnes det någon bestämd skilnad emellan raser och arter i tama tillståndet, så kunde sådana tvifvel ej så ofta komma åter. Man har ofta försäkrat, att våra raser icke visa afvikelser i slägtkaraktererna. För min del tror jag, att detta påstående kan bevisas vara falskt; naturforskares åsigter skilja sig dock betydligt, när det gäller att bestämma dessa slägtkarakterer, då alla sådana bestämningar äro blott empiriska. Enligt den åsigt jag snart skall framställa om slägtenas uppkomst hafva vi icke rätt att vänta att hos våra kulturalster så ofta stöta på afvikelser från slägttypen.

Om vi försöka bestämma graden af olikhet emellan de domesticerade raserna af en och samma art, så råka vi snart i bryderi, derigenom att vi ej veta, om de härstamma från en eller flera arter. Det vore af intresse, om denna fråga blefve något klarare, om det läte bevisa sig, att vindthunden, stöfvaren, gräfsvinshunden, spanieln, bulldoggen, hvilka så strängt fortplanta sin form, att de äro afkomlingar af blott en stamart. Sådana fakta skulle vara särdeles lämpliga att bringa oss till tvifvel om oföränderligheten af många närstående naturliga arter af räf till exempel, som bebo så skilda verldstrakter. Jag tror icke, såsom vi snart skola se, att alla olikheter emellan de särskilda hundraserna hafva uppstått genom domesticering; jag tror att en viss liten del af deras olikheter får skrifvas på räkningen af deras härkomst från skilda arter. Hos andra arter deremot, som blifvit husdjur, kan man antaga, eller till och med bevisa, att alla raser härstamma från en enda vild stamform.

Man har ofta antagit, att menniskan åt sig utvalt sådana växt- och djurarter till domesticering, som hafva en medfödd stark förmåga att variera och att bibehålla sig i olika klimater. Jag vill icke bestrida, att denna förmåga betydligt höjt värdet på våra flesta kulturalster. Men huru kunde en vilde veta, då han begynte tämja ett djur, om detta i följande generationer hade någon benägenhet att variera och kunde tåla vid andra klimater? Eller har åsnans och gåsens ringa förmåga att variera, eller renens ringa förmåga att uthärda värme, eller kamelens att uthärda köld hindrat menniskan att taga dem till husdjur? Jag är fullt öfvertygad, att om man toge andra växt- och djurarter i lika antal som våra domesticerade raser och från lika skilda klasser och trakter ifrån deras naturtillstånd och lät dem fortplanta sig i tamt tillstånd genom en lika lång serie af generationer, skulle de öfverhufvudtaget variera i lika hög grad, som stamarterna till våra raser hafva gjort.

Det är icke möjligt att bestämdt afgöra, huruvida våra af ålder domesticerade växt- och djurraser härstamma från en eller flera arter. Anhängarna af läran om våra husdjurs härstammande från flera urarter åberopa sig hufvudsakligen derpå, att vi redan i de äldsta tider, på de egyptiska monumenten och i pålbygnaderna i Schweiz finna en stor mångfaldighet af tama djur, och att några af dessa gamla raser utomordentligt likna de nu existerande eller till och med äro identiska med dem. Men detta tränger blott civilisationens historia längre tillbaka och visar, att djuren gjordes till husdjur i en mycket äldre tid, än hittills antagits. Pålbyggarna i Schweiz odlade flera slag af hvete och korn, ärter, vallmo och lin; de stodo äfven i förbindelse med andra nationer. Allt detta visar tydligt, såsom Heer anmärker, att de redan i en så långt aflägsen tid hade gjort betydliga framsteg i kultur, och detta förutsätter åter en ännu tidigare, långvarig period af en mindre framskriden civilisation, under hvilken de af åtskilliga folkstammar och i skilda distrikter såsom husdjur använda arterna hafva bildat varieteter och gifvit upphof till skilda raser. Efter upptäckten af flintredskap i de öfre jordlagren hysa alla geologer den åsigt att menniskor funnits till i ett fullkomligt ociviliseradt tillstånd i en oändligt långt aflägsen tid; — och så vidt man vet finnes för det närvarande knappt en enda folkstam så vild, att den icke har åtminstone tama hundar.

Våra flesta husdjurs ursprung skall väl alltid förblifva okändt. Jag vill dock här angifva den slutsats, till hvilken jag kommit efter sorgfälligt samlande af alla kända fakta rörande hushunden från alla delar af jorden, nämligen att flera vilda arter af hundslägtet blifvit tämda och att deras blod nu flyter mer eller mindre blandadt i våra talrika hundrasers ådror. Om fårets och getens ursprung kan jag icke bilda mig någon mening. Efter hvad jag fått mig meddeladt af Blyth angående den indiska puckeloxens lefnadssätt, läte, konstitution och bygnad är det sannolikt, att den härstammar från en annan stamform än våra europeiska nötkreatur, och kompetenta domare anse sig böra härleda de sednare från flera vilda förfäder, vare sig dessa nu förtjena namnet art eller ras. Genom Rütimeyers nya undersökningar kan man numera anse dessa åsigter om våra nötkreatur och den indiska puckeloxen såsom fullkomligt bevisade. Hvad hästen beträffar, är jag böjd att emot några författares åsigt antaga, att alla dess raser härstamma blott från en enda vild stam, och detta af skäl som jag här ej kan anföra. Blyth, hvars åsigt jag måste sätta högre än hvarje annans på grund af hans rika och omfattande forskningar i detta hänseende, tror att alla våra hönsvarieteter härstamma från de vanliga indiska hönsen (Gallus Bankiva). Jag har underhållit exemplar af nästan alla engelska raser, kroaserat dem och sedan undersökt skeletterna och har på detta sätt kommit till samma resultat, och grunderna dertill vill jag närmare utveckla i ett annat arbete. — De olika raserna af ankor och kaniner visa i kroppsbygnad stora afvikelser från hvarandra, men icke desto mindre tala en mängd förhållanden för det antagandet, att de alla härstamma från en vild stamart, vildanden och den vilda kaninen.

Läran om våra olika husdjursrasers härstammande från flera vilda stamformer hafva några skriftställare drifvit till en orimlig ytterlighet. De tro nämligen, att hvarje om ock aldrig så föga afvikande ras, som fortplantar sin karakter, äfven haft sin vilda stamform. I sådant fall måste endast i Europa hafva funnits en hel mängd arter af nötkreatur, många fårarter och några getarter och flera redan inom Storbritannien. En författare anser, att i sistnämda land funnits fordom elfva vilda fårarter, som varit egendomliga för England. Om vi nu taga i betraktande, att Storbritannien för närvarande knappt eger en enda, för detta land egendomlig däggdjursart, att Frankrike eger blott få, som ej förekomma äfven i Tyskland och tvärtom och att förhållandet är detsamma äfven i Ungern, Spanien o. s. v. men att hvart och ett af dessa länder å andra sidan hafva flera för dem egendomliga raser af nötkreatur, af får o. s. v. så måste vi antaga, att i Europa bildats många husdjursstammar; ty hvarifrån skulle de alla hafva kommit, då intet land har några karakteristiska arter, som kunna betraktas såsom särskilda stamformer. Och så är det äfven i Ostindien. Till och med hvad hushunden beträffar, tviflar jag icke, att många ärfda afvikelser måste tagas med i räkningen, ehuru jag antager den härstamma från flera stamarter. Ty hvem kan tro, att i vilda tillståndet djur kunnat lefva nära öfverensstämmande med italienska vindthunden, bulldoggen, mopsen m. fl., hvilka alla så betydligt afvika från alla vilda arter af hundslägtet. Man har framkastat den åsigt, att alla våra hundraser hafva uppkommit genom kroasering af några få stamarter men genom kroasering kunna vi blott erhålla sådana former, som i karakterer stå midt emellan sina föräldrar, och om vi utginge från detta antagande, så måste vi i alla fall medgifva att de mest olika formerna, såsom vidthund och mops till exempel, hafva lefvat i vildt tillstånd. Dessutom har man mycket öfverdrifvit möjligheten att genom kroasering bilda raser. Man känner många fall, som bevisa, att en ras låter modifiera sig genom kroasering af med omsorg valda individer, som visa den åsyftade karakteren, men det torde vara svårt att genom afvel få en ny ras, som står midt emellan två vidt skilda raser eller arter. Sir J. Sebright har anstält särskilda försök i denna riktning men misslyckats. Afkomman af första kroaseringen emellan två rena raser är temligen likformig och stundom, såsom jag funnit hos dufvor, utomordentligt öfverensstämmande och allt synes enkelt nog. Men om dessa bastarder under några generationer paras med hvarandra, så blifva knappt två af afkomlingarna lika hvarandra och då framlysa de stora svårigheterna och man kan knappt vidare hoppas på något resultat. Säkert är, att en mellanras emellan två vidt skilda raser ej kan bildas utan den yttersta sorgfällighet och ett länge fortsatt val af individer till afvel och jag har ej funnit antecknadt ett enda fall, då härigenom en permanent ras blifvit bildad.



\section[Tama dufvor]{Tama dufvor, deras olikheter och ursprung.}

Utgående från den åsigt att det är ändamålsenligast att välja en särskild djurgrupp till föremål för forskning, har jag efter någon öfverläggning dertill valt dufvorna. Jag har samlat alla raser, som jag kunnat köpa eller på annat sätt förskaffa mig, och har dervid fått röna ett välvilligt tillmötesgående från flera håll, isynnerhet af W. Elliot och C. Murray, hvilka sändt mig exemplar, den förre från Ostindien, den sednare från Persien. I detta ämne hafva många arbeten utkommit på flera språk, och vissa bland dem hafva en särskild betydelse för sin höga ålder. Jag har satt mig i förbindelse med flera utmärkta dufälskare och låtit införa mig i två ”pigeon-clubs” i London. Rasernas olikhet är förvånande. Man jemföre t. ex. den engelska brefdufvan och tumletten, betrakte den underbara olikheten i näbbar, som betingar motsvarande olikheter i deras skallar. Den engelska brefdufvan (carrier, Columba tabellaria), isynnerhet hannen, är dessutom märkvärdig genom den underbara utvecklingen af vårtlika utväxter på hufvudet, genom sina förlängda ögonlock, sina vida näsborrar och sin vida munöppning. Tumletten (tumbler, C. gyratrix) har en näbb som i profil ser ut som en sparfnäbb, och den har den egendomliga, strängt ärftliga vanan att i flock stiga till en ansenlig höjd, tumla om i luften och sedan hals öfver hufvud störta ned. Spanska dufvan (runt, C. hispanica) är af ansenligare storlek med stor näbb och stora fötter, några underraser hafva lång hals, andra långa vingar och lång stjert, andra åter en helt egendomlig kort stjert. Indiska dufvan (barb, C. indica) är slägt med brefdufvan, men har mycket kort och bred näbb. Kroppdufvan (pouter, C. gutturosa) har kroppen, vingarna och benen mycket förlängda, och dess oerhördt utvecklade kräfva, som han efter behag kan blåsa upp, må väl väcka förvåning, ja till och med löje. Måsdufvan (turbit, C. turbita) har en mycket kort och kägelformig näbb och en rad uppåt riktade fjädrar å bröstet, och har för sed att ständigt uppdrifva öfre delen af sin matstrupe. Jakobinen (C. cucullata) eller perukdufvan har nackfjädrarna så mycket omvända, att de bilda en peruk, och hennes vingar och stjert äro långa i förhållande till kroppsstorleken. Trumdufvan (trumpeter, glou-glou, C. dasypus) och skrattdufvan\footnote{Skrattdufvan, laugher, är icke Columba risoria, utan en tam, såsom det tyckes i Frankrike och Tyskland och äfven i Sverige okänd dufras.
Ö. a.} kuttra på ett helt annat sätt än de andra raserna, såsom också deras namn antyder. Påfågeldufvan (fantail, C. laticauda) har 30—40 stjertfjädrar i stället för det vanliga antalet 12—14, och bär dessa fjädrar utspärrade och riktade uppåt till den grad, att hos ett godt exemplar hufvud och stjert beröra hvarandra; gumpkörteln är alldeles hopkrympt. Ännu flera mindre utmärkta raser kunde uppräknas.

I de olika rasernas skelett visa ansigtsbenen afvikelser i längd, bredd och krökning. Formen, äfvensom längden och bredden på underkäken förete märkvärdiga olikheter. Antalet af korsbenskotor och svanskotor varierar, likaså antalet refben jemte deras bredd. Särdeles föränderliga äro äfven storleken och formen på luckorna i bröstbenet, äfvensom gaffelbenets vinkelöppning och skänklarnas relativa storlek. Munöppningens vidd, ögonlockens längd, yttre näsöppningarnas och tungans storlek, som icke alltid rättar sig efter näbbens längd, storleken af kräfvan och matstrupens öfre del, gumpkörtelns utveckling eller förkrympning, antalet af de första vingfjädrarna och stjertfjädrarna, vingarnas och stjertens längd i förhållande till hvarandra och till kroppslängden, benens och fotens längd, antalet af hornsköldar på tårna, alla dessa delar af kroppsbygnaden förete stora olikheter hos olika raser. Perioden för den fullständiga fjäderbeklädnaden är äfven likaså föränderlig som dunet, hvarmed ungarna äro beklädda då de lemna ägget. Form och storlek på äggen äro underkastade förändringar. Sättet att flyga är äfven olika, och många raser afgifva olika ljud och visa helt olika sinnesart. Hos vissa raser afvika slutligen äfven hannen och honan något ifrån hvarandra.

Så kunde man utvälja åtminstone tjugu dufvor, hvilka en ornitolog utan betänkande skulle förklara för välbegränsade arter, om man framlade dem för honom såsom vilda fåglar. Jag tror icke en gång att någon ornitolog skulle till samma slägte hänföra den engelska brefdufvan, tumletten, Indiska och Spanska dufvan, kroppdufvan, påfågeldufvan, isynnerhet om man af hvar och en af dessa raser visade honom några ärftliga underraser, hvilka han kunde kalla arter.

Huru stora än olikheterna må vara emellan de olika dufraserna, är jag dock öfvertygad om riktigheten af den bland naturforskarna allmänna meningen, att de alla härstamma från klippdufvan (Columba livia), om man under detta namn inbegriper åtskilliga geografiska raser eller underarter, hvilka blott i underordnade kännetecken afvika från hvarandra. Då några af de skäl, som hafva bestämt mig för denna åsigt, också äro användbara för andra fall, så vill jag här i korthet angifva dem. Om dessa olika raser icke voro varieteter och icke hade sitt ursprung från klippdufvan, så måste de härstamma från 7—8 stamarter; ty det vore omöjligt att erhålla alla våra tama raser genom kroasering af ett mindre antal arter. Huru ville man till exempel få kroppdufvan genom parning af två arter, af hvilka icke åtminstone den ena egde den ofantliga kräfvan. De antagna vilda stamarterna måste vidare hafva varit klippdufvor, d. v. s. sådana som icke lägga ägg i träd, eller icke ens gerna slå sig ned i träd. Men utom Columba livia och dess geografiska underarter känner man blott 2 eller 3 arter klippdufvor, men dessa ega icke ett enda af våra tama dufvors kännetecken. Derföre måste de antagna urstammarna i de trakter der de först tämdes ännu finnas lefvande, men okända af ornitologerna, hvilket synes mycket osannolikt på grund af deras storlek, lefnadssätt och märkvärdiga egenskaper; eller måste de hafva dött ut i vilda tillståndet. Men fåglar som bygga bo på klippor och som flyga väl, äro ej lätta att utrota, och våra vanliga klippdufvor, hvilkas lefnadssätt är lika med våra tama dufvors, hafva ännu icke kunnat utrotas ens på någon af de smärre britiska öarna eller på medelhafskusterna. Det antagna utrotandet af så många arter, som ha lika lefnadssätt med klippdufvan, synes mig derföre vara ett öfveriladt antagande. Dessutom äro alla de ofvannämda, så olika raserna utplanterade i alla verldsdelar och derföre måste väl några af dem hafva kommit till sin hembygd, och dock är icke en af dem förvildad, ehuru fältdufvan, det är klippdufvan i dess föga förändrade form, i några trakter åter blifvit vild. Då nu alla nyare försök visa, att det möter stora svårigheter att bringa ett vildt djur till fortplantning i tama tillståndet, så tvingade oss hypotesen om våra husdufvors flerfaldiga ursprung till det antagandet, att redan i uråldriga tider och af halfciviliserade menniskor 7—8 arter blifvit så fullkomligt tämda, att de bibehållit sin fruktsamhet äfven i fångenskapen.

Ett argument, såsom mig tyckes af stor vigt och äfven användbart i andra fall, är det, att de ofvan uppräknade raserna, ehuru de i allmänhet i konstitution, lefnadssätt, stämma och färg och de flesta delar af sin kroppsbygnad öfverensstämma med klippdufvan, dock i andra delar helt visst afvika från denna; i hela duffamiljen skulle vi förgäfves söka efter en näbb, sådan som den engelska brefdufvans, eller den korthöfdade tumlettens eller indiska dufvans, — eller efter tillbaka riktade fjädrar såsom hos perukdufvan, — eller efter en kräfva såsom hos kroppdufvan eller efter en stjert såsom hos påfågeldufvan. Man måste derföre antaga, att den halfciviliserade menniskan ej allenast tämt flera arter, utan äfven vare sig med afsigt eller af en tillfällighet dertill utvalt särdeles afvikande arter, och att dessa arter sedermera dött ut och försvunnit. Sammanträffandet af så många sällsamma tillfälligheter synes mig i högsta grad osannolikt.

Några omständigheter angående fjäderbeklädnadens färg torde äfven förtjena att här tagas i betraktande. Klippdufvan är skifferblå med hvit öfvergump (hos ostindiska afarten C. intermedia Strickl., blåaktig), har på stjertspetsen ett svart tvärband och på de yttre stjertfjädrarna en hvit yttre rand, och två svarta tvärband på vingarna; några halfvilda och andra såsom det synes fullkomligt vilda underraser hafva dessutom svarta fläckar å vingarna. Dessa kännetecken förekomma förenade icke hos någon annan art af hela familjen. Men nu påträffa vi stundom hos våra tama raser alla dessa kännetecken väl utvecklade, ända till de hvita ränderna på de yttre stjertfjädrarna. Ja, om man parar ihop två eller flera fåglar af olika raser, af hvilka ingendera är blå eller eger något af de angifna kännetecknen, så visa de deraf uppkomna bastarderna en stor benägenhet att plötsligt antaga dessa karakterer. Så sammanparade jag till exempel enfärgadt hvita påfågeldufvor af mycket konstant ras med enfärgadt svarta indiska dufvor, hvilken ras så sällan lemnar blå varieteter, att jag aldrig hört talas derom i England, och jag erhöll en brun, svart och fläckig afkomma. Jag sammanparade nu en indisk dufva och en bläsdufva, en mycket beständig ras af hvit färg, med röd stjert och rödt bläs, och bastarderna voro mörkfärgade och fläckiga. Då jag vidare parade ihop en bastard af påfågeldufva och indisk dufva med en bastard af indisk dufva och bläsdufva erhöll jag en ättling med vackert blå fjäderbeklädnad, hvit öfvergump, dubbelt svart tvärband på vingarna, svart tvärband och hvita sidoränder på stjerten, allt så som hos den vilda klippdufvan. Om alla tama raser härstamma från klippdufvan, kan man förklara dessa fakta ur den bekanta grundsatsen om återgång till stamfädernas karakter. Om vi åter ville förneka detta, måste vi antaga endera af följande förutsättningar, som synas mig mycket osannolika: antingen att alla de antagna olika stamarterna hade varit färgade och tecknade så som klippdufvan (ehuru ingen annan nu lefvande art är det), så att till följd häraf hos alla raser ännu funnes en benägenhet att antaga denna färgteckning; eller att hvar och en, äfven den renaste ras hade under de tolf eller på sin höjd tjugu sista generationerna någon gång blifvit kroaserad med klippdufvan. Jag säger tolf eller tjugu, ty vi känna intet faktum, som kunde understödja det antagande, att en afkomling skulle kunna återtaga sina förfäders kännetecken efter en ännu längre följd af generationer. Om i en ras kroasering med en annan ras egt rum blott en gång, så blir naturligtvis böjelsen att återtaga den sednares karakter så mycket mindre, ju mindre främmande blod ännu flyter i den sednare generationens ådror. Har åter ingen kroasering med främmande ras egt rum och förefinnes likväl hos båda föräldrarna böjelsen att återtaga en karakter som gått förlorad sedan flera generationer, så är, trots allt som kan synas strida häremot, det antagande nödvändigt, att denna benägenhet i oförminskad grad kan bibehållas under en lång rad af generationer. Dessa två fullkomligt skilda fall äro ofta sammanblandade i skrifter öfver ärftligheten.

Slutligen äro alla bastarder, som erhållas genom parning af de olika dufraserna, fullkomligt fruktsamma. Jag kan bekräfta detta efter mina egna försök, som jag med afsigt anstält med de mest olika raser. Deremot skall det vara svårt och kanske omöjligt, att anföra ett fall, då en bastard af två tydligt skilda arter har bibehållit fruktsamheten. Några antaga, att en långvarig domesticering småningom förminskar denna benägenhet för ofruktsamhet. Om man får döma af hundens och några andra husdjurs historia, synes mig denna hypotes hafva stor sannolikhet, om dess giltighet inskränkes till närstående arter. Dock är den ej bekräftad genom något enda direkt försök. Men att utsträcka detta antagande till ett påstående, att en fruktsam afkomma skulle lemnas af arter, ursprungligen så vidt skilda som brefdufvan, tumletten, kroppdufvan och påfågeldufvan, det synes mig ytterst förhastadt.

Dessa grunder, osannolikheten, att menniskan redan i uråldriga tider kunnat bringa till fruktsamhet i tama tillståndet sju till åtta vilda dufarter, som vi nu känna hvarken i vildt eller förvildadt tillstånd, deras kännetecken, som i så många fall afvika från nästan alla duffåglars bildning med undantag af klippdufvans, det tillfälliga återuppträdandet hos alla raser af den blåa färgen och de beskrifna svarta teckningarna, så väl under fortsatt afvel, som under kroasering, bastardernas fullständiga fruktsamhet; alla dessa grunder tvinga mig till den slutsats, att alla våra tama dufvor härstamma från Columba livia och dess geografiska underarter.

Till förmån för denna åsigt vill jag vidare anföra: 1) Att klippdufvan, C. livia, så väl i Europa som i Indien blifvit befunnen lämplig att tämjas och att den i sina vanor och i många delar af sin bygnad öfverensstämmer med alla våra tama raser. 2) Ehuru en engelsk brefdufva eller en tumlett i vissa karakterer vidt skiljer sig från klippdufvan, så är det dock möjligt i dessa båda och några andra, ehuru icke alla fall, att uppställa en nästan oafbruten serie af former emellan de mest skilda bildningar genom att med hvarandra jemföra de olika underraserna af dessa former och isynnerhet de som härstamma från skilda trakter. 3) De kännetecken, som isynnerhet skilja de olika raserna från hvarandra, såsom brefdufvans långa näbb och hudvårtor, tumlettens korta näbb och påfågeldufvans talrika stjertfjädrar, äro hos hvarje ras ytterst föränderliga; förklaringen af detta fenomen uppskjuta vi till kapitlet om urvalet. 4) Man har funnit dufvor hos många folkslag, hvilka vårdat dem med yttersta sorgfällighet ända till passion. Man har i årtusenden tämt dem i alla verldsdelar; den äldsta berättelsen om dem härstammar från den femte egyptiska dynastien, ungefär 3000 år före Kristus, såsom professor Lepsius meddelat mig, men Birch säger, att dufvor finnas redan på en matsedel från den föregående dynastien. Plinius berättar, att romarna betalte stora summor för dufvor, ”ja, det har kommit derhän, säger han, att man räknar deras ras och stamträd”. Omkring år 1600 skattades de i Indien så högt af Akbar Khan, att icke mindre än 20,000 hörde till hofhållningen. Den kunglige historieskrivaren berättar vidare: ”Monarkerna i Iran och Turan sände honom några sällsynta fåglar. Hans majestät har genom rasernas kroasering, en metod som ej förr användts, på ett förvånande sätt förädlat dem”. På samma tid voro holländarna lika passionerade för dufvor som förut de gamla romarna. Dessa betraktelsers stora vigt för förklaringen af de utomordentliga förändringar som dufvorna hafva undergått skall först blifva oss tydlig vid afhandlandet om det naturliga urvalet, då vi äfven få se orsaken hvarföre raserna så ofta hafva ett monströst utseende. För frambringandet af skilda raser är det äfven en mycket gynnande omständighet, att bland dufvorna en hanne gerna för hela lifvet sammanparar sig med en hona, och att följaktligen olika raser kunna hållas tillsammans i ett och samma dufslag.

De tama dufvornas sannolika ursprung har jag behandlat med någon utförlighet (ehuru äfven den är otillräcklig) af det skäl, att då jag började underhålla dufvor för att iakttaga deras olika former, väl vetande huru rena raserna bibehöllo sig, jag sjelf ansåg det för lika så svårt att tro, att alla dessa raser som nu äro husdjur kunde härstamma från en och samma stamfader, som att tro på alla sparfvars eller någon annan stor fågelfamiljs härstammande från en gemensam källa. Isynnerhet frapperades jag af en omständighet, att nämligen nästan alla de som sysselsatt sig med afvel af husdjur och kulturväxter, hvilka jag har rådfrågat eller hvilkas skrifter jag läst, varit fullkomligt öfvertygade om, att de olika raser, med hvilka hvar och en sysselsatte sig, härstammade från lika så många ursprungligen skilda arter. Om man, såsom jag har gjort, frågar en landthushållare om icke Herefordrasen möjligen kunde härstamma från en långhornig ras, eller båda från en gemensam stamform, blir svaret ett hånleende. Jag har aldrig funnit en älskare af dufvor, höns, ankor, kaniner, som icke varit fullt öfvertygad, att hvarje hufvudras härstammar från en särskild stamart. Van Mons visar i sitt verk öfver äpplen och päron, huru omöjligt det är för honom att tro, att olika sorter, till exempel en Ribstonpipping eller ett Codlinäpple skulle kunna uppkomma af frön af ett och samma träd. Och så kunde jag anföra otaliga exempel. Detta låter förklara sig, som jag tror, på ett enkelt sätt. I följd af mångåriga studier hafva dessa personer erhållit en stor känslighet för skilnaden emellan de olika raserna, och ehuru de väl veta, att hvarje ras varierar, då de just genom ett noggrannt urval af sådana små variationer vinna sitt pris försumma de dock att draga allmänna slutsatser och beräkna icke den summa som erhålles genom adderandet, hopandet af små förändringar under loppet af många på hvarandra följande generationer. Och dessa naturforskare, hvilka antaga, att många våra husdjursraser härstamma från samma förfäder, ehuru de ännu mindre känna lagarna om ärftlighet och icke bättre känna mellanlänkarna i den långa serien af afkomlingar, skola icke de häraf lära försigtighet och icke le åt den tanken, att en art i naturtillståndet kan härstamma i rätt nedstigande linie från en annan art?
Fordom hyllade grundsatser för urvalet och deras
följder.

Vi vilja nu i korthet undersöka, huru de domesticerade raserna småningom uppkommit af en eller flera närstående arter. En ringa verkan må dervid tillskrifvas de yttre lefnadsförhållandenas omedelbara inverkan och äfven vanan, men det vore djerft att på deras räkning skrifva olikheterna emellan en arbetshäst och en ridhäst, en vindthund och en tax, en brefdufva och en tumlett. En af de märkvärdigaste egendomligheter, som vi iakttaga hos våra husdjur och odlade växter, är det sätt, hvarpå de lämpa sig icke efter sin egen fördel, utan efter menniskans nytta och nöje. Några för henne nyttiga variationer hafva utan tvifvel plötsligt eller med ens uppstått; många botanister tro till exempel, att kardtisteln (Dipsacus Fullonum) med sina hakar, som i användbarhet öfverträffar hvarje mekanisk inrättning, blott är en varietet af den vanliga Dipsacus, och hela denna variation torde väl hafva uppkommit plötsligt i en frösådd af denna sednare. Så är sannolikt äfven förhållandet med taxen, och det är bekant att det amerikanska anconfåret har uppstått på detta sätt. Men om vi jemföra draghästen med ridhästen, dromedaren med kamelen, fårarter som passa för odlade trakter med sådana som trifvas på bergstrakter, hvilkas ull lämpar sig till helt olika ändamål, om vi jemföra de olika hundraserna, som tjena menniskan hvar och en på sitt sätt, om vi jemföra den uthålliga stridstuppen med andra fredliga och tröga raser, med de ständiga värphönorna (”everlasting layers”) som aldrig vilja kläcka, eller med den lilla prydliga bantamhönan, om vi slutligen betrakta massan af åkerväxter, fruktträd, köks- och trädgårdsväxter, hvilka alla tjena menniskan till nytta eller nöje, ehuru hvar och en på olika tider och på olika sätt, så måste vi väl tänka på någon annan orsak än blott föränderlighet. Vi kunna icke antaga, att alla dessa varieteter uppstått på en gång så fullkomliga och så användbara som vi nu se dem för oss, och i sjelfva verket känna vi mångas historia rätt väl för att veta, att detta icke varit fallet. Nyckeln till gåtan ligger i menniskans accumulativa valförmåga, det vill säga hennes förmåga att genom ett urval till afvel af sådana individer, som ega de åstundade egenskaperna, hos hvarje generation öka dessa egenskaper om ock i ringa grad: naturen lemnar småningom mångahanda afvikelser; menniskan summerar dem i vissa för henne vigtiga riktningar. I denna mening kan man säga, att hon har skapat sina domesticerade raser.

Det stora värdet af denna grundsats för urvalet är icke hypotetiskt; ty det är säkert, att några af våra utmärktaste idkare af boskapsafvel till och med inom en menniskoålder hafva i väsentlig mån modifierat flera raser af nötkreatur och får. För att till fulla värdet uppskatta deras verksamhet måste man läsa några skrifter i detta ämne, hvaraf finnas många, och se sjelfva resultaten. Stuteriegare tala om ett djurs organisation såsom om något fullkomligt plastiskt, som de kunna forma nästan helt och hållet efter sitt behag. Om utrymmet tilläte, kunde jag till bevis anföra många yttranden af de mest sakkunniga fackmän. Youatt, som sannolikt bättre än någon annan var hemmastadd i landthushållningen och sjelf en god djurkännare, yttrar om denna grundsats för urvalet, att den ”satte landthushållaren i stånd att icke blott modifiera karakteren hos sina hjordar, utan att helt och hållet förändra den. Det är den trollstaf, med hvars hjelp han kallar i lifvet hvilken form han behagar”. Lord Sommerville säger med anledning af fårrasernas förädling: ”Det är som hade man tecknat en fulländad form på en vägg och sedan gifvit den lif”. Sir John Sebright plägade säga angående dufvor, att på tre år skulle han frambringa en för honom uppgifven fjäderbeklädnad, men sex år behöfde han för att åstadkomma en önskad form på hufvud och näbb. I Sachsen är denna grundsatsens vigt för afveln af merinofår så väl känd, att den följes handtverksmessigt. Fåren läggas på ett bord och studeras på samma sätt som kännaren studerar en målning. Detta upprepas tre gånger i månaden och fåren antecknas och klassifieras för hvarje gång, så att till slut blott de bästa tagas till afvel.

Hvad man i England har åstadkommit synes af de oerhörda pris, som betalas för djur, hvilka kunna uppvisa ett godt stamträd, och produkterna hafva exporterats till alla verldsdelar. I allmänhet åstadkommes icke förädlingen genom kroasering af skilda raser. Alla de bästa auktoriteter uttala sig strängt emot detta förfarande, som endast kan tillåtas emellan nära beslägtade underraser. Och har en gång en sådan kroasering egt rum, så är ett sorgfälligt urval ännu nödvändigare än i vanliga fall. Vore det vid valet blott fråga om att utvälja en starkt i ögonen fallande variation och använda den till afvel, då vore grundsatsen så lätt begriplig, att det icke vore mödan värdt att tala derom. Men dess vigt består i resultatet af ett genom generationer fortgående hopande af förändringar, som för det oöfvade ögat äro fullkomligt omöjliga att upptäcka, förändringar, som jag till exempel förgäfves sökt upptäcka. Icke en menniska bland tusen har tillräckligt skarpt öga och omdöme att blifva utmärkt i detta fack. Är han i besittning af dessa egenskaper, studerar sitt ämne i åratal och egnar deråt hela sin lefnad med oböjlig ihärdighet, då skall han vinna goda resultat och åstadkomma stora förbättringar; saknar han åter en af dessa egenskaper, så skall han säkerligen misslyckas. Få personer hafva väl en föreställning om, hvilken grad af naturlig begåfning och huru mångårig öfning behöfves för att blifva blott en skicklig dufamatör.

Samma grundsatser följas i trädgårdsodlingen, men förändringarna uppträda ofta hastigare. Dock lär väl ingen tro, att våra ädlaste trädgårdsalster hafva uppstått omedelbarligen ur de vilda urformerna endast genom obetydliga variationer. I några fall kunna vi bevisa, att detta icke är förhållandet, då noggranna anteckningar blifvit förda, men för att gifva ett slående exempel derpå kunna vi åberopa krusbären, som ständigt tilltaga i storlek. Hos många praktväxter varseblifva vi en märkbar förädling, om vi jemföra deras blommor i våra dagar med afbildningar deraf, som äro gjorda för 20 à 30 år sedan. Då en växtras en gång är väl utbildad, utsöker odlaren icke de bästa exemplaren, utan aflägsnar ur sängarna blott de så kallade ”rogues” d. v. s. de som mest afvika från den egendomliga formen. Vid djurafvel eger samma urval äfven rum, ty hvem skulle så illa sköta sina djur, att han använde de sämsta till afvel.

Hos växterna gifves det äfven ett annat medel att iakttaga urvalets accumulativa verkningar, nämligen att jemföra blommornas olikhet hos varieteterna af en art i en blomsterträdgård, att jemföra olikheten i blad, frukthylsor, rotknölar eller andra delar i köksträdgården med olikheterna hos blommorna af samma varieteter, och olikheterna hos frukterna af en arts varieteter i fruktträdgården med blommornas och bladens karakterer i samma varietetsserie. Huru olika äro icke bladen hos våra kålsorter, och dock äro deras blommor så lika! Å andra sidan, huru olika äro icke penséernas blommor, under det att bladen bibehålla samma form! frukterna af våra krusbärssorter, huru olika äro icke de till form, färg, storlek och hårighet, under det att blott helt obetydliga olikheter kunna iakttagas hos blommorna! Härmed vill jag icke säga, att varieteterna i ett hänseende äro väsentligt skilda och i ett annat alldeles icke: detta händer väl sällan och kanske aldrig. Lagarna för organernas ömsesidiga beroende, hvilkas vigt aldrig får förbises, skola alltid föranleda några olikheter, men i allmänhet kan jag icke hysa något tvifvel om, att ett fortsatt urval af obetydliga variationer i blad, blommor eller frukt frambringar sådana raser, som hufvudsakligen i dessa delar afvika från hvarandra.

Här kunde man göra den invändningen, att principen för urvalet först sedan knappa tre fjerdedelar af ett sekel funnit en planmässig användning; säkert är att den först under de sista åren fått en allmännare spridning, och att flera afhandlingar deröfver uppkommit, och i samma mån hafva också resultaten blifvit hastigare och vigtigare. Men det är långt ifrån sant, att denna grundsats är en ny uppfinning. Jag kunde anföra flera bevis, af hvilka visar sig, att redan i gamla skrifter dess vigt vunnit fullt erkännande. Till och med i den engelska historiens råa och barbariska tid infördes ofta utsökta afvelsdjur, och export af sådana var i lag förbjuden; man brukade äfven nedslagta sådana hästar, som ej hade uppnått en viss storlek, en sak som väl låter jemföra sig med trädgårdsodlarens bortrensande af odugliga växter. Grundsatsen för urvalet finner jag äfven bestämdt uppgifven i en gammal kinesisk encyklopedi och bestämda reglor derför finnas uppgifna hos några romerska klassiker. Af några ställen i första Mosebok framgår, att man redan tidigt hade riktat sin uppmärksamhet på husdjurens färg. Vilda folkstammar kroasera stundom ännu sina hundar med vilda hundarter för att förbättra rasen, hvilket enligt Plinii uppgift skett äfven i fordna tider. Vildarna i Sydafrika sammanpara sina dragoxar efter färg och eskimåerna förfara på samma sätt med sina draghundar; Livingstone berättar, huru högt goda djurarter skattas af negrer i det inre Afrika, hvilka aldrig kommit i beröring med européer. Några af de anförda fakta äro visserligen icke bevis för något verkligt urval; men de visa att husdjursafveln redan i äldre tider varit föremål för en synnerlig uppmärksamhet och är det nu äfven hos de råaste vildar. Det vore i sanning också förvånande, om ett sådant urval ej skulle hafva tillvunnit sig uppmärksamhet, då ärftligheten af såväl goda som dåliga egenskaper är så påtaglig.



\section{Omedvetet urval.}

Skickliga män försöka i våra tider att genom planmässigt urval med ett bestämdt mål i sigte bilda nya stammar eller underraser, hvilka skola öfverträffa allt som i den vägen förut åstadkommits. För oss är dock det slags urval vigtigast, som man kan kalla det omedvetna och som är en följd af den omständigheten, att hvar och en söker förskaffa sig de bästa djuren och fortplanta dem. Den som vill hålla sig rapphönshundar försöker naturligtvis att få så goda hundar som möjligt och utväljer sedan de bästa af sina egna hundar till afvel; dervid har han dock ingen afsigt eller förhoppning att åstadkomma en varaktig förbättring af rasen. Det oaktadt kan man antaga, att ett sådant förfarande fortsatt under några hundra år skall förändra och förädla raserna, på samma sätt som Bakewell och Collins m. fl. hafva under sin lifstid åstadkommit väsentliga förändringar i sina hjordar af nötkreatur genom ett lika förfarande ehuru efter en planmässig metod. Långsamma och omärkliga förändringar af detta slag kunde icke blifva märkbara, om man icke sedan lång tid tillbaka hade af de ifrågavarande raserna gjort verkliga mätningar och sorgfälliga anteckningar, hvilka kunna tjena till jemförelse; stundom kan man dock träffa på ännu oförädlade eller föga förändrade individer af samma ras i sådana mindre civiliserade trakter, hvarest deras förädling är mindre långt framskriden. Man har anledning att tro, att King Charles spaniel sedan denna monarks tid har undergått flera oafsigtliga förändringar. Några fullkomligt sakkunniga fackmän hysa den öfvertygelsen, att den långhåriga rapphönshunden (setter) i rätt nedstigande linie härstammar från denna hundras och sannolikt har uppstått genom en långsam förändring af densamma. Det är bekant, att den korthåriga rapphönshunden (pointer) under sista århundradet genomgått stora förändringar, och dessa anser man bero hufvudsakligen på kroasering med räfhunden (C. gallicus, fox-hound); hvad som för oss är af vigt, är att dessa förändringar skett oafsigtligt och långsamt och hafva dock blifvit så ansenliga, att, ehuru den gamla rapphönshunden säkert kommit ifrån Spanien, mr Borrow dock försäkrar, att han i hela Spanien icke sett någon inhemsk hundras, som liknar vår rapphönshund.

Genom ett likartadt urval och en omsorgsfull behandling har man bragt de engelska ridhästarna derhän, att de i storlek och snabbhet öfverträffa sina arabiska stamfäder. Lord Spencer m. fl. har visat, att nötkreaturen i England hafva tilltagit i vigt och bli nu förr fullvuxna än förut. Jemför man de uppgifter, som i gamla skrifter finnes angående brefdufvor och tumletter, med dessa raser, sådana de nu äro i England, Indien och Persien, så kan man i min tanke steg för steg tydligt följa deras utveckling från klippdufvans former till sina nuvarande.

Youatt gifver ett förträffligt exempel på verkningarna af ett långvarigare urval, som man så till vida kan kalla omedvetet, som de dervid verksamma personerna sjelfva aldrig väntade eller önskade det slutliga resultatet, nemligen skapandet af två helt olika stammar. De båda hjordar af Leicester-får, som mr Buckley och mr Burgess underhöllo, ”härstamma från Bakewells ursprungliga stamform och hafva fortplantats rena i mer än 50 år”, såsom Youatt anmärker. ”Bland alla som känna till saken tror icke någon, att egarna till dessa hjordar någonsin inblandat främmande blod i den rena Bakewellska stammen, och dock äro olikheterna emellan dessa båda hjordar nu så stora, att man tror sig se fullkomligt skilda varieteter”.

Funnes det vildar, så barbariska, att de icke hade någon aning om karakterernas ärftlighet hos sina husdjur, så skulle de dock under hungersnöd och andra olyckshändelser, för hvilka vildar så lätt äro utsatta, vara betänkta på att skydda hvarje djur, som vore nyttigt för något särskildt ändamål, och ett på sådant sätt utvaldt djur skulle efterlemna en större afkomma än ett annat af mindre värde, så att redan på detta sätt ett omedvetet urval egde rum. Hvilket värde inbyggarna på Eldslandet sätta på sina djur, se vi under hungersnöd, då de heldre döda och uppäta sina gamla qvinnor än sina hundar.

Hos växterna kan man iakttaga samma gradvisa förädlingsprocess genom bibehållandet af de bästa individer utan afseende på om de visa tillräckliga skiljaktigheter för att redan vid första uppträdandet anses för egna varieteter, eller om de uppstått genom kroasering af två raser eller arter; vi se den äfven i blommornas tilltagande storlek och skönhet hos penséer, dahlier, pelargonier och andra växter i jemförelse med de äldre varieteterna och deras stamformer. Ingen skall vänta sig att få en pensé eller en dahlia af bästa beskaffenhet ur fröen på en vild växt, eller att kunna uppdraga ett päron af bästa sort af fröen på ett vildt päronträd, ehuru han skulle lyckas om detta vore ett förvildadt exemplar af en i trädgård odlad varietet. De redan i den klassiska tiden odlade päronträden synas efter Plinii berättelse hafva varit af temligen underordnadt värde. Jag har i skrifter rörande trädgårdsodlingen läst uttryck af stor förvåning öfver trädgårdsmästarnas stora skicklighet att af så torftigt material få så lysande resultat, men deras konst var utan tvifvel enkel och åtminstone hvad resultatet angår omedveten; den bestod blott deri, att de för hvarje gång utsådde den bästa varieteten och om tillfälligtvis en ny något bättre varietet bildades, så valdes denna till vidare behandling. Men den klassiska tidens trädgårdsodlare, som uppdrogo de bästa päron de kunde åstadkomma, hade ingen aning om, hvilken herrlig frukt vi en gång skulle äta, och dock hafva vi för denna frukt att tacka till en del den omständigheten, att de redan börjat att utvälja och fortplanta de bästa varieteterna.

Den mängd förändringar, som långsamt och oafsigtligt blifvit hopade på våra kulturväxter, förklarar i min tanke det bekanta faktum, att vi i de flesta fall icke kunna igenkänna den vilda moderväxten och följaktligen icke heller uppgifva, hvarifrån de i våra blomster- och fruktträdgårdar längst odlade växterna härstamma. Men om hundrade och tusende år hafva varit nödvändiga att förädla våra kulturväxter till deras nuvarande för menniskan så nyttiga former, så inse vi äfven orsaken, hvarföre hvarken Australien eller Goda Hoppsudden eller något annat af ociviliserade menniskor bebodt land har erbjudit oss någon växt, som förtjenat att odlas. Vi få icke tro, att dessa länder med sin rika vegetation af en egendomlig slump icke af naturen blifvit begåfvade med några till odling användbara örter, utan deras inhemska växter hafva icke genom ett fortsatt urval fått en så hög grad af förädling, att de kunna jemföras med de i de civiliserade länderna under lång tid odlade växterna.

Hvad de ociviliserade folkens husdjur beträffar, få vi icke förbise, att de i allmänhet åtminstone under vissa årstider få sjelfva förskaffa sig sin föda. I olika beskaffade trakter kunna individer af samma art men af något olika bildning och konstitution trifvas olika väl, så att den ena trifves bättre i en trakt, den andra i en annan och på detta sätt kan genom ett slags naturligt urval, såsom vi framdeles få se, bildas tvänne underraser. Detta förklarar måhända till en del en uppgift af några författare, att vildarnas djurraser mera visa karakteren af skilda arter än de civiliserade folkens varieteter.

Enligt ofvan uttalade åsigt om den ytterst vigtiga rol som menniskans urval har spelat förklaras äfven den omständigheten, att våra domesticerade raser i bygnad och lefnadssätt lämpa sig efter menniskans behof och önskan. Den förklarar äfven i min tanke våra domesticerade rasers ofta så abnorma karakter och deras i de yttre kännetecknen vanligen så stora, men i inre delar eller organer relativt så obetydliga afvikelser. Vid sitt urval kan menniskan svårligen fästa afseende vid andra än yttre afvikelser och i sjelfva verket bekymrar hon sig sällan om det inre. Hon kan vidare i sitt val blott upptaga sådana förändringar, som naturen sjelf i ringa grad erbjuder henne. Så skulle ingen hafva försökt att göra en påfågeldufva, om han icke redan förut sett en dufva med ovanligt utbildad stjert, ej heller skulle någon åstadkommit en kroppdufva, innan han funnit en dufva med ovanligt stor kräfva. Ju ovanligare och ju mer afvikande en karakter vid sitt första uppträdande var, desto mer måste den hafva ådragit sig uppmärksamhet. Dock är ett sådant uttryck som ”försök att göra en påfågeldufva” i de flesta fall i högsta grad oriktigt. Ty den som först utvalde till afvel en dufva med något starkare utvecklad stjert har säkerligen icke drömt om hvad som skulle blifva af denna dufvas afkomlingar genom dels omedvetet dels planmässigt urval. Kanske har stamfadern till alla påfågeldufvor blott haft fjorton, något utbredda stjertfjädrar, såsom den Javanesiska påfågeldufvan nu för tiden, eller såsom individer af andra raser, på hvilka man har funnit ända till sjutton stjertfjädrar. Den första kroppdufvan har kanske icke haft sin kräfva mera uppblåst, än måsdufvan nu plägar göra med öfre delen af matstrupen, en vana som af alla dufamatörer lemnas utan afseende, emedan den icke erbjuder någon utgångspunkt för urvalet.

Men man får icke antaga, att det behöfves stora afvikelser för att ådraga sig amatörens uppmärksamhet; han iakttager ytterst små olikheter, och det ligger i menniskans natur att värdera en nyhet som hon har i sin ego, om den ock är obetydlig. Derföre får man icke jemföra värdet af de i början obetydliga individuela olikheterna hos exemplar af samma art med den vigt samma afvikelser få, då en gång mera rena raser af arten äro bildade. Många obetydliga afvikelser kunna hafva förekommit ibland dufvor, hvilka betraktats såsom felaktiga afvikelser från hvarje ras’ fulländade typ, och hafva på den grund förkastats och sådant händer väl ännu. Vår vanliga gås har icke lemnat någon värdefull varietet och vid expositioner af fjäderfä utgifvas derföre den vanliga rasen och Toulouse-rasen för olika raser, ehuru de skilja sig blott till färgen, den mest föränderliga af alla karakterer.

Dessa åsigter förklara vidare en anmärkning som ofta göres, att vi icke veta något om våra rasers ursprung eller historia. Emellertid kan man svårligen säga om en ras, liksom om en språkdialekt, att den haft ett bestämdt ursprung. En person vårdar och använder till afvel en individ med ringa afvikelser i kroppsbygnad eller han nedlägger mer än vanlig omsorg på att sammanpara sina bästa djur och förbättrar derigenom sin afvel; och de förädlade djuren sprida sig långsamt i det närmaste grannskapet. Men då de ännu icke kunna hafva ett bestämdt namn och deras värde ännu ej är särdeles stort, aktar ingen menniska på deras historia. Om de nu genom samma långsamma förfarande ytterligare förädlas, utbreda de sig vidare, uppmärksammas såsom något märkvärdigt och värdefullt och erhålla sannolikt nu först en benämning. I halfciviliserade trakter med dåliga kommunikationer kan utbredandet af en ny underras gå blott långsamt för sig. Men så snart den nya rasens särskilda värdefullare egenskaper en gång vunnit fullt erkännande, skall städse den af mig så kallade principen för det omedvetna urvalet långsamt inverka på samlandet af rasens karakteristiska egenskaper, af hvad slag de än må vara, och detta sker måhända mer på en tid än på en annan, allt efter som en ras stiger eller faller i modet, och mer på en trakt än i en annan allt efter invånarnas civilisationsgrad. Men utsigterna att få noggranna uppgifter öfver dylika långsamma, vexlande och omärkliga förändringar äro ytterst ringa.



\section[Menniskans valförmåga]{Omständigheter, som gynna menniskans valförmåga.}

Jag har nu några ord att säga om de omständigheter, som för menniskans valförmåga äro gynnsamma eller icke. En hög grad af föränderlighet är så till vida uppenbart gynnsam, som den lemnar ett rikligare material till urvalet. Individuela olikheter äro visserligen tillräckliga för att åstadkomma en betydlig förändring i en viss önskad riktning (genom ett med yttersta sorgfällighet bedrifvet accumulativt urval i denna riktning), men då sådana för menniskan uppenbart nyttiga eller behagliga variationer blott tillfälligtvis förekomma, så måste sannolikheten för deras uppträdande ökas med antalet vårdade individer och derföre är detta af högsta vigt för resultatet. Marshall har en gång sagt om fåren i Yorkshire, att ”emedan de vanligen tillhöra fattigt folk och för det mesta äro fördelade i små partier, så kunna de aldrig förädlas”. Å andra sidan hafva sådana personer, som idka handel med trädgårdsalster och följaktligen uppdraga samma växter i stort antal, vanligen större framgång än blotta amatörer. Vill man hålla ett stort antal individer af samma art i en trakt, fordras det såsom vilkor, att de försättas i gynnsamma lefnadsförhållanden, så att de i denna trakt ordentligt fortplanta sig. Är antalet individer af en art ringa, så användas vanligen alla till afvel, hurudana de än må vara, och detta är naturligtvis hinderligt för urvalet. Men den vigtigaste punkten är sannolikt, att djuret eller växten är för egaren af så stor nytta eller så stort värde, att hans uppmärksamhet är noggrannt riktad på hvarje, äfven den ringaste förändring i kroppsbygnad hos alla individer. Är detta icke fallet, blir resultatet intet. Jag har sett framhållas med eftertryck, att af en lycklig slump smultron började variera just då trädgårdsodlarna började fästa sin uppmärksamhet vid växten. Säkerligen hade smultronväxten alltid varierat, sedan den börjat odlas, men man hade förbisett dess obetydliga afvikelser. Då nu trädgårdsodlarna sedermera odlade särskilda växter med större, tidigare eller bättre frukter, uppdrogo nya plantor af deras frön och sedan åter använde de bästa individer och deras frön till vidare odling, då lemnade dessa de många beundransvärda varieteter, som de sista 30 till 40 åren hafva frambragt.

Beträffande djur af olika kön, har lättheten att förekomma kroasering ett vigtigt inflytande på bildningen af nya raser, åtminstone i trakter der redan flera andra raser finnas förut. Här spelar äfven inhägnandet af olika områden en stor rol. Vandrande vildar eller inbyggare på öppna slätter hafva sällan mer än en ras af samma art. Man kan sammanpara två dufvor för hela lifstiden, och detta är en stor beqvämlighet för amatören, ty i följd deraf kunna många raser förädlas och bibehållas rena, ehuru de äro blandade i samma dufslag och denna omständighet har säkerligen märkligt bidragit till bildandet af nya raser. Jag vill äfven tillägga, att dufvorna hastigt föröka sig, och de odugliga individerna kunna utan olägenhet frånskiljas, emedan de kunna användas till föda. Å andra sidan har man svårt att sammanpara kattor till följd af deras nattliga ströftåg, och ehuru barn och qvinnor gerna hålla en katt, ser man sällan en ny ras uppkomma; de nya raser som vi stundom se äro nästan alltid införda från andra trakter. Ehuru jag icke betviflar att benägenheten för variation är olika bland husdjuren, så härrör dock i min tanke sällsyntheten, eller den totala bristen på skilda raser af katt, åsna, påfågel, gås o. s. v. hufvudsakligen derpå, att intet urval blifvit användt: katten är olämplig dertill till följd af svårigheten att åstadkomma parning; åsnan, emedan den hos oss finnes blott i ringa antal och hos fattigt folk, hvilka föga akta på urvalet; påfågeln, emedan den icke är lätt att uppföda och icke hålles i större antal; gåsen, emedan den värderas blott för sitt kött och sina fjädrar. Gåsen tyckes dock hafva en egendomlig oföränderlig organisation, hvilket icke är förhållandet med åsnan, ty i några delar af Spanien och Förenta staterna har åsnan på ett märkvärdigt sätt blifvit förändrad och förädlad.

Några författare hafva påstått, att höjden af variation hos våra kulturalster snart är uppnådd och sedermera icke kan öfverskridas. Det vore något djerft att försäkra, att gränsen redan blifvit uppnådd i något enda fall, ty nästan alla våra djur och växter hafva blifvit i hög grad förädlade på många vis inom en helt ny period och detta innefattar variation. Det vore lika djerft att försäkra, att karakterer som nu hafva uppnått sin yttersta gräns icke skulle kunna under nya lefnadsförhållanden åter börja variera, sedan de i flera århundraden varit oförändrade. Utan tvifvel skall slutligen såsom Wallace anmärkt en gräns uppnås. Det måste till exempel finnas en gräns för landdjurens snabbhet, då denna bestämmes af den friktion som skall öfvervinnas, af kroppsvigten som bäres och kontraktionsförmågan i muskelfibrerna. Men hvad som för oss är af vigt är, att de domesticerade varieteterna af samma art skilja sig från hvarandra i nästan hvarje karakter som menniskan observerat och utvalt, vida mera än de skilda arterna af samma slägte. Isidore Geoffroy S:t Hilaire har bevisat detta om storleken, och så är förhållandet äfven med färg och sannolikt med hårets längd. Hvad hastigheten beträffar, som beror på många kroppsliga karakterer, var Eclipse vida snabbare än någon hästart, och en draghäst är ojemförligt starkare. Så äfven bland växterna; fröna af de olika bön- och majsvarieteterna skilja sig vida mera i storlek än fröna af de skilda arterna i något slägte af samma två familjer. Samma anmärkning gäller om frukter af plommon och ännu mera om melon, så väl som i otaliga andra analoga fall.

Till slut en sammanfattning af hvad som blifvit sagdt angående uppkomsten af alla våra husdjurs och kulturväxters olika raser. Enligt min åsigt äro de yttre lefnadsvilkoren af högsta vigt på grund af deras inverkan på reproduktionssystemet, emedan de härigenom förorsaka föränderlighet. Det är icke sannolikt, att föränderlighet tillkommer alla organiska varelser såsom en inneboende och nödvändig egenskap under alla omständigheter, såsom några antaga. Föränderlighetens verkningar modifieras genom arf och genom regress; den ledes af många okända lagar, isynnerhet lagen om organernas ömsesidiga beroende. Något måste tillskrifvas de yttre lefnadsvilkorens direkta inverkan, en del organernas större eller mindre användning och derigenom blir resultatet särdeles inveckladt. I några fall har sannolikt kroasering af ursprungligen skilda arter haft en väsentlig andel i bildandet af våra förädlade raser. Om en gång i en trakt uppstått flera förädlade raser, så har deras tillfälliga kroasering under samtidigt användande af urval utan tvifvel mäktigt bidragit till bildande af nya raser; men i min tanke har man mycket öfverdrifvit vigten af varieteternas kroasering så väl beträffande djuren, som de växter, hvilka fortplantas med frö. Hos sådana växter deremot, hvilka tidtals fortplantas med sticklingar, knoppar eller dylikt är vigten af kroasering mellan arter eller varieteter ofantligt stor, emedan odlaren här helt och hållet lemnar utan afseende bastardernas föränderlighet och ofruktsamhet; de fall, då växter ej fortplantas genom frön hafva dock ringa betydelse för oss, emedan deras varaktighet ej är stor. Öfver alla dessa orsaker till föränderlighet står dock i verksamhet enligt min öfvertygelse ett fortsatt accumulativt urval, vare sig detta kommer till användning planmässigt och hastigt eller omedvetet och långsamt, men med större effekt.


